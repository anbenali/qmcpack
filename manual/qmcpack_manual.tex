\documentclass[11pt,letterpaper]{report}
\usepackage{qmcpack_manual}
\usepackage{bibtopic}
\bibliographystyle{ieeetr}
\usepackage{amsmath}
\usepackage{amssymb}
\usepackage{delarray}
\usepackage{algorithmic}
\usepackage{algorithm}
\usepackage{makeidx}
\usepackage{fancyhdr}
\usepackage{xcolor}
\usepackage[colorlinks=true,linkcolor=blue,urlcolor=blue]{hyperref} %for urls
\usepackage{tabularx}
\usepackage{placeins}
\usepackage{caption}
\usepackage{graphicx}

% making listing behave properly
%   with setting below, listings now render correctly
%   copy/paste from pdf is still messed up (is this even possible to fix?)
%     -indentation whitespace is not preserved (needed for Python)
%     -copy/paste can result in mangled text
%     -mangling depends on pdf viewer (it is different for acroread and evince)
%     -verbatim suffers from this also

\usepackage{upquote}  % render ' properly
\usepackage{qmcpack_listings}

% set margins for whole document, lots of wasted space at top and bottom originally
\usepackage[left=1.0in,right=1.0in,top=1.0in,bottom=1.0in]{geometry}




\newcommand{\HRule}{\rule{\linewidth}{0.5mm}}
%% \newcommand{\courier}[1]{{\fontfamily{pcr}\selectfont #1}}

% for markup, as needed
\newcommand{\red}[1]{{\color{red} #1}}
\newcommand{\blue}[1]{{\color{blue} #1}}

% hide or show text relevant to developers
\newcommand{\dev}[1]{#1}
%\newcommand{\dev}[1]{}

% efficiently comment out/hide blocks of text for any purpose
\newcommand{\hide}[1]{}


% control display of instructions in the labs
%   normally one only wants to show the 'workstation' way of running the labs
\newif\ifws
\wstrue
%   for the pdf used during the labs, one wants to show the host supercomputer way
%\wsfalse
%  command for switching inline text (do not wrap verbatim environments with this!)
\ifws
\newcommand{\labsw}[2]{#1}
\else
\newcommand{\labsw}[2]{#2}
\fi


\oddsidemargin 0cm
\evensidemargin 0cm
\textwidth 6.5in


% proper rendering of qmcpack
\newcommand{\qmcpack}{{QMCPACK} } % apparently the trailing whitespace is significant

% mathematics convenience commands
\newcommand{\abs}[1]{\lvert #1 \rvert}
\newcommand{\norm}[1]{\lVert #1 \rVert}
\newcommand{\pnorm}[2]{\lVert #1 \rVert_{#2}}
\newcommand{\mean}[1]{\langle #1 \rangle}
\newcommand{\ket}[1]{\lvert #1 \rangle}
\newcommand{\bra}[1]{\langle #1 \rvert}
\newcommand{\expval}[3]{\bra{#1}#2\ket{#3}}
\newcommand{\expvalh}[3]{\bra{#1}\hat{#2}\ket{#3}}
\newcommand{\overlap}[2]{\langle #1 \lvert #2 \rangle}
\newcommand{\operator}[3]{\ket{#1} #2 \bra{#3}}
\newcommand{\idop}{\hat{\mathbb{1}}}
\newcommand{\bs}{\boldsymbol}
\newcommand{\tr}{\text{tr}} % trace
\newcommand{\grad}{\nabla}
\newcommand{\lap}{\nabla^2}  % laplacian

% urls are were too large
% hyperref gives us an easy way to control that
\renewcommand{\UrlFont}{\ttfamily\small}

% latex itself doesn't give us an easy way to deal with \texttt's font size
% so we need to define this command
\usepackage{letltxmacro}
% https://tex.stackexchange.com/q/88001/5764
\LetLtxMacro\oldttfamily\ttfamily
\DeclareRobustCommand{\ttfamily}{\oldttfamily\csname ttsize\endcsname}
\newcommand{\setttsize}[1]{\def\ttsize{#1}}%

% while \texttt use should be sparing (see contributing.tex) its necessary for the
% QMCPACK input XML spec tables, but those are also nearly too big for the page
\setttsize{\footnotesize}

% We have a huge number of overfull boxes, this adds another pass to
% typesetting a paragraph properly
\setlength{\emergencystretch}{3em}


\begin{document}

\input{title.tex}
\newpage
\tableofcontents
\newpage

\begin{btUnit}

\chapter{Introduction}
\label{chap:introduction}

QMCPACK is an open-source, high-performance electronic structure code
that implements numerous Quantum Monte Carlo (QMC) algorithms. Its main
applications are electronic structure calculations of molecular,
periodic 2D, and periodic 3D solid-state systems. Variational Monte
Carlo (VMC), diffusion Monte Carlo (DMC), and a number of other
advanced QMC algorithms are implemented. By directly solving the
Schrodinger equation, QMC methods offer greater accuracy than methods
such as density functional theory but at a trade-off of much greater
computational expense. Distinct from many other correlated many-body
methods, QMC methods are readily applicable to both bulk
(periodic) and isolated molecular systems.

QMCPACK is written in C++ and is designed with the modularity afforded by
object-oriented programming. It makes extensive use of template
metaprogramming to achieve high computational efficiency. Because of the
modular architecture, the addition of new wavefunctions, algorithms,
and observables is relatively straightforward. For parallelization,
QMCPACK uses a fully hybrid (OpenMP,CUDA)/MPI approach to optimize
memory usage and to take advantage of the growing number of cores per
SMP node or graphical processing units (GPUs) and accelerators. High
parallel and computational efficiencies are achievable on the largest
supercomputers. Finally, QMCPACK uses standard file formats for
input and output in XML and HDF5 to facilitate data exchange.

This manual currently serves as an introduction to the essential features
of QMCPACK and as a guide to installing and running it. Over time this
manual will be expanded to include a fuller introduction to QMC
methods in general and to include more of the specialized features in
QMCPACK.
%Test of the bibliography\cite{CeperleyAlderPRL1980}.

\section{Quickstart and a first QMCPACK calculation}
In case you are keen to get started, this section describes how to quickly
build and run QMCPACK on a standard UNIX or Linux-like system. The
autoconfiguring build system usually works without much fuss on these
systems.  If C++, MPI, BLAS/LAPACK, FFTW, HDF5, and CMake are already
installed, QMCPACK can be built and run within five minutes. For
supercomputers, cross-compilation systems, and other computer clusters,
the build system might require hints on the locations of libraries and
which versions to use, typical of any code; see Chapter
\ref{chap:obtaininginstalling}. Section \ref{sec:installexamples}
includes complete examples of installations for common workstations and supercomputers that you can reuse.

To build QMCPACK:

\begin{enumerate}
\item Download the latest QMCPACK distribution from
  \url{http://www.qmcpack.org}.
\item Untar the archive (e.g., \ishell{tar xvf
    qmcpack\_v1.3.tar.gz}).
\item Check the instructions in the README file.
\item Run CMake in a suitable build directory to configure QMCPACK for
  your system: \ishell{cd qmcpack/build; cmake ..}
\item If CMake is unable to find all needed libraries, see Chapter
  \ref{chap:obtaininginstalling} for instructions and specific build
  instructions for common systems.
\item Build QMCPACK: \ishell{make} or \ishell{make -j 16}; use the latter
  for a faster parallel build on a system using, for example, 16 processes.
\item The QMCPACK executable is \ishell{bin/qmcpack}.
\end{enumerate}

QMCPACK is distributed with examples illustrating different
capabilities. Most of the examples are designed to run quickly with
modest resources. We'll run a short diffusion Monte Carlo calculation
of a water molecule:

\begin{enumerate}
\item Go to the appropriate example directory: \ishell{cd
    ../examples/molecules/H2O}.
\item (Optional) Put the QMCPACK binary on your path:\\ \ishell{export PATH=\$PATH:location-of-qmcpack/build/bin}.
\item Run QMCPACK: \ishell{../../../build/bin/qmcpack simple-H2O.xml} or
  \ishell{qmcpack simple-H2O.xml} if you followed the step above.
\item The run will output to the screen and generate a number of files:
\begin{verbatim}
$ls H2O*
H2O.HF.wfs.xml      H2O.s001.scalar.dat H2O.s002.cont.xml
H2O.s002.qmc.xml    H2O.s002.stat.h5    H2O.s001.qmc.xml
H2O.s001.stat.h5    H2O.s002.dmc.dat    H2O.s002.scalar.dat
\end{verbatim}
\item Partially summarized results are in the standard text files with the
  suffixes scalar.dat and dmc.dat. They are viewable with any standard editor.
\end{enumerate}

If you have Python and matplotlib installed, you can use the
\ishell{qmca} analysis utility to produce statistics and plots of the
data. See Chapter \ref{chap:analyzing} for information on analyzing
QMCPACK data.
\begin{verbatim}
export PATH=$PATH:location-of-qmcpack/nexus/bin 
export PYTHONPATH=$PYTHONPATH:location-of-qmcpack/nexus/library
qmca H2O.s002.scalar.dat         # For statistical analysis of the DMC data
qmca -t -q e H2O.s002.scalar.dat # Graphical plot of DMC energy
\end{verbatim}

The last command will produce a graph as per
Fig. \ref{fig:quick_qmca_dmc_trace}. This shows the average energy of
the DMC walkers at each timestep. In a real simulation we would have
to check equilibration, convergence with walker population, time step, etc.

Congratulations, you have completed a DMC calculation with QMCPACK!

\begin{figure}
  \centering
  \includegraphics[width=10cm]{./figures/quick_qmca_dmc_trace.png}
  \caption{Trace of walker energies produced by the qmca tool for a simple
    water molecule example.}
  \label{fig:quick_qmca_dmc_trace}
\end{figure}

\section{Authors and History}
\label{sec:history}
QMCPACK was initially written by Jeongnim Kim while in the group of
Professor David Ceperley at the University of Illinois at
Urbana-Champaign, with later contributations being made at Oak Ridge National Laboratory (ORNL). Over the years, many others have contributed, particularly
students and researchers in the groups of Professor David Ceperley
and Professor Richard M. Martin, as well as staff at Lawrence Livermore
National Laboratory, Sandia National Laboratories, Argonne National
Laboratory, and ORNL.

Additional developers, contributors, and advisors include
Anouar Benali,
Mark A. Berrill,  
David M. Ceperley, 
Simone Chiesa,
Raymond C. III Clay,
Bryan Clark,
Kris T. Delaney,
Kenneth P. Esler,
Paul R. C. Kent,
Jaron T. Krogel,
Ying Wai Li,
Ye Luo,
Jeremy McMinis,
Miguel A. Morales,
William D. Parker,
Nichols A. Romero,
Luke Shulenburger,
Norman M. Tubman,
and Jordan E. Vincent.

If you should be added to this list, please let us know.

Development of QMCPACK has been supported financially by
several grants, including the following:

\begin{itemize}
\item ``Network for ab initio many-body methods: development, education
  and training'' supported through the Predictive
  Theory and Modeling for Materials and Chemical Science program by
  the U.S. Department of Energy Office of Science, Basic Energy
  Sciences
\item ``QMC Endstation,'' supported by Accelerating Delivery of Petascale
  Computing Environment at the DOE Leadership Computing Facility at
  ORNL
\item PetaApps, supported by the US National Science
  Foundation
\item Materials Computation Center (MCC), supported by the
  US National Science Foundation
\end{itemize}


\section{Support and Contacting the Developers}
\label{sec:support}

Questions about installing, applying, or extending QMCPACK can be
posted on the QMCPACK Google group at
\url{https://groups.google.com/forum/#!forum/qmcpack}. You may also
email any of the developers, but we recommend checking the group
first. Particular attention is given to any problem reports.

\section{Performance}
\label{sec:performance}

QMCPACK implements modern Monte Carlo (MC) algorithms, is highly parallel,
and is written using very efficient code for high per-CPU or on-node performance. In particular, the code is highly vectorizable,
giving high performance on modern central processing units (CPUs) and GPUs. We believe QMCPACK
delivers performance either comparable to or better than other QMC
codes when similar calculations are run, particularly for the most
common QMC methods and for large systems. If you find a calculation where this is not the
case, or you simply find performance slower than expected, please post on the Google
group or contact one of the developers. These reports are valuable. If your calculation is
sufficiently mainstream we will optimize QMCPACK to improve
the performance.

\section{Open source license}
\label{sec:license}

QMCPACK is distributed under the University of Illinois at
Urbana-Champaign/National Center for Supercomputing Applications (UIUC/NCSA) Open
Source License. 

\begin{verbatim}
		  University of Illinois/NCSA Open Source License

Copyright (c) 2003, University of Illinois Board of Trustees.
All rights reserved.

Developed by:   
  Jeongnim Kim
  Condensed Matter Physics,
  National Center for Supercomputing Applications, University of Illinois
  Materials computation Center, University of Illinois
  http://www.mcc.uiuc.edu/qmc/

Permission is hereby granted, free of charge, to any person obtaining a
copy of this software and associated documentation files (the
``Software''), to deal with the Software without restriction, including
without limitation the rights to use, copy, modify, merge, publish,
distribute, sublicense, and/or sell copies of the Software, and to
permit persons to whom the Software is furnished to do so, subject to
the following conditions:

        * Redistributions of source code must retain the above copyright 
          notice, this list of conditions and the following disclaimers.
        * Redistributions in binary form must reproduce the above copyright 
          notice, this list of conditions and the following disclaimers in 
          the documentation and/or other materials provided with the 
          distribution.
        * Neither the names of the NCSA, the MCC, the University of Illinois, 
          nor the names of its contributors may be used to endorse or promote 
          products derived from this Software without specific prior written 
          permission.

THE SOFTWARE IS PROVIDED "AS IS", WITHOUT WARRANTY OF ANY KIND, EXPRESS
OR IMPLIED, INCLUDING BUT NOT LIMITED TO THE WARRANTIES OF MERCHANTABILITY, 
FITNESS FOR A PARTICULAR PURPOSE AND NONINFRINGEMENT. IN NO EVENT SHALL 
THE CONTRIBUTORS OR COPYRIGHT HOLDERS BE LIABLE FOR ANY CLAIM, DAMAGES OR 
OTHER LIABILITY, WHETHER IN AN ACTION OF CONTRACT, TORT OR OTHERWISE, 
ARISING FROM, OUT OF OR IN CONNECTION WITH THE SOFTWARE OR THE USE OR 
OTHER DEALINGS WITH THE SOFTWARE.
\end{verbatim}

Copyright is generally believed to remain with the authors of the
individual sections of code. See the various notations in the source code as
well as the code history.

\section{Contributing to QMCPACK}
\label{sec:contributing}

QMCPACK is fully open source, and we welcome contributions. If you are
planning a development, early discussions are encouraged. Please
post on the QMCPACK Google group or contact the developers. We can tell you whether anyone else is working on a similar feature or whether
any related work has been done in the past.  Credit for your
contribution can be obtained, for example, through citation of a paper or by
becoming one of the authors on the next version of the standard
QMCPACK reference citation.

A guide to developing for QMCPACK, including instructions on how to
work with GitHub and make pull requests (contributions) to the main
source are listed on the QMCPACK GitHub wiki:
\url{https://github.com/QMCPACK/qmcpack/wiki}.

Contributions are made under the same license as QMCPACK, the
UIUC/NCSA open source license. If this is problematic, please discuss
with a developer.

Please note the following guidelines for contributions:
\begin{itemize}
\item Additions should be fully synchronized with the latest release
  version and ideally the latest develop branch on github. Merging of code
  developed on older versions is error prone.
\item Code should be cleanly formatted, commented, portable, and accessible to
  other programmers. That is, if you need to use any clever tricks, add a comment
  to note this, why the trick is needed, how it works, etc. Although we like
  high performance, ease of maintenance and accessibility are also
  considerations.
\item Comment your code. You are not only writing it for the compiler
  for also for other humans! (We know this is a repeat of the previous
  point, but it is important enough to repeat.)
\item Write a brief description of the method, algorithms, and inputs and outputs
  suitable for inclusion in this manual.
\item Develop some short tests that exercise the
  functionality that can be used for validation and for examples. We
  can help with this and their integration into the test system.
\end{itemize}

\section{QMCPACK Roadmap}
\label{sec:roadmap}

A general outline of the QMCPACK roadmap is given in Sections 1.7.1 and 1.7.2 . Suggestions
for improvements are welcome, particularly those that would facilitate new
scientific applications. For example, if an interface to a particular
quantum chemical or density functional code would help, this would be
given strong consideration.

\subsection{Code}

We will continue to improve the accessibility and usability of
QMCPACK through combinations of more convenient input parameters, improved
workflow, integration with more quantum chemical and density
functional codes, and a wider range of examples.

In terms of methodological development, we expect to significantly
increase the range of QMC algorithms in QMCPACK in the near future.

Computationally, we are porting QMCPACK to the next generation of
supercomputer systems. The internal changes required to run efficiently on these
systems are expected to benefit \emph{all} platforms due
to improved vectorization, cache utilization, and memory performance.

\subsection{Documentation}

This manual describes the core features of QMCPACK that are
required for routine research calculations, i.e., the VMC and DMC
methods, how to obtain and optimize trial wavefunctions, and simple
observables. Over time this manual will be expanded to include a
broader introduction to QMC methods and to describe more features of
the code.

Because of its history as a research code, QMCPACK contains a variety of
additional QMC methods, trial wavefunction forms, potentials, etc.,
that, although not critical, might be very useful for specialized
calculations or particular material or chemical systems. These
``secret features'' (every code has these) are not actually secret but
simply lack descriptions, example inputs, and tests. You are
encouraged to browse and read the source code to find them. New
descriptions will be added over time but can also be prioritized and
added on request (e.g., if a specialized Jastrow factor would help or
a historical Jastrow form is needed for benchmarking).



\input{features}
\chapter{Obtaining, installing and validating QMCPACK}
\label{chap:obtaininginstalling}

This chapter describes how to obtain, build and validate QMCPACK. This process is designed to be as simple as
possible and should be no harder than building a modern plane-wave density
functional theory code such as Quantum ESPRESSO, QBox, or
VASP. Parallel builds enable a complete
compilation in under 2 minutes on a fast multicore system. If you
are unfamiliar with building codes we suggest working with your system
administrator to install QMCPACK.

\section{Installation steps}
To install QMCPACK, follow the steps listed below. Full details of
each step are given in the referenced sections.
\begin{enumerate}
\item Download the source code, Sections \ref{sec:obrelease} or \ref{sec:obdevelopment}.
\item Verify that you have the required compilers, libraries and tools
  installed, Section \ref{sec:prerequisites}.
\item Run the cmake configure step and build with make, Section
  \ref{sec:cmake} and \ref{sec:cmakequick}. Some examples for common
  systems are given in Section \ref{sec:installexamples}.
\item Run the tests to verify QMCPACK, Section \ref{sec:testing}.
\item Build the ppconvert utility in QMCPACK, Section \ref{sec:buildppconvert}.
\item Download and patch Quantum ESPRESSO. This patch adds the
  pw2qmcpack utility, Section \ref{sec:buildqe}.
\end{enumerate}

Hints for high performance are in Section \ref{sec:buildperformance}. Troubleshooting suggestions are in Section \ref{sec:troubleshoot}.

Note that there are two different QMCPACK executables that can be
produced: the general one, which is the default, and the ``complex''
version which support periodic calculations at arbitrary twist angles and
k-points. This second version is enabled via a cmake configuration
parameter, see Section \ref{sec:cmakeoptions}. The general version
only supports wavefunctions that can be made real. If you run a
calculation that needs the complex version, QMCPACK will stop and inform you.

\section{Obtaining the latest release version}
\label{sec:obrelease}
Major releases of QMCPACK are distributed from
\url{http://www.qmcpack.org}. These releases undergo the most testing. Unless there are
specific reasons we encourage all production calculations to use the
latest release versions.

Releases are usually compressed tar files indicating the version
number, date, and often the source code revision control number
corresponding to the release.

\begin{itemize}
\item Download the latest QMCPACK distribution from \url{http://www.qmcpack.org}.
\item Untar the archive, e.g., \texttt{tar xvf qmcpack\_v1.3.tar.gz}
\end{itemize}

Releases can also be obtained from the 'master' branch of the QMCPACK
git repository, similar to obtaining the development version (Sec. \ref{sec:obdevelopment}).

\section{Obtaining the latest development version}
\label{sec:obdevelopment}
The most recent development version of QMCPACK can be obtained anonymously via
\begin{verbatim}
git clone https://github.com/QMCPACK/qmcpack.git
\end{verbatim}
Once checked-out,
updates can be made via the standard \texttt{git pull}.

The 'develop' branch of the git repository contains the day-to-day development source
with the latest updates, bugfixes etc. This may be useful
for updates to the build system to support new machines, for support
of the latest versions of Quantum ESPRESSO, or for updates to the
documentation.  Note that the development version may not be fully
consistent with the online documentation.  We attempt to keep
the development version fully working. However, please be sure to run the tests and
compare with previous release versions before using for any serious
calculations. We try to keep bugs out, but occasionally they crawl
in! Reports of any breakages are appreciated.

\section{Prerequisites}
\label{sec:prerequisites}
The following are required to build QMCPACK. For workstations, these are available via the standard
package manager. On shared supercomputers this software is usually
installed by default and is often
access via a modules environment - check your system
documentation.

\textbf{Use of the latest versions of all compilers and libraries is
strongly encouraged}, but not absolutely essential. Generally newer versions are faster - see
Section \ref{sec:buildperformance} for performance suggestions.

\begin{itemize}
\item C/C++ compilers such as GNU, Clang, Intel, IBM XL. C++ compilers
  are required to support C++11 standard. Use of recent (``current
  year version'') compilers is strongly encouraged.
\item MPI library such as OpenMPI \url{http://open-mpi.org} or vendor
  optimized MPI.
\item BLAS/LAPACK, numerical and linear algebra libraries. Use
  platform-optimized libraries where available, such as Intel MKL.
  ATLAS or other optimized open-source libraries may also be used
  \url{http://math-atlas.sourceforge.net}
\item CMake, build utility, \url{http://www.cmake.org}
\item Libxml2, XML parser, \url{http://xmlsoft.org}
\item HDF5, portable I/O library, \url{http://www.hdfgroup.org/HDF5/}. Good performance at large scale requires parallel version $>=$ 1.10.
\item BOOST, peer-reviewed portable C++ source libraries, \url{http://www.boost.org}
\item FFTW, FFT library, \url{http://www.fftw.org/}
\end{itemize}

To build the GPU accelerated version of QMCPACK an installation of
NVIDIA CUDA development tools is required. Ensure that this is
compatible with the C and C++ compiler versions you plan to
use. Supported versions are included in the NVIDIA release notes.

Many of the utilities provided with QMCPACK use python (v2). The numpy
and matplotlib libraries are required for full functionality.

Note that the standalone einspline library used by previous versions of QMCPACK
is no longer required. A more optimized version is included
inside. The standalone version should \emph{not} be on any standard
search paths because conflicts between the old and new include files
can result.

\section{Building with CMake}
\label{sec:cmake}
The build system for QMCPACK is based on CMake.  It will autoconfigure
based on the detected compilers and libraries. The most recent
version of CMake has the best detection for the greatest variety of
systems - at the time of writing this means CMake 3.4.3. The much
older CMake 2.8 is known to work, but might not work optimally on your system.

Previously QMCPACK made extensive use of toolchains, but the build system
has since been updated to eliminate the use of toolchain files for
most cases.  The build system is verified to work with GNU, Intel, and IBM XLC
compilers.  Specific compile options can be specified either through
specific environmental or CMake variables.  When the libraries are
installed in standard locations, e.g., /usr, /usr/local, there is no
need to set environmental or cmake variables for the packages.

\subsection{Quick build instructions (try first)}
\label{sec:cmakequick}

If you are feeling lucky and are on a standard UNIX-like system such
as a Linux workstation, the following might quickly give a
working QMCPACK:

The safest quick build option is to specify the C and C++ compilers
through their MPI wrappers. Here we use Intel MPI and Intel
compilers. Move to the build directory, run cmake and make
\begin{verbatim}
cd build
cmake -DCMAKE_C_COMPILER=mpiicc -DCMAKE_CXX_COMPILER=mpiicpc ..
make -j 8
\end{verbatim}
You can increase the ``8'' to the number of cores on your system for
faster builds. Substitute mpicc and mpicxx or other wrapped compiler names to suit
  your system. e.g. With OpenMPI use
\begin{verbatim}
cd build
cmake -DCMAKE_C_COMPILER=mpicc -DCMAKE_CXX_COMPILER=mpicxx ..
make -j 8
\end{verbatim}

If you are feeling particularly lucky, you can skip the compiler specification:
\begin{verbatim}
cd build
cmake ..
make -j 8
\end{verbatim}

The complexities of modern computer hardware and software systems are
such that you should check that the autoconfiguration system has made
good choices and picked optimized libraries and compiler settings
before doing significant production. i.e. Check the details below. We
give examples for a number of common systems in Section \ref{sec:installexamples}.

\subsection{Environment variables}
\label{sec:envvar}
A number of environmental variables affect the build.  In particular
they can control the default paths for libraries, the default
compilers, etc.  The list of environmental variables is given below:
\begin{verbatim}
CXX              C++ compiler
CC               C Compiler
MKL_HOME         Path for MKL
LIBXML2_HOME     Path for libxml2
HDF5_ROOT        Path for HDF5
BOOST_ROOT       Path for Boost
FFTW_HOME        Path for FFTW
\end{verbatim}

\subsection{Configuration options}
\label{sec:cmakeoptions}
In addition to reading the environmental variables, CMake provides a
number of optional variables that can be set to control the build and
configure steps.  When passed to CMake, these variables will take
precedent over the environmental and default variables.  To set them
add -D FLAG=VALUE to the configure line between the cmake command and
the path to the source directory.

\begin{itemize}
\item  Key QMCPACK build options
\begin{verbatim}
QMC_CUDA            Enable CUDA and GPU acceleration (1:yes, 0:no)
QMC_COMPLEX         Build the complex (general twist/k-point) version (1:yes, 0:no)
QMC_MIXED_PRECISION Build the mixed precision (mixing double/float) version
                    (1:yes (GPU default), 0:no (CPU default)).
                    The CPU support is experimental.
                    Use float and double for base and full precision.
                    The GPU support is quite mature.
                    Use always double for host side base and full precision
                    and use float and double for CUDA base and full precision.
ENABLE_SOA          (Experimental) Enable CPU optimization based on Structure-
                    of-Array (SoA) datatypes (1:yes, 0:no (default)).

\end{verbatim}
  \item General build options
\begin{verbatim}
CMAKE_BUILD_TYPE   A variable which controls the type of build
                   (defaults to Release). Possible values are:
                   None (Do not set debug/optmize flags, use
                   CMAKE_C_FLAGS or CMAKE_CXX_FLAGS)
                   Debug (create a debug build)
                   Release (create a release/optimized build)
                   RelWithDebInfo (create a release/optimized build with debug info)
                   MinSizeRel (create an executable optimized for size)
CMAKE_C_COMPILER   Set the C compiler
CMAKE_CXX_COMPILER Set the C++ compiler
CMAKE_C_FLAGS      Set the C flags.  Note: to prevent default
                   debug/release flags from being used, set the CMAKE_BUILD_TYPE=None
                   Also supported: CMAKE_C_FLAGS_DEBUG,
                   CMAKE_C_FLAGS_RELEASE, and CMAKE_C_FLAGS_RELWITHDEBINFO
CMAKE_CXX_FLAGS    Set the C++ flags.  Note: to prevent default
                   debug/release flags from being used, set the CMAKE_BUILD_TYPE=None
                   Also supported: CMAKE_CXX_FLAGS_DEBUG,
                   CMAKE_CXX_FLAGS_RELEASE, and CMAKE_CXX_FLAGS_RELWITHDEBINFO
\end{verbatim}
\item Additional QMCPACK build options
\begin{verbatim}
QMC_INCLUDE         Add extra include paths
QMC_EXTRA_LIBS      Add extra link libraries
QMC_BUILD_STATIC    Add -static flags to build
QMC_DATA            Specify data directory for QMCPACK (currently
                    unused, but likely to be used for future performance tests)
\end{verbatim}
\item libxml2 related
\begin{verbatim}
Libxml2_INCLUDE_DIRS  Specify include directories for libxml2
Libxml2_LIBRARY_DIRS  Specify library directories for libxml2
\end{verbatim}
 \item HDF5 related
\begin{verbatim}
ENABLE_PHDF5    1(default)/0, enables/disable parallel collective IO.
\end{verbatim}
 \item FFTW related
\begin{verbatim}
FFTW_INCLUDE_DIRS   Specify include directories for FFTW
FFTW_LIBRARY_DIRS   Specify library directories for FFTW
\end{verbatim}
 \item CTest related
\begin{verbatim}
MPIEXEC   Specify the mpi wrapper, e.g. srun, aprun, mpirun, etc.
MPIEXEC_NUMPROC_FLAG   Specify the number of mpi processes flag, e.g. "-n", "-np", etc.

\end{verbatim}
\end{itemize}

\subsection{Configure and build using cmake and make}
To configure and build QMCPACK, move to build directory, run cmake and make
\begin{verbatim}
cd build
cmake ..
make -j 8
\end{verbatim}

As you will have gathered, cmake encourages ``out of source'' builds,
where all the files for a specific build configuration reside in their
own directory separate from the source files. This allows multiple
builds to be created from the same source files which is very useful
where the filesystem is shared between different systems. You can also
build versions with different settings (e.g. QMC\_COMPLEX) and
different compiler settings. The build directory does not have to be
called build - use something descriptive such as build\_machinename or
build\_complex. The ``..'' in the cmake line refers to the directory
containing CMakeLists.txt. Update the ``..'' for other build
directory locations.

\subsection{Example configure and build}
\begin{itemize}
\item Set the environments (the examples below assume bash, Intel compilers and MKL library)
\begin{verbatim}
export CXX=icpc
export CC=icc
export MKL_HOME=/usr/local/intel/mkl/10.0.3.020
export LIBXML2_HOME=/usr/local
export HDF5_ROOT=/usr/local
export BOOST_ROOT=/usr/local/boost
export FFTW_HOME=/usr/local/fftw
\end{verbatim}

\item Move to build directory, run cmake and make
\begin{verbatim}
cd build
cmake -D CMAKE_BUILD_TYPE=Release ..
make -j 8
\end{verbatim}
\end{itemize}

\subsection{Build scripts}
It is recommended to create a helper script that contains the
configure line for CMake.  This is particularly useful when avoiding
environmental variables, packages are installed in custom locations,
or if the configure line is long or complex.  In this case it is also
recommended to add "rm -rf CMake*" before the configure line to remove
existing CMake configure files to ensure a fresh configure each time
that the script is called. Deleting all the files in the build
directory is also acceptable. If you do so we recommend to add some sanity
checks in case the script is run from the wrong directory, e.g.,
checking for the existence of some QMCPACK files.

Some build script examples for different systems are given in the
config directory. For example, on Cray systems these scripts might
load the appropriate modules to set the appropriate programming
environment, specific library versions etc.

An example script build.sh is given below. It is much more complex
than usually needed for comprehensiveness:

\begin{verbatim}
export CXX=mpic++
export CC=mpicc
export ACML_HOME=/opt/acml-5.3.1/gfortran64
export HDF5_ROOT=/opt/hdf5
export BOOST_ROOT=/opt/boost

rm -rf CMake*

cmake                                                \
  -D CMAKE_BUILD_TYPE=Debug                         \
  -D Libxml2_INCLUDE_DIRS=/usr/include/libxml2      \
  -D Libxml2_LIBRARY_DIRS=/usr/lib/x86_64-linux-gnu \
  -D FFTW_INCLUDE_DIRS=/usr/include                 \
  -D FFTW_LIBRARY_DIRS=/usr/lib/x86_64-linux-gnu    \
  -D QMC_EXTRA_LIBS="-ldl ${ACML_HOME}/lib/libacml.a -lgfortran" \
  -D QMC_DATA=/projects/QMCPACK/qmc-data            \
  ..
\end{verbatim}

\subsection{Using vendor optimized numerical libraries, e.g. Intel MKL}

Although QMC does not make extensive use of linear algebra, use of
vendor optimized libraries is strongly recommended for highest
performance. BLAS routines are used in the Slater determinant update, the VMC wavefunction optimizer
and to apply orbital coefficients in local basis calculations. Vectorized
math functions are also beneficial, e.g. for the phase factor
computation in solid state calculations. CMake is generally successful
in finding these libraries, but specific combinations can require
additional hints, as described below:

\subsubsection{Using Intel MKL with non-Intel compilers}

To use Intel MKL with, e.g. an MPICH wrapped gcc:
\begin{verbatim}
cmake \
 -DCMAKE_C_COMPILER=mpicc -DCMAKE_CXX_COMPILER=mpicxx \
 -DBLA_VENDOR=Intel10_64lp_seq -DCMAKE_PREFIX_PATH=$MKLROOT/lib \
 ..
\end{verbatim}

MKLROOT is the directory containing the MKL binary, examples, and lib
directories (etc.) and is often /opt/intel/mkl

\subsection{Cross compiling}
Cross-compiling is often difficult but is required on supercomputers
with distinct host and compute processor generations or architectures.
QMCPACK tried to do its best with CMake to facilitate cross-compiling.

\begin{itemize}
  \item On a machine using Cray programming environment, we rely on
      compiler wrappers provided by Cray to set architecture specific
      flags correctly. The CMake configure log should indicate that a
      Cray machine was detected.
  \item If not on a Cray machine, by default we assume building for
    the host architecture. e.g. -xHost is added for the Intel compiler
    and -march=native is added for GNU/Clang compilers.
  \item If -x/-ax or -march is specified by the user in CMAKE\_C\_FLAGS and CMAKE\_CXX\_FLAGS,
    we respect user's intention and do not add any architecture specific flags.
\end{itemize}

The general strategy for cross-compiling should therefore be to
manually set CMAKE\_C\_FLAGS and CMAKE\_CXX\_FLAGS for the target
archicture. Using \texttt{make VERBOSE=1} is a useful way to check the
final compilation options.  If on a Cray machine, selection of the
appropriate programming environment should be sufficient.

\section{Installation instructions for common workstations and
  supercomputers}
\label{sec:installexamples}

This section describes how to build QMCPACK on various common systems
including multiple Linux distributions, Apple OS X, and various
supercomputers. The examples should serve as good starting points for
building QMCPACK on similar machines. For example, the software
environment on modern Crays is very consistent. Note that updates to
operating systems and system software may require small modifications
to these recipes. See Section \ref{sec:buildperformance} for key
points to check to obtain highest performance and
Section \ref{sec:troubleshoot} for troubleshooting hints.

\subsection{Installing on Ubuntu Linux or other apt-get based distributions}
\label{sec:buildubuntu}

The following is designed to obtain a working QMCPACK build on e.g. a
student laptop, starting from a basic Linux installation with none of
the developer tools installed. Fortunately, all the required packages
are available in the default repositories making for a quick
installation. Note that for convenience we use a generic BLAS. For
production a platform optimized BLAS should be used.

\begin{verbatim}
apt-get cmake g++ openmpi-bin libopenmpi-dev libboost-dev
apt-get libatlas-base-dev liblapack-dev libhdf5-dev libxml2-dev fftw3-dev
export CXX=mpiCC
cd build
cmake ..
make -j 8
ls -l bin/qmcpack
\end{verbatim}

For qmca and other tools to function, we install some python libraries:
\begin{verbatim}
sudo apt-get install python-numpy python-matplotlib
\end{verbatim}

\subsection{Installing on CentOS Linux or other yum based distributions}

The following is designed to obtain a working QMCPACK build on e.g. a
student laptop, starting from a basic Linux installation with none of
the developer tools installed. CentOS 7 (Red Hat compatible) is using
gcc 4.8.2. The installation is only complicated by the need to install
another repository to obtain HDF5 packages which are not available by
default. Note that for convenience we use a generic BLAS. For
production a platform optimized BLAS should be used.

\begin{verbatim}
sudo yum install make cmake gcc gcc-c++ openmpi openmpi-devel fftw fftw-devel \
                  boost boost-devel libxml2 libxml2-devel
sudo yum install blas-devel lapack-devel atlas-devel
module load mpi
\end{verbatim}

To setup repoforge as a source for the HDF5 package, go to
\url{http://repoforge.org/use} . Install the appropriate up to date
release package for your OS. By default the CentOS Firefox will offer
to run the installer. The CentOS 6.5 settings were still usable for HDF5 on
CentOS 7 in 2016, but use CentOS 7 versions when they become
available.

\begin{verbatim}
sudo yum install hdf5 hdf5-devel
\end{verbatim}

To build QMCPACK
\begin{verbatim}
module load mpi/openmpi-x86_64
which mpirun
# Sanity check; should print something like   /usr/lib64/openmpi/bin/mpirun
export CXX=mpiCC
cd build
cmake ..
make -j 8
ls -l bin/qmcpack
\end{verbatim}

\subsection{Installing on Mac OS X using Macports}
These instructions assume a fresh installation of macports
and use the gcc 6.1 compiler. Older versions are fine, but it is vital to ensure
matching compilers and libraries are used for all
packages and to force use of what is installed in /opt/local.  Performance should be very reasonable.
Note that we utilize the Apple provided Accelerate framework for
optimized BLAS.

Follow the Macports install instructions \url{https://www.macports.org/}

\begin{itemize}
\item Install Xcode and the Xcode Command Line Tools
\item Agree to Xcode license in Terminal: sudo xcodebuild -license
\item Install MacPorts for your version of OS X
\end{itemize}


Install the required tools:

\begin{verbatim}
sudo port install gcc6
sudo port select gcc mp-gcc6
sudo port install openmpi-devel-gcc6
sudo port select --set mpi openmpi-devel-gcc61-fortran

sudo port install fftw-3 +gcc6
sudo port install libxml2
sudo port install cmake
sudo post install boost +gcc6
sudo port install hdf5 +gcc6

sudo port select --set python python27
sudo port install py27-numpy +gcc6
sudo port install py27-matplotlib  #For graphical plots with qmca
\end{verbatim}

QMCPACK build:
\begin{verbatim}
cd build
cmake -DCMAKE_C_COMPILER=mpicc -DCMAKE_CXX_COMPILER=mpiCXX ..
make -j 6 # Adjust for available core count
ls -l bin/qmcpack
\end{verbatim}

Cmake should pickup the versions of HDF5, libxml (etc.) installed in
/opt/local by macports. If you have other copies of these libraries
installed and wish to force use of a specific version, use the
environment variables detailed in Sec. \ref{sec:envvar}.

This recipe was verified on 1 July 2016 on a Mac running OS X 10.11.5
``El Capitain''.

\subsection{Installing on Mac OS X using Homebrew (brew)}
Homebrew is a package manager for OS X that provides a convenient
route to install all the QMCPACK dependencies. The
following recipe will install the latest available versions of each
package. This was successfully tested under OS X 10.12 ``Sierra'' in December 2017. Note that it is necessary to build the MPI software from
source to use the brew-provided gcc instead of Apple CLANG.

\begin{enumerate}
\item Install Homebrew from \url{http://brew.sh/}
\begin{verbatim}
/usr/bin/ruby -e "$(curl -fsSL
    https://raw.githubusercontent.com/Homebrew/install/master/install)"
\end{verbatim}

\item Install the prerequisites
\begin{verbatim}
brew install gcc # installs gcc 7.2.0 on 2017-12-19
export HOMEBREW_CXX=g++-7
export HOMEBREW_CC=gcc-7
brew install mpich2 --build-from-source
# Build from source required to use homebrew compiled compilers as
# opposed to Apple CLANG. Check "mpicc -v" indicates Homebrew gcc
brew install cmake
brew install fftw
brew install boost
brew install homebrew/science/hdf5
#Note: Libxml2 is not required via brew since OS X already includes it.
\end{verbatim}
\item Configure and build QMCPACK
\begin{verbatim}
cmake -DCMAKE_C_COMPILER=/usr/local/bin/mpicc \
      -DCMAKE_CXX_COMPILER=/usr/local/bin/mpicxx ..
make -j 12
\end{verbatim}
\item Run the short tests. When MPICH is used for the first time, OS
  X will request approval of the network connection for each executable.
\begin{verbatim}
ctest -R short
\end{verbatim}
\end{enumerate}

\subsection{Installing on ANL ALCF Mira/Cetus IBM Blue Gene/Q}
\label{sec:buildbgq}
Mira/Cetus is a Blue Gene/Q supercomputer at Argonne National Laboratory's Argonne Leadership Computing Facility (ANL ALCF).
Mira has 49152 compute nodes and each node has a 16-core PowerPC A2 processor with 16 GB DDR3 memory.
Due to the fact that the login nodes and the compute nodes have different processors with distinct instruction sets,
cross-compiling is required on this platform. See details about using Blue Gene/Q at \url{http://www.alcf.anl.gov/user-guides/compiling-linking}.
On Mira, compilers are loaded via softenv and users need to add +mpiwrapper-bgclang and +cmake in \$HOME/.soft.
In order to build QMCPACK, a toolchain file is provided for setting up CMake and the cmake command should be executed twice.
\textbf{BGClang is required for C++11 support. IBM XL C/C++ compiler should not be used.}

\begin{verbatim}
cd build
cmake -DCMAKE_TOOLCHAIN_FILE=../config/BGQ_Clang++11_ToolChain.cmake ..
cmake -DCMAKE_TOOLCHAIN_FILE=../config/BGQ_Clang++11_ToolChain.cmake ..
make -j 16
ls -l bin/qmcpack
\end{verbatim}

\subsection{Installing on ALCF Theta, Cray XC40}
Theta is a 9.65 petaflops system manufactured by Cray with 3624 compute nodes.
Each node features a second-generation Intel Xeon Phi 7230 processor and 192 GB DDR4 RAM.

\begin{verbatim}
export CRAYPE_LINK_TYPE=dynamic
module unload cray-libsci
module load cray-hdf5-parallel
export BOOST_ROOT=/soft/libraries/boost/1.64.0/intel
cmake ..
make -j 24
ls -l bin/qmcpack
\end{verbatim}

\subsection{Installing on ORNL OLCF Titan Cray XK7 (NVIDIA GPU
  accelerated)}
\label{sec:titanbuildgpu}
Titan is a GPU accelerated supercomputer at Oak Ridge National
Laboratory's  Oak Ridge Leadership Computing Facility  (ORNL OLCF). Each
compute node has a 16 core AMD 2.2GHz Opteron 6274 (Interlagos) and an
NVIDIA Kepler accelerator. The standard Cray software environment is
available, with libraries accessed via modules. The only extra
settings required to build the GPU version are the cudatoolkit module
and specifying -DQMC\_CUDA=1 on the cmake configure line.

Note that on Crays the compiler wrappers ``CC'' and ``cc'' are
used. The build system checks for these and does not (should not) use
the compilers directly.

\begin{verbatim}
module swap PrgEnv-pgi PrgEnv-gnu # Use gnu compilers
module load cudatoolkit           # CUDA for GPU build
module load cray-hdf5
module load cmake
module load fftw
export FFTW_HOME=$FFTW_DIR/..
module load boost
mkdir build_titan_gpu
cd build_titan_gpu
cmake -DQMC_CUDA=1 ..             # Must enable CUDA capabilities
make -j 8
ls -l bin/qmcpack
\end{verbatim}

\subsection{Installing on ORNL OLCF Titan Cray XK7 (CPU version)}
As noted in Section\ref{sec:titanbuildgpu} for the GPU, building on
Crays requires only loading the appropriate library modules.

\begin{verbatim}
export CRAYPE_LINK_TYPE=dynamic
module swap PrgEnv-pgi PrgEnv-gnu # Use gnu compilers
module unload cudatoolkit         # No CUDA for CPU build
module load cray-hdf5
module load cmake
module load fftw
export FFTW_HOME=$FFTW_DIR/..
module load boost
mkdir build_titan_cpu
cd build_titan_cpu
cmake ..
make -j 8
ls -l bin/qmcpack
\end{verbatim}

\subsection{Installing on ORNL OLCF Eos Cray XC30}
Eos is a Cray XC30 with 16 core Intel Xeon E5-2670 processors connected
by the Aries interconnect. The build process is identical to Titan,
except that we use the default Intel programming environment. This is
usually preferred to GNU.
\begin{verbatim}
export CRAYPE_LINK_TYPE=dynamic
module unload cray-libsci
module load cray-hdf5
module load cmake
module load fftw
export FFTW_HOME=$FFTW_DIR/..
module load boost
mkdir build_eos
cd build_eos
cmake ..
make -j 8
ls -l bin/qmcpack
\end{verbatim}

\subsection{Installing on ORNL OLCF SummitDev}
SummitDev is the development cluster for the next GPU accelerated
supercomputer Summit at Oak Ridge National Laboratory's
Leadership Computing Facility  (ORNL OLCF). It has IBM Power8 CPUs and NVIDIA Pascal GPUs.

\subsubsection{Building QMCPACK}
Please note that these build instructions are preliminary as the
software environment is subject to change. QMCPACK can be build with the following commands:
\begin{verbatim}
module load xl
module load essl
module load netlib-lapack
module load hdf5/1.8.18
module load fftw
export FFTW_HOME=$OLCF_FFTW_ROOT
module load python
module load cmake
module load boost
module load cuda
mkdir build_summitdev
cd build_summitdev
cmake -DCMAKE_C_COMPILER="mpixlc" \
      -DCMAKE_CXX_COMPILER="mpixlC" \
      -DBUILD_LMYENGINE_INTERFACE=0 \
      -DQMC_CUDA=1 \
      -DCUDA_ARCH="sm_60" \
      ..
make -j 8
ls -l bin/qmcpack
\end{verbatim}

\subsection{Installing on NERSC Edison Cray XC30}

Edison is a Cray XC30 with dual 12-core Intel "Ivy Bridge" nodes
installed at NERSC. The build settings are identical to eos.

\begin{verbatim}
export CRAYPE_LINK_TYPE=dynamic
module unload cray-libsci
module load boost
module load cmake
module load libxml2
module load cray-hdf5-parallel
cmake ..
make -j 8
ls -l bin/qmcpack
\end{verbatim}
When the above was tested on 15 September 2017, the following module and
software versions were present:
\begin{verbatim}
qmcpack@edison04:trunk> module list
urrently Loaded Modulefiles:
  1) modules/3.2.10.6                              14) rca/2.2.11-6.0.4.0_13.2__g84de67a.ari
  2) intel/17.0.2.174                              15) atp/2.1.1
  3) craype-network-aries                          16) PrgEnv-intel/6.0.4
  4) craype/2.5.12.3                               17) craype-ivybridge
  5) udreg/2.3.2-6.0.4.0_12.2__g2f9c3ee.ari        18) cray-shmem/7.6.0
  6) ugni/6.0.14-6.0.4.0_14.1__ge7db4a2.ari        19) cray-mpich/7.6.0
  7) pmi/5.0.12                                    20) altd/2.0
  8) dmapp/7.1.1-6.0.4.0_46.2__gb8abda2.ari        21) darshan/3.1.4
  9) gni-headers/5.0.11-6.0.4.0_7.2__g7136988.ari  22) boost/1.63
 10) xpmem/2.2.2-6.0.4.0_3.1__g43b0535.ari         23) cmake/3.8.1
 11) job/2.2.2-6.0.4.0_8.2__g3c644b5.ari           24) cray-hdf5-parallel/1.10.0.3
 12) dvs/2.7_2.2.30-6.0.4.1_5.4__gd731684          25) libxml2/2.9.4
 13) alps/6.4.1-6.0.4.0_7.2__g86d0f3d.ari
\end{verbatim}

\subsection{Installing on NERSC Cori, Haswell Partition, Cray XC40}
Cori is a Cray XC40 with 16-core Intel "Haswell" nodes
installed at NERSC.

\begin{verbatim}
export CRAYPE_LINK_TYPE=dynamic
module unload cray-libsci
module load boost
module load cray-hdf5-parallel
module load cmake
mkdir build_cori_hsw
cd build_cori_hsw
cmake ..
make -j 16
ls -l bin/qmcpack
\end{verbatim}

When the above was tested on 29 August 2017, the following module and
software versions were present:

\begin{verbatim}
build_cori_hsw> module list
Currently Loaded Modulefiles:
  1) modules/3.2.10.6                              14) alps/6.4.1-6.0.4.0_7.2__g86d0f3d.ari
  2) nsg/1.2.0                                     15) rca/2.2.11-6.0.4.0_13.2__g84de67a.ari
  3) intel/17.0.2.174                              16) atp/2.1.1
  4) craype-network-aries                          17) PrgEnv-intel/6.0.4
  5) craype/2.5.12                                 18) craype-haswell
  6) udreg/2.3.2-6.0.4.0_12.2__g2f9c3ee.ari        19) cray-shmem/7.6.0
  7) ugni/6.0.14-6.0.4.0_14.1__ge7db4a2.ari        20) cray-mpich/7.6.0
  8) pmi/5.0.12                                    21) altd/2.0
  9) dmapp/7.1.1-6.0.4.0_46.2__gb8abda2.ari        22) darshan/3.1.4
 10) gni-headers/5.0.11-6.0.4.0_7.2__g7136988.ari  23) boost/1.61
 11) xpmem/2.2.2-6.0.4.0_3.1__g43b0535.ari         24) cmake/3.3.2
 12) job/2.2.2-6.0.4.0_8.2__g3c644b5.ari           25) cray-hdf5-parallel/1.10.0.3
 13) dvs/2.7_2.2.31-6.0.4.1_6.1__gb3b87e6
\end{verbatim}

\subsection{Installing on NERSC Cori, Xeon Phi KNL partition, Cray XC40}
The second phase of NERSC's Cori uses Intel
Xeon Phi Knight's Landing (KNL) nodes. The following build recipe ensures that the code
generation is appropriate for the KNL nodes:

\begin{verbatim}
export CRAYPE_LINK_TYPE=dynamic
module swap craype-haswell craype-mic-knl
module unload cray-libsci
module load boost
module load cray-hdf5-parallel
module load cmake
mkdir build_cori_knl
cd build_cori_knl
cmake ..
make -j 16
ls -l bin/qmcpack
\end{verbatim}

When the above was tested on 29 August 2017, the following module and
software versions were present:

\begin{verbatim}
build_cori_knl> module list
Currently Loaded Modulefiles:
  1) modules/3.2.10.6                              10) gni-headers/5.0.11-6.0.4.0_7.2__g7136988.ari  19) cray-shmem/7.6.0
  2) nsg/1.2.0                                     11) xpmem/2.2.2-6.0.4.0_3.1__g43b0535.ari         20) cray-mpich/7.6.0
  3) intel/17.0.2.174                              12) job/2.2.2-6.0.4.0_8.2__g3c644b5.ari           21) altd/2.0
  4) craype-network-aries                          13) dvs/2.7_2.2.31-6.0.4.1_6.1__gb3b87e6          22) darshan/3.1.4
  5) craype/2.5.12                                 14) alps/6.4.1-6.0.4.0_7.2__g86d0f3d.ari          23) boost/1.61
  6) udreg/2.3.2-6.0.4.0_12.2__g2f9c3ee.ari        15) rca/2.2.11-6.0.4.0_13.2__g84de67a.ari         24) cray-hdf5-parallel/1.10.0.3
  7) ugni/6.0.14-6.0.4.0_14.1__ge7db4a2.ari        16) atp/2.1.1                                     25) cmake/3.3.2
  8) pmi/5.0.12                                    17) PrgEnv-intel/6.0.4
  9) dmapp/7.1.1-6.0.4.0_46.2__gb8abda2.ari        18) craype-mic-knl
\end{verbatim}

\subsection{Installing on Windows}
Install the Windows Subsystem for Linux and Bash on Windows.
Open a bash shell and follow the install directions for Ubuntu in Section \ref{sec:buildubuntu}.

\section{Testing and validation of QMCPACK}
\label{sec:testing}
We \textbf{strongly encourage} running the included tests each time
QMCPACK is built. These compare the results from the executable with
known-good mean-field, quantum chemical, and other QMC results.

The tests included with QMCPACK currently mainly test the VMC code with
single determinant wavefunction and simple spline Jastrow
wavefunctions, and for gaussian and periodic spline basis
sets. We check that the known mean
field results are obtained with no Jastrow. When Jastrow functions are
included we test against previous QMC data. The tests are statistical
with a generous 3 $\sigma$ tolerance, however the system sizes are
small, typically $<10$ electrons, so the error bars are typically
small. Limited DMC and optimizer tests included and are scheduled for expansion.

 The ``short'' tests only take a few minutes on a 16
core machine. You can run these tests using the command below in the
build directory:

\begin{verbatim}
ctest -R short   # Run the tests with "short" in their name
\end{verbatim}
The output should be similar to the following:
\begin{verbatim}
Test project build_gcc
      Start  1: short-LiH_dimer_ae-vmc_hf_noj-16-1
 1/44 Test  #1: short-LiH_dimer_ae-vmc_hf_noj-16-1 ..............  Passed   11.20 sec
      Start  2: short-LiH_dimer_ae-vmc_hf_noj-16-1-kinetic
 2/44 Test  #2: short-LiH_dimer_ae-vmc_hf_noj-16-1-kinetic ......  Passed    0.13 sec
..
42/44 Test #42: short-monoO_1x1x1_pp-vmc_sdj-1-16 ...............  Passed   10.02 sec
      Start 43: short-monoO_1x1x1_pp-vmc_sdj-1-16-totenergy
43/44 Test #43: short-monoO_1x1x1_pp-vmc_sdj-1-16-totenergy .....  Passed    0.08 sec
      Start 44: short-monoO_1x1x1_pp-vmc_sdj-1-16-samples
44/44 Test #44: short-monoO_1x1x1_pp-vmc_sdj-1-16-samples .......  Passed    0.08 sec

100% tests passed, 0 tests failed out of 44

Total Test time (real) = 167.14 sec
\end{verbatim}
Note that the number of tests that are run varies between the
standard, complex, and GPU compilations.

The  full set of tests consist of significantly longer versions of the short
tests, as well as tests of the conversion utilities. The runs require
several hours each for improved statistics and a much more
stringent test of the code. To run all the tests simply run ctest in the build
directory:

\begin{verbatim}
ctest            # Run all the tests. This will take several hours.
\end{verbatim}

You can also run verbose tests which direct the QMCPACK
output to the standard output:
\begin{verbatim}
ctest -V -R short   # Verbose short tests
\end{verbatim}

The test system includes specific tests for the complex version of the code.

The input data files for the tests are located in the \texttt{tests} directory.
The system-level test directories are grouped into \texttt{heg}, \texttt{molecules}, and \texttt{solids}, with particular physical systems under each (for example \texttt{molecules/H4\_ae}
\footnote{The suffix `ae' is short for `all-electron' and `pp' is short for `pseudopotential'}).
Under each physical system directory there may be tests for multiple QMC methods or parameter variations.
The numerical comparisons and test definitions are in the \texttt{CMakeLists.txt} file in each physical system directory.

If \textit{all} the QMC tests fail it is likely
that the appropriate mpiexec (or aprun, srun) is not being
called or found. If the QMC runs appear to work but all the other
tests fail it is possible that python is not working on your system -
we suggest checking some of the test console output in \texttt{build/Testing/Temporary/LastTest.log}
or the output files under \texttt{build/tests/}.
%The runs occur in \texttt{build/tests/test\_dir/test\_name}.


Note that because most of these tests are very small, consisting of only a few
electrons, the performance is not representative of larger
calculations. For example, while the calculations might fit in cache,
there will be essentially no vectorization due to the small electron
counts. \textbf{These tests should therefore not be used for any benchmarking or
performance analysis}.

Example runs that can be used for testing performance are described in
Sec. \ref{sec:perftests}

\subsection{Unit tests}

QMCPACK has a set of unit tests.
All of the unit tests can be run with the following command (in the build directory):
\begin{verbatim}
ctest -L unit
\end{verbatim}

The output should look similar to the following:
\begin{verbatim}
Test project qmcpack/build
      Start  1: unit_test_numerics
 1/11 Test  #1: unit_test_numerics ...............   Passed    0.06 sec
      Start  2: unit_test_utilities
 2/11 Test  #2: unit_test_utilities ..............   Passed    0.02 sec
      Start  3: unit_test_einspline
 ...
10/11 Test #10: unit_test_hamiltonian ............   Passed    1.88 sec
      Start 11: unit_test_drivers
11/11 Test #11: unit_test_drivers ................   Passed    0.01 sec

100% tests passed, 0 tests failed out of 11

Label Time Summary:
unit    =   2.20 sec

Total Test time (real) =   2.31 sec
\end{verbatim}

Individual unit test executables can be found in \texttt{build/tests/bin}.
The source for the unit tests is located in the \texttt{tests} directory under each directory in \texttt{src} (e.g. \texttt{src/QMCWavefunctions/tests}).

See Chapter \ref{chap:unit_testing} for more details about unit tests.

\subsection{Integration tests with Quantum Espresso}
\label{sec:integtestqe}
As described in Sec. \ref{sec:buildqe}, it is possible to test entire
workflows of trial wavefunction generation, conversion, and eventual
QMC calculation. A patched QE must be installed so that the
pw2qmcpack converter is available.

By adding \texttt{-D QE\_BIN=your\_QE\_binary\_path} in the cmake command line when building your QMCPACK,
tests named with ``qe-'' prefix will be included in the test set of your build.
You can test the whole pw$\to$pw2qmcpack$\to$qmcpack workflow by
\begin{verbatim}
ctest -R qe
\end{verbatim}
This provides a very solid test of the entire QMC
toolchain for plane wave generated wavefunctions.

\subsection{Performance tests}
\label{sec:perftests}
Performance tests representative of real research runs are included in the
tests/performance directory. They can be used for benchmarking, comparing machine
performance, or assessing optimizations. This is in
contrast to the majority of the conventional integration tests where the particle
counts are too small to be representative. Care is still needed to
remove initialization, I/O, and compute a representative performance
measure.

The ctest integration is sufficient to run the benchmarks and measure
relative performance from version to version of QMCPACK and assess
proposed code changes. Performance tests are prefixed with
``performance''. To obtain highest performance on a particular
platform, you must run the benchmarks in a standalone manner and tune
thread counts, placement, walker count (etc.) This is essential to
fairly compare different machines. Check with the
developers if you are unsure of what is a fair change.

\subsubsection{NiO performance tests}

Follow the instructions in tests/performance/NiO/README to
enable and run the NiO tests.

The NiO tests are for bulk supercells of varying size. The QMC runs consist of short blocks of (i) VMC
without drift (ii) VMC with drift term included (iii) DMC with
constant population. The tests use spline wavefunctions that must be
downloaded as described in the README due to their large size. You
will need to set ``-DQMC\_DATA=YOUR\_DATA\_FOLDER -DENABLE\_TIMERS=1''
when running cmake as
described in the README.

Two sets of wavefunction are tested: spline orbitals with a one and
two body Jastrow functions, and a more complex form with an additional
three body Jastrow function. The Jastrows are the same for each run
and are not reoptimized, as might be done for research purposes.  Runs
in the hundreds of electrons up to low thousands of electrons are representative of
research runs performed in 2017. The largest runs target
future machines and require very large memory.

\begin{table}[h]
\begin{center}
\begin{tabular}{|c|c|c|c|c|}
\hline
\bfseries Performance Test Name&  \bfseries Historical name &\bfseries Atoms& \bfseries Electrons&  \bfseries Electrons/spin \\
\hline
performance-NiO-cpu-a32-e384  & S8 & 32 & 384 & 192 \\
performance-NiO-cpu-a64-e768  & S16 & 64 & 768 & 384 \\
performance-NiO-cpu-a128-e1536 & S32 & 128 & 1536 & 768 \\
performance-NiO-cpu-a256-e3072 & S64 & 256 & 3072 & 1536 \\
performance-NiO-cpu-a512-e6144 & S128 & 512 & 6144 & 3072 \\
performance-NiO-cpu-a1024-e12288& S256 & 1024 & 12288 & 6144 \\
\hline
\end{tabular}
  \caption{System sizes and names for NiO performance tests. GPU performance
    tests are named similarly but have different walker counts.}
  \label{tab:niotests}
\end{center}
\end{table}

\subsection{Troubleshooting tests}
ctest reports briefly pass or fail of tests in printout and also collects all the standard outputs to help investigating how tests fail.
If the ctest execution is completed, look at \texttt{Testing/Temporary/LastTest.log}.
If you manually stop the testing (ctrl+c), look at \texttt{Testing/Temporary/LastTest.log.tmp}.
You can locate the failing tests by searching for the key word `Fail'.

\subsection{Slow testing with OpenMPI}
OpenMPI has a default binding policy making all the threads running on a single core during testing when there are two or fewer MPI ranks.
This significantly increases the testing time. If you are authorized to change the default setting, you can just add `hwloc\_base\_binding\_policy=none' in /etc/openmpi/openmpi-mca-params.conf.

\section{Automated testing of QMCPACK}

The QMCPACK developers run automatic tests of QMCPACK on several
different computer systems,  many on a continuous basis. See reports at
\url{https://cdash.qmcpack.org/CDash/index.php?project=QMCPACK}.
We currently test
the following combinations nightly (workstations) and weekly (supercomputers):

\begin{itemize}
\item On a Linux Intel Xeon workstation:
  \begin{itemize}
  \item GCC 4.8.5 with OpenMPI and CUDA 8.0 (GPU build, run on NVIDIA K40s)
  \item GCC 4.8.5 with OpenMPI with Netlib BLAS
  \item GCC 4.8.5 with OpenMPI with Intel MKL
  \item Intel 2017 with Intel MPI and MKL
  \item Intel 2015 with Intel MPI and MKL and CUDA 8.0 (GPU build, run on NVIDIA K40s)
  \item Intel 2015 with Intel MPI  and MKL
  \end{itemize}
\item On a Linux Intel Knight's Landing workstation:
  \begin{itemize}
  \item Intel 2017 with Intel MPI and MKL
 \item GCC 4.8.5 with Intel MPI and MKL
  \end{itemize}
\item On Eos, a Cray XC30 Intel machine:
  \begin{itemize}
\item The default Intel programming environment and compiler with Cray MPI and Intel MKL
  \end{itemize}

\item On Titan, a Cray XK7 CPU+GPU machine:
  \begin{itemize}
  \item The GCC programming environment and compiler with Cray MPI and CUDA
  \item The GCC programming environment and compiler with Cray MPI
  \end{itemize}
\item On Cetus, an IBM Blue Gene Q machine:
\begin{itemize}
\item Blue Gene Clang 4.0
\end{itemize}
\end{itemize}

\begin{figure}
  \centering
  \includegraphics[width=10cm]{figures/QMCPACK_CDash_CTest_Results_20160129.png}
  \caption{Example test results for QMCPACK, showing data for a
    workstation (Intel, GCC, both CPU and GPU builds) and for two ORNL
    supercomputers. In this example, 4 errors were found. This
    dashboard is accessible at \url{https://cdash.qmcpack.org/}.}
  \label{fig:cdash}
\end{figure}

\section{Building ppconvert, a pseudopotential format converter}
\label{sec:buildppconvert}
QMCPACK includes a utility, ppconvert, to convert between different
pseudopotential formats. Examples include effective core potential
formats (in Gaussians), the UPF format used by Quantum ESPRESSO, and
the XML format used by QMCPACK itself. The utility also enables the
atomic orbitals to recomputed via a numerical density functional
calculation if they need to be reconstructed for use in an
electronic structure calculation.

To build ppconvert follow the instructions in
src/QMCTools/ppconvert/README. Currently ppconvert is not built
automatically although we expect to automate it soon. The makefile
must be updated to refer to suitable C++ compiler and link in
BLAS. Due to the small size of the calculations, optimal settings are
not essential.

\section{Installing and patching Quantum ESPRESSO}
\label{sec:buildqe}
For trial wavefunctions obtained in a plane-wave basis we mainly
support Quantum ESPRESSO. Note that ABINIT and QBox were supported historically
and could be reactivated.

Quantum ESPRESSO currently stores wavefunctions in a non-standard internal
``save'' format. To convert these to a conventional HDF5 format file
we have developed a converter, pw2qmcpack. This is an add on to the
Quantum ESPRESSO distribution.

To simplify the process of patching Quantum ESPRESSO we have developed
a script that will automatically download and patch the source
code. The patches are specific to each version. e.g. To download and
patch QE v5.3.0:
\begin{verbatim}
cd external_codes/quantum_espresso
./download_and_patch_qe5.3.0.sh
\end{verbatim}
After running the patch, you must configure Quantum ESPRESSO with
the HDF5 capability enabled, i.e.
\begin{verbatim}
cd espresso-5.3.0
./configure --with-hdf5 HDF5_DIR=/opt/local   # Specify HDF5 base directory
\end{verbatim}

The complete process is described in external\_codes/quantum\_espresso/README.

The tests involving pw.x and pw2qmcpack.x have been integrated in the test suite of QMCPACK.
By adding \texttt{-D QE\_BIN=your\_QE\_binary\_path} in the cmake command line when building your QMCPACK,
tests named with ``qe-'' prefix will be included in the test set of your build.
You can test the whole pw$\to$pw2qmcpack$\to$qmcpack workflow by
\begin{verbatim}
ctest -R qe
\end{verbatim}
See Sec.\ref{sec:integtestqe} and the testing section for more details.

\section{How to build the fastest executable version of QMCPACK}
\label{sec:buildperformance}
To build the fastest version of QMCPACK we recommend the following:
\begin{itemize}
\item Use the latest C++ compilers available for your
  system. Substantial gains have been made optimizing C++ in recent
  years.
\item Use a vendor optimized BLAS library such as Intel MKL and AMD ACML. Although
  QMC does not make extensive use of linear algebra, it is used in the
  VMC wavefunction optimizer, to apply the orbital coefficients in local basis
  calculations, and in the Slater determinant update.
\item Use a vector math library such as Intel VML.  For periodic
  calculations, the calculation of the structure factor and Ewald
  potential benefit from vectorized evaluation of sin and
  cos. Currently we only autodetect Intel VML, as provided with MKL,
  but support for MASSV and AMD LibM is included via \#defines. See,
  e.g. src/Numerics/e2iphi.h. For
  large supercells, this optimization can gain 10\% in performance.
\end{itemize}

Note that greater speedups of QMC calculations can usually be obtained by
carefully choosing the required statistics for each
investigation. i.e. Do not compute smaller error bars than necessary.

\section{Troubleshooting the installation}
\label{sec:troubleshoot}
Some tips to help troubleshoot installations of QMCPACK:
\begin{itemize}
\item First, build QMCPACK on a workstation that you control, or on any
  system that has a simple and up-to-date set of development
  tools. You can compare the results of cmake and QMCPACK on this
  system with any more difficult systems you encounter.
\item Use up to date development software, particularly a recent
  CMake.
\item Verify that the compilers and libraries that you expect are
  being configured. It is common to have multiple versions
  installed. The configure system will stop at the first version it
  finds which might not be the most recent. If this occurs, specify the appropriate
  directories and files directly (Section
  \ref{sec:cmakeoptions}). e.g. cmake -DCMAKE\_C\_COMPILER=/full/path/to/mpicc -DCMAKE\_CXX\_COMPILER=/full/path/to/mpicxx ..
\item To monitor the compiler and linker settings, use a verbose build, ``make
  VERBOSE=1''. If an individual source file fails to compile you
  can experiment by hand using the output of the verbose build to
  reconstruct the full compilation line.
\end{itemize}

If you still have problems please post to the QMCPACK Google group with full
details, or contact a developer.

\chapter{Running QMCPACK}
\label{chap:running}

QMCPACK requires at least one xml input file, and is invoked via:

{\texttt{qmcpack [command line options] <XML input file(s)>}}

\section{Command line options}
\label{sec:commandline}
QMCPACK offers several command line options which affect how calculations
are performed. If the flag is absent, then the corresponding
option is disabled.

\begin{description}
\item[\texttt{-{}-dryrun}]{ Validate the input file without performing the simulation.
  This is a good way to ensure that QMCPACK will do what you think it will. }
\item[\texttt{-{}-help}]{ Print version information as well as a list of optional
  command-line arguments. }
\item[\texttt{-{}-noprint}]{ Do not print extra information on Jastrow or pseudopotential.
  If this flag is not present, QMCPACK will create several \texttt{.dat} files
  that contain information about pseudopotentials (one file per PP), and jastrow
  factors (one per jastrow factor). These file may be useful for visual inspection
  of the jastrow, for example. }
\item[\texttt{-{}-save\_wfs}]{ Write a \texttt{.h5} file containing the real-space B-spline
  coefficients of the single particle wave functions. See the manual
  \ref{sec:spo_spline} for more information.}
\item[\texttt{-{}-vacuum X}]{deprecated, use `vacuum' input tag described in \ref{chap:simulationcell}. }
\item[\texttt{-{}-version}]{ Print version information and optional arguments.
  Same as \texttt{-{}-help}. }
\end{description}




\section{Input files}
\label{sec:inputs}
The input is one or more XML file(s), documented in chapter~\ref{chap:input_overview}.

\section{Output files}
QMCPACK generates multiple files, documented in chapter~\ref{chap:output_overview}.

\section{Running in parallel}
\label{sec:parallelrunning}

%considerations for mpi, threads, gpu.

\subsection{MPI}
QMCPACK is fully parallelized with MPI. When performing an ensemble job, all
the MPI ranks are first equally divided into groups which perform individual
QMC calculations. Within one calculation, all the walkers are fully distributed
across all the MPI ranks in the group. Since MPI requires distributed memory,
there must be at least one MPI per node. To maximize the efficiency, more facts
should be taken into account. When using MPI+threads on compute nodes with more
than one NUMA domain (e.g., AMD Interlagos CPU on Titan or a node with multiple
CPU sockets), it is recommended to place as many MPI ranks as the number of
NUMA domains if the memory is sufficient. On clusters with more than just one
GPU per node (NVIDIA Tesla K80), it requires to use the same number of MPI
ranks as the number of GPUs per node in order to let each MPI rank take one GPU.

\subsection{Use of OpenMP threads}
\label{sec:openmprunning}
Modern processors integrate multiple identical cores even with hardware threads
on a single die to increase the total performance and maintain a reasonable
power draw. QMCPACK takes advantage of all that compute capability on a
processor by using threads via OpenMP programming model as well as threaded linear algebra libraries. By default, QMCPACK is always built with OpenMP enabled. When launching calculations, users should instruct QMCPACK to create the right number of threads per MPI rank by specifying environmental variable OMP\_NUM\_THREADS. Even in the GPU accelerated version, using threads significantly reduces the time spent on the calculations performed by the CPU.

\subsubsection{Performance consideration}
\label{sec:cpu:performance}
As walkers are the basic units of workload in QMC algorithms, they are loosely coupled and distributed across all the threads. For this reason, the best strategy to run QMCPACK efficiently is to feed enough walkers to the available threads.

In a VMC calculation, the code automatically raises the actual number of walkers per MPI rank to the number of available threads if the user-specified number of walkers is smaller, see ``walkers/mpi=XXX'' in the VMC output.
In a DMC calculation, the target number of walkers should be chosen to be slightly smaller than a multiple of the total number of available threads across all the MPI ranks belongs to this calculation. Since the number of walkers varies from generation to generation, its dynamical value should be slightly smaller or equal to that multiple most of the time.

To achieve better performance, mixed precision version (experimental) has been introduced to the CPU code. The mixed precision CPU code is more aggresive than the GPU version in using single precision (SP) operations. Current implementation utilizes SP on most calculations, except for matrix inversions and reductions where double precision is required to retain high accuracy. All the constant spline data in wavefunction, pseudopotentials and Coulomb potentials are initialized in double precision and later stored in single precision. The mixed precision code is as accurate as the double precision code up to a certain system size. Cross checking and verification of accuracy are encouraged for systems with more than approximately 1500 electrons. The mixed precision code has been tested on solids with real/complex builds with VMC, VMC using drift and DMC runs with wavefunction including single slater determinant and one- and two-body Jastrow factors. Wavefunction optimization will be fixed in the following updates.

\subsubsection{Memory consideration}
When using threads, some memory objects shared by all the threads. Usually these memory are read-only when the walkers are evolving, for instance the ionic distance table and wavefunction coefficients.
If a wavefunction is represented by B-splines, the whole table is shared by all the threads. It usually takes a large chunk of memory when a large primitive cell was used in the simulation. Its actual size is reported as ``MEMORY increase XXX MB BsplineSetReader'' in the output file.
See details about how to reduce it in section~\ref{sec:splinebasis}.

The other memory objects which are distinct for each walker during random walk need to be associated with individual walkers and can not be shared. This part of memory grows linearly as the number of walkers per MPI rank. Those objects include wavefunction values (Slater determinants) at given electronic configurations and electron related distance tables (electron-electron distance table). Those matrices dominate the $N^2$ scaling of the memory usage per walker.

\subsection{Running on GPU machines}
\label{sec:gpurunning}

The GPU version on the NVIDIA CUDA platform is fully incorporated into
the main source code. Commonly used functionalities for
solid-state and molecular systems using B-spline single-particle
orbitals are supported. Use of Gaussian basis sets and three-body
Jastrow functions are not yet supported. A detailed description of the GPU
implementation can be found in Ref. \cite{EslerKimCeperleyShulenburger2012}.

The current GPU implementation assumes one MPI process per GPU. To use
nodes with multiple GPUs, use multiple MPI processes per node.
Vectorization is achieved over walkers, that is, all walkers are
propagated in parallel. In each GPU kernel, loops over electrons,
atomic cores or orbitals are further vectorized to exploit an
additional level of parallelism and to allow coalesced memory access.

%----------------------------------------------------------------------------%

\subsubsection{Performance consideration}
\label{sec:gpu:performance}

The relative speedup of the GPU implementation increases with both the number of electrons and the number of walkers running on a GPU. Typically, 128-256 walkers per GPU utilize sufficient number of threads to operate the GPU efficiently and to hide memory-access latency.

To achieve better performance, current implementation utilizes single precision operations on most GPU calculations, except for matrix inversions and Coulomb interaction where double precision is required to retain high accuracy. The mixed precision GPU code is as accurate as the double precision CPU code up to a certain system size. Cross checking and verification of accuracy are encouraged for systems with more than approximately 1500 electrons.

%------------------------------------------------------------------------------%

\subsubsection{Memory consideration}

In the GPU implementation, each walker has an anonymous buffer on the GPU's global memory to store temporary data associated with the wavefunctions. Therefore, the amount of memory available on a GPU limits the number of walkers and eventually the system size that it can process.

If the GPU memory is exhausted, reduce the number of walkers per GPU.
Coarsening the grids of the B-splines representation (by decreasing the value of meshfactor in the input file) can also lower the memory usage,
at the expense (risk) of obtaining inaccurate results. Proceed with caution if this option has to be considered.
It is also possible to distrubte the B-spline coefficients table between the host and GPU memory, see option Spline\_Size\_Limit\_MB in Sec.~\ref{sec:spo_spline}.


\input{units}

\chapter{Input file overview}
\label{chap:input_overview}

This chapter introduces XML as it is used in the QMCPACK input file.  The focus is on the XML file format itself and the general structure of the input file rather than an exhaustive discussion of all keywords and structure elements.  

QMCPACK uses XML to represent structured data in its input file.  Instead of text blocks like

\begin{shade}
begin project
  id     = vmc
  series = 0
end project

begin vmc
  move     = pbyp
  blocks   = 200
  steps    =  10
  timestep = 0.4
end vmc
\end{shade} 
QMCPACK input looks like
\begin{lstlisting}[style=QMCPXML]
   <project id="vmc" series="0">
   </project>

   <qmc method="vmc" move="pbyp">
      <parameter name="blocks"  >  200 </parameter>
      <parameter name="steps"   >   10 </parameter>
      <parameter name="timestep">  0.4 </parameter>
   </qmc>
\end{lstlisting}
XML elements start with \ixml{<element\_name>}, end with \ixml{</element\_name>}, and can be nested within each other to denote substructure (the trial wavefunction is composed of a Slater determinant and a Jastrow factor, which are each further composed of \ldots).  \ixml{id} and \ixml{series} are attributes of the \ixml{<project/>} element.  XML attributes are generally used to represent simple values, like names, integers, or real values.  Similar functionality is also commonly provided by \ixml{<parameter/>} elements like those previously shown.

The overall structure of the input file reflects different aspects of the QMC simulation: the simulation cell, particles, trial wavefunction, Hamiltonian, and QMC run parameters.  A condensed version of the actual input file is shown as follows:
\begin{lstlisting}[style=QMCPXML]
<?xml version="1.0"?>
<simulation>

  <project id="vmc" series="0">
    ...
  </project>

  <qmcsystem>

    <simulationcell>
      ...
    </simulationcell>

    <particleset name="e">
      ...
    </particleset>

    <particleset name="ion0">
      ...
    </particleset>

    <wavefunction name="psi0" ... >
      ...
      <determinantset>
        <slaterdeterminant>
          ..
        </slaterdeterminant>
      </determinantset>
      <jastrow type="One-Body" ... >
         ...
      </jastrow>
      <jastrow type="Two-Body" ... >
        ...
      </jastrow>
    </wavefunction>

    <hamiltonian name="h0" ... >
      <pairpot type="coulomb" name="ElecElec" ... />
      <pairpot type="coulomb" name="IonIon"   ... />
      <pairpot type="pseudo" name="PseudoPot" ... >
        ...
      </pairpot>
    </hamiltonian>

   </qmcsystem>

   <qmc method="vmc" move="pbyp">
     <parameter name="warmupSteps">   20 </parameter>
     <parameter name="blocks"     >  200 </parameter>
     <parameter name="steps"      >   10 </parameter>
     <parameter name="timestep"   >  0.4 </parameter>
   </qmc>

</simulation>
\end{lstlisting}
The omitted portions (\texttt{...}) are more fine-grained inputs such as the axes of the simulation cell, the number of up and down electrons, positions of atomic species, external orbital files, starting Jastrow parameters, and external pseudopotential files.  


\section{Project}
The \ixml{<project>} tag uses the \ixml{id} and \ixml{series} attributes.
The value of \ixml{id} is the first part of the prefix for output file names.

Output file names also contain the series number, starting at the value given by the
\ixml{series} tag.  After every \ixml{<qmc>} section, the series value will increment, giving each section a unique prefix.

For the input file shown previously, the output files will start with \ishell{vmc.s000}, for example, \ishell{vmc.s000.scalar.dat}.
If there were another \ixml{<qmc>} section in the input file, the corresponding output files would use the prefix \ishell{vmc.s001}.



\section{Random number initialization}

The random number generator state is initialized from the \ixml{random} element using the \ixml{seed} attribute:
\begin{lstlisting}[style=QMCPXML]
<random seed="1000"/>
\end{lstlisting}

If the random element is not present, or the seed value is negative, the seed will be generated from the current time.

To initialize the many independent random number generators (one per thread and MPI process), the seed value is used (modulo 1024) as a starting index into a list of prime numbers.
Entries in this offset list of prime numbers are then used as the seed for the random generator on each thread and process.

If checkpointing is enabled, the random number state is written to an HDF file at the end of each block (suffix: \ishell{.random.h5}).
This file will be read if the \ixml{mcwalkerset} tag is present to perform a restart.
For more information, see the \ixml{checkpoint} element in the QMC methods chapter (\ref{chap:qmcmethods}) and Section \ref{sec:checkpoint_files} on checkpoint and restart files.


\chapter{Specifying the system to be simulated}
\section{Specifying the simulation cell}
\label{chap:simulationcell}

The \texttt{simulationcell} block specifies the geometry of the cell, how the boundary conditions should be handled, and how ewald summation should be broken up.

\begin{table}[h]
\begin{center}
\begin{tabularx}{\textwidth}{l l l l l l }
\hline
\multicolumn{6}{l}{\texttt{simulationcell} element} \\
\hline
\multicolumn{2}{l}{parent elements:} & \multicolumn{4}{l}{\texttt{qmcsystem}}\\
\multicolumn{2}{l}{child  elements:} & \multicolumn{4}{l}{None}\\
\multicolumn{2}{l}{attribute      :} & \multicolumn{4}{l}{}\\
   &   \bfseries parameter name            & \bfseries datatype & \bfseries values & \bfseries default   & \bfseries description \\
\hline
   &   \texttt{lattice}  & 9 floats & any float & Must be specified & Specification of \\
   &                     &        &             &                   & lattice vectors. \\
   &   \texttt{bconds}   & string & ``p'' or ``n''  & ``n n n'' & Boundary conditions \\
   &                     &        &             &           & for each axis. \\
   &   \texttt{vacuum} & float & $\ge 1.0$ & 1.0        & Vacuum scale. \\
   &   \texttt{LR\_dim\_cutoff} & float & float & 15        & Ewald breakup distance. \\
\hline
\end{tabularx}
\end{center}
\end{table}

An example of a \texttt{simulationcell} block is given below:
\begin{shade}
  <simulationcell>
    <parameter name="lattice">
      3.8       0.0       0.0
      0.0       3.8       0.0
      0.0       0.0       3.8
    </parameter>
    <parameter name="bconds">
       p p p
    </parameter>
    <parameter name="LR_dim_cutoff"> 20 </parameter>
  </simulationcell>
\end{shade}

Here, a cubic cell 3.8 bohr on a side will be used.
This simulation will use periodic boundary conditions, and the maximum
$k$ vector will be $20/r_{wigner-seitz}$ of the cell.


\subsection{Lattice}
The cell is specified using 3 lattice vectors.


\subsection{Boundary conditions}
QMCPACK offers the capability to use a mixture of open and periodic boundary conditions.
The \texttt{bconds} parameter expects a single string of three characters separated by
spaces, \textit{e.g.} ``p p p'' for purely periodic boundary conditions. These characters control
the behavior of the $x$, $y$, and $z$, axes, respectively.
Examples of valid \texttt{bconds} include:

\begin{description}
\item[``p p p''] Periodic boundary conditions. Corresponds to a 3D crystal.
\item[``n n n''] Open boundary conditions. Corresponds to an isolated molecule in a vacuum.
\item[``p p n''] Slab geometry. Corresponds to a 2D crystal.
\item[``p n n''] Wire geometry. Corresponds to a 1D crystal.
\end{description}

\subsection{Vacuum}
The vacuum option allows adding a vacuum region in slab or wire boundary conditions
(\texttt{bconds= p p n} or \texttt{bconds= p n n}, respectively). The main use is
to save memory with spline or plane-wave basis trial wavefunctions, because no basis
functions are required inside the vacuum region. For example, a large vacuum region
can be added above and below a graphene sheet without having to generate the trial
wavefunction in such a large box or to have as many splines as would otherwise
be required. Note that the trial wavefunction must still be generated in a
large enough box to sufficiently reduce periodic interactions in the underlying
electronic structure calculation.

With the vacuum option, the box used for Ewald summation increases along the axis labeled \texttt{n} by a factor of \texttt{vacuum}.
Note that all the particles remain in the original box without altering their positions. i.e. Bond lengths are not changed by this option.
The default value is 1, no change to the specified axes.

An example of a \texttt{simulationcell} block using \texttt{vacuum} is given below.
The size of the box along the z-axis increases from 12 to 18 by the vacuum scale of 1.5.
\begin{shade}
  <simulationcell>
    <parameter name="lattice">
      3.8       0.0       0.0
      0.0       3.8       0.0
      0.0       0.0      12.0
    </parameter>
    <parameter name="bconds">
       p p n
    </parameter>
    <parameter name="vacuum"> 1.5 </parameter>
    <parameter name="LR_dim_cutoff"> 20 </parameter>
  </simulationcell>
\end{shade}

\subsection{\texttt{LR\_dim\_cutoff}}
When using periodic boundary conditions direct calculation of the Coulomb energy is
not well behaved. As a result, QMCPACK uses an optimized Ewald summation technique
to compute the Coulomb interaction.\cite{Natoli1995}

In the Ewald summation, the energy is broken into short- and long-ranged terms.
The short-ranged term is computed directly in real space, while the long-ranged term is computed in reciprocal space.
\texttt{LR\_dim\_cutoff} controls where the short-ranged term ends and the long-ranged term begins.
The real-space cutoff, reciprocal-space cutoff, and \texttt{LR\_dim\_cutoff} are related via:
\[
\texttt{LR\_dim\_cutoff} = r_{c} \times k_{c}
\]
where $r_{c}$ is the Wigner-Seitz radius, and $k_{c}$ is the length of the maximum $k$-vector used in the long-ranged term.

\section{Specifying the particle set}
\label{sec:particleset}


The \ixml{particleset} blocks specify the particles in the QMC simulations: their types, attributes (mass, charge, valence), and positions.   

\subsection{Input specification}
\begin{table}[h]
\begin{center}
\begin{tabularx}{\textwidth}{l l l l l X }
\hline
\multicolumn{6}{l}{\texttt{particleset} element} \\
\hline
\multicolumn{2}{l}{Parent elements:} & \multicolumn{4}{l}{\texttt{simulation}}\\
\multicolumn{2}{l}{Child  elements:} & \multicolumn{4}{l}{\texttt{group, attrib}}\\
\multicolumn{2}{l}{Attribute:} & \multicolumn{4}{l}{}\\
   &   \bfseries Name            & \bfseries Datatype & \bfseries Values & \bfseries Default   & \bfseries Description \\
   &   \texttt{name}/\texttt{id}   &  Text              &  \textit{Any}    &  e                & Name of particle set  \\
   &   \texttt{size}$^o$           &  Integer           &  \textit{Any}    &  0                & Number of particles in set \\
   &   \texttt{random}$^o$         &  Text              &  Yes/no          &  No               & Randomize starting positions \\
   &   \texttt{randomsrc}/         &  Text     & \texttt{particleset.name} & \textit{None}     & Particle set to randomize  \\
   &   \texttt{random\_source}$^o$ &                    &                  &                   &                       \\
%   &   \texttt{role}     &  text              &  MC/none         &  none               & (obsolete)                       \\
  \hline
\end{tabularx}
\end{center}
\end{table}

\begin{table}[h]
\begin{center}
\begin{tabularx}{\textwidth}{l l l l l X }
\hline
\multicolumn{6}{l}{\texttt{group} element} \\
\hline
\multicolumn{2}{l}{Parent elements:} & \multicolumn{4}{l}{\texttt{particleset}}\\
\multicolumn{2}{l}{Child  elements:} & \multicolumn{4}{l}{\texttt{parameter, attrib}}\\
\multicolumn{2}{l}{Attribute:} & \multicolumn{4}{l}{}\\
   &   \bfseries Name            & \bfseries Datatype & \bfseries Values & \bfseries Default   & \bfseries Description \\
   &   \texttt{name}               &  Text              &  \textit{Any}    &  e                & Name of particle set  \\
   &   \texttt{size}$^o$           &  Integer           &  \textit{Any}    &  0                & Number of particles in set \\
   &   \texttt{mass}$^o$           &  Real              &  \textit{Any}    &  1                & Mass of particles in set \\
   &   \texttt{unit}$^o$          &  Text              &  au/amu          &  au               & Units for mass of particles \\
\multicolumn{2}{l}{parameters}  & \multicolumn{4}{l}{}\\
   &   \bfseries Name     & \bfseries Datatype & \bfseries Values & \bfseries Default   & \bfseries Description \\
   &   \texttt{charge}    &  Real              &  \textit{Any}    &  0                  & Charge of particles in set \\
   &   \texttt{valence}   &  Real              &  \textit{Any}    &  0                  & Valence charge of particles in set \\
   &   \texttt{atomicnumber} &  Integer        &  \textit{Any}    &  0                  & Atomic number of particles in set \\
  \hline
  \hline
\end{tabularx}
\end{center}
\end{table}

\begin{table}[h]
\begin{center}
\begin{tabularx}{\textwidth}{l l l l l X }
\hline
\multicolumn{6}{l}{\texttt{attrib} element} \\
\hline
\multicolumn{2}{l}{Parent elements:} & \multicolumn{4}{l}{\texttt{particleset,group}}\\
\multicolumn{2}{l}{Attribute:} & \multicolumn{4}{l}{}\\
   &   \bfseries Name            & \bfseries Datatype & \bfseries Values & \bfseries Default   & \bfseries Description \\
   &   \texttt{name}             &  String            &  \textit{Any}    &  \textit{None}    & Name of attrib              \\
   &   \texttt{datatype}         &  String            &  IntArray, realArray, &  \textit{None} & Type of data in attrib \\
   &                             &                    &  posArray, stringArray &             &                        \\
   &   \texttt{size}$^o$         &  String            &  \textit{Any}    &  \textit{None}    & Size of data in attrib \\
  \hline
  \hline
\end{tabularx}
\end{center}
\end{table}

\subsection{Detailed attribute description}

\subsubsection{Required particleset attributes}

\begin{itemize}
\item \ixml{name}/\ixml{id} \\
Unique name for the particle set. Default is ``e" for electrons. ``i" or ``ion0" is typically used for ions. 
\end{itemize}
% Line 192 in ParticleIO/XMLParticleIO.cpp
% Lines 144-145 in QMCApp/ParticleSetPool.cpp

\subsubsection{Optional particleset attributes}

\begin{itemize}
\item \ixml{size} \\
Number of particles in set.
\end{itemize}
% Line 191 in ParticleIO/XMLParticleIO.cpp

%\begin{itemize}
%\item \ixml{role} \\
%What the particles do in the simulation
%\end{itemize}
% Line 146 in QMCApp/ParticleSetPool.cpp

\begin{itemize}
\item \ixml{random} \\
Randomize starting positions of particles. Each component of each particle's position is randomized independently in the range of the simulation cell in that component's direction. 
\end{itemize}
% Line 190 in ParticleIO/XMLParticleIO.cpp
% Line 147 in QMCApp/ParticleSetPool.cpp

\begin{itemize}
\item \ixml{randomsrc}/\ixml{random_source} \\
Specify source particle set around which to randomize the initial positions of this particle set.
\end{itemize}
% Lines 148-149 in QMCApp/ParticleSetPool.cpp

\subsubsection{Required name attributes}

\begin{itemize}
\item \ixml{name}/\ixml{id} \\
Unique name for the particle set group. Typically, element symbols are used for ions and ``u" or ``d" for spin-up and spin-down electron groups, respectively. 
\end{itemize}
% Line 192 in ParticleIO/XMLParticleIO.cpp
% Lines 144-145 in QMCApp/ParticleSetPool.cpp

\subsubsection{Optional group attributes}

\begin{itemize}
\item \ixml{mass} \\
Mass of particles in set.
\end{itemize}
% Line 190 in Particle/ParticleSet.cpp

\begin{itemize}
\item \ixml{unit} \\
Units for mass of particles in set (au[$m_e$ = 1] or amu[$\frac{1}{12}m_{\rm ^{12}C}$ = 1]).
\end{itemize}
% Line 66 in ParticleIO/XMLParticleIO.cpp


%condition appears to be future functionality for different unit types on the position array
%condition must be an integer
% Line 407 in ParticleIO/XMLParticleIO.cpp (reads condition in)
% Line 402 in ParticleIO/XMLParticleIO.cpp (declares utype integer)

\subsection{Example use cases}
\begin{minipage}{\linewidth}
\begin{lstlisting}[style=QMCPXML,caption=Particleset elements for ions and electrons randomizing electron start positions.]
  <particleset name="i" size="2">
    <group name="Li">
      <parameter name="charge">3.000000</parameter>
      <parameter name="valence">3.000000</parameter>
      <parameter name="atomicnumber">3.000000</parameter>
    </group>
    <group name="H">
      <parameter name="charge">1.000000</parameter>
      <parameter name="valence">1.000000</parameter>
      <parameter name="atomicnumber">1.000000</parameter>
    </group>
    <attrib name="position" datatype="posArray" condition="1">
    0.0   0.0   0.0
    0.5   0.5   0.5
    </attrib>
    <attrib name="ionid" datatype="stringArray">
       Li H
    </attrib>
  </particleset>
  <particleset name="e" random="yes" randomsrc="i">
    <group name="u" size="2">
      <parameter name="charge">-1</parameter>
    </group>
    <group name="d" size="2">
      <parameter name="charge">-1</parameter>
    </group>
  </particleset>                 
\end{lstlisting}
\end{minipage}

\begin{minipage}{\linewidth}
\begin{lstlisting}[style=QMCPXML,caption=Particleset elements for ions and electrons specifying electron start positions.]
  <particleset name="e">
    <group name="u" size="4">
      <parameter name="charge">-1</parameter>
      <attrib name="position" datatype="posArray">
	2.9151687332e-01 -6.5123272502e-01 -1.2188463918e-01
	5.8423636048e-01  4.2730406357e-01 -4.5964306231e-03
	3.5228575807e-01 -3.5027014639e-01  5.2644808295e-01
       -5.1686250912e-01 -1.6648002292e+00  6.5837023441e-01
      </attrib>
    </group>
    <group name="d" size="4">
      <parameter name="charge">-1</parameter>
      <attrib name="position" datatype="posArray">
	3.1443445436e-01  6.5068682609e-01 -4.0983449009e-02
       -3.8686061749e-01 -9.3744432997e-02 -6.0456005388e-01
	2.4978241724e-02 -3.2862514649e-02 -7.2266047173e-01
       -4.0352404772e-01  1.1927734805e+00  5.5610824921e-01
      </attrib>
    </group>
  </particleset>
  <particleset name="ion0" size="3">
    <group name="O">
      <parameter name="charge">6</parameter>
      <parameter name="valence">4</parameter>
      <parameter name="atomicnumber">8</parameter>
    </group>
    <group name="H">
      <parameter name="charge">1</parameter>
      <parameter name="valence">1</parameter>
      <parameter name="atomicnumber">1</parameter>
    </group>
    <attrib name="position" datatype="posArray">
      0.0000000000e+00  0.0000000000e+00  0.0000000000e+00
      0.0000000000e+00 -1.4308249289e+00  1.1078707576e+00
      0.0000000000e+00  1.4308249289e+00  1.1078707576e+00
    </attrib>
    <attrib name="ionid" datatype="stringArray">
      O H H 
    </attrib>
  </particleset>
\end{lstlisting}
\end{minipage}

\begin{minipage}{\linewidth}
\begin{lstlisting}[style=QMCPXML,caption=Particleset elements for ions specifying positions by ion type.]
  <particleset name="ion0">
    <group name="O" size="1">
      <parameter name="charge">6</parameter>
      <parameter name="valence">4</parameter>
      <parameter name="atomicnumber">8</parameter>
      <attrib name="position" datatype="posArray">
        0.0000000000e+00  0.0000000000e+00  0.0000000000e+00
      </attrib>
    </group>
    <group name="H" size="2">
      <parameter name="charge">1</parameter>
      <parameter name="valence">1</parameter>
      <parameter name="atomicnumber">1</parameter>
      <attrib name="position" datatype="posArray">
        0.0000000000e+00 -1.4308249289e+00  1.1078707576e+00
        0.0000000000e+00  1.4308249289e+00  1.1078707576e+00
      </attrib>
    </group>
  </particleset>
\end{lstlisting}
\end{minipage}


\chapter{Trial wavefunction specification}
\section{Introduction}
\label{sec:intro_wavefunction}

This section describes the input blocks associated with the specification of the trial wavefunction in a QMCPACK calculation. These sections are contained within the \ixml{<wavefunction> ...  </wavefunction>} xml blocks. \textbf{Users are expected to rely on converters to generate the input blocks described in this section.} The converters and the workflows are designed such that input blocks require minimum modifications from users. Unless the workflow requires modification of wavefunction blocks (e.g., setting the cutoff in a multideterminant calculation), only expert users should directly alter them.
  
The trial wavefunction in QMCPACK has a general product form:
\begin{equation}
\Psi_T(\vec{r}) = \prod_k \Theta_k(\vec{r}) ,
\end{equation}
where each $\Theta_k(\vec{r})$ is a function of the electron coordinates (and possibly ionic coordinates and variational parameters). For problems involving electrons, the overall trial wavefunction must be antisymmetric with respect to electron exchange, so at least one of the functions in the product must be antisymmetric. Notice that, although QMCPACK allows for the construction of arbitrary trial wavefunctions based on the functions implemented in the code (e.g., slater determinants, jastrow functions), the user must make sure that a correct wavefunction is used for the problem at hand. From here on, we assume a standard trial wavefunction for an electronic structure problem 
\begin{equation}
\Psi_T(\vec{r}) =  \textit{A}(\vec{r}) \prod_k \textit{J}_k(\vec{r}),
\end{equation}
where $\textit{A}(\vec{r})$ is one of the antisymmetric functions: (1) slater determinant, (2) multislater determinant, or (3) pfaffian and $\textit{J}_k$ is any of the Jastrow functions (described in Section \ref{sec:jastrow}).  The antisymmetric functions are built from a set of single particle orbitals (\texttt{sposet}). QMCPACK implements four different types of \texttt{sposet}, described in the following section. Each \texttt{sposet} is designed for a different type of calculation, so their definition and generation varies accordingly. 

\section{Single determinant wavefunctions}
\label{sec:singledeterminant}
Placing a single determinant for each spin is the most used ansatz for the antisymmetric part of a trial wavefunction.
The input xml block for \texttt{slaterdeterminant} is give in Listing~\ref{listing:singledet}. A list of options is given in
Table~\ref{table:singledet}.

\begin{table}[h]
\begin{center}
\begin{tabularx}{\textwidth}{l l l l l X }
\hline
\multicolumn{6}{l}{\texttt{slaterdeterminant} element} \\
\hline
\multicolumn{2}{l}{Parent elements:} & \multicolumn{4}{l}{\texttt{determinantset}}\\
\multicolumn{2}{l}{Child  elements:} & \multicolumn{4}{l}{\texttt{determinant}}\\
\multicolumn{2}{l}{Attribute:} & \multicolumn{4}{l}{}\\
   &   \bfseries Name       & \bfseries Datatype & \bfseries Values & \bfseries Default & \bfseries Description \\
   &   \texttt{delay\_rank} &  Integer           &  >0              & 1           &  Number of delayed updates. \\
   &   \texttt{optimize}    &  Text              &  Yes/no          & Yes         &  Enable orbital optimization. \\
  \hline
\end{tabularx}
\end{center}
\caption{Options for the \texttt{slaterdeterminant} xml-block.}
\label{table:singledet}
\end{table}

\begin{lstlisting}[style=QMCPXML,caption=Slaterdeterminant set XML element.\label{listing:singledet}]
  <slaterdeterminant delay_rank="32">
    <determinant id="updet" size="208">
      <occupation mode="ground" spindataset="0">
      </occupation>
    </determinant>
    <determinant id="downdet" size="208">
      <occupation mode="ground" spindataset="0">
      </occupation>
    </determinant>
  </slaterdeterminant>
\end{lstlisting}

Additional information:
\begin{itemize}
\item \ixml{delay\_rank}. This option enables delayed updates of the Slater matrix inverse when particle-by-particle move is used.
By default, \ixml{delay\_rank=1} uses the Fahy's variant~\cite{Fahy1990} of the Sherman-Morrison rank-1 update, which is mostly using memory bandwidth-bound BLAS-2 calls.
With \ixml{delay\_rank>1}, the delayed update algorithm~\cite{Luo2018delayedupdate,McDaniel2017} turns most of the computation to compute bound BLAS-3 calls.
Tuning this parameter is highly recommended to gain the best performance on medium-to-large problem sizes ($>200$ electrons).
We have seen up to an order of magnitude speedup on large problem sizes.
When studying the performance of QMCPACK, a scan of this parameter is required and we recommend starting from 32.
The best \ixml{delay\_rank} giving the maximal speedup depends on the problem size.
Usually the larger \ixml{delay\_rank} corresponds to a larger problem size.
On CPUs, \ixml{delay\_rank} must be chosen as a multiple of SIMD vector length for good performance of BLAS libraries.
The best \ixml{delay\_rank} depends on the processor microarchitecture.
GPU support is under development.
\end{itemize}


\input{spo}
\input{jastrow}
\input{multideterminants}
\input{backflow}
\input{fdlr}
\section{Gaussian Product Wavefunction}
\label{sec:ionwf}

The Gaussian Product wavefunction implements eq.~\ref{eq:gauss_prod_wf}
\begin{equation}
\Psi(\vec{R}) = \prod_{i=1}^N \exp\left[ -\frac{(\vec{R}_i-\vec{R}_i^o)^2}{2\sigma_i^2} \right]
\label{eq:gauss_prod_wf},
\end{equation}
where $\vec{R}_i$ is the position of the $i^{\text{th}}$ quantum particle and $\vec{R}_i^o$ is its center. $\sigma_i$ is the width of the Gaussian orbital around center $i$.

This variational wavefunction enhances single-particle density at chosen spatial locations with adjustable strengths. It is useful whenever such localization is physically relevant yet not captured by other parts of the trial wavefunction. For example, in an electron-ion simulation of a solid, the ions are localized around their crystal lattice sites. This single-particle localization is not captured by the ion-ion Jastrow. Therefore the addition of this localization term will improve the wavefunction. The simplest use case of this wavefunction is perhaps the quantum harmonic oscillator (please see the ``tests/models/sho'' folder for examples).

\subsubsection{Input Specification}

\begin{table}[h]
\begin{center}
\begin{tabular}{l c c c l }
\hline
\multicolumn{5}{l}{Gaussian Product Wavefunction (ionwf)} \\
\hline
\bfseries name & \bfseries datatype & \bfseries values & \bfseries defaults  & \bfseries description \\
\hline
name & text & ionwf & (required) & Unique name for this wavefunction \\
width & floats & 1.0 -1 & (required) & Widths of Gaussian orbitals.\\ 
source & text & ion0 & (required) & Name of classical particle set.\\ 
\hline
\end{tabular}
\end{center}
\end{table}

\FloatBarrier

Additional information:
\begin{itemize}
\item \texttt{width} There must be one width provided for each quantum particle. If a negative width is given, then its corresponding Gaussian orbital is removed. Negative width is useful if one wants to use Gaussian wavefunction for a subset of the quantum particles.
\item \texttt{source} The Gaussian centers must be specified in the form of a classical particle set. This classical particle set is likely the ion positions ``ion0'', hence the name ``ionwf''. However, one may define arbitrary centers using a different particle set. Please refer to examples in `tests/models/sho'.
\end{itemize}

\subsection{Example Use Case}
\begin{lstlisting}
  <qmcsystem>
    <simulationcell>
      <parameter name="bconds">
            n n n
      </parameter>
    </simulationcell>
    <particleset name="e">
      <group name="u" size="1">
        <parameter name="mass">5.0</parameter>
        <attrib name="position" datatype="posArray" condition="0">
          0.0001 -0.0001 0.0002
        </attrib>
      </group>
    </particleset>
    <particleset name="ion0" size="1">
      <group name="H">
        <attrib name="position" datatype="posArray" condition="0">
          0 0 0
        </attrib>
      </group>
    </particleset>
    <wavefunction target="e" id="psi0">
      <ionwf name="iwf" source="ion0" width="0.8165"/>
    </wavefunction>
    <hamiltonian name="h0" type="generic" target="e">
      <extpot type="HarmonicExt" mass="5.0" energy="0.3"/>
      <estimator type="latticedeviation" name="latdev" 
        target="e"    tgroup="u" 
        source="ion0" sgroup="H"/>
    </hamiltonian>
  </qmcsystem>
\end{lstlisting}



\chapter{Hamiltonian and Observables}
\label{chap:hamiltobs}


\qmcpack is capable of the simultaneous measurement of the Hamiltonian and many other quantum operators.  The Hamiltonian attains a special status among the available operators (also referred to as observables) because it ultimately generates all available information regarding the quantum system.  This is evident from an algorithmic standpoint as well since the Hamiltonian (embodied in the projector) generates the imaginary time dynamics of the walkers in DMC and reptation Monte Carlo (RMC). 

This section covers how the Hamiltonian can be specified, component by component, by the user in the XML format native to \qmcpack. It also covers the input structure of statistical estimators corresponding to quantum observables such as the density, static structure factor, and forces.



\section{The Hamiltonian}

The many-body Hamiltonian in Hartree units is given by
\begin{align}
  \hat{H} = -\sum_i\frac{1}{2m_i}\nabla_i^2 + \sum_iv^{ext}(r_i) + \sum_{i<j}v^{qq}(r_i,r_j)   + \sum_{i\ell}v^{qc}(r_i,r_\ell)   + \sum_{\ell<m}v^{cc}(r_\ell,r_m)\:.  
\end{align}
Here, the sums indexed by $i/j$ are over quantum particles, while $\ell/m$ are reserved for classical particles.  Often the quantum particles are electrons, and the classical particles are ions, though \qmcpack is not limited in this way.  The mass of each quantum particle is denoted $m_i$, $v^{qq}/v^{qc}/v^{cc}$ are pair potentials between quantum-quantum/quantum-classical/classical-classical particles, and $v^{ext}$ denotes a purely external potential.

\qmcpack is designed modularly so that any potential can be supported with minimal additions to the code base.  Potentials currently supported include Coulomb interactions in open and periodic boundary conditions, the MPC potential, nonlocal pseudopotentials, helium pair potentials, and various model potentials such as hard sphere, Gaussian, and modified Poschl-Teller.

Reference information and examples for the \texttt{<hamiltonian/>} XML element are provided subsequently.  Detailed descriptions of the input for individual potentials is given in the sections that follow.  

% hamiltonian element
%  dev notes
%    Hamiltonian element read
%      HamiltonianPool::put
%        reads attributes: id name role target 
%          id/name is passed to QMCHamiltonian
%          role selects the primary hamiltonian
%          target associates to quantum particleset
%      HamiltonianFactory::build
%        reads attributes: type source default
%    HamiltonianFactory cloning may be flawed for non-electron systems
%      see HamiltonianFactory::clone
%        aCopy->renameProperty(``e'',qp->getName());
%        aCopy->renameProperty(psiName,psi->getName());
%      the renaming may not work if dynamic particleset.name!=''e''
%   lots of xml inputs are simply ignored if do not explicitly match (fix! here and elsewhere in the build tree)

\FloatBarrier
\begin{table}[h]
\begin{center}
\begin{tabularx}{\textwidth}{l l l l l X }
\hline
\multicolumn{6}{l}{\texttt{hamiltonian} element} \\
\hline
\multicolumn{2}{l}{parent elements:} & \multicolumn{4}{l}{\texttt{simulation, qmcsystem}}\\
\multicolumn{2}{l}{child  elements:} & \multicolumn{4}{l}{\texttt{pairpot extpot estimator constant}(deprecated)}\\
\multicolumn{2}{l}{attributes}  & \multicolumn{4}{l}{}\\
   &   \bfseries name     & \bfseries datatype & \bfseries values & \bfseries default & \bfseries description \\
   & \texttt{name/id}$^o$ &  text              & \textit{anything}& h0                & Unique id for this Hamiltonian instance  \\
   & \texttt{type}$^o$    &  text              &                  & generic           & \textit{No current function}             \\
   & \texttt{role}$^o$    &  text              & primary/extra    & extra             & Designate as primary Hamiltonian or not  \\
   & \texttt{source}$^o$  &  text              & \texttt{particleset.name} & i        & Identify classical \texttt{particleset}           \\
   & \texttt{target}$^o$  &  text              & \texttt{particleset.name} & e        & Identify quantum \texttt{particleset}             \\
   & \texttt{default}$^o$ &  boolean           & yes/no           & yes               & Include kinetic energy term implicitly   \\
  \hline
\end{tabularx}
\end{center}
\end{table}
\FloatBarrier

Additional information:
\begin{itemize}
  \item{\textbf{target:} Must be set to the name of the quantum \texttt{particleset}.  The default value is typically sufficient.  In normal usage, no other attributes are provided.}
\end{itemize}

% All-electron hamiltonian element
\begin{lstlisting}[style=QMCPXML,caption=All electron Hamiltonian XML element.]
<hamiltonian target="e">
  <pairpot name="ElecElec" type="coulomb" source="e" target="e"/>
  <pairpot name="ElecIon"  type="coulomb" source="i" target="e"/>
  <pairpot name="IonIon"   type="coulomb" source="i" target="i"/>
</hamiltonian>
\end{lstlisting}


% Pseudopotential hamiltonian element
\begin{lstlisting}[style=QMCPXML,caption=Pseudopotential Hamiltonian XML element.]
<hamiltonian target="e">
  <pairpot name="ElecElec"  type="coulomb" source="e" target="e"/>
  <pairpot name="PseudoPot" type="pseudo"  source="i" wavefunction="psi0" format="xml">
    <pseudo elementType="Li" href="Li.xml"/>
    <pseudo elementType="H" href="H.xml"/>
  </pairpot>
  <pairpot name="IonIon"    type="coulomb" source="i" target="i"/>
</hamiltonian>
\end{lstlisting}


\section{Pair potentials}

Many pair potentials are supported.  Though only the most commonly used pair potentials are covered in detail in this section, all currently available potentials are listed subsequently.  If a potential you desire is not listed, or is not present at all, feel free to contact the developers.

% pairpot element
\FloatBarrier
\begin{table}[h]
\begin{center}
\begin{tabularx}{\textwidth}{l l l l l X }
\hline
\multicolumn{6}{l}{\texttt{pairpot} factory element} \\
\hline
\multicolumn{2}{l}{parent elements:} & \multicolumn{4}{l}{\texttt{hamiltonian}}\\
\multicolumn{2}{l}{type   selector:} & \multicolumn{4}{l}{\texttt{type} attribute}\\
\multicolumn{2}{l}{type   options: } & \multicolumn{2}{l}{coulomb           } & \multicolumn{2}{l}{Coulomb/Ewald potential}\\
\multicolumn{2}{l}{                } & \multicolumn{2}{l}{pseudo            } & \multicolumn{2}{l}{Semilocal pseudopotential}\\
\multicolumn{2}{l}{                } & \multicolumn{2}{l}{mpc               } & \multicolumn{2}{l}{Model periodic Coulomb interaction/correction}\\
\multicolumn{2}{l}{                } & \multicolumn{2}{l}{cpp               } & \multicolumn{2}{l}{Core polarization potential}\\
\multicolumn{2}{l}{                } & \multicolumn{2}{l}{skpot             } & \multicolumn{2}{l}{\textit{Unknown}}\\
\multicolumn{2}{l}{shared attributes:} & \multicolumn{4}{l}{}\\
   &   \bfseries name     & \bfseries datatype & \bfseries values & \bfseries default   & \bfseries description \\
   &   \texttt{type}$^r$      &  text              & \textit{See above}        & 0                   & Select pairpot type         \\
   &   \texttt{name}$^r$      &  text              & \textit{anything}         & any                 & Unique name for this pairpot\\
   &   \texttt{source}$^r$    &  text              & \texttt{particleset.name} &\texttt{hamiltonian.target}& Identify interacting particles\\
   &   \texttt{target}$^r$    &  text              & \texttt{particleset.name} &\texttt{hamiltonian.target}& Identify interacting particles  \\
   &   \texttt{units}$^o$     &  text              &                           & hartree             & \textit{No current function}  \\
\hline
\end{tabularx}
\end{center}
\end{table}
\FloatBarrier

Additional information:
\begin{itemize}
  \item{\textbf{type:} Used to select the desired pair potential.  Must be selected from the list of type options.}
  \item{\textbf{name:} A unique name used to identify this pair potential.  Block averaged output data will appear under this name in \texttt{scalar.dat} and/or \texttt{stat.h5} files.}
  \item{\textbf{source/target:}  These specify the particles involved in a pair interaction.  If an interaction is between classical (e.g., ions) and quantum (e.g., electrons), \texttt{source}/\texttt{target} should be the name of the classical/quantum \texttt{particleset}.}
  \item{Only \texttt{Coulomb, pseudo}, and \texttt{mpc} are described in detail in the following subsections.  The older or less-used types (\texttt{cpp, skpot}) are not covered.}
  \dev{
  \item{Available only if \texttt{QMC\_CUDA} is not defined: \texttt{skpot}.}
  \item{Available only if \texttt{OHMMS\_DIM==3}: \texttt{mpc, vhxc, pseudo}.}
  \item{Available only if \texttt{OHMMS\_DIM==3} and \texttt{QMC\_CUDA} is not defined: \texttt{cpp}.}
  }
\end{itemize}


% physical read by coulomb potentials
% potential is only for pressure estimator



% pairpot instances

%   do coulomb, pseudo, mpc

\subsection{Coulomb potentials}

The bare Coulomb potential is used in open boundary conditions:
\begin{align}
  V_c^{open} = \sum_{i<j}\frac{q_iq_j}{\abs{r_i-r_j}}\:.
\end{align}

When periodic boundary conditions are selected, Ewald summation is used automatically:
\begin{align}\label{eq:ewald}
  V_c^{pbc} = \sum_{i<j}\frac{q_iq_j}{\abs{r_i-r_j}} + \frac{1}{2}\sum_{L\ne0}\sum_{i,j}\frac{q_iq_j}{\abs{r_i-r_j+L}}\:.
\end{align}
The sum indexed by $L$ is over all nonzero simulation cell lattice vectors.  In practice, the Ewald sum is broken into short- and long-range parts in a manner optimized for efficiency (see Ref.~\cite{Natoli1995}) for details. 

For information on how to set the boundary conditions, consult Section~\ref{chap:simulationcell}.


\FloatBarrier
\begin{table}[h]
\begin{center}
\begin{tabularx}{\textwidth}{l l l l l X }
\hline
\multicolumn{6}{l}{\texttt{pairpot type=coulomb} element} \\
\hline
\multicolumn{2}{l}{parent elements:} & \multicolumn{4}{l}{\texttt{hamiltonian}}\\
\multicolumn{2}{l}{child  elements:} & \multicolumn{4}{l}{\textit{None}}\\
\multicolumn{2}{l}{attributes}  & \multicolumn{4}{l}{}\\
   &   \bfseries name     & \bfseries datatype & \bfseries values & \bfseries default   & \bfseries description \\
   & \texttt{type}$^r$    &  text              & \textbf{coulomb} &                     & Must be coulomb     \\
   & \texttt{name/id}$^r$ &  text              & \textit{anything}&  ElecElec           & Unique name for interaction\\
   & \texttt{source}$^r$  &  text              & \texttt{particleset.name} &\texttt{hamiltonian.target}& Identify interacting particles\\
   & \texttt{target}$^r$  &  text              & \texttt{particleset.name} &\texttt{hamiltonian.target}& Identify interacting particles\\
   & \texttt{pbc}$^o$     &  boolean           & yes/no           & yes                 & Use Ewald summation  \\
   & \texttt{physical}$^o$&  boolean           & yes/no           & yes                 & Hamiltonian(yes)/observable(no) \\
\dev{& \texttt{forces}      &  boolean           & yes/no           & no                  & \textit{Deprecated}             \\ }
  \hline
\end{tabularx}
\end{center}
\end{table}
\FloatBarrier

Additional information
\begin{itemize}
  \item{\textbf{type/source/target:} See description for the previous generic \texttt{pairpot} factory element.}
  \item{\textbf{name:} Traditional user-specified names for electron-electron, electron-ion, and ion-ion terms are \texttt{ElecElec}, \texttt{ElecIon}, and \texttt{IonIon}, respectively.  Although any choice can be used, the data analysis tools expect to find columns in \texttt{*.scalar.dat} with these names.}
  \item{\textbf{pbc}: Ewald summation will not be performed if \texttt{simulationcell.bconds== n n n}, regardless of the value of \texttt{pbc}.  Similarly, the \texttt{pbc} attribute can only be used to turn off Ewald summation if \texttt{simulationcell.bconds!= n n n}.  The default value is recommended.}
  \item{\textbf{physical}: If \texttt{physical==yes}, this pair potential is included in the Hamiltonian and will factor into the \texttt{LocalEnergy} reported by QMCPACK and also in the DMC branching weight.  If \texttt{physical==no}, then the pair potential is treated as a passive observable but not as part of the Hamiltonian itself.  As such it does not contribute to the outputted \texttt{LocalEnergy}.  Regardless of the value of \texttt{physical} output data will appear in \texttt{scalar.dat} in a column headed by \texttt{name}.}
\end{itemize}


\begin{lstlisting}[style=QMCPXML,caption=QMCPXML element for Coulomb interaction between electrons.]
  <pairpot name="ElecElec" type="coulomb" source="e" target="e"/>
\end{lstlisting}

\begin{lstlisting}[style=QMCPXML,caption=QMCPXML element for Coulomb interaction between electrons and ions (all-electron only).]
  <pairpot name="ElecIon"  type="coulomb" source="i" target="e"/>
\end{lstlisting}

\begin{lstlisting}[style=QMCPXML,caption=QMCPXML element for Coulomb interaction between ions.]
  <pairpot name="IonIon"   type="coulomb" source="i" target="i"/>
\end{lstlisting}


\subsection{Pseudopotentials}
\label{sec:nlpp}
\qmcpack supports pseudopotentials in semilocal form, which is local in the radial coordinate and nonlocal in angular coordinates.  When all angular momentum channels above a certain threshold ($\ell_{max}$) are well approximated by the same potential ($V_{\bar{\ell}}\equiv V_{loc}$), the pseudpotential separates into a fully local channel and an angularly nonlocal component:
\begin{align}
  V^{PP} = \sum_{ij}\Big(V_{\bar{\ell}}(\abs{r_i-\tilde{r}_j}) + \sum_{\ell\ne\bar{\ell}}^{\ell_{max}}\sum_{m=-\ell}^\ell \operator{Y_{\ell m}}{\big[V_\ell(\abs{r_i-\tilde{r}_j}) - V_{\bar{\ell}}(\abs{r_i-\tilde{r}_j}) \big]}{Y_{\ell m}} \Big)\:.
\end{align}  
Here the electron/ion index is $i/j$, and only one type of ion is shown for simplicity.

Evaluation of the localized pseudopotential energy $\Psi_T^{-1}V^{PP}\Psi_T$ requires additional angular integrals.  These integrals are evaluated on a randomly shifted angular grid.  The size of this grid is determined by $\ell_{max}$.  See Ref.~\cite{Mitas1991} for further detail. 

\qmcpack uses the FSAtom pseudopotential file format associated with the ``Free Software Project for Atomic-scale Simulations'' initiated in 2002  (see \url{http://www.tddft.org/fsatom/manifest.php} for general information).  The FSAtom format uses XML for structured data.  Files in this format do not use a specific identifying file extension; instead they are simply suffixed with ``\texttt{.xml}.''  The tabular data format of CASINO is also supported.


% FSAtom format links
%   unfortunately none of the surviving links detail the format itself
% http://www.tddft.org/fsatom/index.php
% http://www.tddft.org/fsatom/programs.php
% http://www.tddft.org/fsatom/manifest.php
% http://163.13.111.58/cchu/reference/web/PseudoPotentials%20-%20FSAtom%20Wiki.htm
% http://departments.icmab.es/leem/alberto/xml/pseudo/index.html



% pseudopotential element
%   dev notes
%     attributes name, source, wavefunction, format are read in CoulombFactory.cpp  HamiltonianFactory::addPseudoPotential
%     format==''old'' refers to an old table format that is no longer supported
%     read continues in ECPotentialBuilder::put()
%       if format!=xml/old (i.e. table) qmcpack will attempt to read from *.psf files
%         in this case, <pairpot type=''pseudo'' format=''table''/>, ie there are no elements
%         if particlset groups are Li H (in order), then it looks for Li.psf and H.psf
%         what is the psf format?
%       if format==xml, normal read continues, i.e. <pseudo/> child elements are expected
%         read is not sensitive to particleset group/species ordering
%         child elements not named <pseudo/> are simply ignored (FIX!)
\FloatBarrier
\begin{table}[h]
\begin{center}
\begin{tabularx}{\textwidth}{l l l l l X }
\hline
\multicolumn{6}{l}{\texttt{pairpot type=pseudo} element} \\
\hline
\multicolumn{2}{l}{parent elements:} & \multicolumn{4}{l}{\texttt{hamiltonian}}\\
\multicolumn{2}{l}{child  elements:} & \multicolumn{4}{l}{\texttt{pseudo}}\\
\multicolumn{2}{l}{attributes}  & \multicolumn{4}{l}{}\\
   &   \bfseries name     & \bfseries datatype & \bfseries values & \bfseries default   & \bfseries description \\
   & \texttt{type}$^r$    &  text              & \textbf{pseudo} &                      & Must be pseudo         \\
   & \texttt{name/id}$^r$ &  text              & \textit{anything}&  PseudoPot          & \textit{No current function}\\
   & \texttt{source}$^r$  &  text              & \texttt{particleset.name} &  i                  & Ion \texttt{particleset} name\\
   & \texttt{target}$^r$  &  text              & \texttt{particleset.name} &\texttt{hamiltonian.target}& Electron \texttt{particleset} name  \\
   & \texttt{pbc}$^o$     &  boolean           & yes/no           & yes$^*$             & Use Ewald summation  \\
   & \texttt{forces}      &  boolean           & yes/no           & no                  & \textit{Deprecated}             \\
   &\texttt{wavefunction}$^r$ &  text          & \texttt{wavefunction.name}& invalid    & Identify wavefunction \\
   &   \texttt{format}$^r$    &  text          & xml/table        & table               & Select file format   \\
   & \texttt{algorithm}$^o$   &  text          & batched/default  & default             & Choose NLPP algorithm \\
   & \texttt{DLA}$^o$   &  text          & yes/no  & no             & Use determinant localization approximation \\
  \hline
\end{tabularx}
\end{center}
\end{table}
\FloatBarrier

Additional information:
\begin{itemize}
  \item{\textbf{type/source/target} See description for the generic \texttt{pairpot} factory element.}
  \item{\textbf{name:} Ignored.  Instead, default names will be present in \texttt{*scalar.dat} output files when pseudopotentials are used.  The field \texttt{LocalECP} refers to the local part of the pseudopotential.  If nonlocal channels are present, a \texttt{NonLocalECP} field will be added that contains the nonlocal energy summed over all angular momentum channels.}
  \item{\textbf{pbc:} Ewald summation will not be performed if \texttt{simulationcell.bconds== n n n}, regardless of the value of \texttt{pbc}.  Similarly, the \texttt{pbc} attribute can only be used to turn off Ewald summation if \texttt{simulationcell.bconds!= n n n}.}
  \item{\textbf{format:}  If \texttt{format}==table, QMCPACK looks for \texttt{*.psf} files containing pseudopotential data in a tabular format.  The files must be named after the ionic species provided in \texttt{particleset} (e.g., \texttt{Li.psf} and \texttt{H.psf}). If \texttt{format}==xml, additional \texttt{pseudo} child XML elements must be provided (see the following).  These elements specify individual file names and formats (both the FSAtom XML and CASINO tabular data formats are supported). }
  \item{\textbf{algorithm} The default algorithm evaluates the ratios of wavefunction components together for each qudarature point and then one point after another. The batched algorithm evaluates the ratios of qudarature points together for each wavefunction component and then one component after another. Internally, it uses \texttt{VirtualParticleSet} for quadrature points. Hybrid orbital representation has an extra optimization enabled when using the batched algorithm.}
  \item{\textbf{DLA} Determinant localization approximation (DLA)~\cite{Zen2019} uses only the fermionic part of the wavefunction when calculating NLPP.}
\end{itemize}


\begin{lstlisting}[style=QMCPXML,caption=QMCPXML element for pseudopotential electron-ion interaction (psf files).]
  <pairpot name="PseudoPot" type="pseudo"  source="i" wavefunction="psi0" format="psf"/>
\end{lstlisting}

\begin{lstlisting}[style=QMCPXML,caption=QMCPXML element for pseudopotential electron-ion interaction (xml files).]
  <pairpot name="PseudoPot" type="pseudo"  source="i" wavefunction="psi0" format="xml">
    <pseudo elementType="Li" href="Li.xml"/>
    <pseudo elementType="H" href="H.xml"/>
  </pairpot>
\end{lstlisting}

%\begin{lstlisting}[caption=QMCPXML element for pseudopotential electron-ion interaction (CASINO files).]
%  <pairpot name="PseudoPot" type="pseudo"  source="i" wavefunction="psi0" format="xml">
%    <pseudo elementType="Li" href="Li.data"/>
%    <pseudo elementType="H" href="H.data"/>
%  </pairpot>
%\end{lstlisting}


Details of \texttt{<pseudo/>} input elements are shown in the following.  It is possible to include (or construct) a full pseudopotential directly in the input file without providing an external file via \texttt{href}.  The full XML format for pseudopotentials is not yet covered.

% pseudo element
%   dev notes
%     initial read of href elementType/symbol attributes at ECPotentialBuilder::useXmlFormat()
%     read continues in ECPComponentBuilder
%       format==xml and href==none (not provided) => ECPComponentBuilder::put(cur)
%       format==xml and href==a file => ECPComponentBuilder::parse(href,cur)
%       format==casino => ECPComponentBuilder::parseCasino(href,cur)
%         this reader is tucked away in ECPComponentBuilder.2.cpp
%         nice demonstration of OhmmsAsciiParser here
%         maximum cutoff defined by a 1.e-5 (Ha?) spread in the nonlocal potentials
%     quadrature rules (1-7) set as in J. Chem. Phys. 95 (3467) (1991), see below
%       Rule     # points     lexact
%        1           1          0
%        2           4          2
%        3           6          3
%        4          12          5
%        5          18          5
%        6          26          7
%        7          50         11
%     looks like channels only go from s-g (see ECPComponentBuilder constructor)
%       perhaps not, quadrature rules really do go up to 7 (lexact==11), see SetQuadratureRule()
\FloatBarrier
\begin{table}[h]
\begin{center}
\begin{tabularx}{\textwidth}{l l l l l X }
\hline
\multicolumn{6}{l}{\texttt{pseudo} element} \\
\hline
\multicolumn{2}{l}{parent elements:} & \multicolumn{4}{l}{\texttt{pairpot type=pseudo}}\\
\multicolumn{2}{l}{child  elements:} & \multicolumn{4}{l}{\texttt{header local grid}}\\
\multicolumn{2}{l}{attributes}  & \multicolumn{4}{l}{}\\
   &   \bfseries name     & \bfseries datatype & \bfseries values & \bfseries default   & \bfseries description \\
   & \texttt{elementType/symbol}$^r$&  text    &\texttt{group.name}& none               & Identify ionic species   \\
   & \texttt{href}$^r$    &  text              & \textit{filepath}& none                & Pseudopotential file path\\
   & \texttt{format}$^r$  &  text              & xml/casino       & xml                 & Specify file format\\
   & \texttt{cutoff}$^o$  &  real              &                  &                     & Nonlocal cutoff radius  \\
   & \texttt{lmax}$^o$    &  integer           &                  &                     & Largest angular momentum  \\
   & \texttt{nrule}$^o$   &  integer           &                  &                     & Integration grid order             \\
  \hline
\end{tabularx}
\end{center}
\end{table}
\FloatBarrier


\begin{lstlisting}[style=QMCPXML,caption=QMCPXML element for pseudopotential of single ionic species.]
  <pseudo elementType="Li" href="Li.xml"/>
\end{lstlisting}



\subsection{MPC Interaction/correction}

The MPC interaction is an alternative to direct Ewald summation.  The MPC corrects the exchange correlation hole to more closely match its thermodynamic limit.  Because of this, the MPC exhibits smaller finite-size errors than the bare Ewald interaction, though a few alternative and competitive finite-size correction schemes now exist.  The MPC is itself often used just as a finite-size correction in postprocessing (set \texttt{physical=false} in the input).  


% mpc element
%  dev notes
%    most attributes are read in CoulombPotentialFactory.cpp  HamiltonianFactory::addMPCPotential()
%    user input for the name attribute is ignored and the name is always MPC
%    density G-vectors are stored in ParticleSet: Density_G and DensityReducedGvecs members
%    check the Linear Extrap and Quadratic Extrap output in some real examples (see MPC::init_f_G())
%      what are acceptable values for the discrepancies?
%      check that these decrease as cutoff is increased 
%    commented out code for MPC.dat creation in MPC::initBreakup()
%    short range part is 1/r, MPC::evalSR()
%    long range part is on a spline (VlongSpline), MPC::evalLR()
\FloatBarrier
\begin{table}[h]
\begin{center}
\begin{tabularx}{\textwidth}{l l l l l X }
\hline
\multicolumn{6}{l}{\texttt{pairpot type=mpc} element} \\
\hline
\multicolumn{2}{l}{parent elements:} & \multicolumn{4}{l}{\texttt{hamiltonian}}\\
\multicolumn{2}{l}{child  elements:} & \multicolumn{4}{l}{\textit{None}}\\
\multicolumn{2}{l}{attributes}  & \multicolumn{4}{l}{}\\
   &   \bfseries name     & \bfseries datatype & \bfseries values & \bfseries default   & \bfseries description \\
   & \texttt{type}$^r$    &  text              & \textbf{mpc}     &                     & Must be mpc         \\
   & \texttt{name/id}$^r$ &  text              & \textit{anything}&  MPC                & Unique name for interaction \\
   & \texttt{source}$^r$  &  text              & \texttt{particleset.name} &\texttt{hamiltonian.target}& Identify interacting particles\\
   & \texttt{target}$^r$  &  text              & \texttt{particleset.name} &\texttt{hamiltonian.target}& Identify interacting particles  \\
   & \texttt{physical}$^o$&  boolean           & yes/no           & no                  & Hamiltonian(yes)/observable(no) \\
   &  \texttt{cutoff}     &  real              & $>0$             & 30.0                & Kinetic energy cutoff \\
  \hline
\end{tabularx}
\end{center}
\end{table}
\FloatBarrier
Remarks
\begin{itemize}
  \item{\texttt{physical}:  Typically set to \texttt{no}, meaning the standard Ewald interaction will be used during sampling and MPC will be measured as an observable for finite-size post-correction.  If \texttt{physical} is \texttt{yes}, the MPC interaction will be used during sampling.  In this case an electron-electron Coulomb \texttt{pairpot} element should not be supplied.}
  \item{\textbf{Developer note:} Currently the \texttt{name} attribute for the MPC interaction is ignored.  The name is always reset to \texttt{MPC}.}
\end{itemize}

% MPC correction
\begin{lstlisting}[style=QMCPXML,caption=MPC for finite-size postcorrection.]
  <pairpot type="MPC" name="MPC" source="e" target="e" ecut="60.0" physical="no"/>
\end{lstlisting}



% estimator element
\section{General estimators}

A broad range of estimators for physical observables are available in \qmcpack.  The following sections contain input details for the total number density (\texttt{density}), number density resolved by particle spin (\texttt{spindensity}), spherically averaged pair correlation function (\texttt{gofr}), static structure factor (\texttt{sk}), energy density (\texttt{energydensity}), one body reduced density matrix (\texttt{dm1b}), $S(k)$ based kinetic energy correction (\texttt{chiesa}), forward walking (\texttt{ForwardWalking}), and force (\texttt{Force}) estimators.  Other estimators are not yet covered.

When an \texttt{<estimator/>} element appears in \texttt{<hamiltonian/>}, it is evaluated for all applicable chained QMC runs ({e.g.,} VMC$\rightarrow$DMC$\rightarrow$DMC).  Estimators are generally not accumulated during wavefunction optimization sections.    If an \texttt{<estimator/>} element is instead provided in a particular \texttt{<qmc/>} element, that estimator is only evaluated for that specific section ({e.g.,} during VMC only).


\FloatBarrier
\begin{table}[h]
\begin{center}
\begin{tabularx}{\textwidth}{l l l l l X }
\hline
\multicolumn{6}{l}{\texttt{estimator} factory element} \\
\hline
\multicolumn{2}{l}{parent elements:} & \multicolumn{4}{l}{\texttt{hamiltonian, qmc}}\\
\multicolumn{2}{l}{type   selector:} & \multicolumn{4}{l}{\texttt{type} attribute}\\
\multicolumn{2}{l}{type   options: } & \multicolumn{2}{l}{density           } & \multicolumn{2}{l}{Density on a grid}\\
\multicolumn{2}{l}{                } & \multicolumn{2}{l}{spindensity       } & \multicolumn{2}{l}{Spin density on a grid}\\
\multicolumn{2}{l}{                } & \multicolumn{2}{l}{gofr              } & \multicolumn{2}{l}{Pair correlation function (quantum species)}\\
\multicolumn{2}{l}{                } & \multicolumn{2}{l}{sk                } & \multicolumn{2}{l}{Static structure factor}\\
\multicolumn{2}{l}{                } & \multicolumn{2}{l}{structurefactor   } & \multicolumn{2}{l}{Species resolved structure factor}\\
\multicolumn{2}{l}{                } & \multicolumn{2}{l}{specieskinetic    } & \multicolumn{2}{l}{Species resolved kinetic energy}\\
\multicolumn{2}{l}{                } & \multicolumn{2}{l}{latticedeviation  } & \multicolumn{2}{l}{Spatial deviation between two particlesets}\\
\multicolumn{2}{l}{                } & \multicolumn{2}{l}{momentum          } & \multicolumn{2}{l}{Momentum distribution}\\
\multicolumn{2}{l}{                } & \multicolumn{2}{l}{energydensity     } & \multicolumn{2}{l}{Energy density on uniform or Voronoi grid}\\
\multicolumn{2}{l}{                } & \multicolumn{2}{l}{dm1b              } & \multicolumn{2}{l}{One body density matrix in arbitrary basis}\\
\multicolumn{2}{l}{                } & \multicolumn{2}{l}{chiesa            } & \multicolumn{2}{l}{Chiesa-Ceperley-Martin-Holzmann kinetic energy correction}\\
\multicolumn{2}{l}{                } & \multicolumn{2}{l}{Force             } & \multicolumn{2}{l}{Family of ``force'' estimators (see~\ref{sec:force_est})}\\
\multicolumn{2}{l}{                } & \multicolumn{2}{l}{ForwardWalking    } & \multicolumn{2}{l}{Forward walking values for existing estimators}\\
\multicolumn{2}{l}{                } & \multicolumn{2}{l}{orbitalimages     } & \multicolumn{2}{l}{Create image files for orbitals, then exit}\\
\multicolumn{2}{l}{                } & \multicolumn{2}{l}{flux              } & \multicolumn{2}{l}{Checks sampling of kinetic energy}\\
\multicolumn{2}{l}{                } & \multicolumn{2}{l}{localmoment       } & \multicolumn{2}{l}{Atomic spin polarization within cutoff radius}\\
\dev{
\multicolumn{2}{l}{                } & \multicolumn{2}{l}{Pressure          } & \multicolumn{2}{l}{\textit{No current function}}\\
\multicolumn{2}{l}{shared attributes:} & \multicolumn{4}{l}{}\\
}
   &   \bfseries name     & \bfseries datatype & \bfseries values & \bfseries default   & \bfseries description \\
   &   \texttt{type}$^r$      &  text              & \textit{See above}        & 0                   & Select estimator type         \\
   &   \texttt{name}$^r$      &  text              & \textit{anything}         & any                 & Unique name for this estimator\\
   %&   \texttt{source}$^r$    &  text              & \texttt{particleset.name} &\texttt{hamiltonian.target}& Identify interacting particles\\
   %&   \texttt{target}$^r$    &  text              & \texttt{particleset.name} &\texttt{hamiltonian.target}& Identify interacting particles  \\
   %&   \texttt{units}$^o$     &  text              &                           & hartree             & \textit{No current function}  \\
\hline
\end{tabularx}
\end{center}
\end{table}
\FloatBarrier


%  <estimator type="structurefactor" name="StructureFactor" report="yes"/>
%  <estimator type="nofk" name="nofk" wavefunction="psi0"/>


%\dev{
%\FloatBarrier
%\begin{table}[h]
%\begin{center}
%\begin{tabularx}{\textwidth}{l l l l l X }
%\hline
%\multicolumn{6}{l}{\texttt{estimator type=X} element} \\
%\hline
%\multicolumn{2}{l}{parent elements:} & \multicolumn{4}{l}{\texttt{hamiltonian, qmc}}\\
%\multicolumn{2}{l}{child  elements:} & \multicolumn{4}{l}{\textit{None}}\\
%\multicolumn{2}{l}{attributes}  & \multicolumn{4}{l}{}\\
%   &   \bfseries name     & \bfseries datatype & \bfseries values & \bfseries default   & \bfseries description \\
%   & \texttt{type}$^r$    &  text              & \textbf{X}     &                     & Must be X         \\
%   & \texttt{name}$^r$    &  text              & \textit{anything}&                  & Unique name for estimator \\
%   & \texttt{source}$^o$  &  text              & \texttt{particleset.name} &\texttt{hamiltonian.target}& Identify particles\\
%   & \texttt{target}$^o$  &  text              & \texttt{particleset.name} &\texttt{hamiltonian.target}& Identify particles  \\
%  \hline
%\end{tabularx}
%\end{center}
%\end{table}
%\FloatBarrier
%}


\subsection{Chiesa-Ceperley-Martin-Holzmann kinetic energy correction}

This estimator calculates a finite-size correction to the kinetic energy following the formalism laid out in Ref.~\cite{Chiesa2006}.  The total energy can be corrected for finite-size effects by using this estimator in conjunction with the MPC correction.

\FloatBarrier
\begin{table}[h]
\begin{center}
\begin{tabularx}{\textwidth}{l l l l l X }
\hline
\multicolumn{6}{l}{\texttt{estimator type=chiesa} element} \\
\hline
\multicolumn{2}{l}{parent elements:} & \multicolumn{4}{l}{\texttt{hamiltonian, qmc}}\\
\multicolumn{2}{l}{child  elements:} & \multicolumn{4}{l}{\textit{None}}\\
\multicolumn{2}{l}{attributes}  & \multicolumn{4}{l}{}\\
   &   \bfseries name     & \bfseries datatype & \bfseries values & \bfseries default   & \bfseries description \\
   & \texttt{type}$^r$    &  text              & \textbf{chiesa}            &        & Must be chiesa         \\
   & \texttt{name}$^o$    &  text              & \textit{anything}          & KEcorr & Always reset to KEcorr \\
   & \texttt{source}$^o$  &  text              & \texttt{particleset.name}  & e      & Identify quantum particles\\
   & \texttt{psi}$^o$     &  text              & \texttt{wavefunction.name} & psi0   & Identify wavefunction  \\
  \hline
\end{tabularx}
\end{center}
\end{table}
\FloatBarrier

% kinetic energy correction
\begin{lstlisting}[style=QMCPXML,caption=``Chiesa'' kinetic energy finite-size postcorrection.]
   <estimator name="KEcorr" type="chiesa" source="e" psi="psi0"/>
\end{lstlisting}




\subsection{Density estimator}
The particle number density operator is given by
\begin{align}
  \hat{n}_r = \sum_i\delta(r-r_i)\:.
\end{align}
The \texttt{density} estimator accumulates the number density on a uniform histogram grid over the simulation cell.  The value obtained for a grid cell $c$ with volume $\Omega_c$ is then the average number of particles in that cell:
\begin{align}
  n_c = \int dR \abs{\Psi}^2 \int_{\Omega_c}dr \sum_i\delta(r-r_i)\:.
\end{align}  


\FloatBarrier
\begin{table}[h]
\begin{center}
\begin{tabularx}{\textwidth}{l l l l l X }
\hline
\multicolumn{6}{l}{\texttt{estimator type=density} element} \\
\hline
\multicolumn{2}{l}{parent elements:} & \multicolumn{4}{l}{\texttt{hamiltonian, qmc}}\\
\multicolumn{2}{l}{child  elements:} & \multicolumn{4}{l}{\textit{None}}\\
\multicolumn{2}{l}{attributes}  & \multicolumn{4}{l}{}\\
   &   \bfseries name     & \bfseries datatype & \bfseries values  & \bfseries default   & \bfseries description \\
   & \texttt{type}$^r$      &  text              & \textbf{density}      &                     & Must be density         \\
   & \texttt{name}$^r$      &  text              & \textit{anything}     & any                 & Unique name for estimator \\
   & \texttt{delta}$^o$     &  real array(3)     & $0\le v_i \le 1$      & 0.1 0.1 0.1         & Grid cell spacing, unit coords\\
   & \texttt{x\_min}$^o$    &  real              & $>0$                  & 0                   & Grid starting point in x (Bohr)\\
   & \texttt{x\_max}$^o$    &  real              & $>0$                  &$|\texttt{lattice[0]}|$& Grid ending point in x (Bohr)\\
   & \texttt{y\_min}$^o$    &  real              & $>0$                  & 0                   & Grid starting point in y (Bohr)\\
   & \texttt{y\_max}$^o$    &  real              & $>0$                  &$|\texttt{lattice[1]}|$& Grid ending point in y (Bohr)\\
   & \texttt{z\_min}$^o$    &  real              & $>0$                  & 0                   & Grid starting point in z (Bohr)\\
   & \texttt{z\_max}$^o$    &  real              & $>0$                  &$|\texttt{lattice[2]}|$& Grid ending point in z (Bohr)\\
   & \texttt{potential}$^o$ &  boolean           & yes/no                & no                  & Accumulate local potential, \textit{Deprecated}\\
   & \texttt{debug}$^o$     &  boolean           & yes/no                & no                  & \textit{No current function}\\
  \hline
\end{tabularx}
\end{center}
\end{table}
\FloatBarrier


Additional information:
\begin{itemize}
  \item{\texttt{name}: The name provided will be used as a label in the \texttt{stat.h5} file for the blocked output data.  Postprocessing tools expect \texttt{name="Density."}}
  \item{\texttt{delta}:  This sets the histogram grid size used to accumulate the density: \texttt{delta="0.1 0.1 0.05"}$\rightarrow 10\times 10\times 20$ grid, \texttt{delta="0.01 0.01 0.01"}$\rightarrow 100\times 100\times 100$ grid.  The density grid is written to a \texttt{stat.h5} file at the end of each MC block.  If you request many $blocks$ in a \texttt{<qmc/>} element, or select a large grid, the resulting \texttt{stat.h5} file could be many gigabytes in size.}
  \item{\texttt{*\_min/*\_max}: Can be used to select a subset of the simulation cell for the density histogram grid.  For example if a (cubic) simulation cell is 20 Bohr on a side, setting \texttt{*\_min=5.0} and \texttt{*\_max=15.0} will result in a density histogram grid spanning a $10\times 10\times 10$ Bohr cube about the center of the box.  Use of \texttt{x\_min, x\_max, y\_min, y\_max, z\_min, z\_max} is only appropriate for orthorhombic simulation cells with open boundary conditions.}
  \item{When open boundary conditions are used, a \texttt{<simulationcell/>} element must be explicitly provided as the first subelement of \texttt{<qmcsystem/>} for the density estimator to work.  In this case the molecule should be centered around the middle of the simulation cell ($L/2$) and not the origin ($0$} since the space within the cell, and hence the density grid, is defined from $0$ to $L$).
\end{itemize}


% density estimator
\begin{lstlisting}[style=QMCPXML,caption=Density estimator (uniform grid).]
   <estimator name="Density" type="density" delta="0.05 0.05 0.05"/>
\end{lstlisting}

\subsection{Spin density estimator}

The spin density is similar to the total density described previously.  In this case, the sum over particles is performed independently for each spin component.

\FloatBarrier
\begin{table}[h]
\begin{center}
\begin{tabularx}{\textwidth}{l l l l l X }
\hline
\multicolumn{6}{l}{\texttt{estimator type=spindensity} element} \\
\hline
\multicolumn{2}{l}{parent elements:} & \multicolumn{4}{l}{\texttt{hamiltonian, qmc}}\\
\multicolumn{2}{l}{child  elements:} & \multicolumn{4}{l}{\textit{None}}\\
\multicolumn{2}{l}{attributes}  & \multicolumn{4}{l}{}\\
   & \bfseries name       & \bfseries datatype & \bfseries values  & \bfseries default   & \bfseries description \\
   & \texttt{type}$^r$    &  text              & \textbf{spindensity} &                  & Must be spindensity       \\
   & \texttt{name}$^r$    &  text              & \textit{anything}    & any              & Unique name for estimator \\
   & \texttt{report}$^o$  &  boolean           & yes/no               & no               & Write setup details to stdout \\
\multicolumn{2}{l}{parameters}  & \multicolumn{4}{l}{}\\
   & \bfseries name       & \bfseries datatype & \bfseries values  & \bfseries default   & \bfseries description \\
   & \texttt{grid}$^o$      & integer array(3) & $v_i>0$           &                     & Grid cell count       \\
   & \texttt{dr}$^o$        & real array(3)    & $v_i>0$           &                     & Grid cell spacing (Bohr) \\
   & \texttt{cell}$^o$      & real array(3,3)  & \textit{anything} &                     & Volume grid exists in           \\
   & \texttt{corner}$^o$    & real array(3)    & \textit{anything} &                     & Volume corner location  \\
   & \texttt{center}$^o$    & real array(3)    & \textit{anything} &                     & Volume center/origin location \\
   & \texttt{voronoi}$^o$   & text             &\texttt{particleset.name}&               & \textit{Under development}\\%Ion particleset for Voronoi centers\\
   & \texttt{test\_moves}$^o$& integer         & $>=0$             & 0                   & Test estimator with random moves  \\
  \hline
\end{tabularx}
\end{center}
\end{table}
\FloatBarrier

Additional information:
\begin{itemize}
  \item{\texttt{name}: The name provided will be used as a label in the \texttt{stat.h5} file for the blocked output data.  Postprocessing tools expect \texttt{name="SpinDensity."}}
  \item{\texttt{grid}: The grid sets the dimension of the histogram grid.  Input like \texttt{<parameter name="grid"> 40 40 40 </parameter>} requests a $40 \times 40\times 40$ grid.  The shape of individual grid cells is commensurate with the supercell shape.}
  \item{\texttt{dr}: The {\texttt{dr}} sets the real-space dimensions of grid cell edges (Bohr units).  Input like \texttt{<parameter name="dr"> 0.5 0.5 0.5 </parameter>} in a supercell with axes of length 10 Bohr each (but of arbitrary shape) will produce a $20\times 20\times 20$ grid. The inputted \texttt{dr} values are rounded to produce an integer number of grid cells along each supercell axis.  Either \texttt{grid} or \texttt{dr} must be provided, but not both.}
  \item{\texttt{cell}: When \texttt{cell} is provided, a user-defined grid volume is used instead of the global supercell.  This must be provided if open boundary conditions are used.  Additionally, if \texttt{cell} is provided, the user must specify where the volume is located in space in addition to its size/shape (\texttt{cell}) using either the \texttt{corner} or \texttt{center} parameters.}
  \item{\texttt{corner}: The grid volume is defined as $corner+\sum_{d=1}^3u_dcell_d$ with $0<u_d<1$ (``cell'' refers to either the supercell or user-provided cell).}
  \item{\texttt{center}: The grid volume is defined as $center+\sum_{d=1}^3u_dcell_d$ with $-1/2<u_d<1/2$ (``cell'' refers to either the supercell or user-provided cell).  \texttt{corner/center} can be used to shift the grid even if \texttt{cell} is not specified.  Simultaneous use of \texttt{corner} and \texttt{center} will cause QMCPACK to abort.}
\end{itemize}

% spin density estimators
\begin{lstlisting}[style=QMCPXML,caption=Spin density estimator (uniform grid).]
  <estimator type="spindensity" name="SpinDensity" report="yes">
    <parameter name="grid"> 40 40 40 </parameter>
  </estimator>
\end{lstlisting}

\begin{lstlisting}[style=QMCPXML,caption=Spin density estimator (uniform grid centered about origin).]
  <estimator type="spindensity" name="SpinDensity" report="yes">
    <parameter name="grid">
      20 20 20
    </parameter>
    <parameter name="center">
      0.0 0.0 0.0
    </parameter>
    <parameter name="cell">
      10.0  0.0  0.0
       0.0 10.0  0.0
       0.0  0.0 10.0
    </parameter>
  </estimator>
\end{lstlisting}
   


\subsection{Pair correlation function, $g(r)$}

The functional form of the species-resolved radial pair correlation function operator is
\begin{align}
  g_{ss'}(r) = \frac{V}{4\pi r^2N_sN_{s'}}\sum_{i_s=1}^{N_s}\sum_{j_{s'}=1}^{N_{s'}}\delta(r-|r_{i_s}-r_{j_{s'}}|)\:,
\end{align}
where $N_s$ is the number of particles of species $s$ and $V$ is the supercell volume.  If $s=s'$, then the sum is restricted so that $i_s\ne j_s$.

In QMCPACK, an estimate of $g_{ss'}(r)$ is obtained as a radial histogram with a set of $N_b$ uniform bins of width $\delta r$.  This can be expressed analytically as
\begin{align}
  \tilde{g}_{ss'}(r) = \frac{V}{4\pi r^2N_sN_{s'}}\sum_{i=1}^{N_s}\sum_{j=1}^{N_{s'}}\frac{1}{\delta r}\int_{r-\delta r/2}^{r+\delta r/2}dr'\delta(r'-|r_{si}-r_{s'j}|)\:,
\end{align}
where the radial coordinate $r$ is restricted to reside at the bin centers, $\delta r/2, 3 \delta r/2, 5 \delta r/2, \ldots$.

\FloatBarrier
\begin{table}[h]
\begin{center}
\begin{tabularx}{\linewidth}{l l l l l X }
\hline
\multicolumn{6}{l}{\texttt{estimator type=gofr} element} \\
\hline
\multicolumn{2}{l}{parent elements:} & \multicolumn{4}{l}{\texttt{hamiltonian, qmc}}\\
\multicolumn{2}{l}{child  elements:} & \multicolumn{4}{l}{\textit{None}}\\
\multicolumn{2}{l}{attributes}  & \multicolumn{4}{l}{}\\
   & \bfseries name       & \bfseries datatype & \bfseries values  & \bfseries default   & \bfseries description \\
   & \texttt{type}$^r$    &  text              & \textbf{gofr}     &                     & Must be gofr       \\
   & \texttt{name}$^o$    &  text              & \textit{anything} & any                 & \textit{No current function} \\
   & \texttt{num\_bin}$^r$&  integer           & $>1$              & 20                  & \# of histogram bins \\
   & \texttt{rmax}$^o$    &  real              & $>0$              & 10                  & Histogram extent (Bohr) \\
   & \texttt{dr}$^o$      &  real              & $>0$              & 0.5                 & \textit{No current function} \\%Histogram bin width (Bohr) \\
   & \texttt{debug}$^o$   &  boolean           & yes/no            & no                  & \textit{No current function} \\
   & \texttt{target}$^o$  &  text              &\texttt{particleset.name}&\texttt{hamiltonian.target}& Quantum particles \\   
   & \texttt{source/sources}$^o$&  text array  &\texttt{particleset.name}&\texttt{hamiltonian.target}& Classical particles\\
  \hline
\end{tabularx}
\end{center}
\end{table}
\FloatBarrier

Additional information:
\begin{itemize}
  \item{\texttt{num\_bin:} This is the number of bins in each species pair radial histogram.}
  \item{\texttt{rmax:} This is the maximum pair distance included in the histogram.  The uniform bin width is $\delta r=\texttt{rmax/num\_bin}$.  If periodic boundary conditions are used for any dimension of the simulation cell, then the default value of \texttt{rmax} is the simulation cell radius instead of 10 Bohr.  For open boundary conditions, the volume ($V$) used is 1.0 Bohr$^3$.}
  \item{\texttt{source/sources:} If unspecified, only pair correlations between each species of quantum particle will be measured.  For each classical particleset specified by \texttt{source/sources}, additional pair correlations between each quantum and classical species will be measured.  Typically there is only one classical particleset (e.g., \texttt{source="ion0"}), but there can be several in principle (e.g., \texttt{sources="ion0 ion1 ion2"}).}
  \item{\texttt{target:} The default value is the preferred usage (i.e., \texttt{target} does not need to be provided).}
  \item{Data is output to the \texttt{stat.h5} for each QMC subrun.  Individual histograms are named according to the quantum particleset and index of the pair.  For example, if the quantum particleset is named ``e" and there are two species (up and down electrons, say), then there will be three sets of histogram data in each \texttt{stat.h5} file named \texttt{gofr\_e\_0\_0},  \texttt{gofr\_e\_0\_1}, and  \texttt{gofr\_e\_1\_1} for up-up, up-down, and down-down correlations, respectively.}
\end{itemize}

\begin{lstlisting}[style=QMCPXML,caption=Pair correlation function estimator element.]
  <estimator type="gofr" name="gofr" num_bin="200" rmax="3.0" />
\end{lstlisting}
\begin{lstlisting}[style=QMCPXML,caption=Pair correlation function estimator element with additional electron-ion correlations.]
  <estimator type="gofr" name="gofr" num_bin="200" rmax="3.0" source="ion0" />
\end{lstlisting}


\subsection{Static structure factor, $S(k)$}

Let $\rho^e_{\mathbf{k}}=\sum_j e^{i \mathbf{k}\cdot\mathbf{r}_j^e}$ be the Fourier space electron density, with $\mathbf{r}^e_j$ being the coordinate of the j-th electron.  $\mathbf{k}$ is a wavevector commensurate with the simulation cell.  QMCPACK allows the user to accumulate the static electron structure factor $S(\mathbf{k})$ at all commensurate $\mathbf{k}$ such that $|\mathbf{k}| \leq (LR\_DIM\_CUTOFF) r_c$.  $N^e$ is the number of electrons, \texttt{LR\_DIM\_CUTOFF} is the optimized breakup parameter, and $r_c$ is the Wigner-Seitz radius.  It is defined as follows:
\begin{equation}
S(\mathbf{k}) = \frac{1}{N^e}\langle \rho^e_{-\mathbf{k}} \rho^e_{\mathbf{k}} \rangle\:.
\end{equation}

% has a CUDA counterpart, may be useful to understand difference between cpu and gpu estimators
% see HamiltonianFactory.cpp
%    SkEstimator_CUDA* apot=new SkEstimator_CUDA(*targetPtcl);

\FloatBarrier
\begin{table}[h]
\begin{center}
\begin{tabularx}{\textwidth}{l l l l l X }
\hline
\multicolumn{6}{l}{\texttt{estimator type=sk} element} \\
\hline
\multicolumn{2}{l}{parent elements:} & \multicolumn{4}{l}{\texttt{hamiltonian, qmc}}\\
\multicolumn{2}{l}{child  elements:} & \multicolumn{4}{l}{\textit{None}}\\
\multicolumn{2}{l}{attributes}  & \multicolumn{4}{l}{}\\
   & \bfseries name       & \bfseries datatype & \bfseries values  & \bfseries default   & \bfseries description \\
   & \texttt{type}$^r$    &  text              & sk      &                     & Must be sk       \\
   & \texttt{name}$^r$    &  text              & \textit{anything} & any                 & Unique name for estimator \\
   & \texttt{hdf5}$^o$    &  boolean           & yes/no            & no                  & Output to \texttt{stat.h5} (yes) or \texttt{scalar.dat} (no) \\
  \hline
\end{tabularx}
\end{center}
\end{table}
\FloatBarrier

Additional information:
\begin{itemize}
  \item{\texttt{name:} This is the unique name for estimator instance.  A data structure of the same name will appear in \texttt{stat.h5} output files.}
  \item{\texttt{hdf5:} If \texttt{hdf5==yes}, output data for $S(k)$ is directed to the \texttt{stat.h5} file (recommended usage).  If \texttt{hdf5==no}, the data is instead routed to the \texttt{scalar.dat} file, resulting in many columns of data with headings prefixed by \texttt{name} and postfixed by the k-point index (e.g., \texttt{sk\_0 sk\_1 \ldots sk\_1037 \ldots}).}
  \item{This estimator only works in periodic boundary conditions.  Its presence in the input file is ignored otherwise.}
  \item{This is not a species-resolved structure factor.  Additionally, for $\mathbf{k}$ vectors commensurate with the unit cell, $S(\mathbf{k})$ will include contributions from the static electronic density, thus meaning it wil not accurately measure the electron-electron density response.  }
\end{itemize}

\begin{lstlisting}[style=QMCPXML,caption=Static structure factor estimator element.]
  <estimator type="sk" name="sk" hdf5="yes"/>
\end{lstlisting}



\subsection{Species kinetic energy}
Record species-resolved kinetic energy instead of the total kinetic energy in the \verb|Kinetic| column of scalar.dat. \verb|SpeciesKineticEnergy| is arguably the simplest estimator in QMCPACK. The implementation of this estimator is detailed in \verb|manual/estimator/estimator_implementation.pdf|.

\FloatBarrier
\begin{table}[h]
\begin{center}
\begin{tabularx}{\textwidth}{l l l l l X }
\hline
\multicolumn{6}{l}{\texttt{estimator type=specieskinetic} element} \\
\hline
\multicolumn{2}{l}{parent elements:} & \multicolumn{4}{l}{\texttt{hamiltonian, qmc}}\\
\multicolumn{2}{l}{child  elements:} & \multicolumn{4}{l}{\textit{None}}\\
\multicolumn{2}{l}{attributes}  & \multicolumn{4}{l}{}\\
   & \bfseries name       & \bfseries datatype & \bfseries values  & \bfseries default   & \bfseries description \\
   & \texttt{type}$^r$    &  text              & specieskinetic      &                     & Must be specieskinetic       \\
   & \texttt{name}$^r$    &  text              & \textit{anything} & any                 & Unique name for estimator \\
   & \texttt{hdf5}$^o$    &  boolean           & yes/no            & no                  & Output to \texttt{stat.h5} (yes) \\
  \hline
\end{tabularx}
\end{center}
\end{table}
\FloatBarrier


\begin{lstlisting}[style=QMCPXML,caption=Species kinetic energy estimator element.]
  <estimator type="specieskinetic" name="skinetic" hdf5="no"/>
\end{lstlisting}



\subsection{Lattice deviation estimator}
Record deviation of a group of particles in one particle set (target) from a group of particles in another particle set (source).

\FloatBarrier
\begin{table}[h]
\begin{center}
\begin{tabularx}{\textwidth}{l l l l l X }
\hline
\multicolumn{6}{l}{\texttt{estimator type=latticedeviation} element} \\
\hline
\multicolumn{2}{l}{parent elements:} & \multicolumn{4}{l}{\texttt{hamiltonian, qmc}}\\
\multicolumn{2}{l}{child  elements:} & \multicolumn{4}{l}{\textit{None}}\\
\multicolumn{2}{l}{attributes}  & \multicolumn{4}{l}{}\\
   & \bfseries name       & \bfseries datatype & \bfseries values  & \bfseries default   & \bfseries description \\
   & \texttt{type}$^r$    &  text              & latticedeviation      &                     & Must be latticedeviation       \\
   & \texttt{name}$^r$    &  text              & \textit{anything} & any                 & Unique name for estimator \\
   & \texttt{hdf5}$^o$    &  boolean           & yes/no            & no                  & Output to \texttt{stat.h5} (yes) \\
   & \texttt{per\_xyz}$^o$    &  boolean           & yes/no            & no                  & Directionally resolved (yes) \\
   & \texttt{source}$^r$    &  text           & e/ion0/\dots         & no                  & source particleset \\
   & \texttt{sgroup}$^r$    &  text           & u/d/\dots         & no                  & source particle group \\
   & \texttt{target}$^r$    &  text           & e/ion0/\dots         & no                  & target particleset \\
   & \texttt{tgroup}$^r$    &  text           & u/d/\dots         & no                  & target particle group \\
  \hline
\end{tabularx}
\end{center}
\end{table}
\FloatBarrier

Additional information:
\begin{itemize}
  \item{\texttt{source}: The ``reference'' particleset to measure distances from; actual reference points are determined together with \verb|sgroup|.}
  \item{\texttt{sgroup}: The ``reference'' particle group to measure distances from.} 
  \item{\texttt{source}: The ``target'' particleset to measure distances to.}
  \item{\texttt{sgroup}: The ``target'' particle group to measure distances to. For example, in Listing~\ref{lst:latdev}, the distance from the up electron (``u'') to the origin of the coordinate system is recorded.}
  \item{\texttt{per\_xyz}: Used to record direction-resolved distance. In Listing~\ref{lst:latdev}, the x,y,z coordinates of the up electron will be recorded separately if \texttt{per\_xyz=yes}.}
  \item{\texttt{hdf5}: Used to record particle-resolved distances in the h5 file if \texttt{gdf5=yes}.}
\end{itemize}

\begin{lstlisting}[style=QMCPXML,caption={Lattice deviation estimator element.},label={lst:latdev}]
  <particleset name="e" random="yes">
    <group name="u" size="1" mass="1.0">
       <parameter name="charge"              >    -1                    </parameter>
       <parameter name="mass"                >    1.0                   </parameter>
    </group>
    <group name="d" size="1" mass="1.0">
       <parameter name="charge"              >    -1                    </parameter>
       <parameter name="mass"                >    1.0                   </parameter>
    </group>
  </particleset>
  
  <particleset name="wf_center">
    <group name="origin" size="1">
      <attrib name="position" datatype="posArray" condition="0">
               0.00000000        0.00000000        0.00000000
      </attrib>
    </group>
  </particleset>
  
  <estimator type="latticedeviation" name="latdev" hdf5="yes" per_xyz="yes"
    source="wf_center" sgroup="origin" target="e" tgroup="u"/>
\end{lstlisting}




\subsection{Energy density estimator}
An energy density operator, $\hat{\mathcal{E}}_r$,  satisfies
\begin{align}
  \int dr \hat{\mathcal{E}}_r = \hat{H},
\end{align}
where the integral is over all space and $\hat{H}$ is the Hamiltonian.  In \qmcpack, the energy density is split into kinetic and potential components
\begin{align}
  \hat{\mathcal{E}}_r = \hat{\mathcal{T}}_r + \hat{\mathcal{V}}_r\:, 
\end{align}
with each component given by
\begin{align}
   \hat{\mathcal{T}}_r &=  \frac{1}{2}\sum_i\delta(r-r_i)\hat{p}_i^2 \\  
   \hat{\mathcal{V}}_r &=  \sum_{i<j}\frac{\delta(r-r_i)+\delta(r-r_j)}{2}\hat{v}^{ee}(r_i,r_j)
              + \sum_{i\ell}\frac{\delta(r-r_i)+\delta(r-\tilde{r}_\ell)}{2}\hat{v}^{eI}(r_i,\tilde{r}_\ell) \nonumber\\ 
    &\qquad   + \sum_{\ell< m}\frac{\delta(r-\tilde{r}_\ell)+\delta(r-\tilde{r}_m)}{2}\hat{v}^{II}(\tilde{r}_\ell,\tilde{r}_m)\:.\nonumber
\end{align}
Here, $r_i$ and $\tilde{r}_\ell$ represent electron and ion positions, respectively; $\hat{p}_i$ is a single electron momentum operator; and $\hat{v}^{ee}(r_i,r_j)$, $\hat{v}^{eI}(r_i,\tilde{r}_\ell)$, and $\hat{v}^{II}(\tilde{r}_\ell,\tilde{r}_m)$ are the electron-electron, electron-ion, and ion-ion pair potential operators (including nonlocal pseudopotentials, if present).  This form of the energy density is size consistent; that is, the partially integrated energy density operators of well-separated atoms gives the isolated Hamiltonians of the respective atoms.  For periodic systems with twist-averaged boundary conditions, the energy density is formally correct only for either a set of supercell k-points that correspond to real-valued wavefunctions or a k-point set that has inversion symmetry around a k-point having a real-valued wavefunction.  For more information about the energy density, see Ref.~\cite{Krogel2013}.

In \qmcpack, the energy density can be accumulated on piecewise uniform 3D grids in generalized Cartesian, cylindrical, or spherical coordinates.  The energy density integrated within Voronoi volumes centered on ion positions is also available.  The total particle number density is also accumulated on the same grids by the energy density estimator for convenience so that related quantities, such as the regional energy per particle, can be computed easily.


\FloatBarrier
\begin{table}[h]
\begin{center}
\begin{tabularx}{\textwidth}{l l l l l X }
\hline
\multicolumn{6}{l}{\texttt{estimator type=EnergyDensity} element} \\
\hline
\multicolumn{2}{l}{parent elements:} & \multicolumn{4}{l}{\texttt{hamiltonian, qmc}}\\
\multicolumn{2}{l}{child  elements:} & \multicolumn{4}{l}{\texttt{reference\_points, spacegrid}}\\
\multicolumn{2}{l}{attributes}  & \multicolumn{4}{l}{}\\
   &   \bfseries name     & \bfseries datatype & \bfseries values & \bfseries default   & \bfseries description \\
   & \texttt{type}$^r$    &  text              & \textbf{EnergyDensity}    &                  & Must be EnergyDensity     \\
   & \texttt{name}$^r$    &  text              & \textit{anything}         &                  & Unique name for estimator \\
   & \texttt{dynamic}$^r$ &  text              & \texttt{particleset.name} &                  & Identify electrons \\
   & \texttt{static}$^o$  &  text              & \texttt{particleset.name} &                  & Identify ions  \\
   
  \hline
\end{tabularx}
\end{center}
\end{table}
\FloatBarrier

Additional information:
\begin{itemize}
  \item{\texttt{name:}  Must be unique.  A dataset with blocked statistical data for the energy density will appear in the \texttt{stat.h5} files labeled as \texttt{name}.}
\end{itemize}


\begin{lstlisting}[style=QMCPXML,caption=Energy density estimator accumulated on a $20 \times  10 \times 10$ grid over the simulation cell.]
  <estimator type="EnergyDensity" name="EDcell" dynamic="e" static="ion0">
    <spacegrid coord="cartesian">
      <origin p1="zero"/>
      <axis p1="a1" scale=".5" label="x" grid="-1 (.05) 1"/>
      <axis p1="a2" scale=".5" label="y" grid="-1 (.1) 1"/>
      <axis p1="a3" scale=".5" label="z" grid="-1 (.1) 1"/>
    </spacegrid>
  </estimator>
\end{lstlisting}


\begin{lstlisting}[style=QMCPXML,caption=Energy density estimator accumulated within spheres of radius 6.9 Bohr centered on the first and second atoms in the ion0 particleset.]
  <estimator type="EnergyDensity" name="EDatom" dynamic="e" static="ion0">
    <reference_points coord="cartesian">
      r1 1 0 0 
      r2 0 1 0
      r3 0 0 1
    </reference_points>
    <spacegrid coord="spherical">
      <origin p1="ion01"/>
      <axis p1="r1" scale="6.9" label="r"     grid="0 1"/>
      <axis p1="r2" scale="6.9" label="phi"   grid="0 1"/>
      <axis p1="r3" scale="6.9" label="theta" grid="0 1"/>
    </spacegrid>
    <spacegrid coord="spherical">
      <origin p1="ion02"/>
      <axis p1="r1" scale="6.9" label="r"     grid="0 1"/>
      <axis p1="r2" scale="6.9" label="phi"   grid="0 1"/>
      <axis p1="r3" scale="6.9" label="theta" grid="0 1"/>
    </spacegrid>
  </estimator>
\end{lstlisting}


\begin{lstlisting}[style=QMCPXML,caption=Energy density estimator accumulated within Voronoi polyhedra centered on the ions.]
  <estimator type="EnergyDensity" name="EDvoronoi" dynamic="e" static="ion0">
    <spacegrid coord="voronoi"/>
  </estimator>
\end{lstlisting}



The \texttt{<reference\_points/>} element provides a set of points for later use in specifying the origin and coordinate axes needed to construct a spatial histogramming grid.  Several reference points on the surface of the simulation cell (see Table~\ref{tab:ref_points}), as well as the positions of the ions (see the \texttt{energydensity.static} attribute), are made available by default.  The reference points can be used, for example, to construct a cylindrical grid along a bond with the origin on the bond center. 

\FloatBarrier
\begin{table}[h]
\begin{center}
\begin{tabularx}{\textwidth}{l l l l l X }
\hline
\multicolumn{6}{l}{\texttt{reference\_points} element} \\
\hline
\multicolumn{2}{l}{parent elements:} & \multicolumn{4}{l}{\texttt{estimator type=EnergyDensity}}\\
\multicolumn{2}{l}{child  elements:} & \multicolumn{4}{l}{\textit{None}}\\
\multicolumn{2}{l}{attributes}  & \multicolumn{4}{l}{}\\
   &   \bfseries name     & \bfseries datatype & \bfseries values & \bfseries default   & \bfseries description \\
   &   \texttt{coord}$^r$ &  text              & Cartesian/cell   &                     & Specify coordinate system \\
\multicolumn{2}{l}{body text}  & \multicolumn{4}{l}{}\\
   &                           & \multicolumn{4}{l}{The body text is a line formatted list of points with labels}     \\
  \hline
\end{tabularx}
\end{center}
\end{table}
\FloatBarrier

Additional information
\begin{itemize}
  \item{\texttt{coord:} If \texttt{coord=cartesian}, labeled points are in Cartesian (x,y,z) format in units of Bohr.  If \texttt{coord=cell}, then labeled points are in units of the simulation cell axes.}
  \item{\texttt{body text:}  The list of points provided in the body text are line formatted, with four entries per line (\textit{label} \textit{coor1} \textit{coor2} \textit{coor3}}).  A set of points referenced to the simulation cell is available by default (see Table~\ref{tab:ref_points}).  If \texttt{energydensity.static} is provided, the location of each individual ion is also available (e.g., if \texttt{energydensity.static=ion0}, then the location of the first atom is available with label ion01, the second with ion02, etc.). All points can be used by label when constructing spatial histogramming grids (see the following \texttt{spacegrid} element) used to collect energy densities.    
\end{itemize}


\FloatBarrier
\begin{table}[h]
\begin{center}
\caption{Reference points available by default.  Vectors $a_1$, $a_2$, and $a_3$ refer to the simulation cell axes.  The representation of the cell is centered around \texttt{zero}.\label{tab:ref_points}}
\begin{tabular}{l l l}
\hline
\texttt{label} & \texttt{point} & \texttt{description} \\
\hline
\texttt{zero} & 0 0 0    & Cell center  \\
\texttt{a1}   &  $a_1$   & Cell axis 1  \\
\texttt{a2}   &  $a_2$   & Cell axis 2  \\
\texttt{a3}   &  $a_3$   & Cell axis 3  \\
\texttt{f1p}  &  $a_1$/2 & Cell face 1+ \\
\texttt{f1m}  & -$a_1$/2 & Cell face 1- \\
\texttt{f2p}  &  $a_2$/2 & Cell face 2+ \\
\texttt{f2m}  & -$a_2$/2 & Cell face 2- \\
\texttt{f3p}  &  $a_3$/2 & Cell face 3+ \\
\texttt{f3m}  & -$a_3$/2 & Cell face 3- \\
\texttt{cppp} & $(a_1+a_2+a_3)/2$  & Cell corner +,+,+ \\
\texttt{cppm} & $(a_1+a_2-a_3)/2$  & Cell corner +,+,- \\
\texttt{cpmp} & $(a_1-a_2+a_3)/2$  & Cell corner +,-,+ \\
\texttt{cmpp} & $(-a_1+a_2+a_3)/2$ & Cell corner -,+,+ \\
\texttt{cpmm} & $(a_1-a_2-a_3)/2$  & Cell corner +,-,- \\
\texttt{cmpm} & $(-a_1+a_2-a_3)/2$ & Cell corner -,+,- \\
\texttt{cmmp} & $(-a_1-a_2+a_3)/2$ & Cell corner -,-,+ \\
\texttt{cmmm} & $(-a_1-a_2-a_3)/2$ & Cell corner -,-,- \\
\hline
\end{tabular}
\end{center}
\end{table}
\FloatBarrier



The \texttt{<spacegrid/>} element is used to specify a spatial histogramming grid for the energy density.  Grids are constructed based on a set of, potentially nonorthogonal, user-provided coordinate axes.  The axes are based on information available from \texttt{reference\_points}.  Voronoi grids are based only on nearest neighbor distances between electrons and ions.  Any number of space grids can be provided to a single energy density estimator.


\FloatBarrier
\begin{table}[h]
\begin{center}
\begin{tabularx}{\textwidth}{l l l l l X }
\hline
\multicolumn{6}{l}{\texttt{spacegrid} element} \\
\hline
\multicolumn{2}{l}{parent elements:} & \multicolumn{4}{l}{\texttt{estimator type=EnergyDensity}}\\
\multicolumn{2}{l}{child  elements:} & \multicolumn{4}{l}{\texttt{origin, axis}}\\
\multicolumn{2}{l}{attributes}  & \multicolumn{4}{l}{}\\
   &   \bfseries name     & \bfseries datatype & \bfseries values & \bfseries default   & \bfseries description \\
   &   \texttt{coord}$^r$ &  text              & Cartesian        &                     & Specify coordinate system \\
   &                      &                    & cylindrical      &                     &                           \\
   &                      &                    & spherical        &                     &                           \\
   &                      &                    & Voronoi          &                     &                           \\
  \hline
\end{tabularx}
\end{center}
\end{table}
\FloatBarrier


The \texttt{<origin/>} element gives the location of the origin for a non-Voronoi grid.\\

\FloatBarrier
\begin{table}[h]
\begin{center}
\begin{tabularx}{\textwidth}{l l l l l X }
\hline
\multicolumn{6}{l}{\texttt{origin} element} \\
\hline
\multicolumn{2}{l}{parent elements:} & \multicolumn{4}{l}{\texttt{spacegrid}}\\
\multicolumn{2}{l}{child  elements:} & \multicolumn{4}{l}{\textit{None}}\\
\multicolumn{2}{l}{attributes}  & \multicolumn{4}{l}{}\\
   &   \bfseries name     & \bfseries datatype & \bfseries values & \bfseries default   & \bfseries description \\
   &   \texttt{p1}$^r$      &  text              & \texttt{reference\_point.label}   &    &  Select end point       \\
   &   \texttt{p2}$^o$      &  text              & \texttt{reference\_point.label}   &    &  Select end point       \\
   &   \texttt{fraction}$^o$&  real              &                  &  0                  &  Interpolation fraction \\
  \hline
\end{tabularx}
\end{center}
\end{table}

Additional information:
\begin{itemize}
  \item{\texttt{p1/p2/fraction:} The location of the origin is set to \texttt{p1+fraction*(p2-p1)}.  If only \texttt{p1} is provided, the origin is at \texttt{p1}.}
\end{itemize}
\FloatBarrier


The \texttt{<axis/>} element represents a coordinate axis used to construct the, possibly curved, coordinate system for the histogramming grid.  Three \texttt{<axis/>} elements must be provided to a non-Voronoi \texttt{<spacegrid/>} element.

\FloatBarrier
\begin{table}[h]
\begin{center}
\begin{tabularx}{\textwidth}{l l l l l X }
\hline
\multicolumn{6}{l}{\texttt{axis} element} \\
\hline
\multicolumn{2}{l}{parent elements:} & \multicolumn{4}{l}{\texttt{spacegrid}}\\
\multicolumn{2}{l}{child  elements:} & \multicolumn{4}{l}{\textit{None}}\\
\multicolumn{2}{l}{attributes}  & \multicolumn{4}{l}{}\\
   &   \bfseries name        & \bfseries datatype & \bfseries values & \bfseries default   & \bfseries description \\
   &   \texttt{label}$^r$    &  text              & \textit{See below}&                    &  Axis/dimension label \\
   &   \texttt{grid}$^r$     &  text              &                  & ``0 1"             &  Grid ranges/intervals \\
   &   \texttt{p1}$^r$       &  text              & \texttt{reference\_point.label}   &    &  Select end point     \\
   &   \texttt{p2}$^o$       &  text              & \texttt{reference\_point.label}   &    &  Select end point     \\
   &   \texttt{scale}$^o$    &  real              &                  &                     &  Interpolation fraction\\
  \hline
\end{tabularx}
\end{center}
\end{table}
\FloatBarrier
Additional information:
\begin{itemize}
  \item{\texttt{label:} The allowed set of axis labels depends on the coordinate system (i.e., \texttt{spacegrid.coord}).  Labels are \texttt{x/y/z} for \texttt{coord=cartesian}, \texttt{r/phi/z} for \texttt{coord=cylindrical}, \texttt{r/phi/theta} for \texttt{coord=spherical}.}
  \item{\texttt{p1/p2/scale:} The axis vector is set to \texttt{p1+scale*(p2-p1)}.  If only \texttt{p1} is provided, the axis vector is \texttt{p1}.}
  \item{\texttt{grid:} The grid specifies the histogram grid along the direction specified by \texttt{label}.  The allowed grid points fall in the range [-1,1] for \texttt{label=x/y/z} or [0,1] for \texttt{r/phi/theta}.  A grid of 10 evenly spaced points between 0 and 1 can be requested equivalently by \texttt{grid="0 (0.1) 1"} or  \texttt{grid="0 (10) 1."}  Piecewise uniform grids covering portions of the range are supported, e.g., \texttt{grid="-0.7 (10) 0.0 (20) 0.5."}  }
  \item{Note that \texttt{grid} specifies the histogram grid along the (curved) coordinate given by \texttt{label}.  The axis specified by \texttt{p1/p2/scale} does not correspond one-to-one with \texttt{label} unless \texttt{label=x/y/z}, but the full set of axes provided defines the (sheared) space on top of which the curved (e.g., spherical) coordinate system is built. }
\end{itemize}






\subsection{One body density matrix}
The N-body density matrix in DMC is $\hat{\rho}_N=\operator{\Psi_{T}}{}{\Psi_{FN}}$ (for VMC, substitute $\Psi_T$ for $\Psi_{FN}$).  The one body reduced density matrix (1RDM) is obtained by tracing out all particle coordinates but one:
\begin{align}
  \hat{n}_1 &= \sum_nTr_{R_n}\operator{\Psi_{T}}{}{\Psi_{FN}}\:.
\end{align}
In this formula, the sum is over all electron indices and $Tr_{R_n}(*)\equiv\int dR_n\expval{R_n}{*}{R_n}$ with $R_n=[r_1,...,r_{n-1},r_{n+1},...,r_N]$.  When the sum is restricted over spin-up or spin-down electrons, one obtains a density matrix for each spin species.  The 1RDM computed by \qmcpack is partitioned in this way.

In real space, the matrix elements of the 1RDM are
\begin{align}
  n_1(r,r') &= \expval{r}{\hat{n}_1}{r'} = \sum_n\int dR_n \Psi_T(r,R_n)\Psi_{FN}^*(r',R_n)\:. 
\end{align}

A more efficient and compact representation of the 1RDM is obtained by expanding in the SPOs obtained from a Hartree-Fock or DFT calculation, $\{\phi_i\}$:
\begin{align}\label{eq:dm1b_direct}
  n_1(i,j) &= \expval{\phi_i}{\hat{n}_1}{\phi_j} \nonumber \\
           &= \int dR \Psi_{FN}^*(R)\Psi_{T}(R) \sum_n\int dr'_n \frac{\Psi_T(r_n',R_n)}{\Psi_T(r_n,R_n)}\phi_i(r_n')^* \phi_j(r_n)\:.
\end{align} 

The integration over $r'$ in Eq.~\ref{eq:dm1b_direct} is inefficient when one is also interested in obtaining matrices involving energetic quantities, such as the energy density matrix of Ref.~\cite{Krogel2014} or the related (and more well known) generalized Fock matrix.  For this reason, an approximation is introduced as follows:
\begin{align}
    n_1(i,j) \approx \int dR \Psi_{FN}(R)^*\Psi_T(R)  \sum_n \int dr_n' \frac{\Psi_T(r_n',R_n)^*}{\Psi_T(r_n,R_n)^*}\phi_i(r_n)^* \phi_j(r_n')\:. 
\end{align}
For VMC, FN-DMC, FP-DMC, and RN-DMC this formula represents an exact sampling of the 1RDM corresponding to $\hat{\rho}_N^\dagger$ (see appendix A of Ref.~\cite{Krogel2014} for more detail).




\FloatBarrier
\begin{table}[h]
\begin{center}
\begin{tabularx}{\textwidth}{l l l l l X }
\hline
\multicolumn{6}{l}{\texttt{estimator type=dm1b} element} \\
\hline
\multicolumn{2}{l}{parent elements:} & \multicolumn{4}{l}{\texttt{hamiltonian, qmc}}\\
\multicolumn{2}{l}{child  elements:} & \multicolumn{4}{l}{\textit{none}}\\
\multicolumn{2}{l}{attributes}  & \multicolumn{4}{l}{}\\
   &   \bfseries name     & \bfseries datatype & \bfseries values & \bfseries default   & \bfseries description \\
   & \texttt{type}$^r$    &  text              & \textbf{dm1b}    &                     & Must be dm1b     \\
   & \texttt{name}$^r$    &  text              & \textit{anything}&                     & Unique name for estimator \\
\multicolumn{2}{l}{parameters}  & \multicolumn{4}{l}{}\\
   &   \bfseries name     & \bfseries datatype & \bfseries values & \bfseries default   & \bfseries description \\
   &\texttt{basis}$^r$         &  text array   & sposet.name(s)   &                     & Orbital basis         \\
   &\texttt{integrator}$^o$    &  text         & uniform\_grid    & uniform\_grid       & Integration method    \\
   &                           &               & uniform          &                     &                       \\
   &                           &               & density          &                     &                       \\
   &\texttt{evaluator}$^o$     &  text         & loop/matrix      & loop                & Evaluation method     \\
   &\texttt{scale}$^o$         &  real         & $0<scale<1$      & 1.0                 & Scale integration cell\\
   &\texttt{center}$^o$        &  real array(3)&\textit{any point}&                     & Center of cell        \\
   &\texttt{points}$^o$        &  integer      & $>0$             & 10                  & Grid points in each dim\\
   &\texttt{samples}$^o$       &  integer      & $>0$             & 10                  & MC samples            \\
   &\texttt{warmup}$^o$        &  integer      & $>0$             & 30                  & MC warmup             \\
   &\texttt{timestep}$^o$      &  real         & $>0$             & 0.5                 & MC time step          \\
   &\texttt{use\_drift}$^o$    &  boolean      &  yes/no          & no                  & Use drift in VMC      \\
   &\texttt{check\_overlap}$^o$&  boolean      &  yes/no          & no                  & Print overlap matrix  \\
   &\texttt{check\_derivatives}$^o$& boolean   &  yes/no          & no                  & Check density derivatives \\
   &\texttt{acceptance\_ratio}$^o$&  boolean   &  yes/no          & no                  & Print accept ratio    \\
   &\texttt{rstats}$^o$        &  boolean      &  yes/no          & no                  & Print spatial stats   \\
   &\texttt{normalized}$^o$    &  boolean      &  yes/no          & yes                 & \texttt{basis} comes norm'ed \\
   &\texttt{volume\_normed}$^o$&  boolean      &  yes/no          & yes                 & \texttt{basis} norm is volume \\
   &\texttt{energy\_matrix}$^o$&  boolean      & yes/no           & no                  & Energy density matrix \\
  \hline
\end{tabularx}
\end{center}
\end{table}
\FloatBarrier

Additional information:
\begin{itemize}
  \item{\texttt{name:} Density matrix results appear in \texttt{stat.h5} files labeled according to \texttt{name}.}
  \item{\texttt{basis:} List \texttt{sposet.name}'s.  The total set of orbitals contained in all \texttt{sposet}'s comprises the basis (subspace) onto which the one body density matrix is projected.  This set of orbitals generally includes many virtual orbitals that are not occupied in a single reference Slater determinant.}
  \item{\texttt{integrator:} Select the method used to perform the additional single particle integration.  Options are \texttt{uniform\_grid} (uniform grid of points over the cell), \texttt{uniform} (uniform random sampling over the cell), and \texttt{density} (Metropolis sampling of approximate density, $\sum_{b\in \texttt{basis}}\abs{\phi_b}^2$, is not well tested, please check results carefully!)}.  Depending on the integrator selected, different subsets of the other input parameters are active.
  \item{\texttt{evaluator:} Select for-loop or matrix multiply implementations.  Matrix is preferred for speed.  Both implementations should give the same results, but please check as this has not been exhaustively tested.}
  \item{\texttt{scale:} Resize the simulation cell by scale for use as an integration volume (active for \texttt{integrator=uniform/uniform\_grid}).}
  \item{\texttt{center:} Translate the integration volume to center at this point (active for \texttt{integrator=uniform/\\uniform\_grid}). If \texttt{center} is not provided, the scaled simulation cell is used as is. }
  \item{\texttt{points:} Number of grid points in each dimension for \texttt{integrator=uniform\_grid}.  For example, \texttt{points=10} results in a uniform $10 \times 10 \times 10$ grid over the cell.}
  \item{\texttt{samples:} Sets the number of MC samples collected for each step (active for \texttt{integrator=uniform/\\density}).  }
  \item{\texttt{warmup:} Number of warmup Metropolis steps at the start of the run before data collection (active for \texttt{integrator=density}). }
  \item{\texttt{timestep:} Drift-diffusion time step used in Metropolis sampling (active for \texttt{integrator=density}).}
  \item{\texttt{use\_drift:} Enable drift in Metropolis sampling  (active for \texttt{integrator=density}).}
  \item{\texttt{check\_overlap:} Print the overlap matrix (computed via simple Riemann sums) to the log, then abort.  Note that subsequent analysis based on the 1RDM is simplest if the input orbitals are orthogonal.}
  \item{\texttt{check\_derivatives:} Print analytic and numerical derivatives of the approximate (sampled) density for several sample points, then abort. }
  \item{\texttt{acceptance\_ratio:} Print the acceptance ratio of the density sampling to the log for each step.}
  \item{\texttt{rstats:} Print statistical information about the spatial motion of the sampled points to the log for each step.}
  \item{\texttt{normalized:} Declare whether the inputted orbitals are normalized or not.  If \texttt{normalized=no}, direct Riemann integration over a $200 \times 200 \times 200$ grid will be used to compute the normalizations before use.}
  \item{\texttt{volume\_normed:} Declare whether the inputted orbitals are normalized to the cell volume (default) or not (a norm of 1.0 is assumed in this case).  Currently, B-spline orbitals coming from QE and HEG planewave orbitals native to QMCPACK are known to be volume normalized.}
  \item{\texttt{energy\_matrix:} Accumulate the one body reduced energy density matrix, and write it to \texttt{stat.h5}.  This matrix is not covered in any detail here; the interested reader is referred to Ref.~\cite{Krogel2014}.}
\end{itemize}


\begin{lstlisting}[style=QMCPXML,caption=One body density matrix with uniform grid integration.]
  <estimator type="dm1b" name="DensityMatrices">
    <parameter name="basis"        >  spo_u spo_uv  </parameter>
    <parameter name="evaluator"    >  matrix        </parameter>
    <parameter name="integrator"   >  uniform_grid  </parameter>
    <parameter name="points"       >  4             </parameter>
    <parameter name="scale"        >  1.0           </parameter>
    <parameter name="center"       >  0 0 0         </parameter>
  </estimator>
\end{lstlisting}


\begin{lstlisting}[style=QMCPXML,caption=One body density matrix with uniform sampling.]
  <estimator type="dm1b" name="DensityMatrices">
    <parameter name="basis"        >  spo_u spo_uv  </parameter>
    <parameter name="evaluator"    >  matrix        </parameter>
    <parameter name="integrator"   >  uniform       </parameter>
    <parameter name="samples"      >  64            </parameter>
    <parameter name="scale"        >  1.0           </parameter>
    <parameter name="center"       >  0 0 0         </parameter>
  </estimator>
\end{lstlisting}


\begin{lstlisting}[style=QMCPXML,caption=One body density matrix with density sampling.]
  <estimator type="dm1b" name="DensityMatrices">
    <parameter name="basis"        >  spo_u spo_uv  </parameter>
    <parameter name="evaluator"    >  matrix        </parameter>
    <parameter name="integrator"   >  density       </parameter>
    <parameter name="samples"      >  64            </parameter>
    <parameter name="timestep"     >  0.5           </parameter>
    <parameter name="use_drift"    >  no            </parameter>
  </estimator>
\end{lstlisting}


\begin{lstlisting}[style=QMCPXML,caption={Example \texttt{sposet} initialization for density matrix use.  Occupied and virtual orbital sets are created separately, then joined (\texttt{basis="spo\_u spo\_uv"}).}]
  <sposet_builder type="bspline" href="../dft/pwscf_output/pwscf.pwscf.h5" tilematrix="1 0 0 0 1 0 0 0 1" twistnum="0" meshfactor="1.0" gpu="no" precision="single">
    <sposet type="bspline" name="spo_u"  group="0" size="4"/>
    <sposet type="bspline" name="spo_d"  group="0" size="2"/>
    <sposet type="bspline" name="spo_uv" group="0" index_min="4" index_max="10"/>
  </sposet_builder>
\end{lstlisting}


\begin{lstlisting}[style=QMCPXML,caption={Example \texttt{sposet} initialization for density matrix use.  Density matrix orbital basis created separately (\texttt{basis="dm\_basis"}).}]
  <sposet_builder type="bspline" href="../dft/pwscf_output/pwscf.pwscf.h5" tilematrix="1 0 0 0 1 0 0 0 1" twistnum="0" meshfactor="1.0" gpu="no" precision="single">
    <sposet type="bspline" name="spo_u"  group="0" size="4"/>
    <sposet type="bspline" name="spo_d"  group="0" size="2"/>
    <sposet type="bspline" name="dm_basis" size="50" spindataset="0"/>
  </sposet_builder>
\end{lstlisting}


%  <estimator type="dm1b" name="DensityMatrices">
%     <parameter name="energy_matrix"       >    yes                   </parameter>
%     <parameter name="integrator"          >    uniform_grid            </parameter>
%     <parameter name="points"              >    6                     </parameter>
%     <parameter name="scale"               >    1.0                   </parameter>
%     <parameter name="basis"               >
%        spo_dm
%     </parameter>
%     <parameter name="evaluator"           >    matrix                </parameter>
%     <parameter name="center">
%        0 0 0
%     </parameter>
%     <parameter name="check_overlap"       >    no                    </parameter>
%  </estimator>
%
%  <sposet_builder type="bspline" href="./dft/pwscf_output/pwscf.pwscf.h5" tilematrix="1 0 0 0 1 0 0 0 1" twistnum="0" meshfactor="1.0" gpu="no" precision="single" sort="0">
%    <sposet type="bspline" name="spo_u" size="4" spindataset="0"/>
%    <sposet type="bspline" name="spo_d" size="2" spindataset="1"/>
%    <sposet type="bspline" name="dm_basis" size="50" spindataset="0"/>
%  </sposet_builder>
%  <estimator type="dm1b" name="DensityMatrices">
%    <parameter name="energy_matrix"       >    yes                   </parameter>
%    <parameter name="integrator"          >    uniform_grid          </parameter>
%    <parameter name="points"              >    10                    </parameter>
%    <parameter name="scale"               >    1.0                   </parameter>
%    <parameter name="basis"               >    dm_basis              </parameter>
%    <parameter name="normalized"          >    no                    </parameter>
%    <parameter name="evaluator"           >    matrix                </parameter>
%    <parameter name="center"              >    0 0 0                 </parameter>
%    <parameter name="check_overlap"       >    no                    </parameter>
%    <parameter name="rstats"              >    no                    </parameter>
%  </estimator>
%
%
% found at /psi2/home/development/qmcpack/energy_density_matrix/tests/r6080_edm/02_atoms/runs/O/qmc/vmc.in.xml

%
%  <sposet_builder type="bspline" href="../dft/pwscf_output/pwscf.pwscf.h5" tilematrix="1 0 0 0 1 0 0 0 1" twistnum="0" meshfactor="1.0" gpu="no" precision="single">
%    <sposet type="bspline" name="spo_u"  group="0" size="4"/>
%    <sposet type="bspline" name="spo_d"  group="0" size="2"/>
%    <sposet type="bspline" name="spo_uv" group="0" index_min="4" index_max="10"/>
%  </sposet_builder>
%  <estimator type="dm1b" name="DensityMatrices">
%    <parameter name="basis"        >  spo_u spo_uv  </parameter>
%    <parameter name="energy_matrix">  yes           </parameter>
%    <parameter name="evaluator"    >  matrix        </parameter>
%    <parameter name="center"       >  0 0 0         </parameter>
%    <parameter name="rstats"           >  no        </parameter>
%    <parameter name="acceptance_ratio" >  no        </parameter>
%    <parameter name="check_overlap"    >  no        </parameter>
%    <parameter name="check_derivatives">  no        </parameter>
%    
%    <parameter name="integrator"   >  uniform_grid  </parameter>
%    <parameter name="points"       >  20            </parameter>
%    <parameter name="scale"        >  1.0           </parameter>
% 
%    <!--
%    <parameter name="integrator"   >  uniform       </parameter>
%    <parameter name="samples"      >  14          </parameter>
%    <parameter name="scale"        >  1.0           </parameter>
%    -->
%    
%    <!--
%    <parameter name="integrator"   >  density       </parameter>
%    <parameter name="timestep"     >  1.0           </parameter>
%    <parameter name="use_drift"    >  no            </parameter>
%    <parameter name="samples"      >  1000          </parameter>
%    -->
%    
%    <!--
%    <parameter name="integrator"   >  density       </parameter>
%    <parameter name="timestep"     >  1.0           </parameter>
%    <parameter name="use_drift"    >  yes           </parameter>
%    <parameter name="samples"      >  1000          </parameter>
%    -->
%  </estimator>


\section{Forward-Walking Estimators} \label{sec:forward_walking}
Forward walking is a method for sampling the pure fixed-node distribution $\langle \Phi_0 | \Phi_0\rangle$.  Specifically, one multiplies each walker's DMC mixed estimate for the observable $\mathcal{O}$, $\frac{\mathcal{O}(\mathbf{R})\Psi_T(\mathbf{R})}{\Psi_T(\mathbf{R})}$, by the weighting factor $\frac{\Phi_0(\mathbf{R})}{\Psi_T(\mathbf{R})}$.  As it turns out, this weighting factor for any walker $\mathbf{R}$ is proportional to the total number of descendants the walker will have after a sufficiently long projection time $\beta$.  

To forward walk on an observable, declare a generic forward-walking estimator within a \texttt{<hamiltonian>} block, and then specify the observables to forward walk on and the forward-walking parameters.  Here is a summary.\\  

\begin{table}[h]
\begin{center}
\begin{tabularx}{\textwidth}{l l l l l X }
\hline
\multicolumn{6}{l}{\texttt{estimator type=ForwardWalking} element} \\
\hline
\multicolumn{2}{l}{parent elements:} & \multicolumn{4}{l}{\texttt{hamiltonian, qmc}}\\
\multicolumn{2}{l}{child  elements:} & \multicolumn{4}{l}{\texttt{Observable}}\\
\multicolumn{2}{l}{attributes}  & \multicolumn{4}{l}{}\\
   & \bfseries name       & \bfseries datatype & \bfseries values  & \bfseries default   & \bfseries description \\
   & \texttt{type}$^r$    &  text              & \textbf{ForwardWalking}&                & Must be ``ForwardWalking" \\
   & \texttt{name}$^r$    &  text              & \textit{anything} & any                 & Unique name for estimator \\
  \hline
\end{tabularx}
\end{center}
\end{table}

\begin{table}[h]
\begin{center}
\begin{tabularx}{\textwidth}{l l l l l X }
\hline
\multicolumn{6}{l}{\texttt{Observable} element} \\
\hline
\multicolumn{2}{l}{parent elements:} & \multicolumn{4}{l}{\texttt{estimator, hamiltonian, qmc}}\\
\multicolumn{2}{l}{child  elements:} & \multicolumn{4}{l}{\textit{None}}\\
\multicolumn{2}{l}{attributes}  & \multicolumn{4}{l}{}\\
   & \bfseries name       & \bfseries datatype & \bfseries values  & \bfseries default   & \bfseries description \\
   & \texttt{name}$^r$    &  text              & \textit{anything} & any                 & Registered name of existing estimator on which to forward walk \\
   & \texttt{max}$^r$     &  integer             & $ > 0$     &                     & Maximum projection time in steps (\texttt{max}$=\beta/\tau$)      \\
   & \texttt{frequency}$^r$     &  text              & $\geq 1$      &                & Dump data only for every \texttt{frequency}-th  \\
      &  &               &     &                & to \texttt{scalar.dat} file   \\

  \hline
\end{tabularx}
\end{center}
\end{table}

Additional information:
\begin{itemize}
  \item{\textbf{Cost}:  Because histories of observables up to \texttt{max} time steps have to be stored, the memory cost of storing the nonforward-walked observables variables should be multiplied by $\texttt{max}$.  Although this is not an issue for items such as potential energy, it could be prohibitive for observables such as density, forces, etc.  }
  \item{\textbf{Naming Convention}: Forward-walked observables are automatically named \texttt{FWE\_name\_i}, where \texttt{i} is the forward-walked expectation value at time step \texttt{i}, and \texttt{name} is whatever name appears in the \texttt{<Observable>} block.  This is also how it will appear in the \texttt{scalar.dat} file.  }
\end{itemize}

In the following example case, QMCPACK forward walks on the potential energy for 300 time steps and dumps the forward-walked value at every time step.  

\begin{lstlisting}[style=QMCPXML,caption=Forward-walking estimator element.]
  <estimator name="fw" type="ForwardWalking">
      <Observable name="LocalPotential" max="300" frequency="1"/>
       <!--- Additional Observable blocks go here -->
   </estimator>
\end{lstlisting}






\section{``Force'' estimators} \label{sec:force_est}

% Force estimators added in CoulombPotentialFactory.cpp, HamiltonianFactory::addForceHam
QMCPACK supports force estimation by use of the Chiesa-Ceperly-Zhang (CCZ) estimator.  Currently, open and periodic boundary conditions are supported but for all-electron calculations only.  

Without loss of generality, the CCZ estimator for the z-component of the force on an ion centered at the origin is given by the following expression:
\begin{equation}
F_z = -Z \sum_{i=1}^{N_e}\frac{z_i}{r_i^3}[\theta(r_i-\mathcal{R}) + \theta(\mathcal{R}-r_i)\sum_{\ell=1}^{M}c_\ell r_i^\ell]\:.
\end{equation}

Z is the ionic charge, $M$ is the degree of the smoothing polynomial, $\mathcal{R}$ is a real-space cutoff of the sphere within which the bare-force estimator is smoothed, and $c_\ell$ are predetermined coefficients.  These coefficients are chosen to minimize the weighted mean square error between the bare force estimate and the s-wave filtered estimator.  Specifically, 
\begin{equation}
\chi^2 = \int_0^\mathcal{R}dr\,r^m\,[f_z(r) - \tilde{f}_z(r)]^2\:.
\end{equation}
Here, $m$ is the weighting exponent, $f_z(r)$ is the unfiltered radial force density for the z force component, and $\tilde{f}_z(r)$ is the smoothed polynomial function for the same force density.  The reader is invited to refer to the original paper for a more thorough explanation of the methodology, but with the notation in hand, QMCPACK takes the following parameters.
\FloatBarrier
\begin{table}[h]
\begin{center}
\begin{tabularx}{\textwidth}{l l l l l X }
\hline
\multicolumn{6}{l}{\texttt{estimator type=Force} element} \\
\hline
\multicolumn{2}{l}{parent elements:} & \multicolumn{4}{l}{\texttt{hamiltonian, qmc}}\\
\multicolumn{2}{l}{child  elements:} & \multicolumn{4}{l}{\texttt{parameter}}\\
\multicolumn{2}{l}{attributes}  & \multicolumn{4}{l}{}\\
   & \bfseries name       & \bfseries datatype & \bfseries values  & \bfseries default   & \bfseries description \\
   &   \texttt{mode}$^o$      &  text              & \textit{See above}        & bare          & Select estimator type\\
   &   \texttt{type}$^r$      &  text              &  Force            &               & Must be ``Force"         \\
   &   \texttt{name}$^o$      &  text              & \textit{anything}         & ForceBase     & Unique name for this estimator\\
%   &   \texttt{psi}$^o$       &  text              & \texttt{wavefunction.name}& psi0          & Identify wavefunction\\
   &   \texttt{pbc}$^o$       &  boolean           & yes/no                    & yes           & Using periodic BC's or not\\
   &   \texttt{addionion}$^o$       &  boolean           & yes/no                    & no           & Add the ion-ion force contribution to output force estimate   \\
   \multicolumn{2}{l}{parameters}  & \multicolumn{4}{l}{}\\
   & \bfseries name       & \bfseries datatype & \bfseries values  & \bfseries default   & \bfseries description \\
   &   \texttt{rcut}$^o$      &  real             & $> 0$        & 1.0         & Real-space cutoff $\mathcal{R}$ in bohr\\
   &   \texttt{nbasis}$^o$      &  integer              & $> 0 $           &  2            & Degree of smoothing polynomial $M$ \\
   &   \texttt{weightexp}$^o$      &  integer              &$ > 0$         & 2    & $\chi^2$ weighting exponent $m$\\  
  \hline
\end{tabularx}
\end{center}
\end{table}
\FloatBarrier

Additional information:
\begin{itemize}
  \item{\textbf{Naming Convention}:  The unique identifier \texttt{name} is appended with \texttt{name\_X\_Y} in the \texttt{scalar.dat} file, where \texttt{X} is the ion ID number and \texttt{Y} is the component ID (an integer with x=0, y=1, z=2).  All force components for all ions are computed and dumped to the \texttt{scalar.dat} file.}
  \item{\textbf{Miscellaneous}: Usually, the default choice of \texttt{weightexp} is sufficient.  Different combinations of \texttt{rcut} and  \texttt{nbasis} should be tested though to minimize variance and bias.  There is, of course, a tradeoff, with larger \texttt{nbasis} and smaller \texttt{rcut} leading to smaller biases and larger variances.  }
\end{itemize}

The following is an example use case.  
\begin{lstlisting}[style=QMCPXML]
<estimator name="myforce" type="Force" mode="cep" addionion="yes">
    <parameter name="rcut">0.1</parameter>
    <parameter name="nbasis">4</parameter>
    <parameter name="weightexp">2</parameter>
</estimator>
\end{lstlisting}




\input{methods}
\section{Variational Monte Carlo}
\label{sec:vmc}

\begin{table}[h]
\begin{tabularx}{\textwidth}{l l l l l X }
\hline
\multicolumn{6}{l}{\texttt{vmc} method} \\
\hline
\multicolumn{2}{l}{parameters}  & \multicolumn{4}{l}{}\\
   &   \bfseries name     & \bfseries datatype & \bfseries values & \bfseries default   & \bfseries description \\
   &   \texttt{walkers             } &  integer  & $> 0$   & dep.& Number of walkers per MPI task  \\
   &   \texttt{blocks              } &  integer  & $\ge 0$ & 1   & Number of blocks            \\
   &   \texttt{steps               } &  integer  & $\ge 0$ & 1   & Number of steps per block   \\
   &   \texttt{warmupsteps         } &  integer  & $\ge 0$ & 0   & Number of steps for warming up\\
   &   \texttt{substeps            } &  integer  & $\ge 0$ & 1   & Number of substeps per step \\
   &   \texttt{usedrift            } &  text     & yes, no & yes  & Use the algorithm with drift\\
   &   \texttt{timestep            } &  real     & $> 0$   & 0.1 & Time step for each electron move \\
   &   \texttt{samples             } &  integer  & $\ge 0$ & 0   & Number of walker samples for DMC/optimization\\
   &   \texttt{stepsbetweensamples } &  integer  & $> 0$   & 1   & Period of sample accumulation\\
   &   \texttt{samplesperthread    } &  integer  & $\ge 0$ & 0   & Number of samples per thread  \\
   &   \texttt{storeconfigs        } &  integer  & all values & 0   & Store configurations o  \\
   &   \texttt{blocks\_between\_recompute} &  integer  & $\ge 0$ & dep.  & Wavefunction recompute frequency  \\
  \hline
\end{tabularx}
\end{table}

Additional information:
\begin{itemize}
\item \ixml{walkers}: The initial default number of \ixml{walkers} is one per OpenMP thread or per MPI task if threading is disabled, with a minimum of one per thread. One walker per thread is created in the event fewer \ixml{walkers} than threads are requested. 

\item \ixml{blocks}: This parameter is universal for all the QMC
  methods. The MC processes are divided into a number of
  \ixml{blocks}, each containing a number of steps. At the end of each block,
  the statistics accumulated in the block are dumped into files,
  e.g., \ixml{scalar.dat}. Typically, each block should have a sufficient number of steps that the I/O at the end of each block is negligible
  compared with the computational cost. Each block should not take so
  long that monitoring its progress is difficult. There should be a
  sufficient number of \ixml{blocks} to perform statistical analysis.

\item \ixml{warmupsteps}: \ixml{warmupsteps} are used only for
  equilibration. Property measurements are not performed during
  warm-up steps.

\item \ixml{steps}: \ixml{steps} are the number of energy and other property measurements to perform per block.
  
\item \ixml{substeps}: For each substep, an attempt is made to move each of the electrons once only by either particle-by-particle or an
  all-electron move.  Because the local energy is evaluated only at
  each full step and not each substep, \ixml{substeps} are computationally cheaper
  and can be used to reduce the correlation between property measurements
  at a lower cost.
  
\item \ixml{usedrift}: The VMC is implemented in two algorithms with
  or without drift. In the no-drift algorithm, the move of each
  electron is proposed with a Gaussian distribution. The standard
  deviation is chosen as the time step input. In the drift algorithm,
  electrons are moved by Langevin dynamics.

\item \ixml{timestep}: The meaning of time step depends on whether or not
  the drift is used. In general, larger time steps reduce the
  time correlation but might also reduce the acceptance ratio,
  reducing overall statistical efficiency. For VMC, typically the
  acceptance ratio should be close to 50\% for an efficient
  simulation.

\item \ixml{samples}: Seperate from conventional energy and other
  property measurements, samples refers to storing whole electron
  configurations in memory (``walker samples'') as would be needed by subsequent
  wavefunction optimization or DMC steps. \textit{A standard VMC run to
  measure the energy does not need samples to be set.}

\[
\texttt{samples}=
\frac{\texttt{blocks}\cdot\texttt{steps}\cdot\texttt{walkers}}{\texttt{stepsbetweensamples}}\cdot\texttt{number of MPI tasks}
\]

\item \ixml{samplesperthread}: This is an alternative way to set the target amount of samples and can be useful when preparing a stored population for a subsequent DMC calculation.
\[
\texttt{samplesperthread}=
\frac{\texttt{blocks}\cdot\texttt{steps}}{\texttt{stepsbetweensamples}}
\]

\item \ixml{stepsbetweensamples}: Because samples generated by consecutive steps are correlated, having \ixml{stepsbetweensamples} larger than 1 can be used to reduces that correlation. In practice, using larger substeps is cheaper than using \ixml{stepsbetweensamples} to decorrelate samples. 

\item \ixml{storeconfigs}: If \ixml{storeconfigs} is set to a nonzero value, then electron configurations during the VMC run are saved to files.

\item \ixml{blocks_between_recompute}: Recompute the accuracy critical determinant part of the wavefunction
  from scratch: =1 by default when using mixed precision. =0 (no
  recompute) by default when not using mixed precision. Recomputing
  introduces a performance penalty dependent on system size.
\end{itemize}

An example VMC section for a simple VMC run:
\begin{lstlisting}[style=QMCPXML]
  <qmc method="vmc" move="pbyp">
    <estimator name="LocalEnergy" hdf5="no"/>
    <parameter name="walkers">    256 </parameter>
    <parameter name="warmupSteps">  100 </parameter>
    <parameter name="substeps">  5 </parameter>
    <parameter name="blocks">  20 </parameter>
    <parameter name="steps">  100 </parameter>
    <parameter name="timestep">  1.0 </parameter>
    <parameter name="usedrift">   yes </parameter>
  </qmc>
\end{lstlisting}
Here we set 256 \ixml{walkers} per MPI, have a brief initial equilibration of 100 \ixml{steps}, and then have 20 \ixml{blocks} of 100 \ixml{steps} with 5 \ixml{substeps} each.

The following is an example of VMC section storing configurations (walker samples) for optimization.
\begin{lstlisting}[style=QMCPXML]
  <qmc method="vmc" move="pbyp" gpu="yes">
    <estimator name="LocalEnergy" hdf5="no"/>
    <parameter name="walkers">    256 </parameter>
    <parameter name="samples">    2867200 </parameter>
    <parameter name="stepsbetweensamples">    1 </parameter>
    <parameter name="substeps">  5 </parameter>
    <parameter name="warmupSteps">  5 </parameter>
    <parameter name="blocks">  70 </parameter>
    <parameter name="timestep">  1.0 </parameter>
    <parameter name="usedrift">   no </parameter>
  </qmc>
\end{lstlisting}




\section{Wavefunction Optimization}
\label{sec:optimization}

Optimizing wavefunction is critical in all kinds of real-space quantum Monte Carlo calculations
because it significantly improves both the accuracy and efficiency of computation.
However, it is very difficult to directly adopt deterministic minimization approaches due to the stochastic nature of evaluating quantities with Monte Carlo.
Thanks to the algorithmic breakthrough during the first decade of this century and the tremendous computer power available, 
it becomes feasible to optimize tens of thousands of parameters in a wavefunction for a solid or molecule.
QMCPACK has multiple optimizers implemented based on the state-of-the-art linear method.
We are continually improving our optimizers for the robustness and friendliness and trying to provide a single solution.
Due to the large variation of wavefunction types carrying distinct characteristics, using several optimizer may be needed in some cases.
It is highly suggested to read the recommendation from the experts maintaining these optimizers.

A typical optimization block looks like the following. It starts with method=``linear" and contains three blocks of parameters.
\begin{lstlisting}[style=QMCPXML]
 <loop max="10">
  <qmc method="linear" move="pbyp" gpu="yes">
    <!-- Specify the VMC options -->
    <parameter name="walkers">              256 </parameter>
    <parameter name="samples">          2867200 </parameter>
    <parameter name="stepsbetweensamples">    1 </parameter>
    <parameter name="substeps">               5 </parameter>
    <parameter name="warmupSteps">            5 </parameter>
    <parameter name="blocks">                70 </parameter>
    <parameter name="timestep">             1.0 </parameter>
    <parameter name="usedrift">              no </parameter>
    <estimator name="LocalEnergy" hdf5="no"/>
    ...
    <!-- Specify the correlated sampling options and define the cost function -->
    <parameter name="minwalkers">            0.3 </parameter>
         <cost name="energy">               0.95 </cost>
         <cost name="unreweightedvariance"> 0.00 </cost>
         <cost name="reweightedvariance">   0.05 </cost>
    ...
    <!-- Specify the optimizer options -->
    <parameter name="MinMethod">    OneShiftOnly </parameter>
    ...
  </qmc>
 </loop>
\end{lstlisting}
\begin{itemize}
\item loop is helpful to execute identical optimization blocks repeatedly.
\item The first part is highly identical to a regular VMC block.
\item The second part is to specify the correlated sampling options and define the cost function.
\item The last part is used to specify the options of different optimizers. They can be very distinct from one optimizer to another.
\end{itemize}

\subsection{VMC run for the optimization}
The VMC calculation for the wavefunction optimization has a strict requirement 
that \ixml{samples} or \ixml{samplesperthread} must be specified because of the optimizer needs for the stored samples.
The input parameters of this part are identical to the VMC method.

Recommendations:
\begin{itemize}
\item Run the inclusive VMC calculation correctly and efficiently, because the this part takes significant amount of time in optimization.
For example, make sure the derived steps per block is 1 and use larger substeps to control the correlation between samples.
\item A reasonable starting wavefunction is necessary. A lot of optimization fails because of a bad wavefunction starting point.
The sign of a bad initial wavefunction includes but not limited to very long equilibration time, low acceptance ratio and huge variance.
The first thing to do after a failed optimization is to check the information provided by the VMC calculation via *.scalar.dat files.
\end{itemize}

\subsection{Correlated sampling and Cost function}
After generating the samples with VMC, the derivatives of the wavefunction with respect to the parameters are computed for proposing a new set of parameters by optimizers.
And later, a correlated sampling calculation is performed to quickly evaluate values of the cost function on the old set of parameters and the new set for the further decisions.
The input parameters are listed in the following table.
\begin{table}[h]
\begin{center}
\begin{tabularx}{\textwidth}{l l l l l X }
\hline
\multicolumn{6}{l}{\texttt{linear} method} \\
\hline
\multicolumn{2}{l}{parameters}  & \multicolumn{4}{l}{}\\
   &   \bfseries name     & \bfseries datatype & \bfseries values & \bfseries default   & \bfseries description \\
   &   \texttt{nonlocalpp} &  text     & yes, no & no  & include non-local PP energy in the cost function\\
%   &   \texttt{GEVMethod} &  text     & mixed, H2 & mixed  & methods of generalized eigenvalue problem\\
%   &   \texttt{beta} &  real     & any value & 0.0  & a parameter for GEVMethod\\
%   &   \texttt{use\_nonlocalpp\_deriv} &  text     & yes, no & no  & include the derivatives of non-local PP\\
   &   \texttt{minwalkers} &  real     & 0--1   & 0.3 & lower bound of the effective weight\\
   &   \texttt{maxWeight} &  real     & $>1$   & 1e6 & Maximum weight allowed in reweighting\\
  \hline
\end{tabularx}
\end{center}
\end{table}

Additional information:
\begin{itemize}
\item \ixml{maxWeight}. The default should be good.
\item \ixml{nonlocalpp}. Non-local PP contribution to the local energy depends on the wavefunction.
When a new set of parameter is proposed, this contribution needs to be updated if the cost function consists of local energy.
Fortunately, non-local contribution is chosen small when making a PP for small locality error.
We can ignore its change and avoid the expensive computational cost.
GPU code has a implementation issue that large amount of memory is consumed with this option.
\item \ixml{minwalkers}. A CRITICAL parameter. When the ratio of effective samples to actual number of samples in a reweighting step goes lower than \ixml{minwalkers},
the proposed set of parameters is invalid. % The last set of acceptable parameters is kept.
\end{itemize}

The cost function consists of three components: energy, unreweighted variance and reweighted variance.
\begin{lstlisting}[style=QMCPXML]
     <cost name="energy">                   0.95 </cost>
     <cost name="unreweightedvariance">     0.00 </cost>
     <cost name="reweightedvariance">       0.05 </cost>
\end{lstlisting}

\subsection{Optimizers}
QMCPACK implements a few optimizers having different preference aiming for different priorities.
They can be switched among  `OneShiftOnly' (default), `adaptive' and `quartic' (old) by the following line in the optimization block.
\begin{lstlisting}
<parameter name="MinMethod"> THE METHOD YOU LIKE </parameter>
\end{lstlisting}

\subsubsection{OneShiftOnly}
The OneShiftOnly optimizer targets a fast optimization by moving parameters more aggressively. It works with OpenMP and GPU and can be considered for large systems.
This method relies on the effective weight of correlated sampling rather than the cost function value to justify a new set of parameters.
If effective weight is larger than \ixml{minwalkers}, the new set is taken no matter the cost function value decreases or not.
If a proposed set is rejected, the standard output prints the measured ratio of effective samples to the total number of samples
and adjustment on \ixml{minwalkers} can be made if needed.

\begin{table}[h]
\begin{center}
\begin{tabularx}{\textwidth}{l l l l l X }
\hline
\multicolumn{6}{l}{\texttt{linear} method} \\
\hline
\multicolumn{2}{l}{parameters}  & \multicolumn{4}{l}{}\\
   &   \bfseries name     & \bfseries datatype & \bfseries values & \bfseries default   & \bfseries description \\
   &   \texttt{shift\_i} &  real     & $>0$ & 0.01 & Direct stabilizer added to the Hamiltonian matrix\\
   &   \texttt{shift\_s} &  real     & $>0$ & 1.00 & Initial stabilizer based on the overlap matrix\\
  \hline
\end{tabularx}
\end{center}
\end{table}

Additional information:
\begin{itemize}
\item \ixml{shift_i}. This is the direct term added to the diagonal of the Hamiltonian matrix.
                         More stable but slower optimization with a large value.
\item \ixml{shift_s}. This is the initial value of the stabilizer based on the overlap matrix added to the Hamiltonian matrix.
                         More stable but slower optimization with a large value. The used value is auto-adjusted by the optimizer.
\end{itemize}


Recommendations:
\begin{itemize}
  \item Default \ixml{shift_i}, \ixml{shift_s} should be fine.
  \item For hard cases, increasing \ixml{shift_i} (factor of 5 or 10) can significantly stabilize the optimization by reducing the pace towards the optimal parameter set.
  \item If the VMC energy of the last optimization iterations grows significantly, increase \ixml{minwalkers} closer to 1 and make the optimization stable.
  \item If the first iterations of optimization are rejected on a reasonable initial wavefunction, 
        lower the \ixml{minwalkers} value based on the measured value printed in the standard output to accept the move.
\end{itemize}

It is recommended to use this optimizer in two sections with a very small \ixml{minwalkers} in the first and a large value in the second like the following.
In the very beginning, parameters are far away form optimal values and large changes are proposed by the optimizer.
Having a small \ixml{minwalkers} allows accepting these changes much easier.
When the energy gradually converges, we can have a large \ixml{minwalkers} to avoid risky parameter sets.
\begin{lstlisting}[style=QMCPXML]
 <loop max="6">
  <qmc method="linear" move="pbyp" gpu="yes">
    <!-- Specify the VMC options -->
    <parameter name="walkers">                1 </parameter>
    <parameter name="samples">            10000 </parameter>
    <parameter name="stepsbetweensamples">    1 </parameter>
    <parameter name="substeps">               5 </parameter>
    <parameter name="warmupSteps">            5 </parameter>
    <parameter name="blocks">                25 </parameter>
    <parameter name="timestep">             1.0 </parameter>
    <parameter name="usedrift">              no </parameter>
    <estimator name="LocalEnergy" hdf5="no"/>
    <!-- Specify the optimizer options -->
    <parameter name="MinMethod">    OneShiftOnly </parameter>
    <parameter name="minwalkers">           1e-4 </parameter>
  </qmc>
 </loop>
 <loop max="12">
  <qmc method="linear" move="pbyp" gpu="yes">
    <!-- Specify the VMC options -->
    <parameter name="walkers">                1 </parameter>
    <parameter name="samples">            20000 </parameter>
    <parameter name="stepsbetweensamples">    1 </parameter>
    <parameter name="substeps">               5 </parameter>
    <parameter name="warmupSteps">            2 </parameter>
    <parameter name="blocks">                50 </parameter>
    <parameter name="timestep">             1.0 </parameter>
    <parameter name="usedrift">              no </parameter>
    <estimator name="LocalEnergy" hdf5="no"/>
    <!-- Specify the optimizer options -->
    <parameter name="MinMethod">    OneShiftOnly </parameter>
    <parameter name="minwalkers">            0.5 </parameter>
  </qmc>
 </loop>
\end{lstlisting}

For each optimization step, you will see
\begin{lstlisting}
The new set of parameters is valid. Updating the trial wave function!
\end{lstlisting}
or
\begin{lstlisting}
The new set of parameters is not valid. Revert to the old set!
\end{lstlisting}
Occasional rejection is fine. Frequent rejection indicates potential problems and users should inspect the VMC calculation or change optimization strategy.
To track the progress of optimization, using command ``qmca -q ev *.scalar.dat'' to look at the VMC energy and variance for each optimization step.

\subsubsection{adaptive}

The default setting of the adaptive optimizer is to construct the linear method Hamiltonian and overlap matrices explicitly and add different shifts to the Hamiltonian matrix 
as ``stabilizers''.
The generalized eigenvalue problem is solved for each shift to obtain updates to the wave function parameters.
Then a correlated sampling is performed for each shift's updated wave function and the initial trial wave function
using the middle shift's updated wave function as the guiding function.
The cost function for these wave functions is compared, and the update corresponding to the best cost function is selected.
In the next iteration, the median magnitude of the stabilizers is set to that that generated the best update in the current iteration, thus adapting the magnitude of
the stabilizers automatically.

When the trial wave function contains more than ten thousand parameters, constructing and storing the linear method matrices may become a memory bottleneck. 
To avoid explicit construction of these matrices, the adaptive optimizer implements the block linear method (BLM) approach. \cite{Zhao:2017:blocked_lm}
The BLM tries to find an approximate 
solution $\vec{c}_{opt}$ to the standard LM generalized eigenvalue problem by dividing the variable space into a number of blocks
and making intelligent estimates for which directions within those blocks will be most important for constructing $\vec{c}_{opt}$.
which is then obtained by solving a smaller, more memory-efficient 
eigenproblem in the basis of these supposedly important block-wise directions. 

\begin{table}[h]
\begin{center}
\begin{tabularx}{\textwidth}{l l l l l X }
\hline
\multicolumn{6}{l}{\texttt{linear} method} \\
\hline
\multicolumn{2}{l}{parameters}  & \multicolumn{4}{l}{}\\
   &   \bfseries name     & \bfseries datatype & \bfseries values & \bfseries default   & \bfseries description \\
   %&   \texttt{stepsize} &  real     & 0--1 & 0.25  & Step size for moving parameters\\
   &   \texttt{max\_relative\_change} &  real     & $>0$ & 10.0 & Allowed change in cost function\\
   &   \texttt{max\_param\_change} &  real     & $>0$ & 0.3 & Allowed change in wave function parameter\\
   &   \texttt{shift\_i} &  real     & $>0$ & 0.01 & Initial diagonal      stabilizer added to the Hamiltonian matrix\\
   &   \texttt{shift\_s} &  real     & $>0$ & 1.00 & Initial overlap-based stabilizer added to the Hamiltonian matrix\\
   &   \texttt{chase\_lowest} &  text   & yes, no & yes & Chase the lowest eigenvector in iterative solver\\
   &   \texttt{chase\_closest} &  text   & yes, no & no & Chase the eigenvector closest to initial guess\\
   &   \texttt{block\_lm} &  text   & yes, no & no & Use block linear method\\
   &   \texttt{nblocks} &  integer   & $>0$ &  & \# of blocks in BLM\\
   &   \texttt{nolds} &  integer   & $>0$ &  & \# of old update vectors used in BLM\\
   &   \texttt{nkept} &  integer   & $>0$ &  & \# of eigenvectors to keep per block in BLM\\
  \hline
\end{tabularx}
\end{center}
\end{table}

Additional information:
\begin{itemize}
  \item \ixml{shift_i}.  This is the initial coefficient used to scale the diagonal stabilizer.
                            More stable but slower optimization is expected with a large value.
                            The adaptive method will automatically adjust this value after each linear method iteration.
  \item \ixml{shift_s}.  This is the initial coefficient used to scale the overlap-based stabilizer.
                            More stable but slower optimization is expected with a large value.
                            The adaptive method will automatically adjust this value after each linear method iteration.
  \item \ixml{nblocks}.   This is the number of blocks used in block LM. The amount of memory required to store LM matrices decreases
                            with increased number of blocks. But the error introduced by BLM would increase with number of blocks.  
  \item \ixml{nolds}.     In BLM, the inter-block correlation is accounted for by including a small number of wave function update vectors
                            outside the block. Larger \ixml{nolds} would include more inter-block correlation and more accurate results, but 
                            also higher memory requirements. 
  \item \ixml{nkept}.     This is the number of update directions retained from each block in the BLM. If all directions are retained in each block, 
                            then the BLM becomes equivalent to the standard LM.  Retaining 5 or fewer directions per block is often sufficient.
\end{itemize}

Recommendations:
\begin{itemize}
  \item Default \ixml{shift_i}, \ixml{shift_s} should be fine. 
  \item When there are fewer than about 5,000 variables being optimized, the traditional LM is preferred as it has a lower overhead than the BLM when the number of variables is small.
  \item Initial experience with the BLM suggests that a few hundred blocks and a handful of \ixml{nolds} and \ixml{nkept}
        often provide a good balance between memory use and accuracy.  In general, using fewer blocks should be more accurate but will require more memory.
\end{itemize}

\begin{lstlisting}[style=QMCPXML]
 <loop max="15">
  <qmc method="linear" move="pbyp">
    <!-- Specify the VMC options -->
    <parameter name="walkers">                1 </parameter>
    <parameter name="samples">            20000 </parameter>
    <parameter name="stepsbetweensamples">    1 </parameter>
    <parameter name="substeps">               5 </parameter>
    <parameter name="warmupSteps">            5 </parameter>
    <parameter name="blocks">                50 </parameter>
    <parameter name="timestep">             1.0 </parameter>
    <parameter name="usedrift">              no </parameter>
    <estimator name="LocalEnergy" hdf5="no"/>
    <!-- Specify the correlated sampling options and define the cost function -->
         <cost name="energy">               1.00 </cost>
         <cost name="unreweightedvariance"> 0.00 </cost>
         <cost name="reweightedvariance">   0.00 </cost>
    <!-- Specify the optimizer options -->
    <parameter name="MinMethod">adaptive</parameter>
    <parameter name="max_relative_cost_change">10.0</parameter>
    <parameter name="shift_i"> 1.00 </parameter>
    <parameter name="shift_s"> 1.00 </parameter>
    <parameter name="max_param_change"> 0.3 </parameter>
    <parameter name="chase_lowest"> yes </parameter>
    <parameter name="chase_closest"> yes </parameter>
    <parameter name="block_lm"> no </parameter>
    <!-- Specify the BLM specific options if needed
      <parameter name="nblocks"> 100 </parameter>
      <parameter name="nolds"> 5 </parameter>
      <parameter name="nkept"> 3 </parameter>
    -->
  </qmc>
 </loop>
\end{lstlisting}
%To activate this optimizer, add ``-D BUILD\_LMYENGINE\_INTERFACE=1'' in the CMake command line.

The adaptive optimizer is also able to optimize individual excited states directly. \cite{Zhao:2016:dir_tar}
In this case, it tries to minimize the following function: 
\begin{equation*}
\Omega[\Psi]=\frac{\left<\Psi|\omega-H|\Psi\right>}{\left<\Psi|{\left(\omega-H\right)}^2|\Psi\right>}
\end{equation*}
The global minimum of this function corresponds to the state whose energy lies immediately above the shift parameter $\omega$ in the energy spectrum.
For example, if $\omega$ were placed in between the ground state energy and the first excited state energy and the wave function ansatz was capable of a good
description for the first excited state, then the wave function would be optimized for the first excited state.
It is important to note that, if the ansatz is not capable of a good description of the excited state in question, the optimization may converge to a different
state, as is known to occur in some circumstances for traditional ground state optimizations.
Note also that the ground state can be targeted by this method by choosing $\omega$ to be below the ground state energy, although we should stress that this
is not the same thing as a traditional ground state optimization and will in general give a slightly different wave function.
Excited state targeting requires two additional parameters, as shown in this table.

\begin{table}[h]
\begin{center}
\begin{tabularx}{\textwidth}{l l l l l X }
\hline
\multicolumn{6}{l}{Excited State Targeting} \\
\hline
\multicolumn{2}{l}{parameters}  & \multicolumn{4}{l}{}\\
   &   \bfseries name     & \bfseries datatype & \bfseries values & \bfseries default   & \bfseries description \\
   %&   \texttt{stepsize} &  real     & 0--1 & 0.25  & Step size for moving parameters\\
   &   \texttt{targetExcited} &  text   & yes, no      & no   & Whether to use the excited state targeting optimization\\
   &   \texttt{omega}         &  real   & real numbers & none & Energy shift used to target different excited states\\
  \hline
\end{tabularx}
\end{center}
\end{table}

Excited state recommendations:
\begin{itemize}
  \item Due to the finite variance in any approximate wave function, it is recommended to set $\omega=\omega_0-\sigma$, where $\omega_0$ is placed just
        below the energy of the targeted state and $\sigma^2$ is the energy variance.
  \item In order to obtain an unbiased excitation energy, one should optimize the ground state with the excited state variational principle as well by setting
        \ixml{omega} below the ground state energy.  Note that using the ground state variational principle for the ground state and the excited state variational
        principle for the excited state creates a bias in favor of the ground state. 
\end{itemize}

\subsubsection{quartic}
\textbf{This is an older optimizer method retained for compatibility. We recommend starting with the newest OneShiftOnly or adaptive optimizers.}
The quartic optimizer fits a quartic polynomial to 7 values of the cost function obtained using reweighting along chosen direction and determines the optimal move.
This optimizer is very robust but a bit conservative to accept new steps especially when large parameters changes are proposed.
\begin{table}[h]
\begin{center}
\begin{tabularx}{\textwidth}{l l l l l X }
\hline
\multicolumn{6}{l}{\texttt{linear} method} \\
\hline
\multicolumn{2}{l}{parameters}  & \multicolumn{4}{l}{}\\
   &   \bfseries name     & \bfseries datatype & \bfseries values & \bfseries default   & \bfseries description \\
   %&   \texttt{stepsize} &  real     & 0--1 & 0.25  & Step size for moving parameters\\
   &   \texttt{bigchange} &  real     & $>0$ & 50.0  & Largest parameter change allowed\\
   &   \texttt{alloweddifference} &  real     & $>0$ & 1e-4 & Allowed increased in energy\\
   &   \texttt{exp0} &  real     & any value & -16.0 & Initial value for stabilizer\\
   &   \texttt{stabilizerscale} &  real     & $>0$ & 2.0 & Increase in value of exp0 between iterations\\
   &   \texttt{nstabilizers} &  integer     & $>0$ & 3 & Number of stabilizers to try\\
   &   \texttt{max\_its} &  integer   & $>0$ & 1 & Number of inner loops with same samples\\
  \hline
\end{tabularx}
\end{center}
\end{table}

Additional information:
\begin{itemize}
\item \ixml{exp0}. It is the initial value for stabilizer (shift to diagonal of H). The actual value of stabilizer is $10^{\textrm{exp0}}$.
\end{itemize}

Recommendations:
\begin{itemize}
  \item{For hard cases (e.g. simultaneous optimization of long MSD and 3-Body J), set exp0
to 0 and do a single inner iteration (max its=1) per sample of configurations.}
\end{itemize}

\begin{lstlisting}[style=QMCPXML]
    <!-- Specify the optimizer options -->
    <parameter name="MinMethod">quartic</parameter>
    <parameter name="exp0">-6</parameter>
    <parameter name="alloweddifference"> 1.0e-4 </parameter>
    <parameter name="nstabilizers"> 1 </parameter>
    <parameter name="bigchange">15.0</parameter>
\end{lstlisting}

\subsection{General recommendations}
Here are a few recommendations to make wavefunction optimization easier.
\begin{itemize}
\item All electron wavefunctions are typically more difficult to optimize than pseudopotential ones due to the importance of the wavefunction near the nucleus.
\item Two body Jastrow contributes the largest portion of correlation energy from bare Slater determinants. For this reason, the recommended order for optimizing wavefunction components is two-body, one-body, three-body Jastrow factors and MSD coefficients.
\item For two-body spline Jastrows, always start from a reasonable one. The lack of physically-motivated constraints in the functional form at large distances can cause slow convergence if starting from zero. 
\item One-body spline Jastrow from old calculations can be a good starting point.
\item Three-body polynomial Jastrow can start from zero. It is beneficial to first optimize one-body and two-body Jastrow factors without adding three-body terms in the calculation and then add the three-body Jastrow and optimize all the three components together.
\end{itemize}
\subsubsection{Optimization of CI coefficients}
When storing a CI wavefunction in HDF5 format, the CI coefficients and the $\alpha$ and $\beta$ components of each CI are not in the XML input file. When optimizing the CI coefficients, they will be stored in HDF5 format. 
The optimization header block will have to specify that the new CI coefficients will be saved to HDF5 format. If the tag is not added coefficients will not be saved. 
begin{lstlisting}[style=QMCPXML]
  <qmc method="linear" move="pbyp" gpu="no" hdf5="yes">
\end{lstlisting}

The rest of the optimization block remains the same. 

When running the optimization, the new coefficients will be stored in a *.sXXX.opt.h5 file,  where XXX coressponds to the series number. The H5 file contains only the optimized coefficients. The corresponding *.sXXX.opt.xml  will be updated for each optimization block as follow: 
begin{lstlisting}[style=QMCPXML]
<detlist size="1487" type="DETS" nca="0" ncb="0" nea="2" neb="2" nstates="85" cutoff="1e-2" href="../LiH.orbs.h5" opt_coeffs="LiH.s001.opt.h5"/>
\end{lstlisting}

the opt_coeffs tag will then reference where the new CI coefficients are stored.\\

When restarting the run with the new optimized coeffs, you need to specify the previous hdf5 containing the basis set, orbitals, and MSD, as well as the new optimized coefficients. The code willread the previous data but will rewrite the coefficient that were optimized with the values found in the *.sXXX.opt.h5 file. 
\subsection{General recommendations}
Here are a few recommendations to avoid bad calculations
\begin{itemize}
\item When optimizing CI coefficients, they are optimized with a set of Jastrows. Make sure you maintaining the pair (Jastrows, CI-coefficients) as computed. 
\end{itemize}



\section{Diffusion Monte Carlo}
\label{sec:dmc}
\pagebreak
\begin{table}[h]
\begin{center}
\begin{tabularx}{\textwidth}{l l l l l l }
\hline
\multicolumn{6}{l}{\texttt{dmc} method} \\
\hline
\multicolumn{2}{l}{parameters}  & \multicolumn{4}{l}{}\\
   &   \bfseries name     & \bfseries datatype & \bfseries values & \bfseries default   & \bfseries description \\
   &   \texttt{targetwalkers             } &  integer  & $> 0$ & dep.   & overall total number of walkers \\
   &   \texttt{blocks              } &  integer  & $\ge 0$ & 1   & number of blocks            \\
   &   \texttt{steps               } &  integer  & $\ge 0$ & 1   & number of steps per block   \\
   &   \texttt{warmupsteps         } &  integer  & $\ge 0$ & 0   & number of steps for warming up\\
   &   \texttt{timestep            } &  real     & $> 0$ & 0.1 & time step for each electron move \\
  % &   \texttt{samples             } &  real  & $\ge 0$ & 0   & total number of samples \\
%   &   \texttt{stepsbetweensamples } &  integer  & $> 0$ & 1   & period of the sample accumulation\\
%   &   \texttt{samplesperthread    } &  real  & $\ge 0$ & 0   & number of samples per thread  \\
%   &   \texttt{rewind              } &  integer  & $\ge 0$ & 0   & number of blocks to roll back   \\
%   &   \texttt{storeconfigs        } &  integer  & $\ge 0$ & 0   & whether to store samples  \\
   &   \texttt{checkproperties     } &  integer  & $\ge 0$ & 100   & number of steps between walker updates  \\
%  &   \texttt{recordwalkers       } &  integer  & $\ge 0$ & 0   & number of steps between saving a sample configuration. (only for VMC)  \\
%   &   \texttt{recordconfigs       } &  integer  & $\ge 0$ & 0   & number of steps between dumping a configuration to h5  \\
%   &   \texttt{current             } &  integer  & $\ge 0$ & 0   & current step (only used in optimization runs)  \\
%   &   \texttt{dmcwalkersperthread } &  real  & $\ge 0$ & 0   & number of samples per thread  \\
   &   \texttt{maxcpusecs          } &  real  & $\ge 0$ & 3.6e5   & maximum allowed walltime in seconds \\
   &   \texttt{energyUpdateInterval} &  integer  & $\ge 0$ & 0   & trial energy update interval \\
   &   \texttt{refEnergy           } &  AU  & all values & dep.   & reference energy  \\
   &   \texttt{feedback            } &  double  & $\ge 0$ & 1.0   & population feedback on the trial energy \\
   &   \texttt{useBareTau          } &  option  & yes,no & 0   & do not use effective time step  \\
   &   \texttt{warmupByReconfiguration} &  option  & yes,no & 0   & warm up with a fixed population  \\
 %  &   \texttt{energyBound         } &  double  & $\ge 0$ & 0   & number of samples per thread  \\
   &   \texttt{sigmaBound          } &  double  & $\ge 0$  & 10   & parameter to cutoff large weights  \\
   &   \texttt{killnode            } &  string  & yes/other & no   & kill or reject walkers that cross nodes  \\
  % &   \texttt{benchmark           } &  string  & $\ge 0$ & 0   & number of sample \\
   &   \texttt{reconfiguration     } &  string  & yes/pure/other & no   & fixed population technique  \\
   &   \texttt{branchInterval      } &  integer  & $\ge 0$ & 1   & branching interval \\
   &   \texttt{substeps            } &  integer  & $\ge 0$ & 1   & branching interval \\
   &   \texttt{nonlocalmoves       } &  string  & yes/v0/v1/no & no   & run with tmoves  \\
   &   \texttt{scaleweight         } &  string  & yes/other & yes   & scale weights (CUDA only)  \\
   &   \texttt{MaxAge              } &  double  & $\ge 0$ & 10   & kill persistent walkers  \\
    &   \texttt{MaxCopy             } &  double  & $\ge 0$ &2   & limit population growth \\
   &   \texttt{fastgrad            } &  text  & yes/other & yes   & fast gradients  \\
 %  &   \texttt{printderivs         } &  text  & $\ge 0$ & 0   & number of samples per  thread  \\
 %  &   \texttt{wlen                } &  integer  & $\ge 0$ & 0   & number of samples per  thread  \\
   &   \texttt{maxDisplSq      } &  real  & all values & -1   & maximum particle move  \\
   &   \texttt{storeconfigs        } &  integer  & all values & 0   & store configurations  \\
   &   \texttt{use\_nonblocking    } &  string  & yes/no & yes   & using non-blocking send/recv \\
   &   \texttt{blocks\_between\_recompute} &  integer  & $\ge 0$ & dep.  & wavefunction recompute frequency  \\
  \hline
\end{tabularx}
\end{center}
\end{table}

Additional information:
\begin{itemize}
\item \texttt{targetwalkers}.  A DMC run can be considered a restart run or a new run.  A restart run is considered to be any method block beyond the first one, such as when a DMC method block that follows a VMC block.  Alternatively,  if the user reads in configurations from disk it is also considered a restart run.  In the case of a restart run, the DMC driver will use the configurations from the previous run, and this variable will not be used.  For a new run, if the number of walkers is less than the number of threads, then the number of walkers will be set equal to the number of threads.  

\item \texttt{blocks}. Number of blocks run during an DMC method block.  A block consists of a number of DMC steps (steps), after which all the statistics accumulated in the block are written to disk.

\item \texttt{steps}. Number of diffusion Monte Carlo steps in a block.

\item \texttt{warmupsteps}. Warm-up steps are steps at the beginning of a DMC run in which the 
instantaneous average energy is used to update the trial energy.  During regular steps, E$_{ref}$ is used.

\item \texttt{timestep}. The timestep determines the accuracy of the imaginary time propagator.  Generally, multiple time steps are used to extrapolate to the infinite time step limit.   A good range of timesteps  in which to perform time step extrapolation will typically have  a minimum of 99\% acceptance probability for each step.

\item \texttt{checkproperties}.  When using particle by particle driver, this variable specifies how often to reset all the variables kept in the buffer.

\item \texttt{maxcpusecs}. The default is 100 hours. Once the specified time has elapsed, the program will finalize the simulation even if not all blocks are completed.

\item \texttt{energyUpdateInterval}. The default is to update the trial energy at every step. Otherwise the trial energy is updated every \texttt{energyUpdateInterval} steps.

\[
E_{\text{trial}}=
\textrm{refEnergy}+\textrm{feedback}\cdot(\ln\textrm{targetWalkers}-\ln N)
\]
where $N$ is the current population.

\item \texttt{refEnergy}. The default reference energy is taken from the VMC run that precedes the DMC run. This value is updated to the current mean whenever branching happens.

\item \texttt{feedback}. Variable used to determine how strong to react to population fluctuations when doing population control.  See the equation in energyUpdateInterval for more details.

\item \texttt{useBareTau}. The same time step is used whether a move is rejected to not. The default is to use an effective time step when a move is rejected.

\item \texttt{warmupByReconfiguration}.  Warmup DMC is done with a fixed population

\item \texttt{sigmaBound}.  Determine the branch cutoff to limit wild weights based on the sigma and sigmaBound

\item \texttt{killnode}.  When running fixed-node, if a walker attempts to cross a node, the move will normally be rejected.  If killnode = "yes", then walkers are destroyed when they cross a node.

%\item \texttt{benchmark}. 

\item \texttt{reconfiguration}.  If reconfiguration is "yes", then run with a fixed walker population using the reconfiguration technique.  

\item \texttt{branchInterval}. Number of steps between branching.  The total number of DMC steps in a block will be BranchInterval*Steps.   

\item \texttt{substeps}.  Same as BranchInterval.


\item \texttt{nonlocalmoves}. Evaluate pseudopotentials using one of the nonlocal move algorithms such as t-moves.  Setting nonlocalmoves to `no' will impose the locality approximation. `yes/v0' implements the algorithm in the 2006 Casula paper~\cite{Casula2006} and `v1' implements the v1 algorithm in the 2010 Casula paper~\cite{Casula2010}.
The v1 algorithm is size-consistent and an important advance over the previous v0 non-size-consistent algorithm. Investigating the importance of size consistency is highly recommended.

\item \texttt{scaleweight}. Scaling weight per Umrigar/Nightengale.  CUDA only.

\item \texttt{MaxAge}. Set the weight of a walker to min(currentweight,0.5) after a walker has not moved for MaxAge steps.  Needed if persistent walkers appear during the course of a run.

\item \texttt{MaxCopy}. When determining the number of copies of a walker to branch, set the number of copies equal to min(Multiplicity,MaxCopy).

\item \texttt{fastgrad}. Calculates gradients with either the fast version or the full-ratio version.

\item \texttt{maxDisplSq}.  When running a DMC calculation with particle by particle, this sets the maximum displacement allowed for a single particle move.  All distance displacements larger than the max is rejected.  If initialized to a negative value, it becomes equal to Lattice(LR/rc).

\item \texttt{sigmaBound}.  Determine the branch cutoff to limit wild weights based on the sigma and sigmaBound

%\item \texttt{rewind}. \textit{This input is recorded by QMCDriver.cpp, but is never used anywhere else.}

\item \texttt{storeconfigs}. If storeconfigs is set to a non-zero value, then electron configurations during the DMC run will be saved. This option is disabled for the OpenMP version of DMC.

\item \texttt{blocks\_between\_recompute}. See details in VMC section~\ref{sec:vmc}.

%\item \texttt{recordwalkers}. In VMC this is equivalent for \texttt{stepsbetweensamples}. \textit{This input is not used in DMC.}

%\item \texttt{recordconfigs}. \textit{This input is recorded by QMCDriver.cpp, but is never used anywhere else.}

%\item \texttt{current}. \textit{Only used in QMCLinearOptimize.cpp and QMCOptimize.cpp
%}
%\item \texttt{dmcwalkersperthread}. \textit{This input is only used in VMC.} It is equivalent to \texttt{samplesperthread}.

%\item \texttt{usedrift}. The VMC is implemented in two algorithms with or without drift. In the no-drift algorithm, the move of each electron is proposed with a Gaussian distribution. The standard deviation is chosen as the timestep input. In the drift algorithm, electrons are moved by langevin dynamics.




%\item \texttt{stepsbetweensamples}. Due to the fact that samples generated by consecutive steps might be still correlated. Having stepsbetweensamples larger than 1 reduces that correlation. In practice, using larger substeps is cheaper than using stepsbetweensamples to decorrelate samples.

%\item \texttt{samples}. This is the total amount of samples generated in the current VMC session. This parameter is not important for VMC only calculation but necessary if optimization or DMC follows.
%\[
%\textrm{samples}=
%\frac{\textrm{blocks}\cdot\textrm{steps}\cdot\textrm{walkers}}{\textrm{stepsbetweensamples}}\cdot\textrm{number of MPI tasks}
%\]

%\item \texttt{samplesperthread}. This is an alternative way to set the target amount of samples. More useful in the VMC session preparing the population for the following DMC calculation.
%\[
%\textrm{samplesperthread}=
%\frac{\textrm{blocks}\cdot\textrm{steps}}{\textrm{stepsbetweensamples}}
%\]

\end{itemize}

\begin{lstlisting}[caption=The following is an example of a very simple DMC section. ]
  <qmc method="dmc" move="pbyp" target="e">
    <parameter name="blocks">100</parameter>
    <parameter name="steps">400</parameter>
    <parameter name="timestep">0.010</parameter>
    <parameter name="warmupsteps">100</parameter>
  </qmc>
\end{lstlisting}
The time step should be adjusted for each problem individually.  Please refer to the theory section
on diffusion Monte Carlo.


\begin{lstlisting}[caption=The following is an example of running a simulation that can be restarted . ]
  <qmc method="dmc" move="pbyp"  checkpoint="0">
    <parameter name="timestep">         0.004  </parameter>
    <parameter name="blocks">           100   </parameter>
    <parameter name="steps">            400    </parameter>
  </qmc>
\end{lstlisting}
The checkpoint flag instructs qmcpack to output walker configurations.  This also
works in Variational Monte Carlo.  This will output an h5 file with the name "projectid"."run-number".config.h5.
Check that this file exists before attempting a restart.
To read in this file for a continuation run, specify the following:
\begin{lstlisting}[caption=Restart (read walkers from previous run) ]
 <mcwalkerset fileroot="BH.s002" version="0 6" collected="yes"/>
\end{lstlisting}
BH is the project id and s002 is the calculation number to read in the walkers from the previous run.\\

Combining VMC and DMC in a single run (and wave function optimization can be combined in this way too) is the standard way in which QMCPACK is typical run.   There is no need to run two separate jobs, as method sections can be stacked, and walkers are transferred between them.

\begin{lstlisting}[caption=Combined VMC and DMC run ]
  <qmc method="vmc" move="pbyp" target="e">
    <parameter name="blocks">100</parameter>
    <parameter name="steps">4000</parameter>
    <parameter name="warmupsteps">100</parameter>
    <parameter name="samples">1920</parameter>
    <parameter name="walkers">1</parameter>
    <parameter name="timestep">0.5</parameter>
  </qmc>
  <qmc method="dmc" move="pbyp" target="e">
    <parameter name="blocks">100</parameter>
    <parameter name="steps">400</parameter>
    <parameter name="timestep">0.010</parameter>
    <parameter name="warmupsteps">100</parameter>
  </qmc>
  <qmc method="dmc" move="pbyp" target="e">
    <parameter name="warmupsteps">500</parameter>
    <parameter name="blocks">50</parameter>
    <parameter name="steps">100</parameter>
    <parameter name="timestep">0.005</parameter>
  </qmc>
\end{lstlisting}





\input{reptation}

\chapter{Output overview}
\label{chap:output_overview}

%% Detail contents of output files.
QMCPACK writes several output files which report information about the simulation (e.g. the physical properties such as the energy), as well as information about the computational aspects of the simulation, checkpoints, and restarts.
The types of output files generated depend on the details of a calculation. The list below is not meant to be exhaustive, but rather to highlight some salient features of the more common filetypes. Further detail can be found in the description of the estimator one is interested in computing.


\section{The .scalar.dat file}
\label{sec:scalardat_file}
The most important output file is the \texttt{.scalar.dat} file. This file contains the output of block averaged properties of the system such as the local energy and other estimators.
Each line corresponds to an average over $N_{walkers}*N_{steps}$ samples.
By default, the quantities reported in the \texttt{.scalar.dat} file include:

\begin{description}
\item[LocalEnergy] The local energy.
\item[LocalEnergy\_sq] The local energy squared.
\item[LocalPotential] The local potential energy.
\item[Kinetic] The kinetic energy.
\item[ElecElec] The electron-electron potential energy.
\item[IonIon] The ion-ion potential energy.
\item[LocalECP] The energy due to the pseudopotential/effective core potential.
\item[NonLocalECP] The non-local energy due to the pseudopotential/effective core potential.
\item[MPC] The modified periodic coulomb potential energy.
\item[BlockWeight] The number of MC samples in the block.
\item[BlockCPU] The number of seconds to compute the block.
\item[AcceptRatio] The acceptance ratio.
\end{description}

QMCPACK includes a python utility, \texttt{qmca}, which can be used to process these files. Details and examples are given in chapter~\ref{chap:analyzing}.
\section{The .opt.xml file}
\label{sec:optxml_file}
This file is generated after a VMC wave function optimization, and contains the part of the input file which lists the optimized optimized jastrow factors.
Conveniently, this file is already formatted such it can easily be incorporated into a DMC input file.

\section{The .qmc.xml file}
\label{sec:qmc_file}
This file contains information about the computational aspects of the simulation, for example, which parts of the code are being executed when. This file is only generated in an ensemble run in which qmcpack runs multiple input files.

\section{The .dmc.dat file}
\label{sec:dmc_file}
This file contains information similar to the \texttt{.scalar.dat} file, but also includes extra information about the details of a DMC calculation. For example, information about the walker population.

\begin{description}
\item[Index] The block number.
\item[LocalEnergy] The local energy.
\item[Variance] The variance.
\item[Weight] The number of samples in the block.
\item[NumOfWalkers] The number of walkers times the number of steps.
\item[AvgSentWalkers] The average number of walkers sent. During a DMC simulation walkers may be created or destroyed. At every step, QMCPACK will do some load balancing to ensure that the walkers are evenly distributed across nodes.
\item[TrialEnergy] The trial energy. See \ref{sec:dmc} for an explanation of the trial energy.
\item[DiffEff] The diffusion efficiency.
\item[LivingFraction] The fraction of the walker population from the previous step that survived to the current step.
\end{description}


\section{The .bandinfo.dat file}
\label{sec:bandinfo_file}
This file contains information from the trial wavefunction about the band structure of the system,
including the available $k$-points. This can
be helpful in constructing trial wavefunctions.


\section{Checkpoint and restart files}
\label{sec:checkpoint_files}
\subsection{The .cont.xml file}
This file enables continuation of the run.  It is mostly a copy of the input XML file with the series number incremented, and the \texttt{mcwalkerset} element added to read the walkers from a config file.   The \texttt{.cont.xml} file is always created, but other files it depends on are only present if checkpointing is enabled.

\subsection{The .config.h5 file}
This file contains stored walker configurations.

\subsection{The .random.h5 file}
This file contains the state of the random number generator to allow restarts.
(Older versions used an XML file with a suffix of \texttt{.random.xml}).


\input{analysis}
\chapter{Periodic LCAO for solids}
\label{chap:LCAO}

\section{Introduction}

QMCPACK implements the linear combination of atomic orbitals (LCAO) and Gaussian
basis sets in periodic boundary conditions. This method uses orders of
magnitude less memory than the real-space spline wavefunction. Although
the spline scheme enables very fast evaluation of the wavefunction, it might
require too much on-node memory for a large complex cell. The periodic
Gaussian evaluation provides a fallback that will definitely fit in
available memory but at significantly increased computational
expense. Well-designed Gaussian basis sets should be used to accurately
represent the wavefunction, typically
including both diffuse and high angular momentum functions.

The current implementation is not highly optimized for efficiency but can handle real and complex trial wavefunctions generated by PySCF\cite{Sun2018}, but other codes such as
Crystal can be interfaced on request. Supercell tiling is handled outside QMCPACK through a proper PySCF input generated by Nexus and the Supercell geometry and coefficients of the molecular orbotals are constructed in the converter provided by QMCPACK. This is different from the plane wave/spline route where the tiling is provided in QMCPACK.   

%\subsection{Single Particle Orbitals}
%
%In QMC the many-body trial wavefunction is expressed as the product of an antisymmetric part and a correlating Jastrow factor:
% \begin{equation}
%\Psi_T(\vec{R}) = \mathcal{A}(\vec{R}) \exp\left[\sum_i J_i(\vec{R})\right]
%\end{equation}
%
%Where $\Psi_T(\vec{R})$ is the trial wave function, $\vec{R}$ is a space spin coordinates, $J(\vec{R})$ the jastrow function and $\mathcal{A}(\vec{R})$  the antisymmetric wavefunction. $\mathcal{A}(\vec{R})$  is traditionally obtained from methods such as DFT, Hartree Fock, MCSCF or CI expansion.  Many trial-wavefunctions forms have been explored, but the most popular and effective general form remains the Slater Jastrow form
% \begin{equation}
%\Psi_T(\vec{R}) = \exp\left[\sum_i J_i(\vec{R})\right]\sum_k^M C_kD_k^{\uparrow}(\varphi)D_k^{\downarrow}(\varphi)
%\end{equation}
%Where $D_k^{\downarrow}(\varphi)$ is a slater determinant expressed in terms of single particle orbitals (SPO) $\varphi_i=\sum^{N_b}_l C_l ^i \Phi_l$ . The choice of SPO representation is crucial for QMC as the cost of computing $\Phi_l$ scales linearly with the number of basis functions evaluation.  The scaling grows with the system size and the total evaluation of the N SPOs scales as  $ \mathcal{O}(N)^3$ per Monte Carlo step. In the QMCPACK parallelization scheme, SPOs are stored in read only memory replicated on each node or GPU, limiting the size of the systems to the available memory per node. 
%
%In real space QMC methods it is standard to use a real space b-spline scheme or a closely related method, due to the considerable speedup over plane-waves while retaining simple convergence properties. Use of atomic orbitals and Gaussians that include more physics or chemistry results in much more efficient basis sets, but gives up easy convergence properties. 
%
%\subsubsection{B-splines}
%  3D tricubic B-splines provide a basis in which only
%64 elements are nonzero at any given point in space.
%The one-dimensional cubic B-spline is given by,
%\begin{equation}
%f(x) = \sum_{i'=i-1}^{i+2} b^{i'\!,3}(x)\,\,  p_{i'},
%\label{eq:SplineFunc}
%\end{equation}
%where $b^{i}(x)$ are $p_i$ the piecewise cubic polynomial basis functions
%and $i = \text{floor}(\Delta^{-1} x)$ is the index of
%the first grid point $\le x$.  Constructing a tensor product in each Cartesian
%direction, we can represent a 3D orbital as
%\begin{equation}
%  \phi_n(x,y,z) =
%  \!\!\!\!\sum_{i'=i-1}^{i+2} \!\! b_x^{i'\!,3}(x)
%  \!\!\!\!\sum_{j'=j-1}^{j+2} \!\! b_y^{j'\!,3}(y)
%  \!\!\!\!\sum_{k'=k-1}^{k+2} \!\! b_z^{k'\!,3}(z) \,\, p_{i', j', k',n}.
%\label{eq:TricubicValue}
%\end{equation}
%This allows the rapid evaluation of each orbital in constant time.
%Furthermore, this basis is systematically improvable with a single spacing
%parameter, so that accuracy is not compromised and convergence checks are simple.
%
%Trial wavefunctions for materials are commonly produced using plane wave codes such as Quantum Espresso. The conversion to real space b-splines is straightforward. Compared to directly evaluating Fourier series, b-splines are approximately one order of magnitude faster, with the speedup increasing with system size.
%
%\subsubsection{Linear Combination of Atomic Orbitals (LCAO)}

LCAO schemes use physical considerations to construct a highly
efficient basis set compared with plane waves. Typically only a few tens
of basis functions per atom are required compared with thousands of
plane waves. Many forms of LCAO schemes exist and are being
implemented in QMCPACK. The details of the already-implemented methods
are described in the following section.

\noindent \textbf{GTOs:}
 The Gaussian basis functions follow a radial-angular decomposition of
\begin{equation}
     \phi (\mathbf{r} )=R_{l}(r)Y_{lm}(\theta ,\phi )\:,
\end{equation}
where $ Y_{{lm}}(\theta ,\phi )$ is a spherical harmonic, $l$ and $m$
are the angular momentum and its $z$ component, and $r, \theta, \phi$
are spherical coordinates. In practice, they are atom centered and the
$l$ expansion typically includes 1--3 additional channels compared with
the formally occupied states of the atom (e.g., 4--6 for a nickel atom with
occupied $s$, $p$, and $d$ electron shells.

The evaluation of GTOs within PBC differs slightly from evaluating
GTOs in open boundary conditions (OBCs).  The orbitals are evaluated at
a distance $r$ in the primitive cell (similar to OBC), and then the
contributions of the periodic images are added by evaluating the
orbital at a distance $r+T$, where T is a translation of the cell
lattice vector. This requires loops over the periodic images until the
contributions are orbitals $\Phi$. In the current implementation, the
number of periodic images is an input parameter named
\texttt{PBCimages}, which takes three integers corresponding to the
number of periodic images along the supercell axes (X, Y and Z axes
for a cubic cell). By default these parameters are set to
\texttt{PBCimages= 5 5 5}, but they \textbf{require manual convergence
  checks}. Convergence checks can be performed by checking the total
energy convergence with respect to \texttt{PBCimages}, similar to checks
performed for plane wave cutoff energy and B-spline grids. Use of
diffuse Gaussians might require these parameters to be increased, while
sharply localized Gaussians might permit a decrease. The cost of
evaluating the wavefunction increases sharply as \texttt{PBCimages} is
increased. This input parameter will be replaced by a tolerance
factor and numerical screening in the future.

\section{Generating and using periodic Gaussian-type wavefunctions
  using PySCF}

Similar to any QMC calculation, using periodic GTOs requires the
generation of a periodic trial wavefunction. QMCPACK is currently
interfaced to PySCF, which is a multipurpose electronic structure
written mainly in Python with key numerical functionality implemented
via optimized C and C++ libraries\cite{Sun2018}. Such a wavefunction
can be generated according to the following example for a $2 \times 1 \times 1$ supercell using tiling (kpoints) and a supertwist shifted away from $\Gamma$, leading to a complex wavefunction.  
%Note that the current implementation and examples cover only
%the use of k-points where symmetry allows real coefficients to be
%used.  This allows calculation at $\Gamma$) and, e.g., some high
%symmetry k-points at the Brillouin zone edges.  More general k-points
%requiring complex coefficients will be supported in future releases.

\begin{lstlisting}[style=Python,caption=Example PySCF input for single k-point calculation for a $2 \times 1 \times 1$ carbon supercell.]
#!/usr/bin/env python
import numpy
import h5py
from pyscf.pbc import gto, scf, dft, df
from pyscf.pbc import df

cell = gto.Cell()
cell.a             = '''
         3.37316115       3.37316115       0.00000000
         0.00000000       3.37316115       3.37316115
         3.37316115       0.00000000       3.37316115'''
cell.atom = '''  
   C        0.00000000       0.00000000       0.00000000
   C        1.686580575      1.686580575      1.686580575 
            '''
cell.basis         = 'bfd-vdz'
cell.ecp           = 'bfd'
cell.unit          = 'B'
cell.drop_exponent = 0.1
cell.verbose       = 5
cell.charge        = 0
cell.spin          = 0
cell.build()


sp_twist=[0.07761248, 0.07761248, -0.07761248]

kmesh=[2,1,1]
kpts=[[ 0.07761248,  0.07761248, -0.07761248],[ 0.54328733,  0.54328733, -0.54328733]]


mf = scf.KRHF(cell,kpts)
mf.exxdiv = 'ewald'
mf.max_cycle = 200

e_scf=mf.kernel()

ener = open('e_scf','w')
ener.write('%s\n' % (e_scf))
print('e_scf',e_scf)
ener.close()

title="C_diamond-tiled-cplx"
from PyscfToQmcpack import savetoqmcpack
savetoqmcpack(cell,mf,title=title,kmesh=kmesh,kpts=kpts,sp_twist=sp_twist)

\end{lstlisting}

Note that the last three lines of the file
\begin{lstlisting}[style=Python]
title="C_diamond-tiled-cplx"
from PyscfToQmcpack import savetoqmcpack
savetoqmcpack(cell,mf,title=title,kmesh=kmesh,kpts=kpts,sp_twist=sp_twist)
\end{lstlisting}

contain the title (name of the HDF5 to be used in QMCPACK) and the call to the converter. The title variable will be the name of the
HDF5 file where all the data needed by QMCPACK will be stored.  The
function \textit{savetoqmcpack} will be called at the end of the
calculation and will generate the HDF5 similarly to the nonperiodic
PySCF calculation in Section~\ref{sec:convert4qmc} (convert4qmc). The
function is distributed with QMCPACK and is located in the
qmcpack/src/QMCTools directory under the name
\textit{PyscfToQmcpack.py}. Note that you need to specify the supertwist coordinates that was used with the provided kpoints. The supertwist must match the coordinates of the K-points otherwise the phase factor for the atomic orbital will be incorrect and incorrect results will be obtained. (For more details on how to generate tiling with PySCF and Nexus,  refer to the Nexus guide or the 2019 QMCPACK Workshop material available on github: \url{https://github.com/QMCPACK/qmcpack_workshop_2019} under \textbf{qmcpack\_workshop\_2019/day2\_nexus/pyscf/04\_pyscf\_diamond\_hf\_qmc/}

For the converter in the script to be called properly, you need
to specify the path to the file in your PYTHONPATH such as


\begin{lstlisting}[style=SHELL]
export PYTHONPATH=QMCPACK_PATH/src/QMCTools:$PYTHONPATH
\end{lstlisting}

%When using multiple k-points, it is necessary to expand the k-points into the equivalent supercell, adjust for the phase factor in the coefficient's value due to the translation by the lattice vector, and order the molecular coefficients from each k-point according to their occupation. These operations are all automated in the \textit{savetoqmcpack()} function.\\

%The following example corresponds to the same carbon system ($2 \times 1 \times 1$); however, in this case, we use a primitive simulation cell and a $2 \times 1 \times 1$ k-point mesh.   

%\begin{lstlisting}[style=Python,caption=Example PySCF input for single k-point calculation for a $2 \times 1 \times 1$ carbon supercell.]
%#!/usr/bin/env python

%import numpy
%from pyscf.pbc import gto, scf, dft,df
%kmesh = [2, 1, 1]

%cell = gto.Cell()
%cell.a = '''
%         3.37316115       3.37316115       0.00000000
%         0.00000000       3.37316115       3.37316115
%         3.37316115       0.00000000       3.37316115'''
%cell.atom = '''  
%   C        0.00000000       0.00000000       0.00000000
%   C        1.686580575      1.686580575      1.686580575 
%            '''
%cell.basis='bfd-vtz'
%cell.ecp = 'bfd'%

%cell.unit='B'
%cell.drop_exponent=0.1
%
%cell.verbose = 5
%
%cell.build()

%kpts = cell.make_kpts(kmesh)
%kpts -= kpts[0]

%mydf = df.GDF(cell,kpts)
%mydf.auxbasis = 'weigend'
%mf = scf.KRHF(supcell,kpts).density_fit()

%mf.exxdiv = 'ewald'
%mf.with_df = mydf
%e_scf=mf.kernel()


%title="C_Diamond-211"

%from PyscfToQmcpack import savetoqmcpack
%savetoqmcpack(supcell,mf,title=title,kpts=kpts,kmesh=kmesh)

%\end{lstlisting}



%Note the difference between the 2 input files where\\
%\begin{lstlisting}
%kmesh=[2,1,1]  #k-point mesh
%\end{lstlisting}%

%\begin{lstlisting}
%kpts = cell.make_kpts(kmesh)
%kpts -= kpts[0]
%\end{lstlisting}
%Will generate k-points centered around the $\Gamma$-point and will ensure that the molecular coefficients are %real.\\

%\begin{lstlisting}[style=Python]
%mf = scf.KRHF(supcell,kpts).density_fit()
%\end{lstlisting}
%The computational algorithm chosen in PySCF is \textit{KRHF} instead of \textit{RHF}.

%Finally, to generate the HDF5 file needed by \qmcpack we call the \textit{savetoqmcpack} function.\\
%\begin{lstlisting}[style=Python]
%from PyscfToQmcpack import savetoqmcpack
%savetoqmcpack(supcell,mf,title=title,kpts=kpts,kmesh=kmesh)
%\end{lstlisting}
%In this call, we simply specify the k-point mesh used to force the converter to generate the desired cell. Note that if the parameter \textit{kmesh} is omitted, the converter will still try to ``guess'' it.



To generate QMCPACK input files, you will need to run  \textit{convert4qmc} exactly as specified in Section ~\ref{sec:convert4qmc} for both cases;
\begin{lstlisting}[style=SHELL]
convert4qmc -pyscf C_diamond-tiled-cplx
\end{lstlisting}

This tool can be used with any option described in convert4qmc. Since
the HDF5 contains all the information needed, there is no need to
specify any other specific tag for periodicity. A supercell at
$\Gamma$-point or using multiple k-points will work without further
modification.

Running convert4qmc will generate 3 input files:\\
\begin{lstlisting}[style=QMCPXML,caption=C\_diamond-tiled-cplx.structure.xml. This file contains the geometry of the system.]
<?xml version="1.0"?>
<qmcsystem>
  <simulationcell>
    <parameter name="lattice">
  6.74632230000000e+00  6.74632230000000e+00  0.00000000000000e+00
  0.00000000000000e+00  3.37316115000000e+00  3.37316115000000e+00
  3.37316115000000e+00  0.00000000000000e+00  3.37316115000000e+00
</parameter>
    <parameter name="bconds">p p p</parameter>
    <parameter name="LR_dim_cutoff">15</parameter>
  </simulationcell>
  <particleset name="ion0" size="4">
    <group name="C">
      <parameter name="charge">4</parameter>
      <parameter name="valence">4</parameter>
      <parameter name="atomicnumber">6</parameter>
    </group>
    <attrib name="position" datatype="posArray">
  0.0000000000e+00  0.0000000000e+00  0.0000000000e+00
  1.6865805750e+00  1.6865805750e+00  1.6865805750e+00
  3.3731611500e+00  3.3731611500e+00  0.0000000000e+00
  5.0597417250e+00  5.0597417250e+00  1.6865805750e+00
</attrib>
    <attrib name="ionid" datatype="stringArray">
 C C C C
</attrib>
  </particleset>
  <particleset name="e" random="yes" randomsrc="ion0">
    <group name="u" size="8">
      <parameter name="charge">-1</parameter>
    </group>
    <group name="d" size="8">
      <parameter name="charge">-1</parameter>
    </group>
  </particleset>
</qmcsystem>
  \end{lstlisting}

  As one can see, for both examples, the two-atom primitive cell has been expanded to contain four atoms in a $2 \times 1 \times 1$ carbon cell.
\begin{lstlisting}[style=QMCPXML,caption=C\_diamond-tiled-cplx.wfj.xml. This file contains the trial wavefunction.]
<?xml version="1.0"?>
<qmcsystem>
  <wavefunction name="psi0" target="e">
    <determinantset type="MolecularOrbital" name="LCAOBSet" source="ion0" transform="yes" twist="0.07761248  0.07761248  -0.07761248" href="C_diamond-tiled-cplx.h5" PBCimages="8  8  8">
      <slaterdeterminant>
        <determinant id="updet" size="8">
          <occupation mode="ground"/>
          <coefficient size="52" spindataset="0"/>
        </determinant>
        <determinant id="downdet" size="8">
          <occupation mode="ground"/>
          <coefficient size="52" spindataset="0"/>
        </determinant>

      </slaterdeterminant>
    </determinantset>
    <jastrow name="J2" type="Two-Body" function="Bspline" print="yes">
      <correlation size="10" speciesA="u" speciesB="u">
        <coefficients id="uu" type="Array"> 0 0 0 0 0 0 0 0 0 0</coefficients>
      </correlation>
      <correlation size="10" speciesA="u" speciesB="d">
        <coefficients id="ud" type="Array"> 0 0 0 0 0 0 0 0 0 0</coefficients>
      </correlation>
    </jastrow>
    <jastrow name="J1" type="One-Body" function="Bspline" source="ion0" print="yes">
      <correlation size="10" cusp="0" elementType="C">
        <coefficients id="eC" type="Array"> 0 0 0 0 0 0 0 0 0 0</coefficients>
      </correlation>
    </jastrow>
  </wavefunction>
</qmcsystem>
 \end{lstlisting}
This file contains information related to the trial wavefunction. It is identical to the input file from an OBC calculation to the exception of the following tags:\\
\begin{table}[h]
\begin{center}
\begin{tabularx}{\textwidth}{l l l l l }
\hline
\multicolumn{5}{l}{*.wfj.xml specific tags} \\
\hline
%\multicolumn{2}{l}{Outputfiles}  & \multicolumn{3}{l}{}\\
   &   \bfseries tag     & \bfseries tag type & \bfseries default   & \bfseries description \\
   &   \texttt{twist             } &  3 doubles  &  ( 0 0 0)& Coordinate of the twist to compute\\
   &   \texttt{href             } &  string  & default& Name of the HDF5 file generated by\\ 
   &                              &          &        &  PySCF and used for convert4qmc\\  
   &   \texttt{PBCimages            } &  3 Integer   & 8 8 8  & Number of periodic images to evaluate the orbitals\\
    \hline
    \end{tabularx}
\end{center}
\end{table}

Other files containing QMC methods (such as optimization, VMC, and DMC blocks) will be generated and will behave in a similar fashion regardless of the type of SPO in the trial wavefunction. 




 


\input{selected_ci}

\input{afqmc}
\input{integrals_for_afqmc}



\input{examples}

% labs: import each as a separate chapter for now
\input{lab_qmc_statistics}
\input{lab_qmc_basics}
\input{lab_advanced_molecules}
\input{lab_condensed_matter}
\input{lab_excited}

\input{additional_tools}
\input{external_tools}
\input{contributing}
\input{unit_testing}

\chapter{QMCPACK Design and Feature Documentation}
\label{chap:design_features}

This section contains information on the overall design of QMCPACK.  Also included in this section are detailed explanations/derivations of major features and algorithms present in the code.


\section{QMCPACK Design}
TBD.



\newpage
\section{Feature: Optimized Long-Ranged Breakup (Ewald)}

% Written by Ken Esler as part of the Common codebase used in wfconvert
% Originally titled ``Ewald Breakup for Long-Range Potentials in PIMC''
% PIMC-specific portions have been commented out

Consider a group of particles interacting with long-ranged central
potentials, $v^{\alpha \beta}(|r^{\alpha}_i - r^{\beta}_j|)$, where the Greek superscripts
represent the particle species (eg. $\alpha=\text{electron}$,
$\beta=\text{proton}$), and Roman subscripts refer to particle number
within a species.  We can then write the total interaction energy for
the system as,
\newcommand{\vr}{\mathbf{r}}
\newcommand{\vR}{\mathbf{R}}
\newcommand{\vk}{\mathbf{k}}
\newcommand{\vq}{\mathbf{q}}
\begin{equation}
V = \sum_\alpha \left\{\sum_{i<j} v^{\alpha\alpha}(|\vr^\alpha_i - \vr^\alpha_j|) +
\sum_{\beta<\alpha} 
\sum_{i,j} v^{\alpha \beta}(|\vr^{\alpha}_i - \vr^{\beta}_j|) \right\}
\label{eq:Vperiodic}
\end{equation}
\newcommand{\va}{\mathbf{a}}
\newcommand{\vb}{\mathbf{b}}
\newcommand{\vL}{\mathbf{L}}

\subsection{The Long-Range Problem}
Consider such a system in periodic boundary conditions in a cell
defined by primitive lattice vectors $\va_1$, $\va_2$, and $\va_3$.
Let $\vL \equiv n_1 \va_1 + n_2 \va_2 + n_3\va_3$ be a direct lattice
vector.  Then the interaction energy per cell for the periodic system
is given by
\begin{equation}
\begin{split}
V = & \sum_\vL \sum_\alpha \left\{ 
\overbrace{\sum_{i<j} v^{\alpha\alpha}(|\vr^\alpha_i - \vr^\alpha_j + \vL|)}^{\text{homologous}} +
\overbrace{\sum_{\beta<\alpha} 
\sum_{i,j} v^{\alpha \beta}(|\vr^{\alpha}_i - \vr^{\beta}_j+\vL|)}^{\text{heterologous}}
\right\}  \\
& + \underbrace{\sum_{\vL \neq \mathbf{0}} \sum_\alpha N^\alpha v^{\alpha \alpha} (|\vL|)}_\text{Madelung}
\end{split}
\label{eq:direct},
\end{equation}
where $N^\alpha$ is the number particles of species $\alpha$.
If the potentials $v^{\alpha\beta}(r)$ are indeed long-range, the
summation over direct lattice vectors will not converge in this naive
form.  A solution to the problem was posited by Ewald.  We break the
central potentials into two pieces -- a short range and a long range
part define by
\begin{equation}
v^{\alpha \beta}(r) = v_s^{\alpha\beta}(r) + v_l^{\alpha \beta}(r).
\end{equation}
We will perform the summation over images for the short-range part in
real space, while performing the sum for the long-range part in
reciprocal space.  For simplicity, we choose $v^{\alpha \beta}_s(r)$
so that it is identically zero at the half the box length.  This
eliminates the need to sum over images in real space.


\subsection{Reciprocal-Space Sums}
\subsubsection{Heterologous terms}
We begin with (\ref{eq:direct}), starting with the heterologous terms,
i.e. the terms involving particles of different species.  The
short-range terms are trivial, so we neglect them here.
\begin{equation}
\text{heterologous} = \frac{1}{2} \sum_{\alpha \neq \beta} \sum_{i,j} \sum_\vL
v^{\alpha\beta}_l(\vr_i^\alpha - \vr_j^\beta + \vL)
\end{equation}
We insert the resolution of unity in real space twice,
\begin{eqnarray}
\text{heterologous} & = & \frac{1}{2}\sum_{\alpha \neq \beta} \int_\text{cell} d\vr \, d\vr' \, \sum_{i,j}
\delta(\vr_i^\alpha - \vr) \delta(\vr_j^\beta-\vr') \sum_\vL
v^{\alpha\beta}_l(|\vr - \vr' + \vL|) \\
& = & \frac{1}{2\Omega^2}\sum_{\alpha \neq \beta} \int_\text{cell} d\vr \, d\vr' \, \sum_{\vk, \vk', i, j} e^{i\vk\cdot(\vr_i^\alpha
  - \vr)} e^{i\vk'\cdot(\vr_j^\beta - \vr')} \sum_\vL
v^{\alpha\beta}_l(|\vr - \vr' + \vL|) \nonumber \\
& = & \frac{1}{2\Omega^2} \sum_{\alpha \neq \beta} \int_\text{cell} d\vr \, d\vr'\,
\sum_{\vk, \vk', \vk'', i, j} e^{i\vk\cdot(\vr_i^\alpha - \vr)}
e^{i\vk'\cdot(\vr_j^\beta-\vr')} e^{i\vk''\cdot(\vr -\vr')}
v^{\alpha\beta}_{\vk''}, \nonumber.
\end{eqnarray}
Here, the $\vk$ summations are over reciprocal lattice vectors given
by $\vk = m_1 \vb_1 + m_2\vb_2 + m_3\vb_3$, where
\begin{eqnarray}
\vb_1 & = & 2\pi \frac{\va_2 \times \va_3}{\va_1 \cdot (\va_2 \times
  \va_3)} \nonumber \\
\vb_2 & = & 2\pi \frac{\va_3 \times \va_1}{\va_1 \cdot (\va_2 \times
  \va_3)} \\
\vb_3 & = & 2\pi \frac{\va_1 \times \va_2}{\va_1 \cdot (\va_2 \times
  \va_3)} \nonumber.
\end{eqnarray}
We note that $\vk \cdot \vL = 2\pi(n_1 m_1 + n_2 m_2 + n_3 m_3)$. 

\begin{eqnarray}
v_{k''}^{\alpha \beta} & = & 
\frac{1}{\Omega} \int_{\text{cell}} d\vr'' \sum_\vL
e^{-i\vk''\cdot(|\vr''+\vL|)} v^{\alpha\beta}(|\vr''+\vL|), \\
& = & \frac{1}{\Omega} \int_\text{all space} d\tilde{\vr} \, 
    e^{-i\vk'' \cdot \tilde{\vr}} v^{\alpha\beta}(\tilde{r}), \label{eq:vk}
\end{eqnarray}
where $\Omega$ is the volume of the cell. Here we have used the fact
that summing over all cells of the integral over the cell is
equivalent to integrating over all space.
\begin{equation}
\text{hetero} = \frac{1}{2\Omega^2} \sum_{\alpha \neq \beta}
\int_\text{cell} d\vr \, d\vr' \, \sum_{\vk, \vk', \vk'', i, j}
e^{i(\vk \cdot \vr_i^\alpha + \vk' \cdot\vr_j^\beta)} e^{i(\vk''-\vk)\cdot \vr}
e^{-i(\vk'' + \vk')\cdot \vr'} v^{\alpha \beta}_{\vk''}.
\end{equation}
We have
\begin{equation}
\frac{1}{\Omega} \int d\vr \  e^{i(\vk -\vk')\cdot \vr} =
\delta_{\vk,\vk'},
\end{equation}
Then, performing the integrations we have
\begin{eqnarray}
\text{hetero} = \frac{1}{2} \sum_{\alpha \neq \beta}
\sum_{\vk, \vk', \vk'', i, j}
e^{i(\vk \cdot \vr_i^\alpha + \vk' \cdot\vr_j^\beta)} \delta_{\vk,\vk''}
\delta_{-\vk', \vk''} v^{\alpha \beta}_{\vk''}.
\end{eqnarray}
We now separate the summations, yielding
\begin{equation}
\text{hetero} = \frac{1}{2} \sum_{\alpha \neq \beta} \sum_{\vk, \vk'}
\underbrace{\left[\sum_i e^{i\vk  \cdot \vr_i^\alpha} \rule{0cm}{0.705cm}
    \right]}_{\rho_\vk^\alpha}
\underbrace{\left[\sum_j e^{i\vk' \cdot \vr_j^\beta} \right]}_{\rho_{\vk'}^\beta}
 \delta_{\vk,\vk''} \delta_{-\vk', \vk''} v^{\alpha
  \beta}_{\vk''}.
\end{equation}
Summing over $\vk$ and $\vk'$, we have
\begin{equation}
\text{hetero} = \frac{1}{2} \sum_{\alpha \neq \beta} \sum_{\vk''}
\rho_{\vk''}^\alpha \, \rho_{-\vk''}^\beta v_{k''}^{\alpha \beta}.
\end{equation}
We can simplify the calculation a bit further by rearranging the
sums over species,
\begin{eqnarray}
\text{hetero} & = & \frac{1}{2} \sum_{\alpha > \beta} \sum_{\vk}
\left(\rho^\alpha_\vk \rho^\beta_{-\vk} + \rho^\alpha_{-\vk}
\rho^\beta_\vk\right) v_{k}^{\alpha\beta} \\
& = & \sum_{\alpha > \beta} \sum_\vk \mathcal{R}e\left(\rho_\vk^\alpha
\rho_{-\vk}^\beta\right)v_k^{\alpha\beta} .
\end{eqnarray}


\subsubsection{Homologous Terms}
We now consider the terms involving particles of the same species
interacting with each other.  The algebra is very similar to that
above, with the slight difficulty of avoiding the self-interaction term.
\begin{eqnarray}
\text{homologous} & = & \sum_\alpha \sum_L \sum_{i<j} v_l^{\alpha
  \alpha}(|\vr_i^\alpha - \vr_j^\alpha + \vL|) \\
 & = & \frac{1}{2} \sum_\alpha \sum_L \sum_{i\neq j} v_l^{\alpha
  \alpha}(|\vr_i^\alpha - \vr_j^\alpha + \vL|) 
\end{eqnarray}
\begin{eqnarray}
\text{homologous} & = & \frac{1}{2} \sum_\alpha \sum_L 
\left[
-N^\alpha v_l^{\alpha \alpha}(|\vL|)  + \sum_{i,j} v^{\alpha \alpha}_l(|\vr_i^\alpha - \vr_j^\alpha + \vL|)
  \right] \\
& = & \frac{1}{2} \sum_\alpha \sum_\vk \left(|\rho_k^\alpha|^2 - N
\right) v_k^{\alpha \alpha}
\end{eqnarray}

\subsubsection{Madelung Terms}
Let us now consider the Madelung term for a single particle of species
$\alpha$.  This term corresponds to the interaction of a particle with
all of its periodic images.  
\begin{eqnarray}
v_M^{\alpha} & = & \frac{1}{2} \sum_{\vL \neq \mathbf{0}} v^{\alpha
  \alpha}(|\vL|) \\
& = & \frac{1}{2} \left[ -v_l^{\alpha \alpha}(0) + \sum_\vL v^{\alpha
  \alpha}(|\vL|) \right] \\
& = & \frac{1}{2} \left[ -v_l^{\alpha \alpha}(0) + \sum_\vk v^{\alpha
  \alpha}_\vk \right]  
\end{eqnarray}

\subsubsection{$\vk=\mathbf{0}$ terms}
Thus far, we have neglected what happens at the special point $\vk =
\mathbf{0}$.  For many long-range potentials, such as the coulomb
potential, $v_k^{\alpha \alpha}$ diverges for $k=0$.  However, we
recognize that for a charge-neutral system, the divergent part of the
terms cancel each other.  If all the potential in the system were
precisely coulomb, the $\vk=\mathbf{0}$ terms would cancel precisely,
yielding zero.  For systems involving pseudopotentials, however, it
may be the case the resulting term is finite, but nonzero.  Consider
the terms from $\vk=\mathbf{0}$,
\begin{eqnarray}
V_{k=0} & = & \sum_{\alpha>\beta} N^\alpha N^\beta v^{\alpha \beta}_{k=0}
+ \frac{1}{2} \sum_\alpha \left(N^{\alpha}\right)^2 v^{\alpha\alpha}_{k=0} \\
& = & \frac{1}{2} \sum_{\alpha,\beta} N^\alpha N^\beta v^{\alpha
  \beta}_{k=0}.
\label{eq:kzero}
\end{eqnarray}
Next, we must compute $v^{\alpha \beta}_{k=0}$.  
\begin{equation}
v^{\alpha \beta}_{k=0} = \frac{4 \pi}{\Omega} \int_0^\infty dr\ r^2
v_l^{\alpha \beta}(r)
\end{equation}
We recognize that this integral will not converge because of the
large-$r$ behavior.  However, we recognize that when we do the sum in
(\ref{eq:kzero}), the large-$r$ parts of the integrals will cancel
precisely.  Therefore, we define
\begin{equation}
\tilde{v}^{\alpha \beta}_{k=0} = \frac{4 \pi}{\Omega} 
\int_0^{r_\text{end}} dr\ r^2 v_l^{\alpha \beta}(r),
\end{equation}
where $r_\text{end}$ is some cutoff value after which the potential
tails precisely cancel.

\subsubsection{Neutralizing Background Terms}
For systems with a net charge, such as the one-component plasma
(jellium), we add a uniform background charge which makes the system
neutral.  When we do this, we must add a term which comes from the
interaction of the particle with the neutral background.  It is a
constant term, independent of the particle positions.  In general, we
have a compensating background for each species, which largely cancels
out for neutral systems.
\begin{equation}
V_\text{background} = -\frac{1}{2} \sum_\alpha \left(N^\alpha\right)^2 
v^{\alpha \alpha}_{s\mathbf{0}}
-\sum_{\alpha > \beta} N_\alpha N_\beta
v^{\alpha\beta}_{s\mathbf{0}},
\end{equation}
where $v^{\alpha \beta}_{s\mathbf{0}}$ is given by
\begin{eqnarray}
v^{\alpha \beta}_{s\mathbf{0}} & = & \frac{1}{\Omega} \int_0^{r_c} d^3 r\ 
v^{\alpha \beta}_s(r) \\
& = & \frac{4 \pi}{\Omega} \int_0^{r_c} r^2 v_s(r) \ dr \nonumber
\end{eqnarray}


\subsection{Combining Terms}
Here, we sum all of the terms we computed in the sections above,
\begin{eqnarray}
V & = & \sum_{\alpha > \beta} \left[\sum_{i,j} v_s(|\vr_i^\alpha
  -\vr_j^\beta|) + \sum_\vk \mathcal{R}e\left(\rho_\vk^\alpha
  \rho_{-\vk}^\beta\right)v^{\alpha\beta}_k  -N^\alpha N^\beta
  v^{\alpha \beta}_{s\mathbf{0}}  \right] \nonumber \\
& + & \sum_\alpha \left[ N^\alpha v_M^\alpha + \sum_{i>j} v_s(|\vr_i^\alpha -
  \vr_j^\alpha|) + \frac{1}{2} \sum_\vk \left( |\rho_\vk^\alpha|^2 -
  N\right) v^{\alpha\alpha}_\vk -\frac{1}{2}\left(N_\alpha\right)^2 v_{s\mathbf{0}}^{\alpha\alpha}\right] \nonumber \\
& = & \sum_{\alpha > \beta} \left[\sum_{i,j} v_s(|\vr_i^\alpha
  -\vr_j^\beta|) + \sum_\vk \mathcal{R}e\left(\rho_\vk^\alpha
  \rho_{-\vk}^\beta\right) v^{\alpha \beta}_k   -N^\alpha N^\beta
  v^{\alpha \beta}_{s\mathbf{0}}  +\tilde{V}_{k=0} \right] \\
& + & \sum_\alpha \left[ -\frac{N^\alpha v_l^{\alpha \alpha}(0)}{2}  + \sum_{i>j} v_s(|\vr_i^\alpha -
  \vr_j^\alpha|) + \frac{1}{2} \sum_\vk |\rho_\vk^\alpha|^2 v^{\alpha\alpha}_\vk - \frac{1}{2}\left(N_\alpha\right)^2
  v_{s\mathbf{0}}^{\alpha\alpha} +\tilde{V}_{k=0}\right]  \nonumber
\end{eqnarray}

\subsection {Computing the Reciprocal Potential}
Now we return to (\ref{eq:vk}).  Without loss of generality, we define
for convenience $\vk = k\hat{\mathbf{z}}$.
\begin{equation}
v^{\alpha \beta}_k = \frac{2\pi}{\Omega} \int_0^\infty dr \int_{-1}^1
  d\cos(\theta) \ r^2 e^{-i k r \cos(\theta)} v_l^{\alpha \beta}(r)
\end{equation}
We do the angular integral first.  By inversion symmetry, the
imaginary part of the integral vanishes, yielding
\begin{equation}
v^{\alpha \beta}_k = \frac{4\pi}{\Omega k}\int _0^\infty dr\ r \sin(kr)
v^{\alpha \beta}_l(r).
\label{eq:vkint}
\end{equation}

\subsection{The Coulomb Potential}
For the case of the Coulomb potential, the above integral is not
formally convergent if we do the integral naively. We may remedy the
situation by including a convergence factor, $e^{-k_0 r}$.  For a
potential of the form $v^\text{coul}(r) = q_1 q_2/r$, this yields
\begin{eqnarray}
v^{\text{screened coul}}_k & = & \frac{4\pi q_1 q_2}{\Omega k} \int_0^\infty dr\ \sin(kr)
e^{-k_0r} \\ 
& = & \frac{4\pi q_1 q_2}{\Omega (k^2 + k_0^2)}
\end{eqnarray}
Allowing the convergence factor to tend to zero, we have
\begin{equation}
v_k^\text{coul} = \frac{4 \pi q_1 q_2}{\Omega k^2}
\end{equation}

For more generalized potentials with a coulomb tail, we cannot
evaluate (\ref{eq:vkint}) numerically but must handle the coulomb part
analytically.  In this case, we have
\begin{equation}
v_k^{\alpha \beta} = \frac{4\pi}{\Omega} 
\left\{ \frac{q_1 q_2}{k^2} + \int_0^\infty dr \ r \sin(kr) \left[ v_l^{\alpha \beta}(r) -
  \frac{q_1 q_2}{r} \right] \right\}
\end{equation}

\subsection{Efficient calculation methods}
\subsubsection{Fast computation of $\rho_\vk$}
We wish to quickly calculate the quantity
\begin{equation}
\rho_\vk^\alpha \equiv \sum_i e^{i\vk \cdot r_i^\alpha}
\end{equation}
First, we write 
\begin{eqnarray}
\vk & = & m_1 \vb_1 + m_2 \vb_2 + m_3 \vb_3 \\
\vk \cdot \vr_i^\alpha & = &  m_1 \vb_1 \cdot \vr_i^\alpha + 
m_2 \vb_2 \cdot \vr_i^\alpha + m_3 \vb_3 \cdot \vr_i^\alpha \\
e^{i\vk \cdot r_i^\alpha} & = & 
{\underbrace{\left[e^{i \vb_1 \cdot\vr_i^\alpha}\right]}_{C^{i\alpha}_1}}^{m_1}
{\underbrace{\left[e^{i \vb_2 \cdot\vr_i^\alpha}\right]}_{C^{i\alpha}_2}}^{m_2}
{\underbrace{\left[e^{i \vb_3 \cdot\vr_i^\alpha}\right]}_{C^{i\alpha}_3}}^{m_3}
\end{eqnarray}
Now, we note that
\begin{equation}
[C^{i\alpha}_1]^{m_1} = C^{i\alpha}_1 [C^{i\alpha}]^{(m_1-1)}.
\end{equation}
This allows us to recursively build up an array of the $C^{i\alpha}$s,
and then compute $\rho_\vk$ for all $\vk$-vectors by looping over all
k-vectors, requiring only two complex multiplies per particle per
$\vk$.
\begin{algorithm}
\caption{Algorithm to quickly calculate $\rho_\vk^\alpha$.}
\begin{algorithmic}
\STATE Create list of $\vk$-vectors and corresponding $(m_1, m_2,
m_3)$ indices.
\FORALL{$\alpha \in $ species}
  \STATE Zero out $\rho_\vk^\alpha$
  \FORALL{$i \in $ particles}
    \FOR{$j \in [1\cdots3]$}
      \STATE Compute $C^{i \alpha}_j \equiv e^{i \vb_j \cdot
        \vr^{\alpha}_i}$
       \FOR{$m \in [-m_{\text{max}}\dots m_\text{max}]$}
         \STATE Compute $[C^{i \alpha}_j]^m$ and store in array
       \ENDFOR
    \ENDFOR
     \FORALL{$(m_1, m_2, m_3) \in $ index list}
       \STATE Compute $e^{i \vk \cdot r^\alpha_i} =
         [C^{i\alpha}_1]^{m_1} [C^{i\alpha}_2]^{m_2}
         [C^{i\alpha}_3]^{m_3}$ from array
    \ENDFOR
  \ENDFOR
\ENDFOR
\end{algorithmic}
\end{algorithm}

\subsection{Gaussian Charge Screening Breakup}
This original approach to the short and long-ranged breakup adds an
opposite screening charge of gaussian shape around each point charge.
It then removes the charge in the long-ranged part of the potential.
In this potential,
\begin{equation}
v_{\text{long}}(r) = \frac{q_1 q_2}{r} \text{erf}(\alpha r),
\end{equation}
where $\alpha$ is an adjustable parameter used to control how
short-ranged the potential should be.  If the box size is $L$, a
typical value for $\alpha$ might be $7/(Lq_1 q_2)$. We should note
that this form for the long-ranged potential should also work for any
general potential with a coulomb tail, e.g. pseudo-Hamiltonian
potentials.  For this form of the long-ranged potential, we have in $k$-space
\begin{equation}
v_k = \frac{4\pi q_1 q_2 \exp\left[\frac{-k^2}{4\alpha^2}\right]}{\Omega k^2}.
\end{equation}

\subsection{Optimized Breakup Method}
In this section, we undertake the task of choosing a
long-range/short-range partitioning of the potential which is optimal
in that it minimizes the error for given real and $k$-space cutoffs
$r_c$ and $k_c$.  Here, we modify slightly the method introduced
Natoli and Ceperley\cite{Natoli1995}. We choose $r_c =
\frac{1}{2}\min\{L_i\}$, so that we require the nearest image in
real space summation.  $k_c$ is then chosen so as to satisfy our
accuracy requirements.

Here we modify our notation slightly to accommodate details not
required above.  We restrict our discussion to the interaction of two
particles species (which may be the same), and drop our species
indices.  Thus we are looking for short and long-range potentials
defined by,
\renewcommand{\vs}{v^s}
\newcommand{\vl}{v^\ell}
\begin{equation}
v(r) = \vs(r) + \vl(r)
\end{equation}
Define $\vs_k$ and $\vl_k$ to be the respective Fourier transforms of
the above.  The goal is to choose $v_s(r)$ such that its value and
first two derivatives vanish at $r_c$, while making $\vl(r)$ as smooth as
possible so that $k$-space components, $\vl_k$, are very small for
$k>k_c$.  Here, we describe how to do this is an optimal way.

Define the periodic potential, $V_p$, as 
\begin{equation}
V_p(\vr) = \sum_l v(|\vr + \mathbf{l}|),
\end{equation}
where $\vr$ is the displacement between the two particles and
$\mathbf{l}$ is a lattice vector.  Let us then define our
approximation to this potential, $V_a$, as
\begin{equation}
V_a(\vr) = \vs(r) + \sum_{|\vk| < k_c} \vl_k e^{i\mathbf \vk \cdot \vr}
\end{equation}
Now, we seek to minimize the RMS error over the cell,
\begin{equation}
\chi^2 = \frac{1}{\Omega}\int_\Omega d^3 \mathbf{r} \ 
\left| V_p(\vr) - V_a(\vr)\right|^2 
\end{equation}
We may write
\begin{equation}
V_p(\vr) = \sum_{\vk} v_k e^{i \vk \cdot \vr},
\end{equation}
where 
\begin{equation}
v_k = \frac{1}{\Omega} \int d^3\vr \ e^{-i\vk\cdot\vr}v(r).
\end{equation}
We now need a basis in which to represent the broken up potential.  We
may choose to represent either $\vs(r)$ or $\vl(r)$ in a real-space
basis.  Natoli and Ceperley chose the prior in their paper.  We choose
the latter for a number of reasons.  First, singular potentials are
difficult to represent in a linear basis unless the singularity is
explicitly included.  This requires a separate basis for each type of
singularity.  The short-range potential may have an arbitrary number
of features for $r<r_c$ and still be a valid potential.  By
construction, however, we desire that $\vl(r)$ be smooth in real-space
so that its Fourier transform falls off quickly with increasing $k$.
We therefore expect that, in general, $\vl(r)$ should be
well-represented by fewer basis functions than $\vs(r)$.  Therefore,
we define,
\begin{equation}
\vl(r) \equiv
\begin{cases}
 \sum_{n=0}^{J-1} t_n h_n(r) & \text{for } r \le r_c \\
 v(r) & \text{for } r > r_c.
\end{cases}
\end{equation}
where the $h_n(r)$ are a set of $J$ basis functions.  We require that
the two cases agree on the value and first two derivatives at $r_c$.
We may then define
\begin{equation}
c_{nk} \equiv \frac{1}{\Omega} \int_0^{r_c} d^3 \vr \ e^{-i\vk\cdot\vr} h_n(r).
\end{equation}
Similarly, we define
\begin{equation}
x_k \equiv -\frac{1}{\Omega} \int_{r_c}^\infty d^3\vr \ e^{-i\vk\cdot\vr} v(r)
\end{equation}
Therefore,
\begin{equation}
\vl_k = -x_k + \sum_{n=0}^{J-1} t_n c_{nk} 
\end{equation}
Because $\vs(r)$ goes identically to zero at the box edge, inside the
cell we may write
\begin{equation}
\vs(\vr) = \sum_\vk \vs_k e^{i\vk \cdot \vr}
\end{equation}
We then write
\begin{equation}
\chi^2 = \frac{1}{\Omega} \int_\Omega d^3 \vr \ 
\left| \sum_\vk e^{i\vk \cdot \vr} \left(v_k - \vs_k \right)
-\sum_{|\vk| \le k_c} \vl_k \right|^2
\end{equation}
We see that if we define
\begin{equation}
\vs(r) \equiv v(r) - \vl(r)
\end{equation}
Then
\begin{equation}
\vl_k + \vs_k = v_k,
\end{equation}
which then cancels out all terms for $|\vk| < k_c$.  Then we have
\begin{eqnarray}
\chi^2 & = & \frac{1}{\Omega} \int_\Omega d^3 \vr \ 
\left|\sum_{|\vk|>k_c} e^{i\vk\cdot\vr} 
\left(v_k -\vs_k \right)\right|^2 \\
& = & \frac{1}{\Omega} \int_\Omega d^3 \vr \ 
\left|\sum_{|\vk|>k_c} e^{i\vk\cdot\vr} \vl_k \right|^2 \\ 
& = & 
\frac{1}{\Omega} \int_\Omega d^3 \vr
\left|\sum_{|\vk|>k_c} e^{i\vk\cdot\vr}\left( -x_k + \sum_{n=0}^{J-1} t_n
c_{nk}\right) \right|^2
\end{eqnarray}
We expand the summation,
\newcommand{\ns}{\negthickspace}
\begin{equation}
\chi^2 = \frac{1}{\Omega} \int_\Omega d^3 \vr \ns \ns \ns
\sum_{\{|\vk|,|\vk'|\}>k_c} \ns\ns\ns\ns\ns
 e^{i(\vk-\vk')\cdot \vr}
\left(x_k -\sum_{n=0}^{J-1} t_n c_{nk} \right)
\left(x_k -\sum_{m=0}^{J-1} t_{m} c_{mk'} \right)
\end{equation}
We take the derivative w.r.t. $t_{m}$,
\begin{equation}
\frac{\partial (\chi^2)}{\partial t_{m}} =
\frac{2}{\Omega}\int_\Omega d^3 \vr \ns \ns \ns
\sum_{\{|\vk|,|\vk'|\}>k_c} \ns\ns\ns\ns\ns
 e^{i(\vk-\vk')\cdot \vr}
\left(x_k -\sum_{n=0}^{J-1} t_n c_{nk} \right) c_{mk'}
\end{equation}
We integrate w.r.t. $\vr$, yielding a Kronecker $\delta$.
\begin{equation}
\frac{\partial (\chi^2)}{\partial t_{m}} =
2 \ns\ns\ns\ns\ns\ns\ns 
\sum_{\ \ \ \ \{|\vk|,|\vk'|\}>k_c} \ns\ns\ns\ns\ns\ns\ns \delta_{\vk, \vk'} 
\left(x_k -\sum_{n=0}^{J-1} t_n c_{nk} \right) c_{mk'}
\end{equation}
Summing over $\vk'$ and equating the derivative to zero, we find the
minimum of our error function is given by
\begin{equation}
\sum_{n=0}^{J-1} \sum_{|\vk|>k_c} c_{mk}c_{nk} t_n = 
\sum_{|\vk|>k_c} x_k c_{mk},
\end{equation}
which is equivalent in form to equation (19) in \cite{Natoli1995}, where
we have $x_k$, instead of $V_k$.  Thus, we see that we may optimize
the short-range or long-range potential in simply by choosing to use
$V_k$ or $x_k$ in the above equation.  We now define
\begin{eqnarray}
A_{mn} & \equiv & \sum_{|\vk|>k_c} c_{mk} c_{nk} \\
b_{m} & \equiv & \sum_{|\vk|>k_c} x_k c_{mk}
\end{eqnarray}
Thus, it becomes clear that our minimization equations can be cast in
the canonical linear form,
\newcommand{\bA}{\mathbf{A}}
\newcommand{\bU}{\mathbf{U}}
\newcommand{\bV}{\mathbf{V}}
\newcommand{\bb}{\mathbf{b}}
\newcommand{\bS}{\mathbf{S}}
\begin{equation}
\bA\mathbf{t} = \mathbf{b}.
\end{equation}

\subsubsection{Solution by SVD}
In practice, we note that the matrix $\bA$ frequently becomes singular
in practice.  For this reason, we use the singular value decomposition
to solve for $t_n$.  This factorization decomposes $A$ as
\begin{equation}
\bA = \bU \bS \bV^T,
\end{equation}
where $\bU^T\bU = \bV^T\bV = 1$ and $\bS$ is diagonal.  In this form, we have
\begin{equation}
\mathbf{t} = \sum_{i=0}^{J-1} \left( \frac{\bU_{(i)} \cdot
  \bb}{\bS_{ii}} \right) \bV_{(i)},
\end{equation}
where the parenthesized subscripts refer to columns.  The advantage of
this form is that if $\bS_{ii}$ is zero or very near zero, the
contribution of the $i^{\text{th}}$ of $\bV$, may be neglected, since
it represents a numerical instability and has little physical
meaning.  It represents the fact that the system cannot distinguish
between two linear combinations of the basis functions.  Using the SVD
in this manner is guaranteed to be stable.  This decomposition is
available in LAPACK in the DGESVD subroutine.

\subsubsection{Constraining Values}
Often, we wish to constrain the value of $t_n$ to have a fixed value
to enforce a boundary condition, for example.  To do this, we define
\begin{equation}
\bb' \equiv \vb - t_n \bA_{(n)}.
\end{equation}
We then define $\bA^*$ as $\bA$ with the $n^\text{th}$ row and column
removed, and $\bb^*$ as $\vb'$ with the $n^\text{th}$ element removed.  Then
we solve the reduced equation $\bA^* \mathbf{t}^* = \bb^*$, and
finally insert $t_n$ back into the appropriate place in $\mathbf{t}^*$
to recover the complete, constrained vector $\mathbf{t}$.  This may be
trivially generalized to an arbitrary number of constraints.
\label{sec:contraints}

\subsubsection{The LPQHI basis}
The above discussion was general and independent of the basis used to
represent $\vl(r)$.  In this section, we introduce a convenient basis
of localized interpolant functions, similar to those used for
splines, which have a number of properties which are convenient for
our purposes.  

First, we divide the region from 0 to $r_c$ into $M-1$ subregions,
bounded above and below by points we term {\em knots}, defined by $r_j
\equiv j\Delta$, where $\Delta \equiv r_c/(M-1)$.  We then define
compact basis elements, $h_{j\alpha}$ which span the region
$[r_{j-1},r_{j+1}]$, except for $j=0$ and $j=M$.  For $j=0$, only the
region $[r_0,r_1]$, while for $j=M$, only $[r_{M-1}, r_M]$.  Thus the
index $j$ identify the knot the element is centered on, while $\alpha$
is an integer from 0 to 2 indicating one of three function shapes.
The dual index can be mapped to the single index above by the
relation, $n = 3j + \alpha$.  The basis functions are then defined as
\begin{equation}
h_{j\alpha}(r) = 
\begin{cases}
\ \ \ \, \Delta^\alpha \, \, \sum_{n=0}^5 S_{\alpha n} 
\left( \frac{r-r_j}{\Delta}\right)^n,    & r_j < r \le r_{j+1} \\
(-\Delta)^\alpha \sum_{n=0}^5 S_{\alpha n} 
\left( \frac{r_j-r}{\Delta}\right)^n,    & r_{j-1} < r \le r_j \\
\quad\quad\quad\quad\quad 0, & \text{otherwise},
\end{cases}
\end{equation}
where the matrix $S_{\alpha n}$ is given by
\begin{equation}
S = 
\left[\begin{matrix}
1 & 0 & 0 & -10 & 15 & -6 \\
0 & 1 & 0 & -6  &  8 & -3 \\
0 & 0 & \frac{1}{2} & -\frac{3}{2} & \frac{3}{2} & -\frac{1}{2}
\end{matrix}\right].
\end{equation}
\begin{figure}
\begin{center}
\epsfig{figure=./figures/LPQHI.eps,width=3.5in}
\caption{Basis functions $h_{j0}$, $h_{j1}$, and $h_{j2}$ are shown.
We note at the left and right extremes, the values and first two
derivatives of the functions are zero, while at the center, $h_{j0}$
has a value of 1, $h_{j1}$ has a first derivative of 1, and $h_{j2}$
has a second derivative of 1. \label{fig:LPQHI} }
\end{center}
\end{figure}
Figure~\ref{fig:LPQHI} shows plots of these function shapes.

The basis functions have the property that at the left and right
extremes, i.e. $r_{j-1}$ and $r_{j+1}$, their values and first two
derivatives are zero.  At the center, $r_j$, we have the properties,
\begin{eqnarray}
h_{j0}(r_j)=1, & h'_{j0}(r_j)=0, & h''_{j0}(r_j)= 0 \\
h_{j1}(r_j)=0, & h'_{j1}(r_j)=1, & h''_{j1}(r_j)= 0 \\
h_{j2}(r_j)=0, & h'_{j2}(r_j)=0, & h''_{j2}(r_j)= 1 
\end{eqnarray}
These properties allow the control of the value and first two derivatives
of the represented function at any knot value simply by setting the
coefficients of the basis functions centered around that knot.  Used
in combination with the method described in
section~\ref{sec:contraints} above, boundary conditions can easily be
enforced.  In our case, we wish require that
\begin{equation}
h_{M0} = v(r_c), \ \ h_{M1} = v'(r_c), \ \ \text{and} \ \  h_{M2} = v''(r_c).
\end{equation}
This ensures that $\vs$ and its first two derivatives vanish at $r_c$.

\subsubsection*{Fourier coefficients}
We wish now to calculate the Fourier transforms of the basis
functions, defined as
\begin{equation}
c_{j\alpha k} \equiv \frac{1}{\Omega} \int_0^{r_c} d^3 \vr 
e^{-i \vk \cdot \vr} h_{j\alpha}(r)
\end{equation}
We then may write,
\begin{equation}
c_{j\alpha k} = 
\begin{cases}
\Delta^\alpha \sum_{n=0}^5 S_{\alpha n} D^+_{0 k n}, & j = 0 \\
\Delta^\alpha \sum_{n=0}^5 S_{\alpha n} (-1)^{\alpha+n} D^-_{M k n}, &
j = M \\
\Delta^\alpha \sum_{n=0}^5 S_{\alpha n} 
\left[ D^+_{j k n} + (-1)^{\alpha+n}D^-_{j k n} \right] & \text{otherwise},
\end{cases}
\end{equation}
where
\begin{equation}
D^{\pm}_{jkn} \equiv \frac{1}{\Omega} \int_{r_j}^{r_{j\pm1}} d^3\!\vr \ 
e^{-i\vk \cdot \vr} \left( \frac{r-r_j}{\Delta}\right)^n.
\end{equation}
We then further make the definition that
\renewcommand{\Im}{\text{Im}}
\begin{equation}
D^{\pm}_{jkn} = \pm \frac{4\pi}{k \Omega} 
\left[ \Delta \Im \left(E^{\pm}_{jk(n+1)}\right) + 
r_j \Im \left(E^{\pm}_{jkn}\right)\right]
\end{equation}
It can then be shown that 
\begin{equation}
E^{\pm}_{jkn} =
\begin{cases}
-\frac{i}{k} e^{ikr_j} \left( e^{\pm i k \Delta} - 1 \right) &
\text{if } n=0, \\
-\frac{i}{k} 
\left[ \left(\pm1\right)^n e^{i k (r_j \pm \Delta)} - \frac{n}{\Delta}
E^\pm_{jk(n-1)}  \right] & \text{otherwise}.
\end{cases}
\end{equation}
Note that these equations correct typographical errors present in \cite{Natoli1995}.
\subsubsection{Enumerating $k$-points}
We note that the summations over $k$ which have been ubiquitous in
this paper requires enumeration of the $k$-vectors.  In particular, we
should sum over all $|\vk| > k_c$.  In practice, we must limit our
summation to some finite cutoff value $k_c < |\vk| < k_\text{max}$,
where $k_\text{max}$ should be of order $3000/L$, where $L$ is the
minimum box dimension.  Enumerating these vectors in a naive fashion
even for this finite cutoff would prove quite prohibitive, as it
requires $\sim 10^9$ vectors.

Our first optimization come in realizing that all quantities in this
calculation require only $|\vk|$, and not $\vk$ itself.  Thus, we may
take advantage of the great degeneracy of $|\vk|$.  We create a list
of $(k,N)$ pairs, where $N$ is the number of vectors with magnitude $k$.
We make nested loops over
$n_1$, $n_2$, and $n_3$, yielding $\vk = n_1 \vb_1 + n_2 \vb_2 + n_3
\vb_3$. If $|\vk|$ is in the required range, we check to see if there
is already an entry with that magnitude on our list, incrementing the
corresponding $N$ if there is, or creating a new entry if not.  Doing
so typically saves a factor of $\sim 200$ in storage and computation.

This reduction is still not sufficient for large $k_max$, since it
requires that we still look over $10^9$ entries.  To further reduce
cost, we may pick an intermediate cutoff, $k_\text{cont}$, above which
we will approximate the degeneracy assuming a continuum of
$k$-points.  We stop our exact enumeration at $k_\text{cont}$, and
then add $\sim 1000$ points, $k_i$, uniformly spaced between $k_\text{cont}$
and $k_\text{max}$. We then approximate the degeneracy by
\begin{equation}
N_i = \frac{4 \pi}{3} \frac{\left( k_b^3 -k_a^3\right)}{(2\pi)^3/\Omega},
\end{equation}
where $k_b = (k_i + k_{i+1})/2$ and $k_a = (k_i + k_{i-1})$.  In doing
so, we typically reduce our total number of k-points to sum over $\sim
2500$ from the $10^9$ we had to start.

\subsubsection{Calculating $x_k$'s}
\subsubsection*{The coulomb potential}
For $v(r) = \frac{1}{r}$, $x_k$ is given by
\begin{equation}
x_k^{\text{coulomb}} = -\frac{4 \pi}{\Omega k^2} \cos(k r_c)
\end{equation}

\subsection*{The $1/r^2$ potential}
For $v(r) = \frac{1}{r^2}$, $x_k$ is given by
\begin{equation}
x_k^{1/r^2} = \frac{4 \pi}{\omega k} 
\left[ \text{Si}(k r_c) -\frac{\pi}{2}\right],
\end{equation}
where the {\em sin integral}, $\text{Si}(z)$, is given by
\begin{equation}
\text{Si}(z) \equiv \int_0^z \frac{\sin \ t}{t} dt.
\end{equation}

\subsection*{The $1/r^3$ potential}
For $v(r) = \frac{1}{r^3}$, $x_k$ is given by
\begin{equation}
x_k^{1/r^3} = \frac{4\pi}{\Omega k} 
\left[k\text{Ci}(k r_c) - \frac{\sin(k r_c)}{r_c} \right],
\end{equation}
where the {\em cosine integral}, $\text{Ci}(z)$, is given by
\begin{equation}
\text{Ci}(z) \equiv -\int_z^\infty \frac{\cos t}{t} dt.
\end{equation}

\subsection*{The $1/r^4$ potential}
For $v(r) = \frac{1}{r^4}$, $x_k$ is given by
\begin{equation}
x_k^{1/r^4} = -\frac{4 \pi}{\Omega k} 
\left\{
\frac{k \cos(k r_c)}{2 r_c} + \frac{\sin(k r_c)}{2r_c^2} + \frac{k^2}{2} \left[ \text{Si}(k r_c) - \frac{\pi}{2}\right]\right\}
\end{equation}


%\section{Adapting to PIMC}
%\subsection{Pair actions}
%Let us begin by summarizing what we have done so far.  We began with the many-body Hamiltonian given by 
%\begin{equation}
%\mathcal{H} = \sum_i -\lambda_i \nabla_i^2 + V,
%\end{equation}
%where $V$ is the periodic potential given by (\ref{eq:Vperiodic}), and $\lambda \equiv \frac{\hbar^2}{2m_i}$. 
%
%We approximately solved the action of this Hamiltonian by considering
%the particles pairwise, and solving for the density matrix for the
%density matrix of each pair exactly using the matrix squaring method.
%This yields the the {\em pair action}, defined by
%\begin{equation}
%\rho^{\alpha \beta}(\vr, \vr';\tau) \equiv \rho_0(\vr, \vr';\tau)
%e^{-u^{\alpha \beta}(\vr, \vr';\tau)},
%\end{equation}
%where $\rho_0$ is the {\em free particle} density matrix for species
%$\alpha$ interacting with species $\beta$.  $\rho^{\alpha \beta}$ is
%the density matrix for the pair Hamiltonian
%\begin{equation}
%H^{\alpha\beta} = -\lambda^{\alpha\beta} \nabla^2 + v^{\alpha\beta}(|\vr|),
%\end{equation}
%where $\vr \equiv \vr_i - \vr_j$ and particles $i$ and $j$ are members
%of species $\alpha$ and $\beta$, respectively, and
%$\lambda^{\alpha\beta}$ is given by
%\begin{equation}
%\lambda^{\alpha \beta} = \frac{\hbar^2}{2m_{\alpha}} +
%\frac{\hbar^2}{2m_\beta}.
%\end{equation}
%If the potential $v^{\alpha \beta}(r)$ is long range, then the action,
%$u^{\alpha \beta}(\vr, \vr';\beta)$, will also be long range.  We
%note, however, that the action is not a simple function of the scalar
%$r$, as the potential is.  Experience shows, however, that at large
%distances, the action is well-approximated by
%\begin{eqnarray}
%u^{\alpha\beta}(\vr, \vr';\tau) & \approx & 
%\frac{1}{2} \left[ u^{\alpha\beta}(\vr,\vr;\tau) +
%  u^{\alpha\beta}(\vr',\vr';\tau)\right] \\
%& = & \frac{1}{2} \left[ u^{\alpha\beta}_\text{diag}(r,\tau)+
%u^{\alpha\beta}_\text{diag}(r',\tau)\right]
%\end{eqnarray}
%This is known as the {\em diagonal approximation}.  Thus, as long as
%this approximation is valid at half the minimum box dimension, we may
%break up the diagonal action as we did the potential.  This
%effectively neglects the off-diagonal parts of the action for
%particles more than a half-box length apart, but experience has shown
%that these contributions are usually quite small.  The same
%analysis follows for the $\tau$-derivative the the action, which is
%required to compute the total energy.  Note that PIMC simulation
%requires the pair action at several values of $\tau$, so that in
%practice, we need to do several optimized breakups for each
%$u_\text{diag}^{\alpha\beta}$ and $\dot{u}_\text{diag}^{\alpha\beta}$ and a single breakup for
%each $v^{\alpha\beta}$.
%
%\subsection{Beyond the pair approximation: RPA improvements}
%Consider the limit of a dense gas of charged particles.  We know from
%solid state theory that collective density fluctuations, known as
%plasmons, contribute significantly to the energy spectrum of such a system.
%An approximation to the density matrix determined by considering only
%pairs of particles will neglect these contributions at finite $\tau$.
%As $\tau$ approaches zero, Trotter still guarantees we will approach
%the right limit.
%
%Nonetheless, it is possible to significantly reduce the finite-$\tau$
%timestep error by utilizing a different approximation for the long
%range part of the action.  We begin by defining our effective,
%long-range potential.  As noted above, we may perform an optimized
%breakup on the diagonal action, $u_\text{diag}^{\alpha\beta}(r)$.
%\begin{equation}
%u_\text{diag}^{\alpha\beta}(r) = \hat{u}^{\alpha\beta}_\text{diag}(r) +
%\bar{u}^{\alpha\beta}_\text{diag}(r),
%\end{equation}
%where the $\hat{u}$ and $\bar{u}$ refer to the short and long range
%diagonal actions, respectively, borrowing the notation for short and
%long vowels.
%We subtract the long range part form the total pair action in a
%quasi-primitive approximation by defining
%\begin{equation}
%\bar{u}^{\alpha\beta}_\text{diag}(r) \equiv \tau \bar{v}^{\alpha \beta}(r).
%\end{equation}
%Let $\bar{v}^{\alpha \beta}_k$ represent the Fourier transform the the
%effective potential, $\bar{v}^{\alpha\beta}(r)$.  Finally, let its
%short-range counterpart be defined by 
%\begin{equation}
%\hat{v}^{\alpha \beta}_k \equiv v^{\alpha\beta}_k - \bar{v}^{\alpha\beta}_k
%\end{equation}
%
%Now, we wish to reintroduce a new long range action, which we will
%calculate in $k$-space within the {\em Random Phase Approximation
%  (RPA)}.   We begin with the Bloch equation,
%\begin{equation}
%\dot{\rho} = -\mathcal{H} \rho,
%\end{equation}
%where the dot refers to differentiation w.r.t. $\tau$.  The
%Hamiltonian is given by
%\begin{equation}
%\mathcal{H} = \left[\sum_\alpha \sum_{i\in \alpha} -\lambda_\alpha
%\nabla_i^2\right] + \hat{V} + \bar{V},
%\end{equation}
%where $\hat{V}$ and $\bar{V}$ are the total short and long range
%periodic potentials, respectively.
%Let us now make the partitioning that
%\begin{equation}
%\rho(\vR, \vR';\tau) = \rho_0(\vR, \vR';\tau) e^{-\hat{U}(\vR,
%  \vR';\tau)} e^{-\bar{U}(\vR, \vR';\tau)},
%\end{equation}
%We assume that $\rho_s \equiv \rho_0 e^{-\hat{U}}$ satisfies the Bloch
%equation for the short-range Hamiltonian,
%\begin{equation}
%\mathcal{H}_s = \left[\sum_\alpha \sum_{i\in \alpha} -\lambda_\alpha
%\nabla_i^2\right] + \hat{V}.
%\end{equation}
%In fact, this is only strictly true in the limit that $\tau=0$, but
%this relation will suffice for our present analysis.
%
%
%Recall that $\nabla^2(ab) = a\nabla^2 b + b\nabla^2a +2(\nabla a)
%\cdot (\nabla b)$.  Thus, we have for our Bloch equation,
%\begin{eqnarray}
%-\left [\dot{\rho_s} -\rho_s\dot{\bar{U}}\right] e^{-\bar{U}} & = &
%\sum_{\alpha,\  i\in\alpha} -\lambda_\alpha
%\left[\rho_s \nabla^2_i e^{-\bar{U}} + e^{-\bar{U}} \nabla^2_i \rho_s
%  + 2(\nabla_i \rho_s)\cdot (\nabla_i e^{-\bar{U}}) 
%\right] \nonumber \\ & & + (\hat{V} + \bar{V}) \rho_s e^{-\bar{U}}.
%\end{eqnarray} 
%Subtracting the Bloch equation for the short range part,
%we are left with
%\begin{equation}
%\left[\dot{\bar{U}}-\bar{V}\right] \rho_s e^{-\bar{U}}  = 
%\sum_{\alpha,\  i\in\alpha} -\lambda_\alpha
%\left[\rho_s \nabla^2_i e^{-\bar{U}}
%  + 2(\nabla_i \rho_s)\cdot (\nabla_i e^{-\bar{U}}) 
%\right].
%\end{equation} 
%Recall that
%\begin{eqnarray}
%\nabla e^{-\bar{U}} & = & -\nabla\bar{U}e^{-\bar{U}} \\
%\nabla \rho_0 & = & 0 \ \ \ \ \ \ \ \ \ \ \ \ \ \ \ \ \ \ \ \ \ \ \ \
%\ \ \ \ \ \ \ \ \ 
%\text{ (for $\vR = \vR'$)} \\
%\nabla \rho_s & = & -\rho_s \nabla \hat{U} \\
%\nabla^2 e^{-\bar{U}} & = & 
%\left[(\nabla \bar{U})^2 - \nabla^2 \bar{U}\right] e^{-\bar{U}} 
%\end{eqnarray} 
%We now attempt to solve the Bloch equation under the restriction that
%$\vR = \vR'$, i.e. along the diagonal of the density matrix.  Hence
%let us define
%\begin{eqnarray} 
%\bar{U}(\vR, \vR';\tau) & \equiv &
%  \frac{1}{2}\left[\bar{\mathcal{U}}(\vR;\tau) +
%  \bar{\mathcal{U}}(\vR';\tau) \right] \\ 
% \hat{U}(\vR,\vR';\tau) & \equiv & 
%  \frac{1}{2} \left[\hat{\mathcal{U}}(\vR;\tau) +
%  \hat{\mathcal{U}}(\vR';\tau) \right] 
%\end{eqnarray}
%as the long and short range diagonal actions written as
%functions of only one spatial argument.  Then we have, along the
%diagonal,
%\begin{eqnarray}
%\nabla U   & = & \frac{1}{2} \nabla\mathcal{U} \\
%\nabla^2 U & = & \frac{1}{2} \nabla^2\mathcal{U}.
%\end{eqnarray}
%Substituting back into out Bloch equation,
%\begin{equation}
%\dot{\bar{\mathcal{U}}} = \sum_{\alpha, \ i\in \alpha} -\lambda_\alpha
%\left\{ \frac{1}{4} (\nabla_i \bar{\mathcal{U}})^2 - 
%\frac{1}{2}\nabla_i^2 \bar{\mathcal{U}} + \frac{1}{2} (\nabla_i\hat{\mathcal{U}})
%  \cdot (\nabla_i \bar{\mathcal{U}}) \right\} +\bar{V}
%\end{equation}
%
%We recall that the long range potential, $\bar{V}$, may be written as
%\begin{equation}
%\bar{V} = \sum_\vk \sum_{\alpha} \left[ 
%\frac{1}{2} \left| \rho^\alpha_\vk\right|^2 \bar{v}^{\alpha
%  \alpha}_k + 
%\sum_{\beta < \alpha} \mathcal{R}e \left( \rho^{\alpha}_\vk
%  \rho^\beta_{-\vk} \bar{v}^{\alpha\beta}_k \right)
%\right] 
%\end{equation}
%When we wrote this expression above, we did so to optimize the speed
%of computation.  For the following analysis, we will find it more
%convenient to write
%\begin{equation}
%\bar{V} = \frac{1}{2} \sum_\vk \sum_{\alpha, \beta} \rho_\vk^\alpha
%\rho_{-\vk}^\beta v^{\alpha \beta}_k.
%\end{equation}
%The sum is guaranteed to be real since for every $\vk$, we have a
%corresponding $-\vk$ in the sum.  Hence we need not be concerned by
%taking the real part. We may similarly write $\bar{\mathcal{U}}$ and $\hat{\mathcal{U}}$ in
%terms of $\bar{u}_k^{\alpha\beta}$ and $\hat{u}_k^{\alpha\beta}$.
%
%
%We now proceed to calculate gradients and laplacians.
%Recall that 
%\begin{equation}
%\rho_\vk^\alpha = \sum_{i\in\alpha} e^{i\vk \cdot \vr_i}
%\end{equation}
%\begin{eqnarray}
%\nabla_i \mathcal{U} & = & \frac{1}{2}\sum_\vk \left[ i\vk e^{i\vk \cdot \vr_i} \sum_\alpha
%\rho_{-\vk}^\alpha u^{\alpha \beta}_k + \text{c.c.} \right] \\
%& = & \frac{1}{2} \sum_\vk 2\mathcal{R}e \left[i\vk e^{i\vk \cdot \vr_i} \sum_\alpha
%\rho_{-\vk}^\alpha u^{\alpha \beta}_k\right] \\ 
%& = & \mathcal{R}e \left[ \sum_\vk i\vk e^{i\vk \cdot \vr_i} \sum_\alpha
%\rho_{-\vk}^\alpha u^{\alpha \beta}_k \right] \\
%& = & \sum_\vk i\vk e^{i\vk \cdot \vr_i} \sum_\alpha
%\rho_{-\vk}^\alpha u^{\alpha \beta}_k.
%\end{eqnarray}
%In the last line, we have again recognized that for every $\vk$
%there is a corresponding $-\vk$, so that the sum is purely real.
%
%Next, we compute the Laplacian w.r.t. the $i^{\text{th}}$ particle.
%\begin{eqnarray}
%\nabla^2_i \mathcal{U} & = & \nabla_i \cdot \nabla_i \mathcal{U} \\
%& = & \nabla_i \cdot \sum_\vk i\vk e^{i\vk \cdot \vr_i} \sum_\alpha
%\rho_{-\vk}^\alpha u^{\alpha \sigma_i}_k \\
%& = & \sum_\vk i\vk \cdot \nabla_i \left[ e^{i\vk \cdot \vr_i}
%  \sum_\alpha \rho_{-\vk}^\alpha u^{\alpha\sigma_i}_k \right] \\
%& = & \sum_{\vk} k^2 \left[ u^{\sigma_i \sigma_i}_k - e^{i\vk\cdot\vr_i}\sum_\alpha \rho_{-\vk}
%u_k^{\alpha \sigma_i}\right],
%\end{eqnarray}
%where $\sigma_i$ is the species of the $i^{\text{th}}$ particle.  Now,
%let us sum over all particles,
%\begin{eqnarray}
%\sum_i \lambda_i \nabla^2_i \mathcal{U} & = & \sum_\vk k^2 \left[\sum_\beta N_\beta u_k^{\beta
%  \beta} - \rho_{\vk}^\beta \sum_\alpha \rho_{-\vk} u_k^{\alpha \beta}
%  \right] \\
%& = & \sum_{\vk} k^2 \sum_{\alpha, \beta}
%  \lambda_\beta \left[N^{\alpha}\delta_{\alpha,\beta} -
%  \rho_{-\vk}^{\alpha}\rho_\vk^\beta \right]u_k^{\alpha \beta} 
%%\\
%%& = & \sum_{\vk} k^2 \sum_{\alpha, \beta}
%%  \left(\frac{\lambda_\alpha +\lambda_\beta}{2} \right) \left[N^{\alpha}\delta_{\alpha,\beta} -
%%  \rho_{-\vk}^{\alpha}\rho_\vk^\beta \right]u_k^{\alpha \beta}.
%\end{eqnarray}
%%In the last step, we have added half the sum with $\alpha$ and $\beta$
%%swapped so as to symmetrize the summation.
%Now, let us consider the cross term,
%\begin{eqnarray}
%(\nabla_i \hat{\mathcal{U}}) \cdot ( \nabla_i \bar{\mathcal{U}} ) 
%& = & \left[\sum_\vk i\vk e^{i\vk\cdot \vr_i} \sum_\alpha
%  \rho_{-\vk}^\alpha \hat{u}^{\sigma_i \alpha}_k \right] \cdot
%\left[\sum_\vq i\vq e^{i\vq\cdot \vr_i} \sum_\beta
%  \rho_{-\vk}^\beta \bar{u}^{\sigma_i \beta}_k \right] \nonumber \\
%& = & -\sum_{\vk,\vq} \vk \cdot \vq e^{i(\vk + \vq)\cdot \vr_i}
%\sum_{\alpha, \beta} \rho_{-\vk}^\alpha \rho_{-\vq}^\beta 
%\hat{u}^{\alpha \sigma_i}_k \bar{u}^{\beta \sigma_i}_k
%\end{eqnarray}
%Again, summing over all particles,
%\begin{equation}
%\sum_i (\nabla_i \hat{\mathcal{U}}) \cdot ( \nabla_i \bar{\mathcal{U}} ) =
%-\sum_{\vk, \vq} \vk \cdot \vq \sum_{\alpha, \beta, \gamma}
%\rho_{\vk + \vq}^\gamma \rho_{-\vk}^{\alpha} \rho_{-\vq}^\beta
%\hat{u}^{\alpha \gamma}_k \bar{u}^{\beta \gamma}_k
%\end{equation}
%Similarly,
%\begin{equation}
%\sum_i (\nabla_i \bar{\mathcal{U}})^2 = -\sum_{\vk, \vq} \vk \cdot \vq 
%\sum_{\alpha, \beta, \gamma} \rho^\gamma_{\vk+\vq} \rho^\alpha_{-\vk}
%\rho^\gamma_{-\vq} \bar{u}^{\alpha \gamma}_k \bar{u}^{\beta \gamma}_k
%\end{equation}
%
%The {\em Random Phase Approximation} (RPA) amounts to the assumption
%that $\rho^\gamma_{\vk + \vq} \approx N_\gamma \delta_{\vk + \vq}$. 
%Then we have,
%\begin{eqnarray}
%\sum_i (\nabla_i \hat{\mathcal{U}}) \cdot ( \nabla_i \bar{\mathcal{U}}
%) & \overset{\text{RPA}}{=} &
%\sum_\vk k^2 \sum_{\alpha, \beta, \gamma} N_\gamma \rho^\alpha_{-\vk}
%\rho^\beta_\vk \hat{u}^{\alpha \gamma}_k \bar{u}^{\beta \gamma}_k \\
%\sum_i (\nabla_i \bar{\mathcal{U}})^2 & \overset{\text{RPA}}{=} &
%\sum_\vk k^2 \sum_{\alpha, \beta, \gamma} N_\gamma
%\rho^{\alpha}_{-\vk} \rho^\beta_\vk \bar{u}_k^{\alpha \gamma}
%\bar{u}_k^{\beta \gamma}
%\end{eqnarray}
%We now return to the Bloch equation
%\begin{equation}
%\begin{split}
%\sum_\vk \sum_{\alpha,\beta} & \left\{ 
%\frac{1}{2} \rho_\vk^\alpha
%\rho_{-\vk}^\beta \left(\dot{\bar{u}}_k - \bar{v}_k^{\alpha \beta} \right)
%+\frac{1}{2} \lambda_\alpha k^2 \bar{u}_k^{\alpha \beta}
%\left(\rho_{-\vk}^\alpha 
%  \rho_\vk^\beta - N_\beta \delta_{\alpha,\beta} \right) 
%\right. \\
%& \left. -\sum_\gamma k^2 N_\gamma \lambda_\gamma \rho_{-\vk}^\alpha
%  \rho_\vk^\beta \left[ 
%\frac{1}{4} \hat{u}^{\alpha \gamma}_k \bar{u}^{\beta\gamma}_k +
%\frac{1}{2} \bar{u}^{\alpha \gamma}_k \bar{u}^{\beta \gamma}
%\right]\right\} = 0
%\end{split}
%\end{equation}
%Next, we symmetrize this equation w.r.t $\alpha$ and $\beta$.
%\begin{equation}
%\begin{split}
%\sum_{\vk, \alpha, \beta} & \left\{ \left( \rho^\alpha_{\vk} \rho^\beta_{-\vk} 
%+ \rho^\alpha_{-\vk} \rho^\beta_{\vk} \right) \left[
%\dot{\bar{u}}_k^{\alpha \beta} - \bar{v}_k^{\alpha \beta}
% +k^2 \left(\frac{\lambda_\alpha+\lambda_\beta}{2}\right) 
%\bar{u}_k^{\alpha \beta} \rule{0cm}{0.6cm} \right.\right. \\
%& \ \ \left.\left.+\sum_\gamma \frac{k^2}{2} N^\gamma \left(
%\bar{u}_k^{\alpha \gamma} \bar{u}_k^{\beta \gamma} +
%\hat{u}_k^{\alpha \gamma} \bar{u}_k^{\beta \gamma} +
%\bar{u}_k^{\alpha \gamma} \hat{u}_k^{\beta \gamma}  \right)
%\right]
%- k^2 N^\alpha \delta_{\alpha \beta}
%\right\} = 0
%\end{split}
%\end{equation}
%We require that this expression hold independent of the positions of
%the particles, i.e. independent of the values of $\rho^\alpha_{\vk}$ and
%$\rho^\beta_{\vk}$.  Thus, the equations separate for each value of
%$\vk$, $\alpha$, and $\beta$.  For $\vk \neq 0$,
%\begin{equation}
%\dot{\bar{u}}_k^{\alpha \beta} = \bar{v}_k^{\alpha \beta} 
%- k^2 \left(\frac{\lambda_\alpha+\lambda_\beta}{2}\right)
%\bar{u}_k^{\alpha\beta} -\frac{k^2}{2} \sum_\gamma N_\gamma
%\left(
%\bar{u}_k^{\alpha \gamma} \bar{u}_k^{\beta \gamma} +
%\hat{u}_k^{\alpha \gamma} \bar{u}_k^{\beta \gamma} +
%\bar{u}_k^{\alpha \gamma} \hat{u}_k^{\beta \gamma}
%\right)
%\end{equation}
%Next, we need an equation for the time propagation of
%$\hat{u}_k^{\alpha \beta}$.  Above, we assumed that $\hat{U}$ was the
%solution to the short-range problem.  Our Bloch equation for
%$\hat{mathcal{U}}$ is then given by
%\begin{equation}
%\dot{\hat{\mathcal{U}}} = \sum_i -\lambda_i
%\left\{ \frac{1}{4} (\nabla_i \hat{\mathcal{U}} 
%-\frac{1}{2} \nabla_i^2 \hat{\mathcal{U}} \right\} + \hat{V}
%\end{equation}
%Following the RPA procedure above, we arrive at the following
%equations for $\hat{u}_k^{\alpha\beta}$.
%\begin{equation}
%\dot{\hat{u}}^{\alpha \beta}_k = \hat{v}^{\alpha \beta}_k
%-k^2 \left( \frac{\lambda_\alpha + \lambda_\beta}{2} \right)
%\hat{u}^{\alpha \beta}_k - \frac{k^2}{2} \sum_\gamma N_\gamma
%\hat{u}^{\alpha \gamma}_k \hat{u}^{\beta \gamma}_k.
%\end{equation}
%Hence, for each value of $k$, we have a coupled set of differential
%equations we must solve.  We note that while the equations for
%$\bar{u}$ couple to $\hat{u}$, those for $\hat{u}$ do not couple to
%$\bar{u}$.
%%% \begin{equation}
%%% \begin{split}
%%% \sum_\vk \sum_{\alpha,\beta} 
%%%  & \left\{
%%% \rho_{-\vk}^\alpha \rho_\vk^\beta 
%%% \left[
%%% \dot{\bar{u}}^{\alpha \beta}_k + k^2 \sum_\gamma \lambda_\gamma
%%% N_\gamma 
%%% \left(\frac{\bar{u}^{\alpha \gamma}_k \bar{u}^{\beta \gamma}_k}{4} -
%%% \frac{\hat{u}^{\alpha \gamma}_k \bar{u}^{\beta \gamma}_k}{2}
%%% \right) 
%%% \frac{k^2}{2} \lambda_\beta \bar{u}_k^{\alpha \beta} +
%%% \bar{v}_k^{\alpha \beta}
%%% \right] \right.\\
%%%  & \left. \rule{0pt}{0.6cm}
%%% + \frac{k^2}{2}\lambda_\beta N_\beta \delta_{\alpha,\beta}
%%% \bar{u}_k^{\alpha \beta}
%%% \right\} = 0
%%% \end{split}
%%% \end{equation}
%
%
%%% While this
%%% is correct in the limit that the timestep, $\tau$, goes to zero, it
%%% may incur a substantial error for finite $\tau$.  In this section, we
%%% describe a method to reduce the timestep error of the long range part
%%% of the action by using the Bloch equation combined with the Random
%%% Phase Approximation (RPA).  
%
%%% The Bloch equation may be written,
%%% \begin{equation}
%%% \dot{\rho} = -\mathcal{H} \rho,
%%% \end{equation}
%%% where the dot indicates differentiation with respect to $\tau$.  Now,
%%% we define
%%% \begin{equation}
%%% \rho = \rho_0 e^{-U_s}e^{-U_l}.
%%% \end{equation}
%%% \begin{equation}
%%% \mathcal{H} = \left[ -\lambda \sum_i \nabla_i^2 \right] + V_s + V_l
%%% \end{equation}
%%% The Bloch equation gives us 




\newpage
\section{Feature: Optimized Long-Ranged Breakup (Ewald) 2}

% Written by Simone Chiesa for the FITPN code/tool (Ceperley)
% Originally titled ``Notes on fitnp''

\newcommand{\rv}{\mathbf{r}}
\newcommand{\kv}{\mathbf{k}}
\newcommand{\Rv}{\mathbf{R}}
\newcommand{\Lv}{\mathbf{L}}
\newcommand{\Rc}{\mathcal{R}}
\newcommand{\tV}{\widetilde{V}}
\newcommand{\tW}{\widetilde{W}}
\newcommand{\tc}{\widetilde{c}}
\newcommand{\tY}{\widetilde{Y}}
\newcommand{\Nk}{N_\text{knot}}
\newcommand{\wk}{w_\text{knot}}

Given a lattice of vectors $\Lv$, its associated reciprocal
lattice of vectors $\kv$ and a function $\psi(\rv)$ periodic
on the lattice we define its Fourier transform $\widetilde{\psi}(\kv)$ as
\begin{equation}
\widetilde{\psi}(\kv)=\frac{1}{\Omega}\int_\Omega d\rv \psi(\rv) e^{-i\kv\rv}
\end{equation}
where we indicated both the cell domain and the cell volume by $\Omega$. 
$\psi(\rv)$ can then be expressed as
\begin{equation}
\psi(\rv)=\sum_{\kv} \widetilde{\psi}(\kv)e^{i\kv\rv}
\end{equation}
The potential generated by charges sitting on the lattice positions
at a particular point $\rv$ inside the cell is given by
\begin{equation}
V(\rv)=\sum_{\Lv}v(|\rv+\Lv|)
\end{equation}
and its Fourier transform can be explicitly written as a function of $V$ or $v$
\begin{equation}
\widetilde{V}(\kv)=\frac{1}{\Omega}\int_\Omega d\rv V(\rv) e^{-i\kv\rv}=
\frac{1}{\Omega}\int_{\mathbb{R}^3} d\rv v(\rv) e^{-i\kv\rv}
\end{equation}
where $\mathbb{R}^3$ denotes the whole 3-dimensional space.
We now want to find the best (``best'' to be defined later) approximate 
potential of the form
\begin{equation}
V_a(\rv)=\sum_{k\le k_c} \widetilde{Y}(k) e^{i\kv\rv} + W(r)
\end{equation}
where $W(r)$ has been chosen to go to $0$ smoothly when $r=r_c$, being
$r_c$ lower or equal to the Wigner-Seitz radius of the cell. Note also
the cutoff $k_c$ on the momentum summation.

The best form of $\widetilde{Y}(k)$ and $W(r)$ is given by minimizing
\begin{equation}
  \chi^2=\frac{1}{\Omega}\int d\rv \left(V(\rv)-W(\rv)-
  \sum_{k\le k_c}\widetilde{Y}(k)e^{i\kv\rv}\right)^2
  \label{chi2r}
\end{equation}
or the reciprocal space equivalent
\begin{equation}
  \chi^2=\sum_{k\le k_c}(\tV(k)-\tW(k)-\tY(k))^2+\sum_{k>k_c}(\tV(k)-\tW(k))^2
  \label{chi2k}
\end{equation}
Eq.\ref{chi2k} follows from Eq.\ref{chi2r} and the unitarity
(norm conservation) of the Fourier transform.

This last condition is minimized by
\begin{equation}
\tY(k)=\tV(k)-\tW(k)\qquad \min_{\tW(k)}\sum_{k>k_c}(\tV(k)-\tW(k))^2
\label{mincond}
\end{equation}
We now use a set of basis function $c_i(r)$ vanishing smoothly at $r_c$
to expand $W(r)$ i.e.
\begin{equation}
W(r)=\sum_i t_i c_i(r)\qquad\text{or}\qquad \tW(k)=\sum_i t_i \tc_i(k)
\end{equation}
Inserting the reciprocal space expansion of $\tW$ in the second condition of
Eq.\ref{mincond} and minimizing with respect to $t_i$ leads immediately
to the linear system $\mathbf{A}\mathbf{t}=\mathbf{b}$ where
\begin{center}
\vskip 3mm
\begin{eqnarray}
A_{ij}=\sum_{k>k_c}\tc_i(k)\tc_j(k)\qquad b_j=\sum_{k>k_c} V(k) \tc_j(k)
\label{matrix_elements}
\end{eqnarray}

\end{center}
\vskip 3mm

\subsection{Basis functions}
The basis functions are splines. We define a uniform grid 
with $\Nk$ uniformly spaced knots at position $r_i=i\frac{r_c}{\Nk}$ 
where $i\in[0,\Nk-1]$. On each knot we center $m+1$ piecewise polynomials
$c_{i\alpha}(r)$ with $\alpha\in[0,m]$, defined as
\vskip 3mm
\begin{center}
\begin{eqnarray}
c_{i\alpha}(r)=\begin{cases}
\Delta^\alpha \sum_{n=0}^\mathcal{N} S_{\alpha n}(\frac{r-r_i}{\Delta})^n & r_i<r\le r_{i+1} \\
\Delta^{-\alpha} \sum_{n=0}^\mathcal{N} S_{\alpha n}(\frac{r_i-r}{\Delta})^n & r_{i-1}<r\le r_i \\
0 & |r-r_i| > \Delta
\end{cases}
\label{basisdef}
\end{eqnarray}

\end{center}
\vskip 3mm
These functions and their derivatives are, by construction, continuous and odd (even)
(with respect to $r-r_i\rightarrow r_i-r$) when $\alpha$ is odd (even).
We further ask them to satisfy
\begin{eqnarray}
\left.\frac{d^\beta}{dr^\beta} c_{i\alpha}(r)\right|_{r=r_i}=
\delta_{\alpha\beta} \quad \beta\in[0,m]\\
\left.\frac{d^{\beta}}{dr^{\beta}} c_{i\alpha}(r)\right|_{r=r_{i+1}}=0\quad \beta\in[0,m]
\label{constr}
\end{eqnarray}
(The parity of the functions guarantees that the second constraint is satisfied
at $r_{i-1}$ as well). These constraints have a simple interpretation: the basis functions
and their first $m$ derivatives are $0$ on the boundary of the subinterval where they
are defined; the only function to have a non zero $\beta$-th derivative in $r_i$ is $c_{i\beta}$.
These $2(m+1)$ constraints therefore impose $\mathcal{N}=2m+1$. 
Inserting the definitions of Eq.(\ref{basisdef}) in the constraints of Eq.(\ref{constr})
leads to the set of $2(m+1)$ linear equation that fixes the value of $S_{\alpha n}$: 
\begin{eqnarray}
\Delta^{\alpha-\beta} S_{\alpha\beta} \beta!=\delta_{\alpha\beta}
\label{Smatrix1}\\
\Delta^{\alpha-\beta}\sum_{n=\beta}^{2m+1} S_{\alpha n} \frac{n!}{(n-\beta)!}=0
\end{eqnarray}
One can further simplify inserting the first of these equations into the second and write
the linear system as
\begin{equation}
\sum_{n=m+1}^{2m+1} S_{\alpha n} \frac{n!}{(n-\beta)!}=\begin{cases}
-\frac{1}{(\alpha-\beta)!}& \alpha\ge \beta \\
0 & \alpha < \beta
\end{cases}
\label{Smatrix2}
\end{equation}

\subsection{Fourier components of the basis functions in 3D}
\subsubsection*{$k\ne 0$, non coulomb case.}
We now need to evaluate the Fourier transform $\tc_{i\alpha}(k)$. Let us start
by writing the definition
\begin{equation}
\tc_{i\alpha}(k)=\frac{1}{\omega}\int_\Omega d\rv  e^{-i\kv\rv} c_{i\alpha}(r)
\end{equation}
Because $c_{i\alpha}$ is different from zero only inside the spherical crown
defined by $r_{i-1}<r<r_i$ one can conveniently compute the integral in spherical
coordinates as
\vskip 3mm
\begin{center}
\begin{eqnarray}
\tc_{i\alpha}(k)=\Delta^\alpha\sum_{n=0}^\mathcal{N} S_{\alpha n} \left[
D_{in}^+(k) +\wk(-1)^{\alpha+n}D_{in}^-(k)\right]
\label{fourier_transform}
\end{eqnarray}
\end{center}
\vskip 3mm
where we used the definition $\wk=1-\delta_{i0}$ and
\begin{equation}
D_{in}^\pm(k)=\pm\frac{4\pi}{k\Omega}\Im\left[\int_{r_i}^{r_i\pm\Delta}
dr\left(\frac{r-r_i}{\Delta}\right)^n r e^{ikr}\right]
\label{D+-}
\end{equation}
obtained by integrating the angular part of the Fourier transform.
Using the identity
\begin{equation}
\left(\frac{r-r_i}{\Delta}\right)^n r=\Delta\left(\frac{r-r_i}{\Delta}\right)^{n+1}+\left(\frac{r-r_i}{\Delta}\right)^n r_i
\end{equation}
and the definition
\begin{equation}
E_{in}^\pm(k)=\int_{r_i}^{r_i\pm\Delta}
dr\left(\frac{r-r_i}{\Delta}\right)^n e^{ikr}
\end{equation}
we rewrite Eq.\ref{D+-} as
\begin{center}
\vskip 3mm
\begin{eqnarray}
D_{in}^\pm(k)=\pm\frac{4\pi}{k\Omega}\Im\left[\Delta E_{i(n+1)}^\pm(k)+
r_i E_{in}^\pm(k)\right]
\label{noncoulD+-}
\end{eqnarray}
\end{center}
\vskip 3mm

Finally, using integration by part, one can define $E^\pm_{in}$ recursively
\begin{center}
\vskip 3mm
\begin{eqnarray}
E^\pm_{in}(k)=\frac{1}{ik}\left[(\pm)^ne^{ik(r_i\pm\Delta)}-\frac{n}{\Delta}
E^\pm_{i(n-1)}(k)\right]
\label{nthEpm}
\end{eqnarray}
\end{center}
\vskip 3mm
\noindent
starting from the $n=0$ term
\vskip 3mm
\begin{center}
\begin{eqnarray}
E^\pm_{i0}(k)=\frac{1}{ik}e^{ikr_i}\left(e^{\pm ik\Delta}-1\right)
\label{0thEpm}
\end{eqnarray}
\end{center}
\vskip 3mm
\subsubsection{$k\ne 0$, coulomb case.}
To efficiently treat the coulomb divergence at the origin it is convenient to use
a basis set $c_{i\alpha}^\text{coul}$ of the form 
\begin{equation}
c_{i\alpha}^\text{coul}=\frac{c_{i\alpha}}{r}
\end{equation}
An equation identical to Eq.\ref{D+-} holds but with the modified definition
\begin{equation}
D_{in}^\pm(k)=\pm\frac{4\pi}{k\Omega}\Im\left[\int_{r_i}^{r_i\pm\Delta}
dr\left(\frac{r-r_i}{\Delta}\right)^n e^{ikr}\right]
\end{equation}
which can be simply expressed using $E^\pm_{in}(k)$ as
\vskip 3mm
\begin{center}
\begin{eqnarray}
D_{in}^\pm(k)=\pm\frac{4\pi}{k\Omega}\Im\left[E_{in}^\pm(k)\right]
\label{coulD+-}
\end{eqnarray}
\end{center}
\vskip 3mm
\subsubsection{$k=0$ coulomb and non coulomb case.}
The definitions of $D_{in}(k)$ given so far are clearly incompatible 
with the choice $k=0$ (they involve division by $k$). For the non-coulomb
case the starting definition is
\begin{equation}
D^\pm_{in}(0)=\pm\frac{4\pi}{\Omega}\int_{r_i}^{r_i\pm\Delta}r^2
\left(\frac{r-r_i}{\Delta}\right)^ndr
\end{equation}
Using the definition $I_n^\pm=(\pm)^{n+1}\Delta/(n+1)$ we can express this
as
\begin{center}
\vskip 3mm
\begin{eqnarray}
D^\pm_{in}(0)=\pm\frac{4\pi}{\Omega}\left[\Delta^2 I_{n+2}^\pm
+2r_i\Delta I_{n+1}^\pm+2r_i^2I_n^\pm\right]
\label{noncoul_k=0D+-}
\end{eqnarray}
\end{center}
\vskip 3mm
For the coulomb case one get

\vskip 3mm
\begin{center}
\begin{eqnarray}
D^\pm_{in}(0)=\pm\frac{4\pi}{\Omega}\left(
\Delta I^\pm_{n+1} + r_i I^\pm_n\right)
\label{coul_k=0D+-}
\end{eqnarray}
\end{center}
\vskip 3mm
\subsection{Fourier components of the basis functions in 2D}
Eq.\ref{fourier_transform} still holds provided we define  
\begin{equation}
D^\pm_{in}(k)=\pm\frac{2\pi}{\Omega \Delta^n} \sum_{j=0}^n \binom{n}{j}
(-r_i)^{n-j}\int_{r_i}^{r_i\pm \Delta}\negthickspace \negthickspace 
\negthickspace \negthickspace \negthickspace \negthickspace \negthickspace 
dr r^{j+1-C} J_0(kr)
\label{2DD+-}
\end{equation}
where $C=1(=0)$ for the coulomb(non coulomb) case.
Eq.\ref{2DD+-} is obtained using the integral definition of the 
zero order Bessel function of the first kind 
\begin{equation}
J_0(z)=\frac{1}{\pi}\int_0^\pi e^{iz\cos\theta}d\theta
\end{equation}
and the binomial expansion for $(r-r_i)^n$.
The integrals can be computed recursively using the following identities
\begin{center}
\begin{minipage}{0.7\textwidth}
\begin{align}
&\int dz J_0(z)=\frac{z}{2}\left[\pi J_1(z)H_0(z)+J_0(z)(2-\pi H_1(z))\right]
\label{0thmoment}\\
&\int dz z J_0(z)= z J_1(z)
\label{1stmoment}\\
&\int dz z^n J_0(z)= z^nJ_1(z)+(n-1)x^{n-1}J_0(z)
-(n-1)^2\int dz z^{n-2} J_0(z)
\label{nthmoment}
\end{align}
\end{minipage}
\end{center}
Eq.\ref{nthmoment} is obtained using Eq.\ref{1stmoment}, integration by part and 
the identity $\int J_1(z) dz =-J_0(z)$. In Eq.\ref{0thmoment} $H_0$ and $H_1$ are Struve functions.

\subsection{Construction of the matrix elements}
Using the above equations one can construct the matrix elements in Eq.\ref{matrix_elements}
and proceed solving for the $t_i$. It is sometimes desirable to put some constraints
on the value of $t_i$. For example, when the coulomb potential is concerned one may 
want to set $t_{0}=1$. If the first $g$ variable are constrained by $t_{m}=\gamma_m$ 
with $m=[1,g]$ one can simply redefine Eq.\ref{matrix_elements} as
\begin{equation}
\begin{split}
A_{ij}=&\sum_{k>k_c} \tc_i(k)\tc_j(k)  \quad i,j\notin[1,g] \\
b_j=&\sum_{k>k_c} \left(\tV(k)-\sum_{m=1}^g \gamma_m \tc_m(k)\right)\tc_j(k)\quad j\notin[1,g]
\end{split}
\label{modified_matrix_elements}
\end{equation}



% discussion below of (fortran) routines kept for now (18 Oct 2017)
% possibly these map onto routines in qmcpack also

%\subsection*{The routines}
%\subsubsection*{fitpnnew}
%This routine constructs the $t_i$ and $\tY(k)$. Previously a routine, 
%let us call it {\em shells}, generating a grid of $\kv$ points has to
%be called. {\em shells} stores $\kv$ vectors
%in order of increasing magnitude and defines a shell as the 
%set of vectors having the same magnitude $k$ (in practice their difference 
%in magnitude must be below a given threshold). The total number of
%shells $N_\text{shell}$ has to be large enough to represents $V(\rv)$
%accurately using $\tV(k)$. The number of vector
%in a given shell is called $w(k)$. The following variables are passed as
%input: $\tV(k),k,w(k),N_\text{shell},m,r_c,N_\text{knot},\Omega$ and are called
%\verb!v(0:nk),rk(0:nk),wt(0:nk),nk,m,rad,nknots!. Note that the vectors all
%start from $0$ which corresponds to $k=0$. The number of shells such that
%$k\le k_c$ is also passed as input and called \verb!nf!. Additional input variables 
%are  \verb!coul! a logical variable specifying if the potential is coulombic; 
%\verb!vmad! the exact value of the Madelung constant;
%\verb!t0,t1! logical variables specifying if a constraint has to be put
%on element $t_0$ or $t_1$ and \verb!vt0,vt1! the value at which $t_0$ and $t_1$
%have to be set if corresponding constraints are active. 
%The routine works in this way:
%\begin{itemize}
%\item it calls {\em basis} and gets the coefficients $S_{\alpha n}$ (the $n$-th
%      coefficient of the $\alpha$-th polynomials) for the desired value of $m$.
%\item for every $k$ point, knot $i$ and polynomial $\alpha$ compute $\tc_{i\alpha}(k)$
%      using Eq.\ref{fourier_transform}. $D^\pm_{in}(k)$ is provided by {\em splint3D}
%      or {\em splint2D}. The routine uses \verb!ialpha!$=i(m+1)+\alpha$ (the range of 
%      variability of $\alpha$ and $i$ is specified above Eq.\ref{basisdef}).
%\item Matrix elements are constructed according to Eq.\ref{matrix_elements}
%\item Matrix elements are modified according to Eq.\ref{modified_matrix_elements} 
%      if constraints are active
%\item $t_i$ are computed solving the linear system. $\tY(k)$ are computed.
%\item A comparison with the exact Madelung constant is performed.
%\end{itemize}
%
%\subsubsection*{splint3D}
%Called by {\em fitpnnew}. This routine compute $D^\pm_{in}(k)$ for given 
%$k$ and $i$ and for all
%$n$ (going from $0$ to $\mathcal{N}=2m+1$). $D^\pm_{in}(k)$ are called 
%\verb!ddplus(0:maxn)! and \verb!ddminus(0:maxn)! and are given as output
%by the routine. In input one is required to specify $\mathcal{N},r_i,\Delta,k,\Omega$,
%respectively named \verb!maxn,r,delta,k,vol!. A logical input flag called 
%\verb!coul! specify if the potential is coulombic or not. The routine works
%in this way:
%\begin{itemize}
%\item it checks if $k$ is equal to 0
%\item if $k\ne 0$ then
%  \begin{itemize}
%  \item it computes $E^\pm_{in}(k)$ for the specified $i$ and $k$ using Eqs.\ref{nthEpm} and
%      \ref{0thEpm}. $E^\pm_{in}(k)$ are called \verb!ee(0:maxn,!$\pm$\verb!1)! 
%      (\verb!ee(:,0)! are never used).
%  \item Depending on the value of \verb!coul! either Eq.\ref{noncoulD+-} or Eq.\ref{coulD+-} 
%      is used to construct $D^\pm_{in}(k)$. The prefactor $\frac{4\pi}{k \Omega}$ 
%      is precomputed and called \verb!dnorm!.
%  \end{itemize}
%\item if $k=0$ the code uses either Eq.\ref{noncoul_k=0D+-} or Eq.\ref{coul_k=0D+-}.
%\end{itemize}
%
%\subsubsection*{splint2D}
%Called by {\em fitpnnew}. This routine compute $D^\pm_{in}(k)$ in the 2D case.
%The \verb!i\o! format is identical to {\em splint3D}. Equations from \ref{0thmoment}
%to \ref{nthmoment} are used to generate the required integrals.
%
%
%\subsubsection*{basis}
%Called by {\em fitpnnew}. It computes the coefficients $S_{\alpha n}$ (the $n$-th 
%coefficient of the $\alpha$-th polynomials) using Eqs.\ref{Smatrix1} and \ref{Smatrix2}.
%These coefficients are stored in \verb!s(0:m,0:2m+1)!. $m$ (called \verb!m!) 
%is required in input.
%
%\subsubsection*{computespl}
%This compute $W(r)$ at any $r$. $r$ is named \verb!rpos! internally. It requires
%$m,2m+1,N_\text{knot},S_{\alpha n},r_i,t_i,\Delta$. These are internally called
%\verb!m,maxn,nknots,s(0:m,0:maxn),r(0:nknots),t(0:nknots(m+1)-1),delta!. 
%\verb!coul! is also needed: it is a logical variable 
%to specify if $c_{i\alpha}^\text{coul}(r)$ have to be used instead of $c_{i\alpha}(r)$.
%The value of $W(r)$ is stored in \verb!w!.




\newpage
\section{Feature: Cubic Spline Interpolation}
% Written by Kenneth P .Esler Jr.
% Originally titled ``Cubic Spline Interpolation in 1, 2 and 3 Dimensions''

\newenvironment{DMatrix}{\begin{array}|{*{20}{c}}|}{\end{array}}
\newenvironment{MyMatrix}{\begin{array}({*{20}{c}})}{\end{array}}
\newenvironment{LMatrix}{\begin{array}({*{20}{l}})}{\end{array}}

We present the basic equations and algorithms necessary to
construct and evaluate cubic interpolating splines in one, two, and
three dimensions.  Equations are provided for both natural and
periodic boundary conditions.

\subsection{One Dimension}
Let us consider the problem in which we have a function $y(x)$
specified at a discrete set of points $x_i$, such that $y(x_i) = y_i$.
We wish to construct a piece-wise cubic polynomial interpolating
function, $f(x)$, which satisfies the following conditions:
\begin{itemize}
\item $f(x_i) = y_i$
\item $f'(x_i^-) = f'(x_i^+)$
\item $f''(x_i^-) = f''(x_i+)$
\end{itemize}

\subsubsection{Hermite Interpolants}
In our piecewise representation, we wish to store only the values,
$y_i$, and first derivatives, $y'_i$, of our function at each point
$x_i$, which we call {\em knots}.  Given this data, we wish to
construct the piecewise cubic function to use between $x_i$ and
$x_{i+1}$ which satisfies the above conditions.  In particular, we
wish to find the unique cubic polynomial, $P(x)$ satisfying
\begin{eqnarray}
P(x_i)      & = & y_i      \label{eq:c1} \\
P(x_{i+1})  & = & y_{i+1}  \label{eq:c2} \\
P'(x_i)     & = & y'_i     \label{eq:c3} \\
P'(x_{i+1}) & = & y'_{i+1} \label{eq:c4}
\end{eqnarray}
\begin{eqnarray}
h_i & \equiv & x_{i+1} - {x_i} \\
t & \equiv & \frac{x-x_i}{h_i}.
\end{eqnarray}
We then define the basis functions,
\begin{eqnarray}
p_1(t) & = & (1+2t)(t-1)^2  \label{eq:p1}\\
q_1(t) & = & t (t-1)^2      \\
p_2(t) & = & t^2(3-2t)      \\
q_2(t) & = & t^2(t-1)      \label{eq:q2}
\end{eqnarray}
On the interval, $(x_i, x_{i+1}]$, we define the interpolating
function,
\begin{equation}
P(x) = y_i p_1(t) + y_{i+1}p_2(t) + h\left[y'_i q_1(t) + y'_{i+1} q_2(t)\right]
\end{equation}
It can be easily verified that $P(x)$ satisfies conditions (\ref{eq:c1})
through (\ref{eq:c4}).  It is now left to
determine the proper values for the $y'_i\,$s such that the continuity
conditions given above are satisfied.

By construction, the value of the function and derivative will match
at the knots, i.e.
\begin{equation}
P(x_i^-) = P(x_i^+), \ \ \ \ P'(x_i^-) = P'(x_i^+).
\end{equation}
Then we must now enforce only the second derivative continuity:
\begin{eqnarray}
P''(x_i^-) & = & P''(x_i^+) \\
\frac{1}{h_{i-1}^2}\left[\rule{0pt}{0.3cm}6 y_{i-1} -6 y_i + h_{i-1}\left(2 y'_{i-1} +4 y'_i\right) \right]& = &
\frac{1}{h_i^2}\left[\rule{0pt}{0.3cm}-6 y_i + 6 y_{i+1} +h_i\left( -4 y'_i -2 y'_{i+1} \right)\right] \nonumber
\end{eqnarray}
Let us define
\begin{eqnarray}
\lambda_i & \equiv & \frac{h_i}{2(h_i+h_{i-1})} \\
\mu_i & \equiv & \frac{h_{i-1}}{2(h_i+h_{i-1})}  = \frac{1}{2} - \lambda_i.
\end{eqnarray}
Then we may rearrange,
\begin{equation}
\lambda_i y'_{i-1} + y'_i + \mu_i y'_{i+1} = \underbrace{3 \left[\lambda_i \frac{y_i - y_{i-1}}{h_{i-1}} + \mu_i \frac{y_{i+1}
    - y_i}{h_i} \right] }_{d_i}
\end{equation}
This equation holds for all $0<i<(N-1)$, so we have a tridiagonal set of
equations.  The equations for $i=0$ and $i=N-1$ depend on the boundary
conditions we are using.  
\subsubsection{Periodic boundary conditions}
For periodic boundary conditions, we have
\begin{equation}
\begin{matrix}
y'_0           & +  & \mu_0 y'_1     &   &                   &            & \dots                   & +  \lambda_0 y'_{N-1} & = & d_0 \\
\lambda_1 y'_0 & +  & y'_1           & + &  \mu_1 y'_2       &            & \dots                   &                       & = & d_1 \\
               &    & \lambda_2 y'_1 & + &  y'_2           + & \mu_2 y'_3 & \dots                   &                       & = & d_2 \\
               &    &                &   &  \vdots           &            &                         &                       &   &     \\
\mu_{N-1} y'_0 &    &                &   &                   &            & +\lambda_{N-1} y'_{N-1} & +  y'_{N-2}           & = & d_3 
\end{matrix}
\end{equation}
Or, in matrix form, we have,
\begin{equation}
\begin{MyMatrix}
1         & \mu_0     &    0   &   0           & \dots         &      0        & \lambda_0 \\
\lambda_1 &  1        & \mu_1  &   0           & \dots         &      0        &     0     \\
0         & \lambda_2 &   1    & \mu_2         & \dots         &      0        &     0     \\
\vdots    & \vdots    & \vdots & \vdots        & \ddots        &   \vdots      &  \vdots   \\
0         &   0       &   0    & \lambda_{N-3} &      1        & \mu_{N-3}     &    0      \\
0         &   0       &   0    &   0           & \lambda_{N-2} &      1        & \mu_{N-2} \\
\mu_{N-1} &   0       &   0    &   0           &   0           & \lambda_{N-1} &  1     
\end{MyMatrix}
\begin{MyMatrix} y'_0 \\ y'_1 \\ y'_2 \\ \vdots \\ y'_{N-3} \\ y'_{N-2} \\ y'_{N-1} \end{MyMatrix} =
\begin{MyMatrix} d_0  \\  d_1 \\  d_2 \\ \vdots \\  d_{N-3} \\  d_{N-2} \\  d_{N-1} \end{MyMatrix} .
\end{equation}
The system is tridiagonal except for the two elements in the upper
right and lower left corners.  These terms complicate the solution a
bit, although it can still be done in $\mathcal{O}(N)$ time.  We first
proceed down the rows, eliminating the the first non-zero term in each
row by subtracting the appropriate multiple of the previous row.  At
the same time, we also eliminate the first element in the last row,
shifting the position of the first non-zero element to the right with
each iteration.  When we get to the final row, we will have the value
for $y'_{N-1}$.  We can then proceed back upward, backsubstituting
values from the rows below to calculate all the derivatives.

\subsubsection{Complete boundary conditions}
If we specify the first derivatives of our function at the end points,
we have what is known as {\em complete} boundary conditions.  The
equations in that case are trivial to solve:
\begin{equation}
\begin{MyMatrix}
1         &  0        &    0   &   0           & \dots         &      0        &     0     \\
\lambda_1 &  1        & \mu_1  &   0           & \dots         &      0        &     0     \\
0         & \lambda_2 &   1    & \mu_2         & \dots         &      0        &     0     \\
\vdots    & \vdots    & \vdots & \vdots        & \ddots        &   \vdots      &  \vdots   \\
0         &   0       &   0    & \lambda_{N-3} &      1        & \mu_{N-3}     &    0      \\
0         &   0       &   0    &   0           & \lambda_{N-2} &      1        & \mu_{N-2} \\
0         &   0       &   0    &   0           &   0           &      0        &  1     
\end{MyMatrix}
\begin{MyMatrix} y'_0 \\ y'_1 \\ y'_2 \\ \vdots \\ y'_{N-3} \\ y'_{N-2} \\ y'_{N-1} \end{MyMatrix} =
\begin{MyMatrix} d_0  \\  d_1 \\  d_2 \\ \vdots \\  d_{N-3} \\  d_{N-2} \\  d_{N-1} \end{MyMatrix} .
\end{equation}
This system is completely tridiagonal and we may solve trivially by
performing row eliminations downward, then proceeding upward as
before.

\subsubsection{Natural boundary conditions}
If we do not have information about the derivatives at the boundary
conditions, we may construct a {\em natural spline}, which assumes the
the second derivatives are zero at the end points of our spline.  In
this case our system of equations is the following:
\begin{equation}
\begin{MyMatrix}
1         & \frac{1}{2} &    0   &   0           & \dots         &      0        &     0     \\
\lambda_1 &  1          & \mu_1  &   0           & \dots         &      0        &     0     \\
0         & \lambda_2   &   1    & \mu_2         & \dots         &      0        &     0     \\
\vdots    & \vdots      & \vdots & \vdots        & \ddots        &   \vdots      &  \vdots   \\
0         &   0         &   0    & \lambda_{N-3} &      1        & \mu_{N-3}     &    0      \\
0         &   0         &   0    &   0           & \lambda_{N-2} &      1        & \mu_{N-2} \\
0         &   0         &   0    &   0           &   0           &  \frac{1}{2}  &  1     
\end{MyMatrix}
\begin{MyMatrix} y'_0 \\ y'_1 \\ y'_2 \\ \vdots \\ y'_{N-3} \\ y'_{N-2} \\ y'_{N-1} \end{MyMatrix} =
\begin{MyMatrix} d_0  \\  d_1 \\  d_2 \\ \vdots \\  d_{N-3} \\  d_{N-2} \\  d_{N-1} \end{MyMatrix} ,
\end{equation}
with
\begin{equation}
d_0 = \frac{3}{2} \frac{y_1-y_1}{h_0}, \ \ \ \ \ d_{N-1} = \frac{3}{2} \frac{y_{N-1}-y_{N-2}}{h_{N-1}}.
\end{equation}

\subsection{Bicubic Splines}
It is possible to extend the cubic spline interpolation method to
functions of two variables, i.e. $F(x,y)$.  In this case, we have a
rectangular mesh of points given by $F_{ij} \equiv F(x_i,y_j)$.  In
the case of 1D splines, we needed to store the value of the first
derivative of the function at each point, in addition to the value.
In the case of {\em bicubic splines}, we need to store four
quantities for each mesh point:  
\begin{eqnarray}
F_{ij}    & \equiv & F(x_i, y_i)            \\
F^x_{ij}  & \equiv & \partial_x F(x_i, y_i) \\
F^y_{ij}  & \equiv & \partial_y F(x_i, y_i) \\
F^{xy}    & \equiv & \partial_x \partial_y F(x_i, y_i)
\end{eqnarray}

Consider the point $(x,y)$ at which we wish to interpolate $F$.  We
locate the rectangle which contains this point, such that $x_i <= x <
x_{i+1}$ and $y_i <= x < y_{i+1}$.  Let 
\begin{eqnarray}
h & \equiv & x_{i+1}-x_i \\
l & \equiv & y_{i+1}-y_i \\
u & \equiv & \frac{x-x_i}{h} \\
v & \equiv & \frac{y-y_i}{l}
\end{eqnarray}
Then, we calculate the interpolated value as
\begin{equation}
F(x,y) = 
\begin{MyMatrix}
p_1(u) \\ p_2(u) \\ h q_1(u) \\ h q_2(u) 
\end{MyMatrix}^T
\begin{MyMatrix}
F_{i,j}     & F_{i+1,j}     & F^y_{i,j}      & F^y_{i,j+1}     \\
F_{i+1,j}   & F_{i+1,j+1}   & F^y_{i+1,j}    & F^y_{i+1,j+1}   \\
F^x_{i,j}   & F^x_{i,j+1}   & F^{xy}_{i,j}   & F^{xy}_{i,j+1}  \\
F^x_{i+1,j} & F^x_{i+1,j+1} & F^{xy}_{i+1,j} & F^{xy}_{i+1,j+1} 
\end{MyMatrix}
\begin{MyMatrix}
p_1(v)\\ p_2(v)\\ k q_1(v) \\ k q_2(v)
\end{MyMatrix}
\end{equation}
\subsubsection{Construction bicubic splines}
We now address the issue of how to compute the derivatives that are
needed for the interpolation.  The algorithm is quite simple.  For
every $x_i$, we perform the tridiagonal solution as we did in the 1D
splines to compute $F^y_{ij}$.  Similarly, we perform a tridiagonal
solve for every value of $F^x_{ij}$.  Finally, in order to compute the
cross-derivative we may {\em either} to the tridiagonal solve in the $y$
direction of $F^x_{ij}$, {\em or} solve in the $x$ direction for
$F^y_{ij}$ to obtain the cross-derivatives, $F^{xy}_{ij}$.  Hence,
only minor modifications to the $1D$ interpolations are necessary.

\subsection{Tricubic Splines}
Bicubic interpolation required two four-component vectors and a 4x4
matrix.  By extension, tricubic interpolation requires three
4-component vectors and a 4x4x4 tensor.  We summarize the forms of
these vectors below.
\begin{eqnarray}
h & \equiv & x_{i+1}-x_i \\
l & \equiv & y_{i+1}-y_i \\
m & \equiv & z_{i+1}-z_i \\
u & \equiv & \frac{x-x_i}{h} \\
v & \equiv & \frac{y-y_i}{l} \\
w & \equiv & \frac{z-z_i}{m}
\end{eqnarray}
\begin{eqnarray}
\vec{a} & = & 
\begin{MyMatrix}
p_1(u) & p_2(u) & h q_1(u) & h q_2(u) 
\end{MyMatrix}^T \\
\vec{b} & = & 
\begin{MyMatrix}
p_1(v) & p_2(v) & k q_1(v) & k q_2(v) 
\end{MyMatrix}^T \\
\vec{c} & = & 
\begin{MyMatrix}
p_1(w) & p_2(w) & l q_1(w) & l q_2(w) 
\end{MyMatrix}^T 
\end{eqnarray}
\begin{equation}
\begin{LMatrix}
A_{000} = F_{i,j,k}     & A_{001}=F_{i,j,k+1}     & A_{002}=F^z_{i,j,k}      & A_{003}=F^z_{i,j,k+1}      \\
A_{010} = F_{i,j+1,k}   & A_{011}=F_{i,j+1,k+1}   & A_{012}=F^z_{i,j+1,k}    & A_{013}=F^z_{i,j+1,k+1}    \\
A_{020} = F^y_{i,j,k}   & A_{021}=F^y_{i,j,k+1}   & A_{022}=F^{yz}_{i,j,k}   & A_{023}=F^{yz}_{i,j,k+1}   \\
A_{030} = F^y_{i,j+1,k} & A_{031}=F^y_{i,j+1,k+1} & A_{032}=F^{yz}_{i,j+1,k} & A_{033}=F^{yz}_{i,j+1,k+1} \\
                        &                         &                          &                            \\
A_{100} = F_{i+1,j,k}     & A_{101}=F_{i+1,j,k+1}     & A_{102}=F^z_{i+1,j,k}      & A_{103}=F^z_{i+1,j,k+1}      \\
A_{110} = F_{i+1,j+1,k}   & A_{111}=F_{i+1,j+1,k+1}   & A_{112}=F^z_{i+1,j+1,k}    & A_{113}=F^z_{i+1,j+1,k+1}    \\
A_{120} = F^y_{i+1,j,k}   & A_{121}=F^y_{i+1,j,k+1}   & A_{122}=F^{yz}_{i+1,j,k}   & A_{123}=F^{yz}_{i+1,j,k+1}   \\
A_{130} = F^y_{i+1,j+1,k} & A_{131}=F^y_{i+1,j+1,k+1} & A_{132}=F^{yz}_{i+1,j+1,k} & A_{133}=F^{yz}_{i+1,j+1,k+1} \\
                        &                         &                          &                            \\
A_{200} = F^x_{i,j,k}      & A_{201}=F^x_{i,j,k+1}      & A_{202}=F^{xz}_{i,j,k}      & A_{203}=F^{xz}_{i,j,k+1}    \\
A_{210} = F^x_{i,j+1,k}    & A_{211}=F^x_{i,j+1,k+1}    & A_{212}=F^{xz}_{i,j+1,k}    & A_{213}=F^{xz}_{i,j+1,k+1}  \\
A_{220} = F^{xy}_{i,j,k}   & A_{221}=F^{xy}_{i,j,k+1}   & A_{222}=F^{xyz}_{i,j,k}     & A_{223}=F^{xyz}_{i,j,k+1}   \\
A_{230} = F^{xy}_{i,j+1,k} & A_{231}=F^{xy}_{i,j+1,k+1} & A_{232}=F^{xyz}_{i,j+1,k}   & A_{233}=F^{xyz}_{i,j+1,k+1} \\
                        &                         &                          &                                      \\
A_{300} = F^x_{i+1,j,k}      & A_{301}=F^x_{i+1,j,k+1}      & A_{302}=F^{xz}_{i+1,j,k}    & A_{303}=F^{xz}_{i+1,j,k+1}   \\
A_{310} = F^x_{i+1,j+1,k}    & A_{311}=F^x_{i+1,j+1,k+1}    & A_{312}=F^{xz}_{i+1,j+1,k}  & A_{313}=F^{xz}_{i+1,j+1,k+1} \\
A_{320} = F^{xy}_{i+1,j,k}   & A_{321}=F^{xy}_{i+1,j,k+1}   & A_{322}=F^{xyz}_{i+1,j,k}   & A_{323}=F^{xyz}_{i+1,j,k+1}  \\
A_{330} = F^{xy}_{i+1,j+1,k} & A_{331}=F^{xy}_{i+1,j+1,k+1} & A_{332}=F^{xyz}_{i+1,j+1,k} & A_{333}=F^{xyz}_{i+1,j+1,k+1} 
\end{LMatrix}
\end{equation}
Now, we can write
\begin{equation}
F(x,y,z) = \sum_{i=0}^3 a_i \sum_{j=0}^3 b_j \sum_{k=0}^3 c_k \ A_{i,j,k} 
\end{equation}
The appropriate derivatives of $F$ may be computed by a generalization
of the method used for bicubic splines above.




\newpage
\section{Feature: B-spline Orbital Tiling (Band Unfolding)}

% Written by Kenneth P .Esler Jr.
% Originally titled ``Generalized band unfolding for quantum Monte Carlo simulation of solids''

In continuum quantum Monte Carlo simulations, it is necessary to
evaluate the electronic orbitals of a system at real-space positions
hundreds of millions of times.  It has been found that if
these orbitals are represented in a localized, B-spline basis, each
evaluation takes a small, constant time which is independent of system
size.

Unfortunately, the memory required for storing the B-spline grows with
the second power of the system size.  If we are studying perfect
crystals, however, this can be reduced to linear scaling if we {\em
  tile} the primitive cell.  In this approach, 
%implemented in the CASINO QMC simulation suite, 
a supercell is constructed by tiling the
primitive cell $N_1 \times N_2 \times N_3$ in the three lattice
directions.  The orbitals are then represented in real space only in
the primitive cell, and an $N_1 \times N_2 \times N_3$ k-point mesh.
To evaluate an orbital at any point in the supercell, it is only
necessary to wrap that point back into the primitive cell, evaluate
the spline, and then multiply the phase factor,
$e^{-i\mathbf{k}\cdot\mathbf{r}}$.  

Here, we show that this approach can be generalized to a tiling
constructed with a $3\times 3$ nonsingular matrix of integers, of which
the above approach is a special case.  This generalization brings with
it a number of advantages.  The primary reason for performing
supercell calculations in QMC is to reduce finite-size errors.  These
errors result from three sources:  1) the quantization of the crystal
momentum;  2) the unphysical periodicity of the exchange-correlation
hole of the electron; and 3) the kinetic-energy contribution from the
periodicity of the long-range jastrow correlation functions.  The first
source of error can be largely eliminated by twist averaging.  If the
simulation cell is large enough that XC hole does not ``leak'' out of
the simulation cell, the second source can be eliminated either
through use of the MPC interaction or the {\em a postiori} correction
of Chiesa et. al.  

The satisfaction of the leakage requirement is controlled by whether
the minimum distance, $L_\text{min}$ from one supercell image to the
next is greater than the width of the XC hole.  Therefore, given a
choice, it is best to use a cell which is as nearly cubic as possible,
since this choice maximizes $L_\text{min}$ for a given number of
atoms.  Most often, however, the primitive cell is not cubic.  In
these cases, if we wish to choose the optimal supercell to reduce
finite size effects, we cannot utilize the simple primitive tiling
scheme.  In the generalized scheme we present, it is possible to
choose far better supercells (from the standpoint of finite-size
errors), while retaining the storage efficiency of the original tiling
scheme.

\subsection{The mathematics}
\renewcommand{\vp}{\mathbf{a}^\text{p}}
\renewcommand{\vs}{\mathbf{a}^\text{s}} 
\renewcommand{\Smat}{\mathbf{S}}
Consider the set of primitive lattice vectors, $\{\vp_1, \vp_2,
\vp_3\}$.  We may write these vectors in a matrix, $\mathbf{L}_p$, whose
rows are the primitive lattice vectors.  Consider a non-singular
matrix of integers, $\Smat$.  A corresponding set of supercell lattice
vectors, $\{\vs_1, \vs_2, \vs_3\}$, can be constructed by the matrix
product 
\begin{equation}
\vs_i = S_{ij} \vp_j
\end{equation}
If the primitive cell contains $N_p$ atoms, the supercell will then
contain $N_s = |\det(\Smat)| N_p$ atoms.

\subsection{Example: FeO}
As an example, consider the primitive cell for antiferromagnetic FeO
(wustite) in the rocksalt structure.  The primitive vectors, given in
units of the lattice constant, are given by
\newcommand{\xv}{\hat{\mathbf{x}}} 
\newcommand{\yv}{\hat{\mathbf{y}}}
\newcommand{\zv}{\hat{\mathbf{z}}}
\begin{eqnarray}
\vs_1 & = & \frac{1}{2}\xv + \frac{1}{2}\yv +      \ \   \zv \\
\vs_2 & = & \frac{1}{2}\xv +      \ \   \yv + \frac{1}{2}\zv \\
\vs_3 & = &   \ \      \xv + \frac{1}{2}\yv + \frac{1}{2}\zv 
\end{eqnarray}
This primitive cell contains two iron atoms and two oxygen atoms. It
is a very elongated cell with acute angles, and thus has a short
minimum distance between adjacent images.

The smallest cubic cell consistent with the AFM ordering can be
constructed with the matrix
\begin{equation}
\Smat = \left[\begin{array}{rrr}
  -1 & -1 &  3 \\
  -1 &  3 & -1 \\
   3 & -1 & -1 
  \end{array}\right]
\end{equation}
This cell has $2\det(\Smat) = 32$ iron atoms and 32 oxygen atoms.  In
this example, we may perform the simulation in the 32-iron supercell,
while storing the orbitals only in the 2-iron primitive cell, for a
savings of a factor of 16.  
%On current multicore supercomputers, with
%1-2GB RAM per core, this is literally the difference between be able
%to perform the simulation or not.

\subsubsection{The k-point mesh}
In order to be able to use the generalized tiling scheme, we need to
have the appropriate number of bands to occupy in the supercell.
This may be achieved by appropriately choosing the k-point mesh.  In
this section, we explain how these points are chosen.  

For simplicity, let us assume that the supercell calculation will be
performed at the $\Gamma$-point.  We may lift this restriction very
easily later.  The fact that supercell calculation is performed at
$\Gamma$ implies that the k-points used in the primitive-cell
calculation must be $\mathbf{G}$-vectors of the superlattice.  This
still leaves us with an infinite set of vectors.  We may reduce this
set to a finite number by considering that the orbitals must form an
linearly independent set.  Orbitals with k-vectors $\mathbf{k}^p_1$
and $\mathbf{k}^p_2$ will differ by at most a constant factor if
$\mathbf{k}^p_1 - \mathbf{k}^p_2 = \mathbf{G}^p$, where $\mathbf{G}^p$
is a reciprocal lattice vector of the primitive cell.  

Combining these two considerations gives us a prescription for
generating our k-point mesh.  The mesh may be taken to be the set of
k-point which are G-vectors of the superlattice, reside within the
first Brillouin zone (FBZ) of the primitive lattice, whose members do
not differ a G-vector of the primitive lattice.  Upon constructing
such a set, we find that the number of included k-points is equal to
$|\det(\Smat)|$, precisely the number we need.  This can by considering
the fact that the supercell has a volume $|\det(\Smat)|$ times that of
the primitive cell.  This implies that the volume of the supercell's
FBZ is $|\det(\Smat)|^{-1}$ times that of the primitive cell.  Hence,
$|\det(\Smat)|$ G-vectors of the supercell will fit in the FBZ of the
primitive cell.  Removing duplicate k-vectors, which differ from
another by a reciprocal lattice vector, avoids double-counting vectors
which lie on zone faces.

\subsubsection{Formulae}
\newcommand{\Amat}{\mathbf{A}} 
\newcommand{\Bmat}{\mathbf{B}} 
\renewcommand{\vk}{\mathbf{k}}
\newcommand{\vt}{\mathbf{t}}

Let $\Amat$ be the matrix whose rows are the direct lattice vectors,
$\{\mathbf{a}_i\}$.  The, let the matrix $\Bmat$ be defined as
$2\pi(\Amat^{-1})^\dagger$.  Its rows are the primitive reciprocal
lattice vectors.  Let $\Amat_p$ and $\Amat_s$ represent the primitive
and superlattice matrices, respectively, and similarly for their
reciprocals.  Then we have
\begin{eqnarray}
\Amat_s & = & \Smat \Amat_p \\
\Bmat_s & = & 2\pi\left[(\Smat \Amat_p)^{-1}\right]^\dagger \\
        & = & 2\pi\left[\Amat_p^{-1} \Smat{-1}\right]^\dagger \\
        & = & 2\pi(\Smat^{-1})^\dagger (\Amat_p^{-1})^\dagger \\
        & = & (\Smat^{-1})^\dagger \Bmat_p
\end{eqnarray}  
Consider a k-vector, $\vk$.  It may be alternatively be written in
basis of reciprocal lattice vectors as $\vt$.  
\begin{eqnarray}
\vk & = & (\vt^\dagger \Bmat)^\dagger \\
    & = & \Bmat^\dagger \vt           \\
\vt & = & (\Bmat^\dagger)^{-1} \vk    \\
    & = & (\Bmat^{-1})^\dagger \vk    \\
    & = & \frac{\Amat \vk}{2\pi}
\end{eqnarray}
We may then express a twist vector of the primitive lattice, $\vt_p$ in terms
of the superlattice.
\begin{eqnarray}
\vt_s & = & \frac{\Amat_s \vk}{2\pi}                           \\
      & = & \frac{\Amat_s \Bmat_p^\dagger \vt_p}{2\pi}         \\
      & = & \frac{\Smat \Amat_p \Bmat_p^\dagger \vt_p}{2\pi}   \\
      & = & \frac{2\pi \Smat \Amat_p \Amat_p^{-1} \vt_p}{2\pi} \\
      & = & \Smat \vt_p
\end{eqnarray}
This gives the simple result that twist-vectors transform in precisely
the same way as direct lattice vectors.




\newpage
\section{Feature: Hybrid orbital representation}

% Written by Kenneth P. Esler, Jr.
% Document originally included in QMCPACK at src/QMCWaveFunctions/AtomicOrbital.tex
% Originally titled ``Hybrid orbital representation''

\renewcommand{\vr}{\mathbf{r}}

\begin{equation}
\phi(\vr) = \sum_{\ell=0}^{\ell_\text{max}} \sum_{m=-\ell}^\ell Y_\ell^m (\hat{\Omega})
u_{\ell m}(r),
\end{equation}
where $u_{lm}(r)$ are complex radial functions represented in some
radial basis (e.g. splines).

\subsection{Real Spherical Harmonics}
\renewcommand{\Re}{\rm Re}
\renewcommand{\Im}{\rm Im}
If $\phi(\vr)$ can be written as purely real, we can change the
representation so that
\begin{equation}
\phi(\vr) = \sum_{l=0}^{l_\text{max}} \sum_{m=-\ell}^\ell Y_{\ell m}(\hat{\Omega})
\bar{u}_{lm}(r),
\end{equation}
where $\bar{Y}_\ell^m$ are the {\em real} spherical harmonics defined by
\begin{equation}
Y_{\ell m} = \begin{cases}
Y_\ell^0 & \mbox{if } m=0\\
{1\over 2}\left(Y_\ell^m+(-1)^m \, Y_\ell^{-m}\right) \ = \Re\left[Y_\ell^m\right]
%\sqrt{2} N_{(\ell,m)} P_\ell^m(\cos \theta) \cos m\varphi 
& \mbox{if } m>0 \\
{1\over i 2}\left(Y_\ell^{-m}-(-1)^{m}\, Y_\ell^{m}\right) = \Im\left[Y_\ell^{-m}\right]
%\sqrt{2} N_{(\ell,m)} P_\ell^{-m}(\cos \theta) \sin m\varphi 
&\mbox{if } m<0.
\end{cases}
\end{equation}
We need then to relate $\bar{u}_{\ell m}$ to $u_{\ell m}$.  We wish
to express,
\begin{equation}
\Re\left[\phi(\vr)\right] = \sum_{\ell=0}^{\ell_\text{max}} \sum_{m=-\ell}^\ell
\Re\left[Y_\ell^m (\hat{\Omega}) u_{\ell m}(r)\right]
\end{equation}
in terms of $\bar{u}_{\ell m}(r)$ and $Y_{\ell m}$.
\begin{eqnarray}
\Re\left[Y_\ell^m u_{\ell m}\right] & = & \Re\left[Y_\ell^m\right]
\Re\left[u_{\ell m}\right] - \Im\left[Y_\ell^m\right] \Im\left[u_{\ell m}\right]
\end{eqnarray}
For $m>0$,
\begin{equation}
\Re\left[Y_\ell^m\right] = Y_{\ell m} \qquad \text{and} \qquad \Im\left[Y_\ell^m\right] = Y_{\ell\,-m}.
\end{equation}
For $m<0$,
\begin{equation}
\Re\left[Y_\ell^m\right] = (-1)^m Y_{\ell\, -m} \qquad \text and \qquad \Im\left[Y_\ell^m\right] = -(-1)^m Y_{\ell m}.
\end{equation}
Then for $m > 0$,
\begin{eqnarray}
\bar{u}_{\ell m} & = & \Re\left[u_{\ell m}\right] + (-1)^m \Re\left[u_{\ell\,-m}\right] \\
\bar{u}_{\ell\, -m} & = & -\Im\left[u_{\ell m}\right] + (-1)^m \Im\left[u_{\ell\,-m}\right].
\end{eqnarray}


\subsection{Projecting to atomic orbitals}

% Written by Ken Esler as part of the Common codebase used in wfconvert
% Originally titled ``Notes on projecting to atomic orbitals''
% Dated July 19, 2009

\renewcommand{\vr}{\mathbf{r}}
\newcommand{\vI}{\mathbf{I}}
\renewcommand{\vk}{\mathbf{k}}
\newcommand{\vG}{\mathbf{G}}

%\subsubsection{Form for orbitals}
Inside a muffin tin, orbitals are represented as product of spherical
harmonics and 1D radial functions, primarily represented by splines.
For a muffin tin centered at $\vI$, 
\begin{equation}
\phi_n(\vr) = \sum_{\ell,m} Y_\ell^m(\hat{\vr -\vI})
u_{lm}\left(\left|\vr - \vI\right|\right) \label{eq:ulm}
\end{equation}
Let use consider the case that our original representation for
$\phi(\vr)$ is of the form
\begin{equation}
\phi_{n,\vk}(\vr) = \sum_\vG c_{\vG+\vk}^n e^{i(\vG + \vk)\cdot \vr}
\end{equation}
Recall that
\begin{equation}
e^{i\vk\cdot\vr} = 4\pi \sum_{\ell,m} i^\ell j_\ell(|\vr||\vk|)
Y_\ell^m(\hat{\vk}) \left[Y_\ell^m(\hat{\vr})\right]^*.
\end{equation}
Conjugating,
\begin{equation}
e^{-i\vk\cdot\vr} = 4\pi\sum_{\ell,m} (-i)^\ell j_\ell(|\vr||\vk|)
\left[Y_\ell^m(\hat{\vk})\right]^* Y_\ell^m(\hat{\vr}).
\end{equation}
Setting $\vk \rightarrow -k$,
\begin{equation}
e^{i\vk\cdot\vr} = 4\pi\sum_{\ell,m} i^\ell j_\ell(|\vr||\vk|)
\left[Y_\ell^m(\hat{\vk})\right]^* Y_\ell^m(\hat{\vr}).
\end{equation}

Then,
\begin{equation}
e^{i\vk\cdot(\vr-\vI)} = 4\pi\sum_{\ell,m} i^\ell j_\ell(|\vr-\vI||\vk|)
\left[Y_\ell^m(\hat{\vk})\right]^* Y_\ell^m(\hat{\vr-\vI}).
\end{equation}

\begin{equation}
e^{i\vk\cdot\vr} = 4\pi e^{i\vk\cdot\vI} \-\sum_{\ell,m} i^\ell j_\ell(|\vr-\vI||\vk|)
\left[Y_\ell^m(\hat{\vk})\right]^* Y_\ell^m(\hat{\vr-\vI}).
\end{equation}

Then
\begin{equation}
\phi_{n,\vk}(\vr) =  \sum_\vG 4\pi c_{\vG+\vk}^n
e^{i(\vG+\vk)\cdot\vI} \sum_{\ell,m}
  i^\ell j_\ell(|\vG +\vk||\vr-\vI|)
  \left[Y_\ell^m(\hat{\vG+\vk})\right]^*
Y_\ell^m(\hat{\vr - \vI})
\end{equation}
Comparing to (\ref{eq:ulm}),
\begin{equation}
u_{\ell m}^n(r) = 4\pi i^\ell \sum_G c_{\vG+\vk}^n e^{i(\vG+\vk)\cdot\vI}  j_\ell\left(|\vG + \vk|r|\right)
\left[Y_\ell^m(\hat{\vG + \vk})\right]^*.
\end{equation}
If we had adopted the opposite sign convention for Fourier transforms
(as is unfortunately the case in wfconvert), we would have
\begin{equation}
u_{\ell m}^n(r) = 4\pi (-i)^\ell \sum_G c_{\vG+\vk}^n e^{-i(\vG+\vk)\cdot\vI}  j_\ell\left(|\vG + \vk|r|\right)
\left[Y_\ell^m(\hat{\vG + \vk})\right]^*.
\end{equation}




\newpage
\section{Feature: Electron-electron-ion Jastrow factor}

% Written by Kenneth P. Esler, Jr.
% Document originally included in QMCPACK at src/QMCWaveFunctions/Jastrow/eeI_Jastrow.tex
% Originally titled ``Electron-electron-ion Jastrow factor''

\newcommand{\riI}{r_{iI}}
\newcommand{\briI}{\mathbf{r}_{iI}}
\newcommand{\rjI}{r_{jI}}
\newcommand{\brjI}{\mathbf{r}_{jI}}
\newcommand{\rij}{r_{ij}}
\newcommand{\brij}{\mathbf{r}_{ij}}
%\section{Form of the Jastrow}
The general form of the 3-body Jastrow we describe here depends on the
three inter-particle distances, $(\rij, \riI, \rjI)$.
\begin{equation}
J_3 = \sum_{I\in\text{ions}} \sum_{i,j \in\text{elecs};i\neq j} U(\rij, \riI,
\rjI)
\end{equation}
Note that we constrain the form of $U$ such that
$U(\rij, \riI,\rjI) = U(\rij, \rjI,\riI)$, so as to preserve the
particle symmetry of the wave function.  We then compute the gradient as
\begin{equation}
\nabla_i J_3 =  \sum_{I\in\text{ions}} \sum_{j \neq i}
\left[\frac{\partial U(\rij, \riI,\rjI)}{\partial\rij}
  \frac{\mathbf{r}_i - \mathbf{r}_j}{|\mathbf{r}_i - \mathbf{r}_j|} 
+ \frac{\partial U(\rij, \riI,\rjI)}{\partial\riI}
  \frac{\mathbf{r}_i - \mathbf{I}}{|\mathbf{r}_i - \mathbf{I}|}  \right]
\end{equation}
To compute the laplacian, we take
\begin{eqnarray}
\nabla_i^2 J_3 & = & \nabla_i \cdot \left(\nabla_i J_3\right) \\
& = & \sum_{I\in\text{ions}} \sum_{j\neq i } \left[
\frac{\partial^2 U}{\partial \rij^2} + \frac{2}{\rij} \frac{\partial
  U}{\partial \rij} + 2 \frac{\partial^2 U}{\partial \rij \partial
  \riI}\frac{\brij\cdot\briI}{\rij\riI} +\frac{\partial^2 U}{\partial
  \riI^2}
+ \frac{2}{\riI}\frac{\partial U}{\partial \riI} \nonumber
\right]
\end{eqnarray}
We now wish to compute the gradient of these terms w.r.t. the ion position, $I$.
\begin{equation}
\nabla_I J_3 = -\sum_{j\neq i} \left[ \frac{\partial U(\rij, \riI,\rjI)}{\partial\riI}
  \frac{\mathbf{r}_i - \mathbf{I}}{|\mathbf{r}_i - \mathbf{I}|} 
+\frac{\partial U(\rij, \riI,\rjI)}{\partial\rjI}
  \frac{\mathbf{r}_j - \mathbf{I}}{|\mathbf{r}_j - \mathbf{I}|} \right]
\end{equation}
For the gradient w.r.t. $i$ of the gradient w.r.t. $I$, the result is a tensor,
\begin{eqnarray}
\nabla_I \nabla_i J_3 & = & \nabla_I \sum_{j \neq i}
\left[\frac{\partial U(\rij, \riI,\rjI)}{\partial\rij}
  \frac{\mathbf{r}_i - \mathbf{r}_j}{|\mathbf{r}_i - \mathbf{r}_j|} 
+ \frac{\partial U(\rij, \riI,\rjI)}{\partial\riI}
  \frac{\mathbf{r}_i - \mathbf{I}}{|\mathbf{r}_i - \mathbf{I}|}  \right] \\\nonumber \\\nonumber
& = & -\sum_{j\neq i} \left[ 
\frac{\partial^2 U}{\partial \rij \riI} \hat{\mathbf{r}}_{ij} \otimes
\hat{\mathbf{r}}_{iI} + \left(\frac{\partial^2 U}{\partial \riI^2} -
\frac{1}{\riI} \frac{\partial U}{\partial \riI}\right)
\hat{\mathbf{r}}_{iI} \otimes \hat{\mathbf{r}}_{iI} \right. + \\\nonumber
& & \left. \qquad \ \ \  \frac{\partial^U}{\partial \rij \rjI} \hat{\mathbf{r}}_{ij} \otimes \hat{\mathbf{r}}_{jI} + \frac{\partial^2 U}{\partial \riI \partial \rjI}
\hat{\mathbf{r}}_{iI}\otimes \hat{\mathbf{r}}_{jI}  +
\frac{1}{\riI} \frac{\partial U}{\partial \riI} \overleftrightarrow{\mathbf{1}}\right]
\end{eqnarray}

\begin{eqnarray}
\nabla_I \nabla_i J_3 & = & \nabla_I \sum_{j \neq i}
\left[\frac{\partial U(\rij, \riI,\rjI)}{\partial\rij}
  \frac{\mathbf{r}_i - \mathbf{r}_j}{|\mathbf{r}_i - \mathbf{r}_j|} 
+ \frac{\partial U(\rij, \riI,\rjI)}{\partial\riI}
  \frac{\mathbf{r}_i - \mathbf{I}}{|\mathbf{r}_i - \mathbf{I}|}  \right] \\\nonumber 
& = & \sum_{j\neq i} \left[ -\frac{\partial^2 U}{\partial \rij \partial \riI} \hat{\mathbf{r}}_{ij} \otimes \hat{\mathbf{r}}_{iI} +
\left(-\frac{\partial^2 U}{\partial \riI^2}  + \frac{1}{\riI}\frac{\partial U}{\partial \riI} \right) 
\hat{\mathbf{r}}_{iI} \otimes \hat{\mathbf{r}}_{iI} - \frac{1}{\riI}\frac{\partial U}{\partial \riI} \overleftrightarrow{\mathbf{1}}
\right]
\end{eqnarray}
For the laplacian,
\begin{eqnarray}
\nabla_I \nabla_i^2 J_3 & = & \nabla_I\left[\nabla_i \cdot \left(\nabla_i J_3\right)\right] \\
& = & \nabla_I \sum_{j\neq i } \left[
\frac{\partial^2 U}{\partial \rij^2} + \frac{2}{\rij} \frac{\partial
  U}{\partial \rij} + 2 \frac{\partial^2 U}{\partial \rij \partial
  \riI}\frac{\brij\cdot\briI}{\rij\riI} +\frac{\partial^2 U}{\partial
  \riI^2}
+ \frac{2}{\riI}\frac{\partial U}{\partial \riI} \nonumber
\right] \\
& = & \sum_{j\neq i } 
\left[ \frac{\partial^3 U}{\partial r_{iI} \partial^2 r_{ij}} +
\frac{2}{r_{ij}} \frac{\partial^2 U}{\partial r_{iI} \partial r_{ij}}
+ 2\left(\frac{\partial^3 U}{\partial \rij \partial^2 \riI} -\frac{1}{\riI} \frac{\partial^2 U}{\partial \rij \partial \riI}\right)\frac{\brij\cdot\briI}{\rij\riI} + \frac{\partial^3 U}{\partial^3 \riI} - \frac{2}{\riI^2} \frac{\partial U}{ \partial \riI} + \frac{2}{\riI} \frac{\partial^2 U}{\partial^2 \riI}
\right] \frac{\mathbf{I} - \mathbf{r}_i}{|\mathbf{I} - \mathbf{r}_i|} + \nonumber \\\nonumber 
 & & \sum_{j\neq i } \left[ \frac{\partial^3U}{\partial \rij^2 \partial \rjI} + \frac{2}{\rij}\frac{\partial^2 U}{\partial \rjI \partial \rij} 
+ 2\frac{\partial^3 U}{\partial \rij \partial \riI \partial \rjI}\frac{\brij\cdot\briI}{\rij\riI}
+\frac{\partial^3 U}{\partial \riI^2 \partial \rjI} + \frac{2}{\riI}\frac{\partial^2 U}{\partial \riI \partial \rjI} \right] 
\frac{\mathbf{I} - \mathbf{r}_j}{|\mathbf{r}_j - \mathbf{I}|} + \\\nonumber 
& & \sum_{j\neq i } \left[ -\frac{2}{\riI}\frac{\partial^2 U}{\partial \rij \partial \riI}\right] \frac{\mathbf{r}_{ij}}{r_{ij}}
\end{eqnarray}




\newpage
\section{Feature: Reciprocal-Space Jastrow Factors}

% Written by Kenneth P. Esler, Jr.
% Document originally included in QMCPACK at src/QMCWaveFunctions/Jastrow/kSpaceJastrowNotes.tex
% Originally titled ``Notes on Reciprocal-Space Jastrow Factors''

\renewcommand{\vG}{\mathbf{G}}
\renewcommand{\vr}{\mathbf{r}}
\renewcommand{\vI}{\mathbf{I}}

\subsection{Two-body Jastrow}
\begin{equation}
J_2 = \sum_{\vG\neq \mathbf{0}}\sum_{i\neq j} a_\vG e^{i\vG\cdot(\vr_i-\vr_j)}
\end{equation}
This may be rewritten as
\begin{eqnarray}
J_2 & = & \sum_{\vG\neq \mathbf{0}}\sum_{i\neq j} a_\vG e^{i\vG\cdot\vr_i}e^{-i\vG\cdot\vr_j} \\
& = & \sum_{\vG\neq \mathbf{0}} a_\vG \left\{
\underbrace{\left[\sum_i e^{i\vG\cdot\vr_i} \right]}_{\rho_\vG}
\underbrace{\left[\sum_j e^{-i\vG\cdot\vr_j} \right]}_{\rho_{-\vG}}  -1 \right\}
\end{eqnarray}
The $-1$ is just a constant term and may be subsumed into the $a_\vG$
coefficient by a simple redefinition.  This leaves a simple, but
general, form:
\begin{equation}
J_2 = \sum_{\vG\neq\mathbf{0}} a_\vG \rho_\vG \rho_{-\vG}
\end{equation}
We may now further constrain this on physical grounds.  First, we
recognize that $J_2$ should be real.  Since $\rho_{-\vG} =
\rho_\vG^*$, it follows that $\rho_{\vG}\rho_{-\vG} = |\rho_\vG|^2$ is
real, so that $a_\vG$ must be real.  Furthermore, we group the $\vG$'s
into $(+\vG, -\vG)$ pairs, and sum over only the positive vectors to
save time.

\subsection{One-body Jastrow}
The one-body Jastrow has a similar form, but depends on the
displacement from the electrons to the ions in the system.
\begin{equation}
J_1 = \sum_{\vG\neq\mathbf{0}} \sum_{\alpha}
\sum_{i\in\vI^\alpha}\sum_{j\in\text{elec.}} b^{\alpha}_\vG
  e^{i\vG\cdot(\vI^{\alpha}_i - \vr_j)},
\end{equation}
where $\alpha$ denotes the different ionic species.
We may rewrite this in terms of $\rho^{\alpha}_\vG$, 
\begin{equation}
J_1 = \sum_{\vG\neq\mathbf{0}} \left[\sum_\alpha b^\alpha_\vG
  \rho_\vG^\alpha\right] \rho_{-\vG},
\end{equation}
where
\begin{equation}
\rho^\alpha_\vG = \sum_{i\in\vI^\alpha} e^{i\vG\cdot\vI^\alpha_i}.
\end{equation}
We note that in the above equation, for a single configuration of the
ions, the sum in brackets can be rewritten as a single constant.  This
implies that the per-species one-body coefficients, $b^\alpha_\vG$, are
underdetermined for single configuration of the ions.  In general, if
we have $N$ species, we need $N$ linearly independent ion
configurations to uniquely determine $b^{\alpha}_\vG$.  For this
reason, we will drop the $\alpha$ superscript of $b_\vG$ for now.  

If we do desire to find a reciprocal space one-body Jastrow that is
transferable to systems with different ion positions and $N$ 
ionic species, we must perform compute $b_\vG$ for $N$ different ion
configurations.  We may then construct $N$ equations at each value of
$\vG$ to solve for the $N$ unknown values, $b^\alpha_\vG$.

In the two-body case, $a_\vG$ was constrained to be real by the fact
that $\rho_\vG \rho_{-\vG}$ was real.  However, in the one-body case,
there is no such guarantee about $\rho^\alpha_\vG \rho_\vG$.
Therefore, in general, $b_\vG$ may be complex.

\subsection{Symmetry considerations}
For a crystal, many of the $\vG$-vectors will be equivalent by
symmetry.  It is useful then, to divide the $\vG$-vectors into
symmetry-related groups and then to require that they share a common
coefficient.  Two vectors, $\vG$ and $\vG'$, may be considered
symmetry related if, for all $\alpha$ and $\beta$,
\begin{equation}
\rho^\alpha_\vG \rho^\beta_{-\vG} = \rho^\alpha_{\vG'} \rho^\beta_{-\vG'}. 
\end{equation}
For the one-body term, we may also omit from our list of $\vG$-vectors
those for which all species structure factors are zero.  This is
equivalent to saying that, if we are tiling a primitive cell, we
should include only the $\vG$-vectors of the primitive cell, and not
the supercell.  Note that this is not the case for the two-body term,
since the exchange-correlation hole should not have the periodicity of
the primitive cell.

\subsection{Gradients and Laplacians}
\begin{eqnarray}
\nabla_{\vr_i} J_2 & = & \sum_{\vG \neq 0} a_\vG \left[\left(\nabla_{\vr_i}\rho_\vG\right) \rho_{-\vG} + \text{c.c.}\right] \\
& = & \sum_{\vG\neq \mathbf{0}} 2\vG a_\vG \mathbf{Re}\left(i e^{i\vG\cdot\vr_i} \rho_{-\vG} \right) \\
& = & \sum_{\vG\neq \mathbf{0}} -2\vG a_\vG\mathbf{Im}\left(e^{i\vG\cdot\vr_i} \rho_{-\vG} \right)
\end{eqnarray}
The Laplacian is then given by
\begin{eqnarray}
  \nabla^2 J_2 & = & \sum_{\vG\neq\mathbf{0}} a_\vG \left[\left(\nabla^2 \rho_\vG\right) \rho_{-\vG} + \text{c.c.} 
  + 2\left(\nabla \rho_\vG)\cdot(\nabla \rho_{-\vG}\right)\right] \\
& = & \sum_{\vG\neq\mathbf{0}} a_\vG \left[ -2G^2\mathbf{Re}(e^{i\vG\cdot\vr_i}\rho_{-\vG}) + 
    2\left(i\vG e^{i\vG\cdot\vr_i}\right) \cdot \left(-i\vG e^{-i\vG\cdot\vr_i}\right)
\right] \\
& = & 2 \sum_{\vG\neq\mathbf{0}} G^2 a_\vG  \left[-\mathbf{Re}\left(e^{i\vG\cdot\vr_i}\rho_{-\vG}\right) + 1\right] 
%  \nabla^2_{\vr_i} J_2 & = & \nabla_{\vr_i} \cdot \nabla_{\vr_i} J_2 \\
%  & = & -2\sum_{\vG \neq \mathbf{0}} a_\vG \vG \cdot \nabla_{\vr_i} \mathbf{Im}\left(e^{i\vG\cdot\vr_i} \rho_{-\vG}\right)
%  & = & -2\sum_{\vG \neq \mathbf{0}} a_\vG \vG \cdot \mathbf{Im}\left(i\vG e^{i\vG\cdot\vr_i}\rho_{-\vG} -i\vG \right)
\end{eqnarray}


\chapter{Development Guide}
\label{chap:developguide}

The section gives guidance on how to extend the functionality of QMCPACK. Future examples will likely include topics such as the addition of a jastrow function or add a new QMC method.

\input{coding_standards}

\section{Scalar estimator implementation}
\subsection{Introduction: Life of a specialized OperatorBase}

Almost all observables in QMCPACK are implemented as specialized derived classes of the OperatorBase base class. Each observable is instantiated in HamiltonianFactory and added to QMCHamiltonian for tracking. QMCHamiltonian tracks two types of observables: main and auxiliary. Main observables contribute to the local energy. These observables are elements of the simulated Hamiltonian such as kinetic or potential energy. Auxiliary observables are expectation values of matrix elements that do not contribute to the local energy. These Hamiltonians do not affect the dynamics of the simulation. In the code, the main observables are labeled by ``physical'' flag; the auxiliary observables have ``physical'' set to false.

\subsubsection{Initialization}
When an \verb|<estimator type="est_type" name="est_name" other_stuff="value"/>| tag is present in the \verb|<hamiltonian/>| section, it is first read by HamiltonianFactory. In general, the \verb|type| of the estimator will determine which specialization of OperatorBase should be instantiated, and a derived class with \verb|myName="est_name"| will be constructed. Then, the put() method of this specific class will be called to read any other parameters in the \verb|<estimator/>| XML node. Sometimes these parameters will instead be read by HamiltonianFactory because it can access more objects than OperatorBase.

\subsubsection{Cloning}
When \verb|OpenMP| threads are spawned, the estimator will be cloned by the \verb|CloneManager|, which is a parent class of many QMC drivers. 
\begin{lstlisting}[style=C++]
// In CloneManager.cpp
#pragma omp parallel for shared(w,psi,ham)
for(int ip=1; ip<NumThreads; ++ip)
{
  wClones[ip]=new MCWalkerConfiguration(w);
  psiClones[ip]=psi.makeClone(*wClones[ip]);
  hClones[ip]=ham.makeClone(*wClones[ip],*psiClones[ip]);
}
\end{lstlisting}
In the preceding snippet, \verb|ham| is the reference to the estimator on the master thread. If the implemented estimator does not allocate memory for any array, then the default constructor should suffice for the \verb|makeClone| method.
\begin{lstlisting}[style=C++]
// In SpeciesKineticEnergy.cpp
OperatorBase* SpeciesKineticEnergy::makeClone(ParticleSet& qp, TrialWaveFunction& psi)
{
  return new SpeciesKineticEnergy(*this);
}
\end{lstlisting}
If memory is allocated during estimator construction (usually when parsing the XML node in the \verb|put| method), then the \verb|makeClone| method should perform the same initialization on each thread.
\begin{lstlisting}[style=C++]
OperatorBase* LatticeDeviationEstimator::makeClone(ParticleSet& qp, TrialWaveFunction& psi)
{
  LatticeDeviationEstimator* myclone = new LatticeDeviationEstimator(qp,spset,tgroup,sgroup);
  myclone->put(input_xml);
  return myclone;
}
\end{lstlisting}

\subsubsection{Evaluate}
After the observable class (derived class of OperatorBase) is constructed and prepared (by the put() method), it is ready to be used in a QMCDriver. A QMCDriver will call \verb|H.auxHevaluate(W,thisWalker)| after every accepted move, where H is the QMCHamiltonian that holds all main and auxiliary Hamiltonian elements, W is a MCWalkerConfiguration, and thisWalker is a pointer to the current walker being worked on. As shown in the following, this function goes through each auxiliary Hamiltonian element and evaluates it using the current walker configuration. Under the hood, observables are calculated and dumped to the main particle set's property list for later collection.

\begin{lstlisting}[style=C++]
// In QMCHamiltonian.cpp
// This is more efficient. 
// Only calculate auxH elements if moves are accepted.
void QMCHamiltonian::auxHevaluate(ParticleSet& P, Walker_t& ThisWalker)
{
#if !defined(REMOVE_TRACEMANAGER)
  collect_walker_traces(ThisWalker,P.current_step);
#endif
  for(int i=0; i<auxH.size(); ++i)
  {
    auxH[i]->setHistories(ThisWalker);
    RealType sink = auxH[i]->evaluate(P);
    auxH[i]->setObservables(Observables);
#if !defined(REMOVE_TRACEMANAGER)
    auxH[i]->collect_scalar_traces();
#endif
    auxH[i]->setParticlePropertyList(P.PropertyList,myIndex);
  }
}
\end{lstlisting}

For estimators that contribute to the local energy (main observables), the return value of evaluate() is used in accumulating the local energy. For auxiliary estimators, the return value is not used (\verb|sink| local variable above); only the value of Value is recorded property lists by the setObservables() method as shown in the preceding code snippet. By default, the setObservables() method will transfer \verb|auxH[i]->Value| to \verb|P.PropertyList[auxH[i]->myIndex]|. The same property list is also kept by the particle set being moved by QMCDriver. This list is updated by \verb|auxH[i]->setParticlePropertyList(P.PropertyList,myIndex)|, where myIndex is the starting index of space allocated to this specific auxiliary Hamiltonian in the property list kept by the target particle set P.

\subsubsection{Collection}
The actual statistics are collected within the QMCDriver, which owns
an EstimatorManager object. This object (or a clone in the case of
multithreading) will be registered with each mover it owns. For each mover
(such as VMCUpdatePbyP derived from QMCUpdateBase), an accumulate() call
is made, which by default, makes an accumulate(walkerset) call to the
EstimatorManager it owns. Since each walker has a property set, EstimatorManager uses that local copy to calculate statistics. The EstimatorManager performs block averaging and file I/O.

\subsection{Single scalar estimator implementation guide}
Almost all of the defaults can be used for a single scalar observable. With any luck, only the put() and evaluate() methods need to be implemented. As an example, this section presents a step-by-step guide for implementing a \verb|SpeciesKineticEnergy| estimator that calculates the kinetic energy of a specific species instead of the entire particle set. For example, a possible input to this estimator can be:

\verb|<estimator type="specieskinetic" name="ukinetic" group="u"/>|

\verb|<estimator type="specieskinetic" name="dkinetic" group="d"/>|\:.

This should create two extra columns in the \inlinecode{scalar.dat} file that contains the kinetic energy of the up and down electrons in two separate columns. If the estimator is properly implemented, then the sum of these two columns should be equal to the default \verb|Kinetic| column.

\subsubsection{Barebone}

The first step is to create a barebone class structure for this simple scalar estimator. The goal is to be able to instantiate this scalar estimator with an XML node and have it print out a column of zeros in the \inlinecode{scalar.dat} file. 

To achieve this, first create a header file ``SpeciesKineticEnergy.h" in the QMCHamiltonians folder, with only the required functions declared as follows: 

\begin{lstlisting}[style=C++]
// In SpeciesKineticEnergy.h
#ifndef QMCPLUSPLUS_SPECIESKINETICENERGY_H
#define QMCPLUSPLUS_SPECIESKINETICENERGY_H

#include <Particle/WalkerSetRef.h>
#include <QMCHamiltonians/OperatorBase.h>

namespace qmcplusplus
{

class SpeciesKineticEnergy: public OperatorBase
{
public:
  
  SpeciesKineticEnergy(ParticleSet& P):tpset(P){ };
  
  bool put(xmlNodePtr cur);         // read input xml node, required
  bool get(std::ostream& os) const; // class description, required
  
  Return_t evaluate(ParticleSet& P);
  inline Return_t evaluate(ParticleSet& P, std::vector<NonLocalData>& Txy)
  { // delegate responsity inline for speed
    return evaluate(P);
  } 
  
  // pure virtual functions require overrider
  void resetTargetParticleSet(ParticleSet& P) { }                         // required
  OperatorBase* makeClone(ParticleSet& qp, TrialWaveFunction& psi); // required

private:
  ParticleSet& tpset;

}; // SpeciesKineticEnergy

} // namespace qmcplusplus
#endif
\end{lstlisting}

Notice that a local reference \verb|tpset| to the target particle set \verb|P| is saved in the constructor. The target particle set carries much information useful for calculating observables. Next, make ``SpeciesKineticEnergy.cpp," and make vacuous definitions.
\begin{lstlisting}[style=C++]
// In SpeciesKineticEnergy.cpp
#include <QMCHamiltonians/SpeciesKineticEnergy.h>
namespace qmcplusplus
{

bool SpeciesKineticEnergy::put(xmlNodePtr cur)
{
  return true;
} 

bool SpeciesKineticEnergy::get(std::ostream& os) const
{ 
  return true;
}

SpeciesKineticEnergy::Return_t SpeciesKineticEnergy::evaluate(ParticleSet& P)
{
  Value = 0.0;
  return Value;
}

OperatorBase* SpeciesKineticEnergy::makeClone(ParticleSet& qp, TrialWaveFunction& psi)
{
  // no local array allocated, default constructor should be enough
  return new SpeciesKineticEnergy(*this);
}

} // namespace qmcplusplus
\end{lstlisting}

Now, head over to HamiltonianFactory and instantiate this observable if an XML node is found requesting it. Look for ``gofr" in HamiltonianFactory.cpp, for example, and follow the if block.
\begin{lstlisting}[style=C++]
// In HamiltonianFactory.cpp
#include <QMCHamiltonians/SpeciesKineticEnergy.h>
else if(potType =="specieskinetic")
{        
  SpeciesKineticEnergy* apot = new SpeciesKineticEnergy(*target_particle_set);
  apot->put(cur);
  targetH->addOperator(apot,potName,false);
}
\end{lstlisting}
The last argument of addOperator() (i.e., the \verb|false| flag) is \textbf{crucial}. This tells QMCPACK that the observable we implemented is not a physical Hamiltonian; thus, it will not contribute to the local energy. Changes to the local energy will alter the dynamics of the simulation. Finally, add ``SpeciesKineticEnergy.cpp" to HAMSRCS in ``CMakeLists.txt" located in the QMCHamiltonians folder. Now, recompile QMCPACK and run it on an input that requests \verb|<estimator type="specieskinetic" name="ukinetic"/>| in the \verb|hamiltonian| block. A column of zeros should appear in the \inlinecode{scalar.dat} file under the name ``ukinetic."

\subsubsection{Evaluate}
The evaluate() method is where we perform the calculation of the desired observable. In a first iteration, we will simply hard-code the name and mass of the particles.
\begin{lstlisting}[style=C++]
// In SpeciesKineticEnergy.cpp
#include <QMCHamiltonians/BareKineticEnergy.h> // laplaician() defined here
SpeciesKineticEnergy::Return_t SpeciesKineticEnergy::evaluate(ParticleSet& P)
{
  std::string group="u";
  RealType minus_over_2m = -0.5;
  
  SpeciesSet& tspecies(P.getSpeciesSet());
  
  Value = 0.0;
  for (int iat=0; iat<P.getTotalNum(); iat++)
  {
    if (tspecies.speciesName[ P.GroupID(iat) ] == group)
    {
      Value += minus_over_2m*laplacian(P.G[iat],P.L[iat]);
    }
  }
  return Value;
  
  // Kinetic column has:
  // Value = -0.5*( Dot(P.G,P.G) + Sum(P.L) );
}
\end{lstlisting}
\textit{Voila}---you should now be able to compile QMCPACK, rerun, and see that the values in the ``ukinetic'' column are no longer zero. Now, the only task left to make this basic observable complete is to read in the extra parameters instead of hard-coding them.

\subsubsection{Parse extra input}
The preferred method to parse extra input parameters in the XML node is to implement the put() function of our specific observable. Suppose we wish to read in a single string that tells us whether to record the kinetic energy of the up electron (group=``u") or the down electron (group=``d"). This is easily achievable using the OhmmsAttributeSet class,
\begin{lstlisting}[style=C++]
// In SpeciesKineticEnergy.cpp
#include <OhmmsData/AttributeSet.h>
bool SpeciesKineticEnergy::put(xmlNodePtr cur)
{ 
  // read in extra parameter "group"
  OhmmsAttributeSet attrib;
  attrib.add(group,"group");
  attrib.put(cur);
  
  // save mass of specified group of particles
  SpeciesSet& tspecies(tpset.getSpeciesSet());
  int group_id  = tspecies.findSpecies(group);
  int massind   = tspecies.getAttribute("mass");
  minus_over_2m = -1./(2.*tspecies(massind,group_id));
  
  return true;
}
\end{lstlisting}
where we may keep ``group'' and ``minus\_over\_2m'' as local variables to our specific class.
\begin{lstlisting}[style=C++]
// In SpeciesKineticEnergy.h
private:
  ParticleSet& tpset;
  std::string  group;
  RealType minus_over_2m;
\end{lstlisting}
Notice that the previous operations are made possible by the saved reference to the target particle set. Last but not least, compile and run a full example (i.e., a short DMC calculation) with the following XML nodes in your input:

\verb|<estimator type="specieskinetic" name="ukinetic" group="u"/>|

\verb|<estimator type="specieskinetic" name="dkinetic" group="d"/>|\:.\\

Make sure the sum of the ``ukinetic" and ``dkinetic" columns is \textbf{exactly} the same as the Kinetic columns at \textbf{every block}.

For easy reference, a summary of the complete list of changes follows:
\begin{lstlisting}[style=C++]
// In HamiltonianFactory.cpp
#include "QMCHamiltonians/SpeciesKineticEnergy.h"
else if(potType =="specieskinetic")
{
	SpeciesKineticEnergy* apot = new SpeciesKineticEnergy(*targetPtcl);
	apot->put(cur);
	targetH->addOperator(apot,potName,false);
}
\end{lstlisting}
\begin{lstlisting}[style=C++]
// In SpeciesKineticEnergy.h
#include <Particle/WalkerSetRef.h>
#include <QMCHamiltonians/OperatorBase.h>

namespace qmcplusplus
{

class SpeciesKineticEnergy: public OperatorBase
{
public:

  SpeciesKineticEnergy(ParticleSet& P):tpset(P){ };

  // xml node is read by HamiltonianFactory, eg. the sum of following should be equivalent to Kinetic
  // <estimator name="ukinetic" type="specieskinetic" target="e" group="u"/>
  // <estimator name="dkinetic" type="specieskinetic" target="e" group="d"/>
  bool put(xmlNodePtr cur);         // read input xml node, required
  bool get(std::ostream& os) const; // class description, required
  
  Return_t evaluate(ParticleSet& P);
  inline Return_t evaluate(ParticleSet& P, std::vector<NonLocalData>& Txy)
  { // delegate responsity inline for speed
    return evaluate(P);
  } 
  
  // pure virtual functions require overrider
  void resetTargetParticleSet(ParticleSet& P) { }                         // required
  OperatorBase* makeClone(ParticleSet& qp, TrialWaveFunction& psi); // required
  
private:
  ParticleSet&       tpset; // reference to target particle set
  std::string        group; // name of species to track
  RealType   minus_over_2m; // mass of the species !! assume same mass
  // for multiple species, simply initialize multiple estimators
  
}; // SpeciesKineticEnergy

} // namespace qmcplusplus
#endif
\end{lstlisting}
\begin{lstlisting}[style=C++]
// In SpeciesKineticEnergy.cpp
#include <QMCHamiltonians/SpeciesKineticEnergy.h>
#include <QMCHamiltonians/BareKineticEnergy.h> // laplaician() defined here
#include <OhmmsData/AttributeSet.h>

namespace qmcplusplus
{

bool SpeciesKineticEnergy::put(xmlNodePtr cur)
{
  // read in extra parameter "group"
  OhmmsAttributeSet attrib;
  attrib.add(group,"group");
  attrib.put(cur);
  
  // save mass of specified group of particles
  int group_id  = tspecies.findSpecies(group);
  int massind   = tspecies.getAttribute("mass");
  minus_over_2m = -1./(2.*tspecies(massind,group_id)); 

  return true;
}

bool SpeciesKineticEnergy::get(std::ostream& os) const
{ // class description
  os << "SpeciesKineticEnergy: " << myName << " for species " << group;
  return true;
}

SpeciesKineticEnergy::Return_t SpeciesKineticEnergy::evaluate(ParticleSet& P)
{
  Value = 0.0;
  for (int iat=0; iat<P.getTotalNum(); iat++)
  {
    if (tspecies.speciesName[ P.GroupID(iat) ] == group)
    {
      Value += minus_over_2m*laplacian(P.G[iat],P.L[iat]);
    }
  }
  return Value;
}

OperatorBase* SpeciesKineticEnergy::makeClone(ParticleSet& qp, TrialWaveFunction& psi)
{ //default constructor
  return new SpeciesKineticEnergy(*this);
}

} // namespace qmcplusplus
\end{lstlisting}

\subsection{Multiple scalars}
It is fairly straightforward to create more than one column in the \inlinecode{scalar.dat} file with a single observable class. For example, if we want a single SpeciesKineticEnergy estimator to simultaneously record the kinetic energies of all species in the target particle set, we only have to write two new methods: addObservables() and setObservables(), then tweak the behavior of evaluate(). First, we will have to override the default behavior of addObservables() to make room for more than one column in the \inlinecode{scalar.dat} file as follows,
\begin{lstlisting}[style=C++]
// In SpeciesKineticEnergy.cpp
void SpeciesKineticEnergy::addObservables(PropertySetType& plist, BufferType& collectables)
{
  myIndex = plist.size();
  for (int ispec=0; ispec<num_species; ispec++)
  { // make columns named "$myName_u", "$myName_d" etc.
    plist.add(myName + "_" + species_names[ispec]);
  }
}
\end{lstlisting}
where ``num\_species'' and ``species\_name'' can be local variables initialized in the constructor. We should also initialize some local arrays to hold temporary data.
\begin{lstlisting}[style=C++]
// In SpeciesKineticEnergy.h
private:
  int num_species;
  std::vector<std::string> species_names;
  std::vector<RealType> species_kinetic,vec_minus_over_2m;
  
// In SpeciesKineticEnergy.cpp
SpeciesKineticEnergy::SpeciesKineticEnergy(ParticleSet& P):tpset(P)
{
  SpeciesSet& tspecies(P.getSpeciesSet());
  int massind = tspecies.getAttribute("mass");

  num_species = tspecies.size();
  species_kinetic.resize(num_species);
  vec_minus_over_2m.resize(num_species);
  species_names.resize(num_species);
  for (int ispec=0; ispec<num_species; ispec++)
  {
    species_names[ispec] = tspecies.speciesName[ispec];
    vec_minus_over_2m[ispec] = -1./(2.*tspecies(massind,ispec));   
  }
}
\end{lstlisting}
Next, we need to override the default behavior of \icode{setObservables()} to transfer multiple values to the property list kept by the main particle set, which eventually goes into the \inlinecode{scalar.dat} file.
\begin{lstlisting}[style=C++]
// In SpeciesKineticEnergy.cpp
void SpeciesKineticEnergy::setObservables(PropertySetType& plist)
{ // slots in plist must be allocated by addObservables() first
  copy(species_kinetic.begin(),species_kinetic.end(),plist.begin()+myIndex);
}
\end{lstlisting}
Finally, we need to change the behavior of evaluate() to fill the local vector ``species\_kinetic'' with appropriate observable values.
\begin{lstlisting}[style=C++]
SpeciesKineticEnergy::Return_t SpeciesKineticEnergy::evaluate(ParticleSet& P)
{
  std::fill(species_kinetic.begin(),species_kinetic.end(),0.0);

  for (int iat=0; iat<P.getTotalNum(); iat++)
  {
    int ispec = P.GroupID(iat);
    species_kinetic[ispec] += vec_minus_over_2m[ispec]*laplacian(P.G[iat],P.L[iat]);
  }
  
  Value = 0.0; // Value is no longer used
  return Value;
}
\end{lstlisting}
That's it! The SpeciesKineticEnergy estimator no longer needs the ``group'' input and will automatically output the kinetic energy of every species in the target particle set in multiple columns. You should now be able to run with 
\verb|<estimator type="specieskinetic" name="skinetic"/>| and check that the sum of all columns that start with ``skinetic'' is equal to the default ``Kinetic'' column.

\subsection{HDF5 output}
If we desire an observable that will output hundreds of scalars per simulation step (e.g., SkEstimator), then it is preferred to output to the \inlinecode{stat.h5} file instead of the \inlinecode{scalar.dat} file for better organization. A large chunk of data to be registered in the \inlinecode{stat.h5} file is called a ``Collectable'' in QMCPACK. In particular, if a OperatorBase object is initialized with \verb|UpdateMode.set(COLLECTABLE,1)|, then the ``Collectables'' object carried by the main particle set will be processed and written to the \inlinecode{stat.h5} file, where ``UpdateMode'' is a bit set (i.e., a collection of flags) with the following enumeration:
\begin{lstlisting}[style=C++]
// In OperatorBase.h
///enum for UpdateMode
enum {PRIMARY=0,
  OPTIMIZABLE=1,
  RATIOUPDATE=2,
  PHYSICAL=3,
  COLLECTABLE=4,
  NONLOCAL=5,
  VIRTUALMOVES=6
};
\end{lstlisting}

As a simple example, to put the two columns we produced in the previous section into the \inlinecode{stat.h5} file, we will first need to declare that our observable uses ``Collectables.''
\begin{lstlisting}[style=C++]
// In constructor add: 
hdf5_out = true;
UpdateMode.set(COLLECTABLE,1);
\end{lstlisting}
Then make some room in the ``Collectables'' object carried by the target particle set.
\begin{lstlisting}[style=C++]
// In addObservables(PropertySetType& plist, BufferType& collectables) add:
if (hdf5_out)
{
  h5_index = collectables.size();
  std::vector<RealType> tmp(num_species);
  collectables.add(tmp.begin(),tmp.end());
}
\end{lstlisting}
Next, make some room in the \inlinecode{stat.h5} file by overriding the registerCollectables() method.
\begin{lstlisting}[style=C++]
// In SpeciesKineticEnergy.cpp
void SpeciesKineticEnergy::registerCollectables(std::vector<observable_helper*>& h5desc, hid_t gid) const
{
  if (hdf5_out)
  {
    std::vector<int> ndim(1,num_species);
    observable_helper* h5o=new observable_helper(myName);
    h5o->set_dimensions(ndim,h5_index);
    h5o->open(gid);
    h5desc.push_back(h5o);
  }
}
\end{lstlisting}
Finally, edit evaluate() to use the space in the ``Collectables'' object.
\begin{lstlisting}[style=C++]
// In SpeciesKineticEnergy.cpp
SpeciesKineticEnergy::Return_t SpeciesKineticEnergy::evaluate(ParticleSet& P)
{
  RealType wgt = tWalker->Weight; // MUST explicitly use DMC weights in Collectables!
  std::fill(species_kinetic.begin(),species_kinetic.end(),0.0);

  for (int iat=0; iat<P.getTotalNum(); iat++)
  {
    int ispec = P.GroupID(iat);
    species_kinetic[ispec] += vec_minus_over_2m[ispec]*laplacian(P.G[iat],P.L[iat]);
    P.Collectables[h5_index + ispec] += vec_minus_over_2m[ispec]*laplacian(P.G[iat],P.L[iat])*wgt;
  }

  Value = 0.0; // Value is no longer used
  return Value;
}
\end{lstlisting}
There should now be a new entry in the \inlinecode{stat.h5} file containing the same columns of data as the \inlinecode{stat.h5} file. After this check, we should clean up the code by
\begin{enumerate}
\item making ``hdf5\_out'' and input flag by editing the put() method and
\item disabling output to \inlinecode{scalar.dat} when the ``hdf5\_out'' flag is on.
\end{enumerate}



\section{Estimator output}
\subsection{Estimator definition}
For simplicity, consider a local property $O(\bs{R})$, where $\bs{R}$ is the collection of all particle coordinates. An \textit{estimator} for $O(\bs{R}) $ is a weighted average over walkers:
\begin{align}
E[O] = \left(\sum\limits_{i=1}^{N^{tot}_{walker}} w_i O(\bs{R}_i) \right) / \left( \sum \limits_{i=1}^{N^{tot}_{walker}} w_i \right). \label{eq:estimator}
\end{align}
$N^{tot}_{walker}$ is the total number of walkers collected in the entire simulation. Notice that $N^{tot}_{walker}$ is typically far larger than the number of walkers held in memory at any given simulation step. $w_i$ is the weight of walker $i$.

In a VMC simulation, the weight of every walker is 1.0. Further, the number of walkers is constant at each step. Therefore, Equation~\ref{eq:estimator} simplifies to
\begin{align}
E_{VMC}[O] = \frac{1}{N_{step}N_{walker}^{ensemble}} \sum_{s,e} O(\bs{R}_{s,e})\:.
\end{align}
Each walker $\bs{R}_{s,e}$ is labeled by \textit{step index} s and \textit{ensemble index} e.

In a DMC simulation, the weight of each walker is different and may change from step to step. Further, the ensemble size varies from step to step. Therefore, Equation~\ref{eq:estimator} simplifies to
\begin{align}
E_{DMC}[O] = \frac{1}{N_{step}} \sum_{s} \left\{ \left(\sum_e w_{s,e} O(\bs{R}_{s,e})  \right) / \left( \sum \limits_{e} w_{s,e} \right)  \right\}\:.
\end{align}

We will refer to the average in the $\{\}$ as \textit{ensemble average} and to the remaining averages as \textit{block average}. The process of calculating $O(\bs{R})$ is \textit{evaluate}.

\subsection{Class relations}
A large number of classes are involved in the estimator collection process. They often have misleading class or method names. Check out the document gotchas in the following list:
\begin{enumerate}
\item \icode{EstimatorManager} is an unused copy of \icode{EstimatorManagerBase}. \icode{EstimatorManagerBase} is the class used in the QMC drivers. (PR \#371 explains this.)
\item \icode{EstimatorManagerBase::Estimators} is completely different from \icode{QMCDriver::Estimators}, which is subtly different from \icode{OperatorBase::Estimators}. The first is a list of pointers to \icode{ScalarEstimatorBase}. The second is the master estimator (one per MPI group). The third is the slave estimator that exists one per OpenMP thread.
\item \icode{QMCHamiltonian} is NOT a parent class of \icode{OperatorBase}. Instead, \icode{QMCHamiltonian} owns two lists of \icode{OperatorBase} named \icode{H} and \icode{auxH}.
\item \icode{QMCDriver::H} is NOT the same as \icode{QMCHamiltonian::H}. The first is a pointer to a \icode{QMCHamiltonian}. \icode{QMCHamiltonian::H} is a list.
\item \icode{EstimatorManager::stopBlock(std::vector)} is completely different from \icode{EstimatorManager::}
\icode{stopBlock(RealType)}, which is the same as \icode{stopBlock(RealType, true)} but that is subtly different from \icode{stopBlock(RealType, false)}. The first three methods are intended to be called by the master estimator, which exists one per MPI group. The last method is intended to be called by the slave estimator, which exists one per OpenMP thread.
\end{enumerate}

\subsection{Estimator output stages}
%In QMCPACK, evaluation is done by \icode{OperatorBase}; ensemble average is done either by a ``CloneDriver'' (e.g. \icode{VMCSingleOMP}, \icode{DMCOMP}) or \icode{ScalarEstimatorBase}; block average is done by \icode{ScalarEstimatorBase} or \icode{EstimatorManagerBase}. Walkers can be accessed by ``CloneDriver'' and \icode{OperatorBase} but not by \icode{EstimatorManagerBase} or \icode{ScalarEstimatorBase}. Output files can be accessed by the latter two classes but not the former two. Therefore, in order to output estimators to file, data must be transferred from \textit{evaluate} classes to \textit{average} classes.

Estimators take four conceptual stages to propagate to the output files: evaluate, load ensemble, unload ensemble, and collect. They are easier to understand in reverse order.

\subsubsection{Collect stage}
File output is performed by the master \icode{EstimatorManager} owned by \icode{QMCDriver}. The first 8+ entries in \icode{EstimatorManagerBase::AverageCache} will be written to \icode{scalar.dat}. The remaining entries in \icode{AverageCache} will be written to \icode{stat.h5}. File writing is triggered by \icode{EstimatorManagerBase}\\ \icode{::collectBlockAverages} inside \icode{EstimatorManagerBase::stopBlock}.

\begin{lstlisting}
// In EstimatorManagerBase.cpp::collectBlockAverages
  if(Archive)
  {
    *Archive << std::setw(10) << RecordCount;
    int maxobjs=std::min(BlockAverages.size(),max4ascii);
    for(int j=0; j<maxobjs; j++)
      *Archive << std::setw(FieldWidth) << AverageCache[j];
    for(int j=0; j<PropertyCache.size(); j++)
      *Archive << std::setw(FieldWidth) << PropertyCache[j];
    *Archive << std::endl;
    for(int o=0; o<h5desc.size(); ++o)
      h5desc[o]->write(AverageCache.data(),SquaredAverageCache.data());
    H5Fflush(h_file,H5F_SCOPE_LOCAL);
  }
\end{lstlisting}

\icode{EstimatorManagerBase::collectBlockAverages} is triggered from the master-thread estimator via either \icode{stopBlock(std::vector)} or \icode{stopBlock(RealType, true)}. Notice that file writing is NOT triggered by the slave-thread estimator method \icode{stopBlock(RealType, false)}.

\begin{lstlisting}
// In EstimatorManagerBase.cpp
void EstimatorManagerBase::stopBlock(RealType accept, bool collectall)
{
  //take block averages and update properties per block
  PropertyCache[weightInd]=BlockWeight;
  PropertyCache[cpuInd] = MyTimer.elapsed();
  PropertyCache[acceptInd] = accept;
  for(int i=0; i<Estimators.size(); i++)
    Estimators[i]->takeBlockAverage(AverageCache.begin(),SquaredAverageCache.begin());
  if(Collectables)
  { 
    Collectables->takeBlockAverage(AverageCache.begin(),SquaredAverageCache.begin());
  }
  if(collectall)
    collectBlockAverages(1);
}
\end{lstlisting}

\begin{lstlisting}
// In ScalarEstimatorBase.h
template<typename IT>
inline void takeBlockAverage(IT first, IT first_sq)
{
  first += FirstIndex;
  first_sq += FirstIndex;
  for(int i=0; i<scalars.size(); i++)
  {
    *first++ = scalars[i].mean();
    *first_sq++ = scalars[i].mean2();
    scalars_saved[i]=scalars[i]; //save current block
    scalars[i].clear();
  }
}
\end{lstlisting}

At the collect stage, \icode{ScalarEstimatorBase::scalars} must be populated with ensemble-averaged data. Two derived classes of \icode{ScalarEstimatorBase} are crucial: \icode{LocalEnergyEstimator} will carry \icode{Properties}, where as \icode{CollectablesEstimator} will carry \icode{Collectables}.

\subsubsection{Unload ensemble stage}
\icode{LocalEnergyEstimator::scalars} are populated by
\icode{ScalarEstimatorBase::accumulate}, whereas
\icode{CollectablesEstimator::scalars} are populated by
\icode{CollectablesEstimator::} \icode{accumulate_all}. Both
accumulate methods are triggered by
\icode{EstimatorManagerBase::accumulate}. One confusing aspect about
the unload stage is that \icode{EstimatorManagerBase::accumulate} has
a master and a slave call signature. A slave estimator such as
\icode{QMCUpdateBase::Estimators} should unload a subset of
walkers. Thus, the slave estimator should call
\icode{accumulate(W,it,it_end)}. However, the master estimator, such
as \icode{SimpleFixedNodeBranch::myEstimator}, should unload data from
the entire walker ensemble. This is achieved by calling
\icode{accumulate(W)}.

\begin{lstlisting}
void EstimatorManagerBase::accumulate(MCWalkerConfiguration& W)
{ // intended to be called by master estimator only
  BlockWeight += W.getActiveWalkers();
  RealType norm=1.0/W.getGlobalNumWalkers();
  for(int i=0; i< Estimators.size(); i++)
    Estimators[i]->accumulate(W,W.begin(),W.end(),norm);
  if(Collectables)//collectables are normalized by QMC drivers
    Collectables->accumulate_all(W.Collectables,1.0);
}
\end{lstlisting}

\begin{lstlisting}
void EstimatorManagerBase::accumulate(MCWalkerConfiguration& W
 , MCWalkerConfiguration::iterator it
 , MCWalkerConfiguration::iterator it_end)
{ // intended to be called slaveEstimator only
  BlockWeight += it_end-it;
  RealType norm=1.0/W.getGlobalNumWalkers();
  for(int i=0; i< Estimators.size(); i++)
    Estimators[i]->accumulate(W,it,it_end,norm);
  if(Collectables)
    Collectables->accumulate_all(W.Collectables,1.0);
}
\end{lstlisting}

\begin{lstlisting}
// In LocalEnergyEstimator.h
inline void accumulate(const Walker_t& awalker, RealType wgt)
{ // ensemble average W.Properties
  // expect ePtr to be W.Properties; expect wgt = 1/GlobalNumberOfWalkers
  const RealType* restrict ePtr = awalker.getPropertyBase();
  RealType wwght= wgt* awalker.Weight;
  scalars[0](ePtr[LOCALENERGY],wwght);
  scalars[1](ePtr[LOCALENERGY]*ePtr[LOCALENERGY],wwght);
  scalars[2](ePtr[LOCALPOTENTIAL],wwght);
  for(int target=3, source=FirstHamiltonian; target<scalars.size(); ++target, ++source)
    scalars[target](ePtr[source],wwght);
}
\end{lstlisting}

\begin{lstlisting}
// In CollectablesEstimator.h
inline void accumulate_all(const MCWalkerConfiguration::Buffer_t& data, RealType wgt)
{ // ensemble average W.Collectables
  // expect data to be W.Collectables; expect wgt = 1.0
  for(int i=0; i<data.size(); ++i)
    scalars[i](data[i], wgt);
}
\end{lstlisting}

At the unload ensemble stage, the data structures \icode{Properties} and \icode{Collectables} must be populated by appropriately normalized values so that the ensemble average can be correctly taken. \icode{QMCDriver} is responsible for the correct loading of data onto the walker ensemble.

\subsubsection{Load ensemble stage}
\icode{Properties} in the MC ensemble of walkers \icode{QMCDriver::W} is populated by \icode{QMCHamiltonian}\\ \icode{::saveProperties}. The master \icode{QMCHamiltonian::LocalEnergy}, \icode{::KineticEnergy}, and \icode{::Observables} must be properly populated at the end of the evaluate stage.
\begin{lstlisting}
// In QMCHamiltonian.h
  template<class IT>
  inline
  void saveProperty(IT first)
  { // expect first to be W.Properties
    first[LOCALPOTENTIAL]= LocalEnergy-KineticEnergy;
    copy(Observables.begin(),Observables.end(),first+myIndex);
  }
\end{lstlisting}

\icode{Collectables}'s load stage is combined with its evaluate stage.

\subsubsection{Evaluate stage}

The master \icode{QMCHamiltonian::Observables} is populated by slave \icode{OperatorBase}
\icode{::setObservables}. However, the call signature must be \icode{OperatorBase::setObservables}
\icode{(QMCHamiltonian::} \\\icode{Observables)}. This call signature is enforced by \icode{QMCHamiltonian::evaluate} and \icode{QMCHamiltonian::} \\\icode{auxHevaluate}.

\begin{lstlisting}
// In QMCHamiltonian.cpp
QMCHamiltonian::Return_t
QMCHamiltonian::evaluate(ParticleSet& P)
{
  LocalEnergy = 0.0;
  for(int i=0; i<H.size(); ++i)
  {
    myTimers[i]->start();
    LocalEnergy += H[i]->evaluate(P);
    H[i]->setObservables(Observables);
#if !defined(REMOVE_TRACEMANAGER)
    H[i]->collect_scalar_traces();
#endif
    myTimers[i]->stop();
    H[i]->setParticlePropertyList(P.PropertyList,myIndex);
  }
  KineticEnergy=H[0]->Value;
  P.PropertyList[LOCALENERGY]=LocalEnergy;
  P.PropertyList[LOCALPOTENTIAL]=LocalEnergy-KineticEnergy;
  // auxHevaluate(P);
  return LocalEnergy;
}
\end{lstlisting}

\begin{lstlisting}
// In QMCHamiltonian.cpp
void QMCHamiltonian::auxHevaluate(ParticleSet& P, Walker_t& ThisWalker)
{
#if !defined(REMOVE_TRACEMANAGER)
  collect_walker_traces(ThisWalker,P.current_step);
#endif
  for(int i=0; i<auxH.size(); ++i)
  {
    auxH[i]->setHistories(ThisWalker);
    RealType sink = auxH[i]->evaluate(P);
    auxH[i]->setObservables(Observables);
#if !defined(REMOVE_TRACEMANAGER)
    auxH[i]->collect_scalar_traces();
#endif
    auxH[i]->setParticlePropertyList(P.PropertyList,myIndex);
  }
}
\end{lstlisting}

\subsection{Estimator use cases}

\subsubsection{VMCSingleOMP pseudo code}
\begin{lstlisting}
bool VMCSingleOMP::run()
{
  masterEstimator->start(nBlocks);
  for (int ip=0; ip<NumThreads; ++ip)
    Movers[ip]->startRun(nBlocks,false);  // slaveEstimator->start(blocks, record)
  
  do // block
  {
    #pragma omp parallel
    {
      Movers[ip]->startBlock(nSteps);  // slaveEstimator->startBlock(steps)
      RealType cnorm = 1.0/static_cast<RealType>(wPerNode[ip+1]-wPerNode[ip]);
      do // step
      {
        wClones[ip]->resetCollectables();
        Movers[ip]->advanceWalkers(wit, wit_end, recompute);
        wClones[ip]->Collectables *= cnorm;
        Movers[ip]->accumulate(wit, wit_end);
      } // end step
      Movers[ip]->stopBlock(false);  // slaveEstimator->stopBlock(acc, false)
    } // end omp
    masterEstimator->stopBlock(estimatorClones);  // write files
  } // end block
  masterEstimator->stop(estimatorClones);
}
\end{lstlisting}

\subsubsection{DMCOMP  pseudo code}
\begin{lstlisting}
bool DMCOMP::run()
{
  masterEstimator->setCollectionMode(true);
  
  masterEstimator->start(nBlocks);
  for(int ip=0; ip<NumThreads; ip++)
    Movers[ip]->startRun(nBlocks,false);  // slaveEstimator->start(blocks, record)
  
  do // block
  {
    masterEstimator->startBlock(nSteps);
    for(int ip=0; ip<NumThreads; ip++)
      Movers[ip]->startBlock(nSteps);  // slaveEstimator->startBlock(steps)
    
    do // step
    {
      #pragma omp parallel
      {
      wClones[ip]->resetCollectables();
      // advanceWalkers
      } // end omp
      
      //branchEngine->branch
      { // In WalkerControlMPI.cpp::branch
      wgt_inv=WalkerController->NumContexts/WalkerController->EnsembleProperty.Weight;
      walkers.Collectables *= wgt_inv;
      slaveEstimator->accumulate(walkers);
      }
      masterEstimator->stopBlock(acc)  // write files
    }  // end for step
  }  // end for block
  
  masterEstimator->stop();
}
\end{lstlisting}

\subsection{Summary}

Two ensemble-level data structures, \icode{ParticleSet::Properties} and \icode{::Collectables}, serve as intermediaries between evaluate classes and output classes to \icode{scalar.dat} and \icode{stat.h5}. \icode{Properties} appears in both \icode{scalar.dat} and \icode{stat.h5}, whereas \icode{Collectables} appears only in \icode{stat.h5}. \icode{Properties} is overwritten by \icode{QMCHamiltonian::Observables} at the end of each step. \icode{QMCHamiltonian::Observables} is filled upon call to \icode{QMCHamiltonian::evaluate} and \icode{::auxHevaluate}. \icode{Collectables} is zeroed at the beginning of each step and accumulated upon call to \icode{::auxHevaluate}.

Data are output to \icode{scalar.dat} in four stages: evaluate, load, unload, and collect. In the evaluate stage, \icode{QMCHamiltonian::Observables} is populated by a list of \icode{OperatorBase}. In the load stage, \icode{QMCHamiltonian::Observables} is transfered to \icode{Properties} by \icode{QMCDriver}. In the unload stage, \icode{Properties} is copied to \icode{LocalEnergyEstimator::scalars}. In the collect stage, \icode{LocalEnergyEstimator::scalars} is block-averaged to \icode{EstimatorManagerBase}\\ \icode{::AverageCache} and dumped to file. For \icode{Collectables}, the evaluate and load stages are combined in a call to \icode{QMCHamiltonian::auxHevaluate}. In the unload stage, \icode{Collectables} is copied to \icode{CollectablesEstimator::scalars}. In the collect stage, \icode{CollectablesEstimator}\\ \icode{::scalars} is block-averaged to \icode{EstimatorManagerBase::AverageCache} and dumped to file.

\subsection{Appendix: dmc.dat}

\begin{sloppypar}
There is an additional data structure, \icode{ParticleSet::EnsembleProperty}, that is managed by \icode{WalkerControlBase::EnsembleProperty} and directly dumped to \icode{dmc.dat} via its own averaging procedure. \icode{dmc.dat} is written by \icode{WalkerControlBase::measureProperties}, which is called by \icode{WalkerControlBase::branch}, which is called by \icode{SimpleFixedNodeBranch}\\ \icode{::branch}, for example.
\end{sloppypar}
\input{backflow_implementation}
\section{Adding a wavefunction}
\label{sec:adding_wavefunction}

The total wavefunction is stored in \texttt{TrialWaveFunction} as a product of all
the components.  Each component derives from \texttt{WaveFunctionComponent}.
The code contains an example of a wavefunction component for a Helium atom using a simple form and
is described in Section \ref{sec:helium_wavefunction_example}.


\subsection{Mathematical preliminaries}

The wavefunction evaluation functions compute the log of the wavefunction,
the gradient and the Laplacian of the log of the wavefunction.
Expanded, the gradient and Laplacian are
\begin{eqnarray}
G &=& \grad \log(\psi) = \frac{\grad \psi}{\psi} \\
L &=& \lap \log(\psi) = \frac{\lap \psi}{\psi} - \frac{\grad \psi}{\psi} \cdot \frac{\grad \psi}{\psi} \\
                &=& \frac{\lap \psi}{\psi} - G \cdot G
\end{eqnarray}

However, the local energy formula needs $\frac{\lap \psi}{\psi}$.  The conversion from the Laplacian of the log of the wavefunction to the local energy value is performed
in \texttt{QMCHamiltonians/BareKineticEnergy.h} (i.e. $L + G \cdot G$.)


\subsection{Wavefunction evaluation}

The process for creating a new wavefunction component class is to derive
from WaveFunctionComponent and implement a number pure virtual functions.
To start most of them can be empty.

The following four functions evaluate the wavefunction values and spatial derivatives:

\begin{description}
\item{\texttt{evaluateLog}} Computes the log of the wavefunction and the gradient
and Laplacian (of the log of the wavefunction) for all particles.
The input is the\texttt{ParticleSet}(\texttt{P}) (of the electrons).
The return value is the log of wavefunction, and the gradient is in \texttt{G} and Laplacian in \texttt{L}.

\item{\texttt{ratio}} Computes the wavefunction ratio (not the log) for a single particle move ($\psi_{new}/\psi_{old}$).
The inputs are the \texttt{ParticleSet}(\texttt{P}) and the particle index (\texttt{iat}).

\item{\texttt{evalGrad}} Computes the gradient for a given particle.
The inputs are the \texttt{ParticleSet}(\texttt{P}) and the particle index (\texttt{iat}).

\item{\texttt{ratioGrad}} Computes the wavefunction ratio and the gradient at the new position for a single particle move.
The inputs are the \texttt{ParticleSet}(\texttt{P}) and the particle index (\texttt{iat}).
The output gradient is in \texttt{grad\_iat};
\end{description}

The \texttt{updateBuffer} function needs to be implemented, but to start it can simply
call \texttt{evaluateLog}.

The \texttt{put} function should be implemented to read parameter specifics from the input XML file.

\subsection{Function use}

For debugging it can be helpful to know the under what conditions the various
routines are called.

The VMC and DMC loops initialize the walkers by calling \texttt{evaluateLog}.
For all-electron moves, each timestep advance calls \texttt{evaluateLog}.
If the \texttt{use\_drift} parameter is no, then only the wavefunction value is used for sampling.
The gradient and Laplacian are used for computing the local energy.

For particle-by-particle moves, each timestep advance
\begin{enumerate}
\item calls \texttt{evalGrad}
\item computes a trial move
\item calls \texttt{ratioGrad} for the wavefunction ratio and the gradient at the trial position.
(If the \texttt{use\_drift} parameter is no, the \texttt{ratio} function is called instead.)
\end{enumerate}


The following example shows part of an input block for VMC with all-electron moves and drift.

\begin{verbatim}
<qmc method="vmc" target="e" move="alle">
  <parameter name="use_drift">yes</parameter>
</qmc>
\end{verbatim}



\subsection{Particle distances}

The \texttt{ParticleSet} parameter in these functions refers to the electrons.
The distance tables that store the inter-particle distances are stored as an array.
The first distance table is
always for election-electron distances (in general, the first table is for distances between the same particles).  It is accessed with the following.
\begin{verbatim}
const DistanceTableData *ee_table = P.DistTables[0];
\end{verbatim}
From there the distances and displacements can obtained.

The lower triangle for the electron-electron table should be used.
It is the only part of the distance table valid during particle-by-particle updates.

To get the ion-electron distances, add the ion \texttt{ParticleSet} using \texttt{addTable}
and save the returned index.
Use that index into distance table array to get the ion-electron distance table.

In \texttt{ratioGrad}, the new distances are stored in the \texttt{Temp\_r} and \texttt{Temp\_dr}
members of the distance tables.

\subsection{Setup}

A builder processes XML input, creates the wavefunction, and adds it to \texttt{targetPsi}.
Builders derive from \texttt{WaveFunctionComponentBuilder}.

The new builder hooks into the XML processing in \texttt{WaveFunctionFactory.cpp} in the \texttt{build} function.


\subsection{Caching values}
The \texttt{acceptMove} and \texttt{restore} methods are called on accepted and rejected moves for
the component to update cached values.


\subsection{Threading}
The \texttt{makeClone} function needs to be implemented to work correctly with OpenMP threading.
There will be one copy of the component created for each thread.
If there is no extra storage, calling the copy constructor will be sufficient.
If there are cached values, the clone call may need to create space.


\subsection{Parameter optimization}

The \texttt{checkInVariables}, \texttt{checkOutVariables}, and \texttt{resetParameters} functions manage the variational parameters.
Optimizable variables also need to be registered when the XML is processed.

Variational parameter derivatives are computed in the \texttt{evaluateDerivatives} function.
The first output value is an array with parameter derivatives of log of the wavefunction.
The second output values is an array with parameter derivatives of
the Laplacian divided by the wavefunction (and not the Laplacian of the log of the wavefunction)
The kinetic energy term contains a $-1/2m$ factor.
The $1/m$ factor is applied in \texttt{TrialWaveFunction.cpp}, but the $-1/2$ is not and must be accounted for in this function.


%See \texttt{QMCWaveFunctions/Jastrow/DiffTwoBodyJastrowOrbital.h} for an
%application.
%Also note that it calls into the radial functions for derivatives, but
%it expects the derivative of the logs.

%The conversion is
%\begin{equation}
%\frac{\partial }{\partial B} = \frac{\partial}{\partial B} \lap \log(\psi) +
%2 \frac{\grad \psi}{\psi} \frac{\partial}{\partial B} \frac{\grad \psi}{\psi}
%\end{equation}

\subsection{Helium Wavefunction Example}
\label{sec:helium_wavefunction_example}
The code contains an example of a wavefunction component for a Helium atom using STO orbitals and a Pade Jastrow.

The wavefunction is
\begin{equation}
\psi = \frac{1}{\sqrt{\pi}} \exp(-Z r_1) \exp(-Z r_2) \exp(A / (1 + B r_{12}))
\end{equation}
where $Z = 2$ is the nuclear charge, $A=1/2$ is the electron-electron cusp, and $B$ is a variational parameter.
The electron-ion distances are $r_1$ and $r_2$, and $r_{12}$ is the electron-electron distance.
The wavefunction is the same as the one expressed with built-in components in \texttt{examples/molecules/He/he\_simple\_opt.xml}.

The code is in \texttt{src/QMCWaveFunctions/ExampleHeComponent.cpp}.
The builder is in \texttt{src/QMCWaveFunctions/ExampleHeBuilder.cpp}.
The input file is in \texttt{examples/molecules/He/he\_example\_wf.xml}.
A unit test compares results from the wavefunction evaluation functions for
consistency in \texttt{src/QMCWaveFunctions/tests/test\_example\_he.cpp}.

The recommended approach for creating a new wavefunction component is to copy the example and the unit test.
Implement the evaluation functions and ensure the unit test passes.



%% \renewcommand{\chaptername}{}
%% \renewcommand{\thechapter}{}
\chapter*{References}
\addcontentsline{toc}{chapter}{References}
\begin{btSect}{bibliography}
\btPrintCited
\end{btSect}
\end{btUnit}

\appendix
\chapter{Derivation of twist averaging efficiency}
\label{sec:app_ta_efficiency}
In this appendix we derive the relative statistical efficiency of 
twist averaging with an irreducible (weighted) set of k-points 
versus using uniform weights over an unreduced set of k-points 
(\emph{e.g.} a full Monkhorst-Pack mesh).

Consider the weighted average of a set of statistical variables 
$\{x_m\}$ with weights $\{w_m\}$:
\begin{align}
  x_{TA} = \frac{\sum_mw_mx_m}{\sum_mw_m}
\end{align} 
If produced by a finite quantum Monte Carlo run at a set of 
twist angles/k-points $\{k_m\}$, each variable mean $\mean{x_m}$ 
has a statistical error bar $\sigma_m$ and we can also obtain 
the statistical error bar of the mean of the twist averaged 
quantity $\mean{x_{TA}}$:
\begin{align}
  \sigma_{TA} = \frac{\left(\sum_mw_m^2\sigma_m^2\right)^{1/2}}{\sum_mw_m}
\end{align}
The error bar of each individual twist $\sigma_m$, is related to the 
autocorrelation time $\kappa_m$,  intrinsic variance $v_m$, and number 
of post-equilibration Monte Carlo steps $N_{step}$ in the following way:
\begin{align}
  \sigma_m^2=\frac{\kappa_mv_m}{N_{step}}
\end{align}
In the setting of twist averaging, the autocorrelation time and 
variance for different twist angles are often very similar across 
twists and we have
\begin{align}
  \sigma_m^2=\sigma^2=\frac{\kappa v}{N_{step}}
\end{align} 
If we define the total weight as $W$, \emph{i.e.} $W\equiv\sum_{m=1}^Mw_m$, 
for the weighted case with $M$ irreducible twists the error bar is
\begin{align}
  \sigma_{TA}^{weighted}=\frac{\left(\sum_{m=1}^Mw_m^2\right)^{1/2}}{W}\sigma
\end{align}
For uniform weighting with $w_m=1$ the number of twists is $W$ and 
we have
\begin{align}
  \sigma_{TA}^{uniform}=\frac{1}{\sqrt{W}}\sigma
\end{align}
We are interested in comparing the efficiency of choosing weights 
uniformly or based on the irreducible multiplicity of each twist angle 
for a given target error bar $\sigma_{target}$.  The number of Monte Carlo  
steps required to reach this target for uniform weighting is
\begin{align}
  N_{step}^{uniform} = \frac{1}{W}\frac{\kappa v}{\sigma_{target}^2}
\end{align}
while for non-uniform weighting we have
\begin{align}\label{eq:weighted_step}
  N_{step}^{weighted} &= \frac{\sum_{m=1}^Mw_m^2}{W^2}\frac{\kappa v}{\sigma_{target}^2} \nonumber\\
                  &=\frac{\sum_{m=1}^Mw_m^2}{W}N_{step}^{uniform}
\end{align}
The Monte Carlo efficiency is defined as 
\begin{align}
  \xi = \frac{1}{\sigma^2t}
\end{align}
where $\sigma$ is the error bar and $t$ is the total cpu time required 
for the Monte Carlo run.  

The main advantage made possible by irreducible twist weighting is to 
reduce the equilibration time overhead by having fewer twists, and 
hence fewer Monte Carlo runs to equilibrate.  In the context of twist 
averaging, the total cpu time for a run can be considered to be
\begin{align}
  t=N_{twist}(N_{eq}+N_{step})t_{step}
\end{align}
where $N_{twist}$ is the number of twists, $N_{eq}$ is the number of Monte 
Carlo steps required to reach equilibrium, $N_{step}$ is the number 
of Monte Carlo steps included in the statisical averaging as before, 
and $t_{step}$ is the wall clock time required to complete a single 
Monte Carlo step. For uniform weighting $N_{twist}=W$ while for irreducible 
weighting $N_{twist}=M$.

We can now calculate the relative efficiency ($\eta$) of irreducible vs. 
uniform twist weighting with the aim of obtaining a target error bar 
$\sigma_{target}$:
\begin{align}
  \eta &= \frac{\xi_{TA}^{weighted}}{\xi_{TA}^{uniform}} \nonumber \\
       &= \frac{\sigma_{target}^2t_{TA}^{uniform}}{\sigma_{target}^2t_{TA}^{weighted}} \nonumber \\
       &= \frac{W(N_{eq}+N_{step}^{uniform})}{M(N_{eq}+N_{step}^{weighted})} \nonumber \\
       &= \frac{W(N_{eq}+N_{step}^{uniform})}{M(N_{eq}+\frac{\sum_{m=1}^Mw_m^2}{W}N_{step}^{uniform})} \nonumber \\
       &= \frac{W}{M}\frac{1+f}{1+\frac{\sum_{m=1}^Mw_m^2}{W}f}
\end{align}
In this last expression, $f$ is the ratio of the number of usable 
Monte Carlo steps to the number that must be discarded during equilibration 
($f=N_{step}^{uniform}/N_{eq}$) and as before $W=\sum_mw_m$, which is the number of 
twist angles in the uniform weighting case.  It is important to recall 
that $N_{step}^{uniform}$ in $f$ is defined relative to uniform weighting; it is 
the number of Monte Carlo steps required to reach a target accuracy in the 
case of uniform twist weights.

The formula for $\eta$ above can be easily changed with the help of 
Eq. \ref{eq:weighted_step} to reflect the 
number of Monte Carlo steps obtained in an irreducibly weighted run 
instead.  A good exercise is to consider runs one has already completed 
with either uniform or irreducible weighting and calculate the 
expected efficiency change had the opposite type of weighting been used.

The break even point $(\eta=1)$ can be found at a usable step fraction of 
\begin{align}
  f=\frac{W-M}{M\frac{\sum_{m=1}^Mw_m^2}{W}-W}
\end{align}

The relative efficiency $(\eta)$ above is useful to consider in view of certain 
scenarios.  An important case is where the number of required sampling 
steps is no larger than the number of equilibration steps, \emph{i.e.} 
$f\approx 1$.  For a very simple case with 8 uniform twists with 
irreducible multiplicities of $w_m\in\{1,3,3,1\}$ ($W=8$, $M=4$), the 
relative efficiency of irreducible vs. uniform weighting is 
$\eta=\frac{8}{4}\frac{2}{1+20/8}\approx 1.14$.  In this case, 
irreducible weighting is about $14$\% more efficient than uniform weighting.

Another interesting case is where the number of sampling steps you can 
reach with uniform twists before wall clock time runs out is small 
relative to the number of equilibration steps ($f\rightarrow 0$). 
In this limit $\eta\approx W/M$.  For our 8 uniform twist example, this would 
result in a relative efficiency of $\eta=8/4=2$ making irreducible 
weighting twice as efficient.

A final case of interest is where the equilibration time is short 
relative to the available sampling time $(f\rightarrow\infty)$, 
giving $\eta\approx W^2/(M\sum_{m=1}^Mw_m^2)$.  Again for our simple example 
we find $\eta=8^2/(4\times 20)\approx 0.8$, with uniform weighting being 
$25$\% more efficient than irreducible.  

For this simple example, the crossover point for irreducible weighting being 
more efficient than uniform weighting is $f<2$, \emph{i.e.} when the 
available sampling period is less than twice the length of the equilibration 
period.  The expected efficiency ratio and crossover point should be checked 
for the particular case under consideration to inform the choice between   
twist averaging methods.




\begin{btUnit}
\chapter{QMCPACK papers}
% Uset the bibtopic package to generate a bibliography local to this btsection below.
% btPrintAll creates the bibliography citing all entries in the files qmcpack_papers.bib
\begin{btSect}{qmcpack_papers}

  The following is a list of all papers, theses, and book chapters
  known to use QMCPACK. Please let the developers know if your paper
  is missing, if you know of other works, or an entry is incorrect. We
  list papers whether they cite QMCPACK directly or not. This list
  will be placed on the \url{http://www.qmcpack.org} website.

\btPrintAll

\end{btSect}
 
\end{btUnit}
\end{document}
